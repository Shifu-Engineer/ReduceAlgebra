
% From hearn@rand.orgSat Sep 16 16:53:08 1995
% Date: Fri, 15 Sep 95 10:10:44 -0700
% From: Tony Hearn <hearn@rand.org>
% To: John Fitch <jpff@maths.bath.ac.uk>, Arthur Norman <acn1@cam.ac.uk>,
%     Winfried Neun <neun@sc.zib-berlin.de>
% Subject: Final announcement
% 
% I hope this is ok:

\documentstyle[11pt]{article}
\setlength{\parindent}{0pt}
\setlength{\parskip}{6pt}
\raggedbottom
\setlength{\textwidth}{6.2in}
\setlength{\oddsidemargin}{0.2in}
\setlength{\evensidemargin}{0in}
\setlength{\textheight}{8.7in}
\setlength{\topmargin}{-0.5in}
\newlength{\redboxwidth}
\setlength{\redboxwidth}{4in}
\newcommand{\REDUCE}{REDUCE}
\pagestyle{empty}
\begin{document}
\begin{center}
\fbox{\rule[-0.2cm]{0cm}{0.8cm}{\Large \bf ANNOUNCING REDUCE 3.6}}%
 \rule[-0.31cm]{3pt}{1.0cm}\\[-0.6mm]
 \hspace*{1pt}\rule{82mm}{3pt}\\[0.2in]
\end{center}

Version 3.6 of REDUCE is now available for distribution.  This is the
first major update since the release of {\REDUCE} 3.5 in October 1993.  As
is usual for a new release, a large number of bugs and awkward features
(including those documented in the {\em patches.red} file available from the
REDUCE Network Library) have been corrected.  Taken together with the many
new features that have been added, {\REDUCE} 3.6 represents a significant
enhancement over previous versions.

In addition to the capabilities of the original release of {\REDUCE} 3.5,
this new version supports, among other things:

\vspace*{-0.1in}
\begin{itemize}
\item {definite integration}
\item {noncommutative Gr\"obner bases}
\item {expanded special function handling}
\item {improved solve capabilities}
\item {improved trigonometric simplification}
\item {linear algebra and linear programming}
\item {matrix normal forms}
\item {operations on sets}
\item {residue computations.}
\end{itemize}

The {\REDUCE} algebraic mode has been improved substantially since the
last release, and in particular offers improved rule list capabilities,
including free operators, conditional binding of variables and better
matching facilities for quotients.  The {\REDUCE} graphics interface has
also been improved.  For example, a user is now able to plot an implicitly
defined function.

A large number of different people are responsible for these improvements.
Other special purpose packages contributed by users include:

\begin{itemize}
\item {APPLYSYM: Infinitesimal symmetries of differential equations}

\item {BOOLEAN: Boolean algebra}

\item {DUMMY: Canonical form of expressions with dummy variables}

\item {FPS: Calculation of formal power series}

\item {INVBASE: Computation of involutive bases}

\item {NCPOLY: Non--commutative polynomial ideals}

\item {NORMFORM: Computation of matrix normal forms}

\item {RANDPOLY: A random polynomial generator}

\item {XCOLOR: Color factor in non-abelian gauge field theories}

\item {XIDEAL: Gr\"obner Bases for exterior algebra}

\item {ZEILBERG: Indefinite and definite summation}

\item{ZTRANS: Calculations with the $Z$ and inverse $Z$ transform.}
\end{itemize}

Updated documentation includes an improved User's Manual in {\LaTeX}
format, a more detailed online help system for MS/Windows and Unix/X11
Systems and a bibliography listing over 800 references to REDUCE-related
publications.

A complete information package is obtainable from:
% \vspace*{-0.1in}
\begin{quote}
       REDUCE Secretary \\
       RAND \\
       1700 Main Street \\
       P.O. Box 2138 \\
       Santa Monica CA 90407-2138 U.S.A. \\
       Telephone: +1-310-393-0411 Ext. 7681 \\
       Facsimile: +1-310-393-4818 \\
       Electronic Mail: reduce@rand.org
\end{quote}
\vspace*{-0.1in}

If you have e-mail access to the Internet, you can also obtain current
information by sending the message {\em send info-package} to
reduce-netlib@rand.org, reduce-netlib@can.nl or
reduce-netlib@pi.cc.u-tokyo.ac.jp.  The single line message can either be
the subject of the message or the body.  This message is answered by an
automated server for the REDUCE network library.  The library will in time
contain any packages made available since the release of REDUCE 3.6 and
patches to correct any bugs that may be discovered.  Further information
on this library, as well as instructions on how to join the REDUCE
electronic forum, can be obtained by including the word {\em help} on a
separate line in the message.

The same information is available from an Internet gopher server with
the address info.rand.org.  The network library files are in a ``REDUCE
Library'' directory under the directory ``Publicly Available Software''.
The relevant URL is gopher://info.rand.org/11/software/reduce .

A World Wide Web REDUCE server with URL http://www.rrz.uni-koeln.de/REDUCE/
is also supported.  In addition to general information about REDUCE, this
server has pointers to the network library, the demonstration versions,
examples of REDUCE programming, a set of manuals, and the REDUCE online
help system.
\end{document}

