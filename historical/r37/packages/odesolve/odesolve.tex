\documentstyle[11pt,reduce]{article}
\date{}
\title{ODESOLVE}
\author{Malcolm A.H. MacCallum \\ Queen Mary and Westfield College, London \\
Email: mm@maths.qmw.ac.uk \\[0.1in]
Other contributors: Francis Wright, Alan Barnes}
\begin{document}
\maketitle

\index{ODESOLVE package}
\index{ordinary differential equations}
The ODESOLVE package is a solver for ordinary differential equations.
At the present time it has very limited capabilities,

\begin{enumerate}
\item it can handle only a single scalar equation presented as an
algebraic expression or equation, and
\item it can solve only first-order equations of simple types,
linear equations with constant coefficients and Euler equations.
\end{enumerate}

\noindent These solvable types are exactly those for
which Lie symmetry techniques give no useful information.

\section{Use}
The only top-level function the user should normally invoke is:
\ttindex{ODESOLVE}

\vspace{.1in}
\begin{tabbing}
{\tt ODESOLVE}(\=EXPRN:{\em expression, equation}, \\
\>VAR1:{\em variable}, \\
\>VAR2:{\em variable}):{\em list-algebraic}
\end{tabbing}
\vspace{.1in}

\noindent {\tt ODESOLVE} returns a list containing an equation (like solve):

\begin{description}
\item[EXPRN] is a single scalar expression such that EXPRN = 0 is the
ordinary differential equation (ODE for short) to be solved,
or is an equivalent equation.
\item[VAR1] is the name of the dependent variable.
\item[VAR2] is the name of the independent variable
\end{description}

\noindent (For simplicity these will be called y and x in the sequel)
The returned value is a list containing the equation giving the
general solution of the ODE (for simultaneous equations this will be a
list of equations eventually). It will contain occurrences of the
\index{ARBCONST operator}
operator {\tt ARBCONST} for the arbitrary constants in the general solution.
The arguments of {\tt ARBCONST} should be new, as with {\tt ARBINT} etc.
in SOLVE. A counter {\tt !!ARBCONST} is used to arrange this (similar to the
way {\tt ARBINT} is implemented).

Some other top-level functions may be of use elsewhere, especially:
\ttindex{SORTOUTODE}

\vspace{.1in}
\noindent{\tt SORTOUTODE}(EXPRN:{\em algebraic}, Y:{\em var}, X:{\em var}):
{\em expression}
\vspace{.1in}

\noindent which finds the order and degree of the EXPRN as a differential
equation for Y with respect to Y and sets the linearity and highest
derivative occurring in reserved variables ODEORDER, ODEDEGREE,
\ttindex{ODEORDER} \ttindex{ODEDEGREE} \ttindex{ODELINEARITY}
\ttindex{HIGHESTDERIV}
ODELINEARITY and HIGHESTDERIV. An expression equivalent to the ODE is
returned, or zero if EXPRN (equated to 0) is not an ODE in the
given vars.

\section{Tracing}

Some rudimentary tracing is provided and is activated by the switch TRODE
\index{tracing ! ODESOLVE}
\ttindex{TRODE}
(analogous to TRFAC and TRINT)

\section{Comments}

The intention in the long run is to develop a rather general and
powerful ordinary differential equation solver incorporating the
methods detailed below.  At present the program has not been optimized
for efficiency and much work remains to be done to convert algebraic
mode procedures to more efficient symbolic mode replacements.

No attempt is made to extend the REDUCE integrator, although this is
in some sense a problem of ODEs.  Thus the equation $\frac{dy}{dx} = g(x)$ will
be solved if and only if $\int g(x) dx$ succeeds.

The available and planned coverage is as follows:

\begin{itemize}
\item First-order equations: (first degree unless otherwise stated)

\begin{itemize}
\item Quadrature of $\frac{df}{dx} = g(x)$
\item Linear equations
\item Separable equations
\item (Algebraically) homogeneous equations
\item Equations reducible to the previous case by linear transformations
\item Exact equations
\item Bernoulli equations
\end{itemize}

The above are already implemented. Further 1st order cases are not:
\begin{itemize}
\item Riccati equations using Schmidt's methods and other special cases
\item Hypotheses on the integrating factor following Char (SYMSAC 81)
or Shtokhamer, Glinos and Caviness.
\item Higher degree cases
\end{itemize}
\item Linear equations of higher order
\begin{itemize}
\item Constant coefficients case for driving terms solvable by
variation of parameters using the integrator
(Choice of method is discussed in the source of module lccode).
\end{itemize}
The above is already implemented. Further higher order methods are not:
\begin{itemize}
\item More complex driving terms via Laplace transforms (?)
\item  Variable coefficients: Watanabe (EUROSAM 84) methods
including Kovacic's algorithm as extended by Singer
\item  Factorization of operators as in Schwarz's ISSAC-89 paper or
Berkovich's 1990 book
\item  Other methods based on Galois theory (see Ulmer's preprints
from Karlsruhe, 1989, 1990 and Singer's 1989 review) or
other ways of hunting Liouvillian solutions (see Singer's
review in J. Symb. Comp., 1990).
\end{itemize}
\item Non-linear equations of order 2 and higher
\begin{itemize}
\item Lie algebra of point symmetries e.g. using Wolf's CRACK now available
in REDUCE
\item  Other special ansatze (see Wolf. op. cit), in particular
contact transformations for 2nd order cases
\end{itemize}
\item Possibly (?) exploitation of Cartan's methods for equivalence of
differential equations.
\end{itemize}
\end{document}
