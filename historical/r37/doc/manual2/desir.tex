\chapter[DESIR: Linear Homogeneous DEs]%
        {DESIR: Differential linear homogeneous equation solutions in the
                neighbourhood of irregular and regular singular points}
\label{DESIR}
\typeout{[DESIR: Linear Homogeneous DEs]}

{\footnotesize
\begin{center}
C. Dicrescenzo, F. Richard--Jung, E. Tournier \\
Groupe de Calcul Formel de Grenoble \\
laboratoire TIM3 \\
France \\[0.05in]
e--mail: dicresc@afp.imag.fr
\end{center}
}

\ttindex{DESIR}

This software enables the basis of formal solutions to be computed for an
ordinary homogeneous differential equation with polynomial coefficients
over Q of any order, in the neighbourhood of zero (regular or irregular
singular point, or ordinary point).

This software can be used in two ways, directly via the \f{DELIRE}
procedure, or interactively with the \f{DESIR} procedure.  The basic
procedure is the f{DELIRE} procedure which enables the solutions of a
linear homogeneous differential equation to be computed in the
neighbourhood of zero.

The \f{DESIR} procedure is a procedure without argument whereby
\f{DELIRE} can be called without preliminary treatment to the data,
that is to say, in an interactive autonomous way. This procedure also
proposes some transformations on the initial equation. This allows one
to start comfortably with an equation which has a non zero singular
point, a polynomial right-hand side and parameters.

\noindent{\tt delire(x,k,grille,lcoeff,param)}

This procedure computes formal solutions of a linear homogeneous
differential equation with polynomial coefficients over Q and of any
order, in the neighbourhood of zero, regular or irregular singular
point.  {\tt x} is the variable, {\tt k} is the number of desired
terms (that is for each formal series in $x_t$ appearing in polysol,
$a_0+a_1 x_t+a_2 x_t^2+\ldots + a_n x_t^n+ \ldots$ we compute the
$k+1$ first coefficients $a_0$, $a_1$ to $a_k$.  The coefficients of
the differential operator as polynomial in $x^{grille}$.  In general
grille is 1.  The argument {\tt lcoeff} is a list of coefficients of
the differential operator (in increasing order of differentiation) and
{\tt param} is a list of parameters.  The procedure returns the list
of general solutions.

\begin{verbatim}
lcoeff:={1,x,x,x**6};

                          6
        lcoeff := {1,x,x,x }

param:={};

        param := {}

sol:=delire(x,4,1,lcoeff,param);

                  4       3        2
                xt  - 4*xt  + 12*xt  - 24*xt + 24
sol := {{{{0,1,-----------------------------------,1},{
                               12

             }}},

                            4                3
        {{{0,1,(6*log(xt)*xt  - 18*log(xt)*xt

                                2
                 + 36*log(xt)*xt  - 36*log(xt)*xt

                       4       3
                 - 5*xt  + 9*xt  - 36*xt + 36)/36,0},{}

          }},

              1
        {{{-------,1,
                4
            4*xt

                  4       3        2
            361*xt  + 4*xt  + 12*xt  + 24*xt + 24
           ---------------------------------------,10},
                             24

          {}}}}
\end{verbatim}

