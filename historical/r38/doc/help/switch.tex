\section{General Switches}

\begin{Introduction}{SWITCHES}
Switches are set on or off using the commands \nameref{on} or
\nameref{off}, respectively.
The default setting of the switches described in this section is
\nameref{off} unless stated otherwise. 
\end{Introduction}

\begin{Switch}{ALGINT}
\index{integration}
When the \name{algint} switch is on, the algebraic integration module (which
must be loaded from the REDUCE library) is used for integration.

\begin{Comments}
Loading \name{algint} from the library automatically turns on the
\name{algint} switch.  An error message will be given if \name{algint} is
turned on when the \name{algint} has not been loaded from the library.
\end{Comments}
\end{Switch}


\begin{Switch}{ALLBRANCH}
When \name{allbranch} is on, the operator \nameref{solve} selects all 
branches of solutions. 
When \name{allbranch} is off, it selects only the principal
branches.  Default is \name{on}.

\begin{Examples}

solve(log(sin(x+3)),x);    &
\begin{multilineoutput}{6cm}
\{X=2*ARBINT(1)*PI - ASIN(1) - 3,
 X=2*ARBINT(1)*PI + ASIN(1) + PI - 3\}
\end{multilineoutput}\\
off allbranch; \\
solve(log(sin(x+3)),x);    &
                 {X=ASIN(1) - 3}
\end{Examples}

\begin{Comments}
\nameref{arbint}(1) indicates an arbitrary integer, which is given a
unique identifier by REDUCE, showing that there are infinitely many
solutions of this type.  When \name{allbranch} is off, the single
canonical solution is given.
\end{Comments}
\end{Switch}


\begin{Switch}{ALLFAC}
\index{output}
The \name{allfac} switch, when on, causes REDUCE to factor out automatically
common products in the output of expressions.  Default is \name{on}.

\begin{Examples}
x + x*y**3 + x**2*cos(z);  &      X*(COS(Z)*X + Y^{3}  + 1) \\
off allfac; \\
x + x*y**3 + x**2*cos(z);  &      COS(Z)*X^{2} + X*Y^{3}  + X
\end{Examples}

\begin{Comments}
The \name{allfac} switch has no effect when \name{pri} is off.  Although the
switch setting stays as it was, printing behavior is as if it were off.
\end{Comments}
\end{Switch}


% \begin{Switch}{ALLPREC}
%
% \end{Switch}
%
%
\begin{Switch}{ARBVARS}
\index{solve}
When \name{arbvars} is on, the solutions of singular or underdetermined
systems of equations are presented in terms of arbitrary complex variables
(see \nameref{arbcomplex}). Otherwise, the solution is parametrized in
terms of some of the input variables. Default is \name{on}.

\begin{Examples}
solve({2x + y,4x + 2y},{x,y});                          &
                   \{\{x= - \rfrac{arbcomplex(1)}{2},y=arbcomplex(1)\}\} \\
solve({sqrt(x)+ y**3-1},{x,y});				&
		   \{\{y=arbcomplex(2),x=y^6  - 2*y^3  + 1\}\} \\
off arbvars; \\
solve({2x + y,4x + 2y},{x,y});                          &
                   \{\{x= - \rfrac{y}{2}\}\} \\
solve({sqrt(x)+ y**3-1},{x,y});				&
		   \{\{x=y^6  - 2*y^3  + 1\}\} \\
\end{Examples}
\begin{Comments}

With \name{arbvars} off, the return value \name{\{\{\}\}} means that the
equations given to \nameref{solve} imply no relation among the input
variables.
\end{Comments}
\end{Switch}


\begin{Switch}{BALANCED\_MOD}
\index{modular}
\nameref{modular} numbers are normally produced in the range [0,...\meta{n}), 
where
\meta{n} is the current modulus.  With \name{balanced\_mod} on, the range
[-\meta{n}/2,\meta{n}/2], or more precisely
[-floor((\meta{n}-1)/2), ceiling((\meta{n}-1)/2)], is used instead.

\begin{Examples}
setmod 7; & 1 \\
on modular; \\
4; & 4 \\
on balanced_mod; \\
4; & -3
\end{Examples}
\end{Switch}


\begin{Switch}{BFSPACE}
\index{output}\index{floating point}
Floating point numbers are normally printed in a compact notation (either
fixed point or in scientific notation if \nameref{SCIENTIFIC\_NOTATION}
has been used).  In some (but not all) cases, it helps comprehensibility
if spaces are inserted in the number at regular intervals.  The switch
\name{bfspace}, if on, will cause a blank to be inserted in the number after
every five characters.
\begin{Examples}
on rounded; \\
1.2345678;  & 1.2345678 \\
on bfspace; \\
1.2345678;  & 1.234 5678
\end{Examples}
\begin{Comments}
\name{bfspace} is normally off.
\end{Comments}
\end{Switch}


\begin{Switch}{COMBINEEXPT}
\index{exponent simplification}
REDUCE is in general poor at surd simplification.  However, when the
switch \name{combineexpt} is on, the system attempts to combine
exponentials whenever possible.
\begin{TEX}
\begin{Examples}
3^(1/2)*3^(1/3)*3^(1/6); & SQRT(3)*3^{\rfrac{1}{3}}*3^{\rfrac{1}{6}} \\
on combineexpt; \\
ws; & 3
\end{Examples}
\end{TEX}
\begin{INFO}
{\begin{Examples}
3^(1/2)*3^(1/3)*3^(1/6); & SQRT(3)*3^{1/3}*3^{1/6} \\
on combineexpt; \\
ws; & 1
\end{Examples}}
\end{INFO}

\end{Switch}


\begin{Switch}{COMBINELOGS}
\index{logarithm}
In many cases it is desirable to expand product arguments of logarithms,
or collect a sum of logarithms into a single logarithm.  Since these are
inverse operations, it is not possible to provide rules for doing both at
the same time and preserve the REDUCE concept of idempotent evaluation.
As an alternative, REDUCE provides two switches \nameref{expandlogs} and
\name{combinelogs} to carry out these operations.
\begin{Examples}
on expandlogs; \\
log(x*y); & LOG(X) + LOG(Y) \\
on combinelogs; \\
ws; & LOG(X*Y)
\end{Examples}

\begin{Comments}
At the present time, it is possible to have both switches on at once,
which could lead to infinite recursion.  However, an expression is
switched from one form to the other in this case.  Users should not rely
on this behavior, since it may change in the next release.
\end{Comments}

\end{Switch}


\begin{Switch}{COMP}
\index{compiler}
When \name{comp} is on, any succeeding function definitions are compiled
into a faster-running form.  Default is \name{off}.

\begin{Examples}
\explanation{The following procedure finds Fibonacci numbers recursively.
Create a new file ``refib" in your current directory with the following
lines in it:} \\[6mm]
\begin{multilineinput}
procedure refib(n);
   if fixp n and n >= 0 then
     if n <= 1 then 1
       else refib(n-1) + refib(n-2)
    else rederr "nonnegative integer only";

end;
\end{multilineinput}\\
\explanation{Now load REDUCE and run the following:}\\
on time;                     &             Time: 100 ms \\

in "refib"$                  &             Time: 0 ms \\

                             &              REFIB \\

                             &              Time: 260 ms \\

                             &              Time: 20 ms \\

refib(80);                   &             37889062373143906 \\

                             &              Time: 14840 ms \\

on comp;                     &             Time: 80 ms \\

in "refib"$                  &             Time: 20 ms \\

                             &              REFIB \\

                              &             Time: 640 ms \\

refib(80);                   &             37889062373143906 \\

                               &            Time: 10940 ms
\end{Examples}
\begin{Comments}
Note that the compiled procedure runs faster.  Your time messages will
differ depending upon which system you have.  Compiled functions remain so
for the duration of the REDUCE session, and are then lost.  They must be
recompiled if wanted in another session.  With the switch \nameref{time} on
as shown above, the CPU time used in executing the command is returned in
milliseconds.  Be careful not to leave \name{comp} on unless you want it,
as it makes the processing of procedures much slower.

\end{Comments}
\end{Switch}


\begin{Switch}{COMPLEX}
\index{complex}
When the \name{complex} switch is on, full complex arithmetic is used in
simplification, function evaluation, and factorization.  Default is \name{off}.

\begin{Examples}

factorize(a**2 + b**2);      &    \{\{A^{2}  + B^{2},1\}\} \\
on complex; \\

factorize(a**2 + b**2);      &    \{\{A + I*B,1\},\{A - I*B,1\}\} \\

(x**2 + y**2)/(x + i*y);     &    X - I*Y \\

on rounded;             &

    *** Domain mode COMPLEX changed to COMPLEX\_FLOAT \\

sqrt(-17);                   &    4.12310562562*I \\

log(7*i);                    &    1.94591014906 + 1.57079632679*I
\end{Examples}

\begin{Comments}
Complex floating-point can be done by turning on \nameref{rounded} in
addition to \name{complex}.  With \name{complex} off however, REDUCE knows
that \IFTEX{$i$}{i} is the square root of \IFTEX{$-1$}{-1} but will not
carry out more complicated complex operations.  If you want complex
denominators cleared by multiplication by their conjugates, turn on the
switch \nameref{rationalize}.
\end{Comments}
\end{Switch}


\begin{Switch}{CREF}
\index{cross reference}
The switch \name{cref} invokes the CREF cross-reference program that
processes a set of procedure definitions to produce a summary of their
entry points, undefined procedures, non-local variables and so on.  The
program will also check that procedures are called with a consistent
number of arguments, and print a diagnostic message otherwise.

The output is alphabetized on the first seven characters of each function
name.

To invoke the cross-reference program, \name{cref} is first turned on.
This causes the program to load and the cross-referencing process to
begin.  After all the required definitions are loaded, turning \name{cref}
off will cause a cross-reference listing to be produced.

\begin{Comments}

Algebraic procedures in REDUCE are treated as if they were symbolic, so
that algebraic constructs will actually appear as calls to symbolic
functions, such as \name{aeval}.
\end{Comments}

\end{Switch}


\begin{Switch}{CRAMER}
\index{matrix}\index{linear system}\index{solve}
When the \name{cramer} switch is on, \nameref{matrix} inversion 
and linear equation
solving (operator \nameref{solve}) is done by Cramer's rule, through exterior
multiplication. Default is \name{off}.

\begin{Examples}
on time;                     &             Time: 80 ms \\
off output;                  &             Time: 100 ms \\[4mm]
\begin{multilineinput}
mm := mat((a,b,c,d,f),(a,a,c,f,b),(b,c,a,c,d), (c,c,a,b,f),
          (d,a,d,e,f));
\end{multilineinput} &             Time: 300 ms \\
inverse := 1/mm;             &             Time: 18460 ms \\
on cramer;                   &             Time: 80 ms \\
cramersinv := 1/mm;          &             Time: 9260 ms
\end{Examples}

\begin{Comments}
Your time readings will vary depending on the REDUCE version you use.
After you invert the matrix, turn on \nameref{output} and ask for one of
the elements of the inverse matrix, such as \name{cramersinv(3,2)}, so that
you can see the size of the expressions produced.

Inversion of matrices and the solution of linear equations with dense
symbolic entries in many variables is generally considerably faster with
\name{cramer} on.  However, inversion of numeric-valued matrices is
slower.  Consider the matrices you're inverting before deciding whether to
turn \name{cramer} on or off.  A substantial portion of the time in matrix
inversion is given to formatting the results for printing.  To save this
time, turn \name{output} off, as shown in this example or terminate the
expression with a dollar sign instead of a semicolon.  The results are
still available to you in the workspace associated with your prompt
number, or you can assign them to an identifier for further use.
\end{Comments}
\end{Switch}


\begin{Switch}{DEFN}
\index{lisp}
When the switch \name{defn} is on, the Standard Lisp equivalent of the
input statement or procedure is printed, but not evaluated.  Default is
\name{off}.

\begin{Examples}

on defn; \\

17/3;                        &         (AEVAL (LIST 'QUOTIENT 17 3)) \\

df(sin(x),x,2);          
         &         (AEVAL (LIST 'DF (LIST 'SIN 'X) 'X 2)) \\
\begin{multilineinput}
procedure coshval(a);
   begin scalar g;
      g := (exp(a) + exp(-a))/2;
      return g
   end;
\end{multilineinput} &
\begin{multilineoutput}{7cm}
(AEVAL 
  (PROGN 
    (FLAG '(COSHVAL) 'OPFN) 
    (DE COSHVAL (A) 
      (PROG (G) 
        (SETQ G 
          (AEVAL 
             (LIST 
                'QUOTIENT 
                (LIST 
                   'PLUS 
                   (LIST 'EXP A) 
                   (LIST 'EXP (LIST 'MINUS A))) 
                2))) 
       (RETURN G)))) ) 
\end{multilineoutput} \\

coshval(1);                 &       (AEVAL (LIST 'COSHVAL 1)) \\

off defn; \\

coshval(1);                 &   Declare COSHVAL operator? (Y or N) \\

n \\
\begin{multilineinput}
procedure coshval(a);
   begin scalar g;
      g := (exp(a) + exp(-a))/2;
      return g
   end;
\end{multilineinput} &       COSHVAL \\

on rounded; \\

coshval(1);                 &       1.54308063482
\end{Examples}

\begin{Comments}
The above function \name{coshval} finds the hyperbolic cosine (cosh) of its
argument.  When \name{defn} is on, you can see the Standard Lisp equivalent
of the function, but it is not entered into the system as shown by the
message \name{Declare COSHVAL operator?}.  It must be reentered with
\name{defn} off to be recognized.  This procedure is used as an example; a
more efficient procedure would eliminate the unnecessary local variable
with
\begin{verbatim}
      procedure coshval(a);
         (exp(a) + exp(-a))/2;
\end{verbatim}

\end{Comments}
\end{Switch}


\begin{Switch}{DEMO}
\index{interactive}\index{output}
The \name{demo} switch is used for interactive files, causing the system
to pause after each command in the file until you type a \key{Return}.
Default is \name{off}.

\begin{Comments}
The switch \name{demo} has no effect on top level interactive
statements.  Use it when you want to slow down operations in a file so
you can see what is happening.

You can either include the \name{on demo} command in the file, or enter
it from the top level before bringing in any file.  Unlike the
\nameref{pause} command, \name{on demo} does not permit you to interrupt
the file for questions of your own.

\end{Comments}
\end{Switch}


\begin{Switch}{DFPRINT}
\index{output}\index{derivative}
When \name{dfprint} is on, expressions in the differentiation operator
\nameref{df} are printed in a more ``natural'' notation, with the
differentiation variables appearing as subscripts.  In addition, if the
switch \nameref{noarg} is on (the default), the arguments of the
differentiated operator are suppressed.

\begin{Examples}
operator f; \\
df(f x,x); & DF(F(X),X); \\
on dfprint; \\
ws;  & F_{X} \\
df(f(x,y),x,y); & F_{X}_{,}_{Y} \\
off noarg; \\
ws; & F(X,Y)_{X}
\end{Examples}

\end{Switch}


\begin{Switch}{DIV}
\index{output}
When \name{div} is on, the system divides any simple factors found in
the denominator of an expression into the numerator.  Default is \name{off}.

\begin{Examples}

on div; \\

a := x**2/y**2;             &      A := X^{2} *Y^{-2} \\

b := a/(3*z);               &      B := \rfrac{1}{3}*X^{2} *Y^{-2}  *Z^{-1} \\

off div; \\

a;                           &    \rfrac{X^{2}}{Y^{2}}\\

b;                           &   \rfrac{X^{2}}{3*Y^{2} *Z}
                                       
\end{Examples}

\begin{Comments}
The \name{div} switch only has effect when the \nameref{pri} switch is on.
When \name{pri} is off, regardless of the setting of \name{div}, the
printing behavior is as if \name{div} were off.
\end{Comments}
\end{Switch}


\begin{Switch}{ECHO}
\index{output}
The \name{echo} switch is normally off for top-level entry, and on when files
are brought in.  If \name{echo} is turned on at the top level, your input
statements are echoed to the screen (thus appearing twice).  Default
\name{off} (but note default \name{on} for files).

\begin{Comments}
If you want to display certain portions of a file and not others, use the
commands \name{off echo} and \name{on echo} inside the file.  If you want
no display of the file, use the input command

	\name{in} {\em filename}\name{\$}

rather than using the semicolon delimiter.

Be careful when you use commands within a file to generate another file.
Since \name{echo} is on for files, the output file echoes input statements
(unlike its behavior from the top level).  You should explicitly turn off
\name{echo} when writing output, and turn it back on when you're done.
\end{Comments}
\end{Switch}


\begin{Switch}{ERRCONT}
\index{error handling}
When the \name{errcont} switch is on, error conditions do not stop file
execution.  Error messages will be printed whether \name{errcont} is on or
off.

Default is \name{off}.

\begin{Comments}
\begin{TEX}
The table below shows REDUCE behavior under the settings of \name{errcont} and
\name{int} :

  \begin{center}
  \begin{tabular}{|l|l|p{9.5cm}|}
\hline
\multicolumn{3}{|c|}{Behavior in Case of Error in Files}\\
\hline
\multicolumn{1}{|c|}{errcont} &
  \multicolumn{1}{c|}{int} &
    \multicolumn{1}{c|}{Behavior when errors in files are encountered}\\
\hline
off & off &
Message is printed and parsing continues, but no further statements are 
executed; no commands from keyboard accepted except \verb|bye| or
\verb|end| \\
off & on &
Message is printed, and you are asked if you wish to continue.  (This is the
default behavior) \\
on & off &
Message is printed, and file continues to execute without pause \\
on & on &
Message is printed, and file continues to execute without pause\\
\hline
  \end{tabular}
  \end{center}
\end{TEX}
\begin{INFO}
The following describes what happens when an error occurs in a file under
each setting of \name{errcont} and \name{int}:

Both off:  Message is printed and parsing continues, but no further
statements are executed; no commands from keyboard accepted except bye or
end;

\name{errcont} off, \name{int} on:  Message is printed, and you are asked
if you wish to continue. (This is the default behavior);

\name{errcont} on, \name{int} off:  Message is printed, and file continues
to execute without pause;

Both on: Message is printed, and file continues to execute without pause.
\end{INFO}
\end{Comments}
\end{Switch}


\begin{Switch}{EVALLHSEQP}
\index{equation}
Under normal circumstances, the right-hand-side of an \nameref{equation}
is evaluated but not the left-hand-side.  This also applies to any
substitutions made by the \nameref{sub} operator.  If both sides are to be
evaluated, the switch \name{evallhseqp} should be turned on.

\end{Switch}


\begin{Switch}{EXP}
\index{simplification}
When the \name{exp} switch is on, powers and products of expressions are
expanded.  Default is \name{on}.

\begin{Examples}
(x+1)**3;                    &     X^{3} + 3*X^{2} + 3*X + 1 \\
(a + b*i)*(c + d*i);         &     A*C + A*D*I + B*C*I - B*D \\
off exp; \\
(x+1)**3;                    &     (X + 1)^{3} \\
(a + b*i)*(c + d*i);         &       (A + B*I)*(C + D*I) \\
length((x+1)**2/(y+1));      &     2
\end{Examples}

\begin{Comments}
Note that REDUCE knows that \IFTEX{$i^2 = -1$}{i^2 = -1}.
When \name{exp} is off, equivalent expressions may not simplify to the same
form, although zero expressions still simplify to zero.  Several operators
that expect a polynomial argument behave differently when \name{exp} is
off, such as \nameref{length}.  Be cautious about leaving \name{exp} off.
\end{Comments}
\end{Switch}


\begin{Switch}{EXPANDLOGS}
\index{logarithm}
In many cases it is desirable to expand product arguments of logarithms,
or collect a sum of logarithms into a single logarithm.  Since these are
inverse operations, it is not possible to provide rules for doing both at
the same time and preserve the REDUCE concept of idempotent evaluation.
As an alternative, REDUCE provides two switches \name{expandlogs} and
\nameref{combinelogs} to carry out these operations.  Both are off by default.
\begin{Examples}
on expandlogs; \\
log(x*y); & LOG(X) + LOG(Y) \\
on combinelogs; \\
ws; & LOG(X*Y)
\end{Examples}

\begin{Comments}
At the present time, it is possible to have both switches on at once,
which could lead to infinite recursion.  However, an expression is
switched from one form to the other in this case.  Users should not rely
on this behavior, since it may change in the next release.
\end{Comments}

\end{Switch}


\begin{Switch}{EZGCD}
\index{greatest common divisor}\index{polynomial}
When \name{ezgcd} and \nameref{gcd} are on, greatest common divisors are
computed using the EZ GCD algorithm that uses modular arithmetic (and is
usually faster).  Default is \name{off}.

\begin{Comments}
As a side effect of the gcd calculation, the expressions involved are
factored, though not the heavy-duty factoring of \nameref{factorize}. The
EZ GCD algorithm was introduced in a paper by J. Moses and D.Y.Y. Yun in
\meta{Proceedings of the ACM}, 1973, pp. 159-166.

Note that the \nameref{gcd} switch must also be on for \name{ezgcd} to have
effect.
\end{Comments}
\end{Switch}


\begin{Switch}{FACTOR}
\index{output}
When the \name{factor} switch is on, input expressions and results are
automatically factored.

\begin{Examples}

on factor; \\

aa := 3*x**3*a + 6*x**2*y*a + 3*x**3*b + 6*x**2*y*b\\
+ x*y*a + 2*y**2*a + x*y*b + 2*y**2*b;
			     &     AA := (A + B)*(3*X^{2} + Y)*(X + 2*Y) \\
off factor; \\
aa;                          &
     3*A*X^{3} + 6*A*X^{2}*Y + A*X*Y + 2*A*Y^{2}  + 3*B*X^{3} + 6*B*X^{2}*Y\\
+ B*X*Y + 2*B*Y^{2} \\
on factor; \\
ab := x**2 - 2;              &     AB := X^{2} - 2
\end{Examples}

\begin{Comments}
REDUCE factors univariate and multivariate polynomials with 
integer coefficients, finding any factors that also have integer coefficients.
The factoring is done by reducing multivariate problems to univariate
ones  with symbolic coefficients, and then solving the univariate ones modulo 
small primes.  The results of these calculations are merged to
determine the factors of the original polynomial.  The factorizer normally
selects evaluation points and primes using a random number generator.  
Thus, the detailed factoring behavior may be different each time any
particular problem is tackled.

When the \name{factor} switch is turned on, the \nameref{exp} switch is
turned off, and when the \name{factor} switch is turned off, the
\nameref{exp} switch is turned on, whether it was on previously or not.

When the switch \nameref{trfac} is on, informative messages are generated at
each call to the factorizer.  The \nameref{trallfac} switch causes the
production of a more verbose trace message.  It takes precedence over
\name{trfac} if they are both on.

To factor a polynomial explicitly and store the results, use the operator 
\nameref{factorize}.
\end{Comments}
\end{Switch}


\begin{Switch}{FAILHARD}
\index{integration}
When the \name{failhard} switch is on, the integration operator \nameref{int} 
terminates with an error message if the integral cannot be done in closed 
terms.
Default is off.

\begin{Comments}
Use the \name{failhard} switch when you are dealing with complicated integrals
and want to know immediately if REDUCE was unable to handle them.  The
integration operator sometimes returns a formal integration form that is
more complicated than the original expression, when it is unable to
complete the integration.
\end{Comments}
\end{Switch}


\begin{Switch}{FORT}
\index{FORTRAN}
When \name{fort} is on, output is given Fortran-compatible syntax.  Default
is \name{off}.

\begin{Examples}
on fort; \\
df(sin(7*x + y),x);          &     ANS=7.*COS(7*X+Y) \\
on rounded; \\
b := log(sin(pi/5 + n*pi));  &
	       B=LOG(SIN(3.14159265359*N+0.628318530718))
\end{Examples}

\begin{Comments}
REDUCE results can be written to a file (using \nameref{out}) and used as data
by Fortran programs when \name{fort} is in effect.  \name{fort} knows about
correct statement length, continuation characters, defining a symbol when 
it is first used, and other Fortran details.

The \nameref{GENTRAN} package offers many more possibilities than the
\name{fort} switch.  It produces Fortran (or C or Ratfor) code from REDUCE
procedures or structured specifications, including facilities for producing
double precision output.
\end{Comments}
\end{Switch}

\begin{Switch}{FORTUPPER}
\index{FORTRAN}
When \name{fortupper} is on, any Fortran-style output appears in upper case.
Default is \name{off}.

\begin{Examples}
on fort; \\
df(sin(7*x + y),x);          &     ans=7.*cos(7*x+y) \\
on fortupper; \\
df(sin(7*x + y),x);          &     ANS=7.*COS(7*X+Y) \\
\end{Examples}
\end{Switch}


\begin{Switch}{FULLPREC}
\index{precision}\index{rounded}
Trailing zeroes of rounded numbers to the full system precision are
normally not printed.  If this information is needed, for example to get a
more understandable indication of the accuracy of certain data, the switch
\name{fullprec} can be turned on.

\begin{Examples}
on rounded; \\
1/2;  & 0.5 \\
on fullprec; \\
ws; & 0.500000000000
\end{Examples}

\begin{Comments}
This is just an output options which neither influences
the accuracy of the computation nor does it give additional
information about the precision of the results.
See also \nameref{scientific_notation}.
\end{Comments}
\end{Switch}

\begin{Switch}{FULLROOTS}
\index{solve}\index{polynomial}
Since roots of cubic and quartic polynomials can often be very
messy, a switch \name{fullroots} controls the production
of results in closed form. \nameref{solve} will apply the
formulas for explicit forms for degrees 3 and 4 only if
\name{fullroots} is \name{on}. Otherwise the result forms
are built using \nameref{root\_of}. Default is \name{off}.
\end{Switch}

\begin{Switch}{GC}
\index{memory}
With the \name{gc} switch, you can turn the garbage collection messages on
or off.  The form of the message depends on the particular Lisp used for
the REDUCE implementation.

\begin{Comments}
See \nameref{reclaim} for an explanation of garbage collection.  REDUCE does
garbage collection when needed even if you have turned the notices off.
\end{Comments}
\end{Switch}


\begin{Switch}{GCD}
\index{greatest common divisor}\index{rational expression}
When \name{gcd} is on, common factors in numerators and denominators of
expressions are canceled.  Default is \name{off}.

\begin{Examples}
\begin{multilineinput}

(2*(f*h)**2 - f**2*g*h - (f*g)**2 - f*h**3 + f*h*g**2
   - h**4 + g*h**3)/(f**2*h - f**2*g - f*h**2 + 2*f*g*h
   - f*g**2 - g*h**2 + g**2*h);
\end{multilineinput}         &
\rfrac{F^{2}*G^{2} + F^{2}*G*H - 2*F^{2}*H^{2} - F*G^{2}*H + F*H^{3} - G*H^{3} +
 H^{4}}
       {F^{2}*G - F^{2}*H + F*G^{2} - 2*F*G*H + F*H^{2} - G^{2}*H + G*H^{2}} \\
on gcd; \\
ws;                          &       \rfrac{F*G + 2*F*H + H^{2}}{F + G} \\
e2 := a*c + a*d + b*c + b*d; &      E2 := A*C + A*D + B*C + B*D \\
off exp;  \\
e2;                          &      (A + B)*(C + D)
\end{Examples}

\begin{Comments}
Even with \name{gcd} off, a check is automatically made for common variable
and numerical products in the numerators and denominators of expression,
and the appropriate cancellations made.  Thus the example demonstrating the
use of \name{gcd} is somewhat complicated.  Note when \nameref{exp} is off,
\name{gcd} has the side effect of factoring the expression.
\end{Comments}
\end{Switch}


\begin{Switch}{HORNER}
\index{output}\index{polynomial}
When the \name{horner} switch is on, polynomial expressions are printed
in Horner's form for faster and safer numerical evaluation. Default
is \name{off}. The leading variable of the expression is selected as
Horner variable. To select the Horner variable explicitly use the 
\nameref{korder} declaration.

\begin{Examples}
on horner;\\
(13p-4q)^3;&
( - 64)*q^3  + p*(624*q^2  + p*(( - 2028)*q + p*2197))\\
korder q;\\
ws;&
2197*p^3  + q*(( - 2028)*p^2  + q*(624*p + q*(-64)))
\end{Examples}
\end{Switch}


\begin{Switch}{IFACTOR}
\index{integer}\index{factorize}
When the \name{ifactor} switch is on, any integer terms appearing as a result
of the \nameref{factorize} command are factored themselves into primes.  Default
is \name{off}.  If the argument of \name{factorize} is an integer,
\name{ifactor} has no effect, since the integer is always factored.

\begin{Examples}
factorize(4*x**2 + 28*x + 48);    &    \{\{4,1\},\{X + 4,1\},\{X + 3,1\}\} \\
factorize(22587);                 &    \{\{3,1\},\{7529,1\}\} \\
on ifactor; \\
factorize(4*x**2 + 28*x + 48);    &    \{\{2,2\},\{X + 4,1\},\{X + 3,1\}\} \\
factorize(22587);                 &    \{\{3,1\},\{7529,1\}\} \\
\end{Examples}

\begin{Comments}
Constant terms that appear within nonconstant
polynomial factors are not factored.

The \name{ifactor} switch affects only factoring done specifically
with \nameref{factorize}, not on factoring done automatically when the
\nameref{factor} switch is on.
\end{Comments}
\end{Switch}


\begin{Switch}{INT}
\index{interactive}
The \name{int} switch specifies an interactive mode of operation.  Default
\name{on}.

\begin{Comments}
There is no reason to turn \name{int} off during interactive calculations,
since there are no benefits to be gained.  If you do have \name{int} off
while inputting a file, and REDUCE finds an error, it prints the message
``Continuing with parsing only."  In this state, REDUCE accepts only
\nameref{end}\name{;} or \nameref{bye}\name{;} from the keyboard;
everything else is ignored, even the command \name{on int}.
\end{Comments}
\end{Switch}


\begin{Switch}{INTSTR}
\index{output}
If \name{intstr} (for ``internal structure'') is on, arguments of an
operator are printed in a more structured form.

\begin{Examples}
operator f; \\
f(2x+2y); & F(2*X + 2*Y) \\
on intstr; \\
ws; & F(2*(X + Y))
\end{Examples}

\end{Switch}


\begin{Switch}{LCM}
\index{rational expression}
The \name{lcm} switch instructs REDUCE to compute the least common multiple
of denominators whenever rational expressions occur.  Default is \name{on}.

\begin{Examples}
off lcm; \\
z := 1/(x**2 - y**2) + 1/(x-y)**2;  
			     &     Z := \rfrac{2*X*(X - Y)}{X^{4} - 2*X^{3}*Y + 
2*X*Y^{3} - Y^{4}} \\
on lcm; \\
z;                           &     \rfrac{2*X*(X - Y)}{X^{4} - 2*X^{3}*Y + 2*X*Y
^{3} - Y^{4}} \\
zz := 1/(x**2 - y**2) + 1/(x-y)**2;
			     &     ZZ := \rfrac{2*X}{X^{3} - X^{2}*Y - X*Y^{2} +
 Y^{3}} \\
on gcd; \\
z;                           &     \rfrac{2*X}{X^{3} - X^{2}*Y - X*Y^{2} + Y^{3}
}
\end{Examples}

\begin{Comments}
Note that \name{lcm} has effect only when rational expressions are first
combined.  It does not examine existing structures for simplifications on
display.  That is shown above when \IFTEX{$z$}{z} is entered with
\name{lcm} off.  It remains unsimplified even after \name{lcm} is turned
back on.  However, a new variable containing the same expression is
simplified on entry.  The switch \nameref{gcd} does examine existing
structures, as shown in the last example line above.

Full greatest common divisor calculations become expensive if work with
large rational expressions is required.  A considerable savings of time
can be had if a full gcd check is made only when denominators are combined,
and only a partial check for numerators.  This is the effect of the \name{lcm}
switch.
\end{Comments}
\end{Switch}


\begin{Switch}{LESSSPACE}
\index{output}
You can turn on the switch \name{lessspace} if you want fewer
blank lines in your output.
\end{Switch}


\begin{Switch}{LIMITEDFACTORS}
\index{factorize}\index{polynomial}
To get limited factorization in cases where it is too expensive to use
full multivariate polynomial factorization, the switch
\name{limitedfactors} can be turned on.  In that case, only ``inexpensive''
factoring operations, such as square-free factorization, will be used
when \nameref{factorize} is called.

\begin{Examples}
a := (y-x)^2*(y^3+2x*y+5)*(y^2-3x*y+7)$ \\
factorize a; &
\begin{multilineoutput}{7cm}
\{- 3*X*Y + Y^{2} + 7,1\}
\{2*X*Y + Y^{3} + 5,1\},
\{X - Y,2\}\}
\end{multilineoutput} \\
on limitedfactors; \\
factorize a; &
\begin{multilineoutput}{7cm}
\{- 6*X^{2}*Y^{2} - 3*X*Y^{4} + 2*X*Y^{3} - X*Y + Y^{5} + 7*Y^{3} + 5*Y^{2} + 35,1\},
\{X - Y,2\}\}
\end{multilineoutput}
\end{Examples}

\end{Switch}


\begin{Switch}{LIST}
The \name{list} switch causes REDUCE to print each term in any sum on
separate lines.

\begin{Examples}
x**2*(y**2 + 2*y) + x*(y**2 + z)/(2*a);
			    &     \rfrac{X*(2*A*X*Y^{2} + 4*A*X*Y + Y^{2} +
Z)}{2*A} \\
on list; \\
ws;                         &
\begin{multilineoutput}{6cm}
(X*(2*A*X*Y^{2}
  + 4*A*X*Y
  + Y^{2}
  + Z))/(2*A)
\end{multilineoutput}
\end{Examples}
\end{Switch}


\begin{Switch}{LISTARGS}
\index{list}\index{argument}\index{operator}
If an operator other than those specifically defined for lists is given a
single argument that is a list, then the result of this operation will be
a list in which that operator is applied to each element of the list.
This process can be inhibited globally by turning on the switch
\name{listargs}.

\begin{Examples}
log {a,b,c}; & {LOG(A),LOG(B),LOG(C)} \\
on listargs; \\
log {a,b,c}; & LOG({A,B,C})
\end{Examples}

\begin{Comments}
It is possible to inhibit such distribution for a specific operator by
using the declaration \nameref{listargp}.  In addition, if an operator has
more than one argument, no such distribution occurs, so \name{listargs}
has no effect.
\end{Comments}

\end{Switch}


\begin{Switch}{MCD}
\index{rational expression}
When \name{mcd} is on, sums and differences of rational expressions are put
on a common denominator.  Default is \name{on}.

\begin{Examples}
a/(x+1) + b/5;               &         \rfrac{5*A + B*X + B}{5*(X + 1)} \\
off mcd; \\
a/(x+1) + b/5;               &        (X + 1)^{-1}*A + 1/5*B \\
1/6 + 1/7;                   &        13/42
\end{Examples}

\begin{Comments}
Even with \name{mcd} off, rational expressions involving only numbers are still
put over a common denominator.  

Turning \name{mcd} off is useful when explicit negative powers are needed,
or if no greatest common divisor calculations are desired, or when
differentiating complicated rational expressions.  Results when \name{mcd}
is off are no longer in canonical form, and expressions equivalent to zero
may not simplify to 0.  Some operations, such as factoring cannot be done
while \name{mcd} is off.  This option should therefore be used with some
caution.  Turning \name{mcd} off is most valuable in intermediate parts of
a complicated calculation, and should be turned back on for the last stage.
\end{Comments}
\end{Switch}


\begin{Switch}{MODULAR}
\index{modular}
When \name{modular} is on, polynomial coefficients are reduced by the
modulus set by \nameref{setmod}.  If no modulus has been set, \name{modular}
has no effect.

\begin{Examples}
setmod 2;                    &          1 \\
on modular;                             \\
(x+y)**2;                    &          X^{2} + Y^{2} \\
145*x**2 + 20*x**3 + 17 + 15*x*y;
			     &          X^{2} + X*Y + 1
\end{Examples}

\begin{Comments}
Modular operations are only conducted on the coefficients, not the
exponents.  The modulus is not restricted to being prime.  When the modulus
is prime, division by a number not relatively prime to the modulus results
in a \meta{Zero divisor} error message.  When the modulus is a composite
number, division by a power of the modulus results in an error message, but
division by an integer which is a factor of the modulus does not.
The representation of modular number can be influenced by 
\nameref{balanced\_mod}.
\end{Comments}
\end{Switch}


\begin{Switch}{MSG}
\index{output}
When \name{msg} is off, the printing of warning messages is suppressed.  Error
messages are still printed.

\begin{Comments}
Warning messages include those about redimensioning an \nameref{array} 
or declaring an \nameref{operator} where one is expected.  
\end{Comments}
\end{Switch}


\begin{Switch}{MULTIPLICITIES}
\index{solve}
When \nameref{solve} is applied to a set of equations with multiple roots,
solution multiplicities are normally stored in the global variable
\nameref{root\_multiplicities} rather than the solution list.  If you want
the multiplicities explicitly displayed, the switch \name{multiplicities}
should be turned on.  In this case, \name{root_multiplicities} has no value.

\begin{Examples}
solve(x^2=2x-1,x); & {X=1} \\
root_multiplicities; & {2} \\
on multiplicities; \\
solve(x^2=2x-1,x); & {X=1,X=1} \\
root_multiplicities; &
\end{Examples}

\end{Switch}


\begin{Switch}{NAT}
\index{output}
When \name{nat} is on, output is printed to the screen in natural form, with
raised exponents.  \name{nat} should be turned off when outputting expressions
to a file for future input.  Default is \name{on}.

\begin{Examples}
(x + y)**3;                  &          X^{3} + 3*X^{2}*Y + 3*X*Y^{2} + Y^{3} \\
off nat; \\
(x + y)**3;                  &          X**3 + 3*X**2*Y + 3*X*Y**2 + Y**3\$ \\
on fort; \\
(x + y)**3;                  &          ANS=X**3+3.*X**2*Y+3.*X*Y**2+Y**3
\end{Examples}
\begin{Comments}
With \name{nat} off, a dollar sign is printed at the end of each expression.
An output file written with \name{nat} off is ready to be read into REDUCE
using the command \nameref{in}.
\end{Comments}
\end{Switch}


\begin{Switch}{NERO}
\index{output}
When \name{nero} is on, zero assignments (such as matrix elements) are not
printed.

\begin{Examples}
matrix a;
a := mat((1,0),(0,1));       & \begin{multilineoutput}{6cm}
A(1,1) := 1
A(1,2) := 0
A(2,1) := 0
A(2,2) := 1
\end{multilineoutput}\\
on nero; \\
a;                           &  \begin{multilineoutput}{6cm}
MAT(1,1) := 1
MAT(2,2) := 1
\end{multilineoutput}\\
a(1,2);                      & \explanationo{nothing is printed.} \\
b := 0;                     &  \explanationo{nothing is printed.} \\
off nero; \\
b := 0;                     &          B := 0
\end{Examples}
\begin{Comments}
\name{nero} is often used when dealing with large sparse matrices, to avoid
being overloaded with zero assignments.
\end{Comments}
\end{Switch}


\begin{Switch}{NOARG}
\index{output}\index{derivative}
When \nameref{dfprint} is on, expressions in the differentiation operator
\nameref{df} are printed in a more ``natural'' notation, with the
differentiation variables appearing as subscripts.  When \name{noarg}
is on (the default), the arguments of the differentiated operator are also
suppressed.

\begin{Examples}
operator f; \\
df(f x,x); & DF(F(X),X); \\
on dfprint; \\
ws;  & F_{X} \\
off noarg; \\
ws; & F(X)_{X}
\end{Examples}

\end{Switch}


\begin{Switch}{NOLNR}
\index{integration}
When \name{nolnr} is on, the linear properties of the integration operator
\nameref{int} are suppressed if the integral cannot be found in closed terms.

\begin{Comments}
REDUCE uses the linear properties of integration to attempt to break down
an integral into manageable pieces.  If an integral cannot be found in
closed terms, these pieces are returned.  When the \name{nolnr} switch is off,
as many of the pieces as possible are integrated.  When it is on, if any piece
fails, the rest of them remain unevaluated.
\end{Comments}
\end{Switch}


%\begin{Switch}{NORNDBF}
%
%***** To be added *****
%
%\end{Switch}


\begin{Switch}{NOSPLIT}
\index{output}
Under normal circumstances, the printing routines try to break an expression
across lines at a natural point.  This is a fairly expensive process.  If
you are not overly concerned about where the end-of-line breaks come, you
can speed up the printing of expressions by turning off the switch
\name{nosplit}.  This switch is normally on.

\end{Switch}


\begin{Switch}{NUMVAL}
\index{rounded}
With \nameref{rounded} on, elementary functions with numerical arguments
will return a numerical answer where appropriate.  If you wish to inhibit
this evaluation, \name{numval} should be turned off.  It is normally on.

\begin{Examples}
on rounded; \\
cos 3.4; & - 0.966798192579 \\
off numval; \\
cos 3.4; & COS(3.4)
\end{Examples}

\end{Switch}


\begin{Switch}{OUTPUT}
\index{output}
When \name{output} is off, no output is printed from any REDUCE calculation.
The calculations have their usual effects other than printing.  Default is
\name{on}.

\begin{Comments}
Turn output \name{off} if you do not wish to see output when executing
large files, or to save the time REDUCE spends formatting large expressions
for display.  Results are still available with \nameref{ws}, or in their
assigned variables. 
\end{Comments}
\end{Switch}


\begin{Switch}{OVERVIEW}
\index{factorize}
When \name{overview} is on, the amount of detail reported by the factorizer
switches \nameref{trfac} and \nameref{trallfac} is reduced.

\end{Switch}


\begin{Switch}{PERIOD}
\index{output}\index{integer}
When \name{period} is on, periods are added after integers in
Fortran-compatible output (when \nameref{fort} is on).  There is no effect
when \name{fort} is off.  Default is \name{on}.
\end{Switch}


\begin{Switch}{PRECISE}
\index{simplification}\index{square root}
When the \name{precise} switch is on, simplification of roots of even
powers returns absolute values, a more precise answer mathematically.
Default is \name{on}.

\begin{Examples}
sqrt(x**2);                  &        X \\
(x**2)**(1/4);               &        SQRT(X) \\
on precise; \\
sqrt(x**2);                  &        ABS(X) \\
(x**2)**(1/4);               &        SQRT(ABS(X))
\end{Examples}

\begin{Comments}
In many types of mathematical work, simplification of powers and surds can
proceed by the fastest means of simplifying the exponents arithmetically.
When it is important to you that the positive root be returned, turn
\name{precise} on.  One situation where this is important is when graphing
square-root expressions such as \IFTEX{$\sqrt{x^2+y^2}$}{sqrt(x^2+y^2)} to
avoid a spike caused by REDUCE simplifying
\IFTEX{$\sqrt{y^2}$}{sqrt(y^2)} to \IFTEX{$y$}{y} when \IFTEX{$x$}{x} is
zero.
\end{Comments}
\end{Switch}


\begin{Switch}{PRET}
\index{output}
When \name{pret} is on, input is printed in standard REDUCE format and then
evaluated.

\begin{Examples}
on pret; \\
 (x+1)^3;                     &
\begin{multilineoutput}{6cm}
 (x + 1)**3;
X^{3} + 3*X^{2} + 3*X + 1
\end{multilineoutput} \\
\begin{multilineinput}

procedure fac(n);
   if not (fixp(n) and n>=0)
     then rederr "Choose nonneg. integer only"
    else for i := 0:n-1 product i+1;
\end{multilineinput}     &
\begin{multilineoutput}{8cm}
procedure fac n;
   if not (fixp n and n>=0)
     then rederr "Choose nonneg. integer only"
    else for i := 0:n - 1 product i + 1;

FAC
\end{multilineoutput}\\

fac 5;                    & \begin{multilineoutput}{6cm}
fac 5;
120
\end{multilineoutput}

\end{Examples}

\begin{Comments}
Note that all input is converted to lower case except strings (which keep
the same case) all operators with a single argument have had the
parentheses removed, and all infix operators have had a space added on each
side.  In addition, syntactical constructs like
\name{if}\ldots\name{then}\ldots\name{else} are printed in a standard format.
\end{Comments}
\end{Switch}


\begin{Switch}{PRI}
\index{output}
When \name{pri} is on, the declarations \nameref{order} and \nameref{factor} can
be used, and the switches \nameref{allfac}, \nameref{div}, \nameref{rat}, 
and \nameref{revpri} take effect when they are on.  Default is \name{on}.

\begin{Comments}
Printing of expressions is faster with \name{pri} off.  The expressions are
then returned in one standard form, without any of the display options that
can be used to feature or display various parts of the expression.  You can
also gain insight into REDUCE's representation of expressions with
\name{pri} off.
\end{Comments}
\end{Switch}


\begin{Switch}{RAISE}
\index{input}\index{character}
When \name{raise} is on, lower case letters are automatically converted to
upper case on input.  \name{raise} is normally on.

\begin{Comments}
This conversion affects the internal representation of the letter, and is
independent of the case with which a letter is printed, which is normally
lower case.
\end{Comments}

\end{Switch}


\begin{Switch}{RAT}
\index{output}
When the \name{rat} switch is on, and kernels have been selected to display
with the \nameref{factor} declaration, the denominator is printed with each
term rather than one common denominator at the end of an expression.

\begin{Examples}
(x+1)/x + x**2/sin y;        
          &     \rfrac{SIN(Y)*X + SIN(Y) + X^{3}}{SIN(Y)*X} \
\
factor x; \\
(x+1)/x + x**2/sin y;        
          &     \rfrac{X^{3} + X*SIN(Y) + SIN(Y)}{X*SIN(Y)} \
\
on rat;  \\
(x+1)/x + x**2/sin y;       
          &     \rfrac{X^{2}}{SIN(Y)} + 1 + X^{-1}
\end{Examples}

\begin{Comments}
The \name{rat} switch only has effect when the \nameref{pri} switch is on.
When \name{pri} is off, regardless of the setting of \name{rat}, the
printing behavior is as if \name{rat} were off.  \name{rat} only has
effect upon the display of expressions, not their internal form.
\end{Comments}
\end{Switch}


\begin{Switch}{RATARG}
\index{rational expression}\index{polynomial}
When \name{ratarg} is on, rational expressions can be given to operators
such as \nameref{coeff} and \nameref{lterm} that normally require
polynomials in one of their arguments.  When \name{ratarg} is off, rational
expressions cause an error message.

\begin{Examples}
aa := x/y**2 + 1/x + y/x**2; 
    &    AA := \rfrac{X^{3} + X*Y^{2} + Y^{3}}{X^{2}*Y^{2}} \\
coeff(aa,x);                 &
   ***** \rfrac{X^{3} + X*Y^{2} + Y^{3}}{X^{2}*Y^{2}} invalid as POLYNOMIAL\\
on ratarg; \\
coeff(aa,x);                
    & \{\rfrac{Y}{X^{2}},\rfrac{1}{X^{2}},0,\rfrac{1}{X^{2}*Y^{2}}\}
\end{Examples}
\end{Switch}


\begin{Switch}{RATIONAL}
\index{rational expression}\index{polynomial}
When \name{rational} is on, polynomial expressions with rational coefficients
are produced.

\begin{Examples}
x/2 + 3*y/4;                 &       \rfrac{2*X + 3*Y}{4} \\
(x**2 + 5*x + 17)/2;         &       \rfrac{X^{2} + 5*X + 17}{2} \\
on rational; \\
x/2 + 3y/4;                  &       \rfrac{1}{2}*(X + \rfrac{3}{2}*Y) \\
(x**2 + 5*x + 17)/2;         &       \rfrac{1}{2}*(X^{2} + 5*X + 17)
\end{Examples}

\begin{Comments}
By using \name{rational}, polynomial expressions with rational
coefficients can be used in some commands that expect polynomials.  With
\name{rational} off, such a polynomial becomes a rational expression, with
denominator the least common multiple of the denominators of the rational
number coefficients.  % The \nameref{factorize} command does not accept
% polynomials with rational coefficients.
\end{Comments}
\end{Switch}


\begin{Switch}{RATIONALIZE}
\index{rational expression}\index{simplification}\index{complex}
When the \name{rationalize} switch is on, denominators of rational expressions
that contain complex numbers or root expressions are simplified by 
multiplication by their conjugates.

\begin{Examples}
qq := (1+sqrt(3))/(sqrt(3)-7);  &   QQ := \rfrac{SQRT(3) + 1}{SQRT(3) - 7} \\
on rationalize; \\
qq;                             &   \rfrac{- 4*SQRT(3) - 5}{23} \\
2/(4 + 6**(1/3));               &     \rfrac{6^{2/3} - 4*6^{1/3} + 16}{35} \\
(i-1)/(i+3);                    &   \rfrac{2*I - 1}{5} \\
off rationalize; \\
(i-1)/(i+3);                    &   \rfrac{I - 1}{I + 3}
\end{Examples}

\end{Switch}


\begin{Switch}{RATPRI}
\index{output}\index{rational expression}
When the \name{ratpri} switch is on, rational expressions and fractions are
printed as two lines separated by a fraction bar, rather than in a linear
style.  Default is \name{on}.

\begin{Examples}
3/17;                        &            \rfrac{3}{17} \\
2/b + 3/y;                   &            \rfrac{3*B + 2*Y}{B*Y} \\
off ratpri; \\
3/17;                        &            3/17 \\
2/b + 3/y;                   &            (3*B + 2*Y)/(B*Y)
\end{Examples}
\end{Switch}


\begin{Switch}{REVPRI}
\index{output}
When the \name{revpri} switch is on, terms are printed in reverse order from
the normal printing order.

\begin{Examples}
x**5 + x**2 + 18 + sqrt(y);  &    SQRT(Y) + X^{5} + X^{2} + 18 \\
a + b + c + w;               &    A + B + C + W \\
on revpri; \\
x**5 + x**2 + 18 + sqrt(y);  &    17 + X^{2} + X^{5} + SQRT(Y) \\
a + b + c + w;               &    W + C + B + A
\end{Examples}

\begin{Comments}
Turn \name{revpri} on when you want to display a polynomial in ascending
rather than descending order.
\end{Comments}
\end{Switch}


\begin{Switch}{RLISP88}
\index{lisp}
Rlisp '88 is a superset of the Rlisp that has been traditionally used for
the support of REDUCE.  It is fully documented in the book Marti, J.B.,
``{RLISP} '88:  An Evolutionary Approach to Program Design and Reuse'',
World Scientific, Singapore (1993).  It supports different looping
constructs from the traditional Rlisp, and treats ``-'' as a letter unless
separated by spaces.  Turning on the switch \name{rlisp88} converts to
Rlisp '88 parsing conventions in symbolic mode, and enables the use of
Rlisp '88 extensions.  Turning off the switch reverts to the traditional
Rlisp and the previous mode ( (\nameref{symbolic} or \nameref{algebraic})
in force before \name{rlisp88} was turned on.

\end{Switch}


\begin{Switch}{ROUNDALL}
\index{rounded}\index{rational expression}\index{floating point}
In \nameref{rounded} mode, rational numbers are normally converted to a
floating point representation.  If \name{roundall} is off, this conversion
does not occur.  \name{roundall} is normally \name{on}.

\begin{Examples}
on rounded; \\
1/2; & 0.5 \\
off roundall; \\
1/2; \rfrac{1}{2}
\end{Examples}

\end{Switch}


\begin{Switch}{ROUNDBF}
When \nameref{rounded} is on, the normal defaults cause underflows to be
converted to zero.  If you really want the small number that results in
such cases, \name{roundbf} can be turned on.

\begin{Examples}
on rounded; \\
exp(-100000.1^2); & 0 \\
on roundbf; \\
exp(-100000.1^2); & 1.18441281937E-4342953505
\end{Examples}

\begin{Comments}
If a polynomial is input in \nameref{rounded} mode at the default
precision into any \nameref{roots} function, and it is not possible to
represent any of the coefficients of the polynomial precisely in the
system floating point representation, the switch \name{roundbf} will be
automatically turned on.  All rounded computation will use the internal
bigfloat representation until the user subsequently turns \name{roundbf}
off. (A message is output to indicate that this condition is in effect.)
\end{Comments}
\end{Switch}


\begin{Switch}{ROUNDED}
\index{floating point}
When \name{rounded} is on, floating-point arithmetic is enabled, with
precision initially at a system default value, which is usually 12 digits.
The precise number can be found by the command \nameref{precision}(0).
\begin{Examples}
pi;                          &            PI \\
35/217;                      &            \rfrac{5}{31} \\
on rounded; \\
pi;                          &            3.14159265359 \\
35/217;                      &            0.161 \\
sqrt(3);                     &            1.73205080756
\end{Examples}
\begin{Comments}
If more than the default number of decimal places are required, use the
\nameref{precision} command to set the required number.
\end{Comments}
\end{Switch}


\begin{Switch}{SAVESTRUCTR}
\index{STRUCTR OPERATOR}
When \name{savestructr} is on, results of the \nameref{structr} command are
returned as a list whose first element is the representation for the
expression and the remaining elements are equations showing the
relationships of the generated variables.

\begin{Examples}
off exp; \\
structr((x+y)^3 + sin(x)^2);                          &
\begin{multilineoutput}{6cm}
ANS3
   where
      ANS3 := ANS1^{3} + ANS2^{2}
      ANS2 := SIN(X)
      ANS1 := X + Y
\end{multilineoutput}\\
ans3;                        &     ANS3 \\
on savestructr; \\
structr((x+y)^{3} + sin(x)^{2});                         &
{ANS3,ANS3=ANS1^{3} + ANS2^{2},ANS2=SIN(X),ANS1=X + Y} \\
ans3 where rest ws; &
(X + Y)^{3} + SIN(X)^{2}
\end{Examples}
\begin{Comments}
In normal operation, \nameref{structr} is only a display command.  With
\name{savestructr} on, you can access the various parts of the expression
produced by \name{structr}.

The generic system names use the stem \name{ANS}.  You can change this to your
own stem by the command \nameref{varname}.  REDUCE adds integers to this stem
to make unique identifiers.
\end{Comments}
\end{Switch}


\begin{Switch}{SOLVESINGULAR}
\index{solve}
When \name{solvesingular} is on, singular or underdetermined systems of
linear equations are solved, using arbitrary real, complex or integer
variables in the answer.  Default is \name{on}.

\begin{Examples}
solve({2x + y,4x + 2y},{x,y});                          &
                   \{\{X= - \rfrac{ARBCOMPLEX(1)}{2},Y=ARBCOMPLEX(1)\}\} \\
solve({7x + 15y - z,x - y - z},{x,y,z});                &
\begin{multilineoutput}{6cm}
\{\{X=\rfrac{8*ARBCOMPLEX(3)}{11}
  Y= - \rfrac{3*ARBCOMPLEX(3)}{11}
  Z=ARBCOMPLEX(3)\}\}
\end{multilineoutput}\\
off solvesingular; \\
solve({2x + y,4x + 2y},{x,y});                          &
                   ***** SOLVE given singular equations \\
solve({7x + 15y - z,x - y - z},{x,y,z});                &
                   ***** SOLVE given singular equations
\end{Examples}
\begin{Comments}

The integer following the identifier \nameref{arbcomplex} above is assigned by
the system, and serves to identify the variable uniquely.  It has no other
significance.
\end{Comments}
\end{Switch}


\begin{Switch}{TIME}
\index{time}
When \name{time} is on, the system time used in executing each REDUCE
statement is printed after the answer is printed.

\begin{Examples}
on time;                     &       Time: 4940 ms \\
df(sin(x**2 + y),y);         &
\begin{multilineoutput}{6cm}
COS(X  + Y^{2})
Time: 180 ms
\end{multilineoutput}\\
solve(x**2 - 6*y,x);         &
\begin{multilineoutput}{6cm}
\{X= - SQRT(Y)*SQRT(6),
 X=SQRT(Y)*SQRT(6)\}
Time: 320 ms
\end{multilineoutput}
\end{Examples}
\begin{Comments}
When \name{time} is first turned on, the time since the beginning of the
REDUCE session is printed.  After that, the time used in computation,
(usually in milliseconds, though this is system dependent) is printed after
the results of each command.  Idle time or time spent typing in commands is
not counted.  If \name{time} is turned off, the first reading after it is
turned on again gives the time elapsed since it was turned off.  The time
printed is CPU or wall clock time, depending on the system.
\end{Comments}
\end{Switch}


\begin{Switch}{TRALLFAC}
\index{factorize}
When \name{trallfac} is on, a more detailed trace of factorizer calls is
generated.

\begin{Comments}
The \name{trallfac} switch takes precedence over \nameref{trfac} if they are
both on. \name{trfac} gives a factorization trace with less detail in it.
When the \nameref{factor} switch is on also, all input polynomials are sent to
the factorizer automatically and trace information is generated.  The
\nameref{out} command saves the results of the factoring, but not the trace.
% You need to use the \name{dribble} command to save the trace text to a file.
\end{Comments}
\end{Switch}


\begin{Switch}{TRFAC}
\index{factorize}
When \name{trfac} is on, a narrative trace of any calls to the factorizer is
generated.  Default is \name{off}.

\begin{Comments}
When the switch \nameref{factor} is on, and \name{trfac} is on, every input
polynomial is sent to the factorizer, and a trace generated.  With
\name{factor} off, only polynomials that are explicitly factored with the
command \nameref{factorize} generate trace information.

% Use the \name{dribble} or \name{logfile} command to save the trace text to
% a file.
The \nameref{out} command saves the results of the factoring, but not
the trace.  The \nameref{trallfac} switch gives trace information to a
greater level of detail.
\end{Comments}
\end{Switch}

\begin{Switch}{TRIGFORM}
\index{solve}\index{polynomial}
When \nameref{fullroots} is on, \nameref{solve} will compute the
roots of a cubic or quartic polynomial is closed form. When 
\name{trigform} is on, the roots will be expressed by trigonometric
forms. Otherwise nested surds are used. Default is \name{on}.
\end{Switch}


\begin{Switch}{TRINT}
\index{integration}
When \name{trint} is on, a narrative tracing various steps in the
integration process is produced.

\begin{Comments}
%Use the \name{dribble} or \name{logfile} command to save this text to a file.
The \nameref{out} command saves the results of the integration, but not the
trace.
\end{Comments}
\end{Switch}

\begin{Switch}{TRNONLNR}
\index{solve}
When \name{trnonlnr} is on, a narrative tracing various steps in
the process for solving non-linear equations is produced.

\begin{Comments}
\name{trnonlnr} can only be used after the solve package has been loaded
(e.g., by an explicit call of \nameref{load\_package}).  The \nameref{out}
command saves the results of the equation solving, but not the trace.
\end{Comments}
\end{Switch}


\begin{Switch}{VAROPT}
\index{solve}
When \name{varopt} is on, the sequence of variables is optimized by
\nameref{solve} with respect to execution speed. Otherwise, the sequence
given in the call to \nameref{solve} is preserved. Default is \name{on}.

In combination with the switch \nameref{arbvars}, \name{varopt} can be used
to control variable elimination.

\begin{Examples}
off arbvars; \\
solve({x+2z,x-3y},{x,y,z});				&
		   \{\{y=\rfrac{x}{3},z= - \rfrac{x}{2}\}\} \\
solve({x*y=1,z=x},{x,y,z});				&
		   \{\{z=x,y=\rfrac{1}{x}\}\} \\
off varopt; \\
solve({x+2z,x-3y},{x,y,z});				&
		   \{\{x= - 2*z,y= - \rfrac{2*z}{3}\}\} \\
solve({x*y=1,z=x},{x,y,z});				&
		   \{\{y=\rfrac{1}{z},x=z\}\} \\
\end{Examples}
\end{Switch}

