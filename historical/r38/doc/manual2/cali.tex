\chapter[CALI: Commutative Algebra]{CALI: Computational Commutative Algebra}
\label{CALI}
\typeout{{CALI: Computational Commutative Algebra}}

{\footnotesize
\begin{center}
Hans-Gert Gr\"abe \\
Institut f\"ur Informatik, Universit\"at Leipzig\\
Augustusplatz 10 -- 11\\
04109 Leipzig, Germany \\[0.05in]
e--mail: graebe@informatik.uni-leipzig.de
\end{center}
}

\ttindex{CALI}

This package contains algorithms for computations in commutative algebra
closely related to the Gr\"obner algorithm for ideals and modules.  Its
heart is a new implementation of the Gr\"obner algorithm that also allows
for the computation of syzygies.  This implementation is also applicable to
submodules of free modules with generators represented as rows of a matrix.
As main topics CALI contains facilities for
\begin{itemize}
\item defining rings, ideals and modules,

\item computing Gr\"obner bases and local standard bases,

\item computing syzygies, resolutions and (graded) Betti numbers,

\item computing (now also weighted) Hilbert series, multiplicities,
independent sets, and dimensions,

\item computing normal forms and representations,

\item computing sums, products, intersections, quotients, stable
quotients, elimination ideals etc.,

\item primality tests, computation of radicals, unmixed radicals,
equidimensional parts, primary decompositions etc. of ideals and
modules,

\item advanced applications of Gr\"obner bases (blowup, associated graded
ring, analytic spread, symmetric algebra, monomial curves etc.),

\item applications of linear algebra techniques to zero dimensional
	ideals, as {\em e.g.\ }the FGLM change of term orders, border bases
	and affine and projective ideals of sets of points,

\item splitting polynomial systems of equations mixing factorisation and
the Gr\"obner algorithm, triangular systems, and different versions of the
extended Gr\"obner factoriser.

\end{itemize}

There is more extended documentation on this package elsewhere, which
includes facilities for tracing and switches to control its behaviour.

