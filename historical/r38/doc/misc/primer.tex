\documentstyle[11pt,makeidx]{article}
\newcommand{\reduce}{\small REDUCE}
\newcommand{\ttindex}[1]{\index{#1@{\tt #1}}}
\title{REDUCE Symbolic Mode Primer}
\date{}
\author{
H. Melenk \\[0.05in]
Konrad--Zuse--Zentrum \\
f\"ur Informationstechnik Berlin \\
Takustrasse 7 \\
14195 Berlin-Dahlem \\
Germany \\[0.05in]
Email:  melenk@zib.de \\[0.05in]}

\makeindex
\begin{document}
\maketitle
\section{Introduction}

This document should explain some essential technical details for
symbolic mode programming in {\reduce} for those who
have some experience in {\reduce} algebraic programming
and need to write a program in symbolic mode. For a general
introduction to {\reduce} please consult the {\reduce} User's
Manual or one of the available books, e.g.\  ``Algebraic
computing with REDUCE" by Malcolm MacCallum and Francis
Wright (Oxford Press).

This text cannot be complete, as the set of facilities
available in {\reduce} is so rich that it would take years
to describe all and months to read and understand such text.
So a good policy for entering the business of symbolic mode
programming is to study the source files - the
liberal {\reduce} distribution policy simplifies this -
and to understand those parts which contribute to the
field which one would like to use. This text tries to
collect in one place some of the wide spread basic information
in order to facilitate your walk through the {\reduce} mountains.

When should you write a program in symbolic mode?
Symbolic programs are not {\it a priori} better than algebraic
programs - the essential thing is the mathematics which they
implement. A common prejudice is that symbolic programs
are more ``efficient". This is true only if you can
save in symbolic mode substantial algebraic evaluation
steps. As long as most of the
computing time is needed for a few big calculations steps
(integrals, equation solving, polynomial gcd etc.)
you will gain nothing when calling the same procedures
from symbolic mode. However, if you need structures
which are not present in algebraic mode or if you
can design an evaluation avoiding lots of conversions,
the step to symbolic mode programming is justified.

As it is very difficult to design non trivial but short
examples for symbolic mathematical programming no
attempt has been made in that direction. The examples
in this text all are silly - please look at the
sources of {\reduce} for meaningful material. The following pieces
of the sources are good points for first reading as they
provide a substantial functionality within a few pages
of code:
\begin{enumerate}
\item module {\tt polrep} in package {\tt poly}: routines
   addf, addd and addm, the heart of standard form and
   standard quotient arithmetic,
\item module {\tt det} of package {\tt matrix}: internal
   arithmetic with symbolic entities (standard quotients)
   and clever local data structure,
\item module {\tt rational} in package {\tt poly}: implementation
   of a typical {\reduce} domain,
\item module {\tt maxmin} in package {\tt alg}: a typical
   simplification routine for an operator,
\item module {\tt algbool} in package {\tt alg}: demonstrates
   how to supply ``small" pieces of symbolic code for
   algebraic use.
\end{enumerate}

For symbolic mode programming you will urgently need the
{\em Standard LISP Report} which describes
the basic LISP functions; these will be available under all
{\reduce} implementations and they guarantee an optimum
of portability. However, in the course of the
years {\reduce} has imported some additional functions on the
LISP level -- they have been implemented on top of Standard LISP
and live in the module {\tt support}\index{support} of 
the package {\tt rlisp.red}.
In order to prevent the reinvention of the wheel these functions
are described in the appendix as an extension to the
{\em Standard LISP Report}.

The description is based on the recent version of REDUCE. Some of the
described features are not available in earlier versions.
\section{Very short course on RLISP}

\subsection{What is RLISP}

As you will know {\reduce} is based on the programming
language LISP, or to be more exact, on the LISP dialect
{\tt Standard LISP}. Fortunately
\ttindex{LISP} \ttindex{Standard LISP}
you need not learn the syntax of this language with its
large number of brackets - the {\reduce}
language is used for algebraic programming as well as
for symbolic programs and in that mode it allows you to
use all of the rich LISP data structures. It is LISP
semantics in high level {\reduce} syntax. In the following
it is expected that you are familiar with the {\reduce}
programming language as far as it is used for algebraic
mode programs. You should know how to write a (recursive) procedure
with parameters and local variables, how to build loops,
while blocks, {\tt if-then-else} clauses and {\tt begin-end} resp $<< - >>$
blocks. If not, please study the {\reduce} manual first
or have a look at the source files of the {\reduce}
kernel - this is the best method for learning how to
program in RLISP.

\subsection{Modes, function classes}

The symbols {\tt symbolic} (or equivalent {\tt lisp})
\ttindex{symbolic} \ttindex{LISP}
and {\tt algebraic} \ttindex{algebraic}
 play a double role: as a statement
they switch the evaluation mode globally to {\em symbolic}
or {\em algebraic mode} respectively, as a {\em mode prefix} 
\index{mode prefix} they tell the
{\reduce} evaluator that one isolated evaluation should
be done in the named mode. The scope of this prefix use
can be rather narrow (e.g.\ one single expression) or
cover a larger piece of code up to a complete procedure.
Typically procedures in symbolic modules should be tagged
``symbolic" explicitly although this might be redundant
information in a totally symbolic context. If a procedure
needs an evaluation mode different from the actual global
one the {\em mode prefix} must be set.

In symbolic mode there are two additional procedure types
available, {\tt macro} \ttindex{macro} and {\tt smacro}\ttindex{smacro}. 
The discussion of {\tt macro}s is beyond the scope of
this document - their use requires extensive knowledge of
LISP. On the other hand {\tt smacro}s are frequently use
in {\reduce} and you will see lots of them in the sources.
An {\tt smacro} (an abbreviation for ``substitution macro")
is an ordinary procedure tagged with
the symbol ``smacro" and usually with a rather small body.
At source read time (or better: at {\reduce} translator
time) any call for an {\tt smacro} will be replaced
literally by the body of the {\tt smacro} procedure
which saves execution time for the call protocol at
run time. So the purpose of {\tt smacro}s is twofold:
encapsulation of frequently used pieces of code and
an increased efficiency (avoiding function calls).
Example:
\begin{verbatim}
     smacro procedure my_diff(x,y); x-y;

     symbolic procedure hugo(a,b); my_diff(a,b);
\end{verbatim}
Here the formal function call {\em my\_diff} in the
procedure literally is replaced by the body of the
{\tt smacro} where $x$ and $y$ are replaced by $a$ and
$b$. Obviously this translation can be done only if
the {\tt smacro} declaration is entered in a {\reduce}
session before the first reference. And later changes
of an {\tt smacro} don't affect previously translated
references.

Sometimes in older {\reduce} sources you will find
the symbol {\tt expr}\ttindex{expr} in front of a procedure declaration.
This is more or less useless as the absence of {\tt macro}
or {\tt smacro} symbols indicates that a procedure is
of type {\tt expr} - the default Standard LISP procedure type.


\subsection{Evaluation model, symbols, and variables}

The main difference between algebraic and symbolic
mode lies in the {\tt evaluation model}\ttindex{evaluation model}:
\begin{itemize}
\item In algebraic mode a symbol stands for itself
as unknown as long as no value is assigned; after
an assignment it plays the role of a representative
for that value just like a variable in a standard
programming language:
\begin{verbatim}
        1: x;
        X
        2: x:=y+1$
        3: x;
        Y + 1
\end{verbatim}
 
In symbolic mode there 
is a clear barrier between the role of a symbol as
variable of the programming language RLISP,
a named item which represents some variable value,
and the role to be part of an algebraic expression. If you mean
the {\em symbol x} you must write $'x$; without
the quote tag $x$ is considered as {\em variable x}
and it will be asked for its value which is NOT $'x$
initially. Uninitialized variables cause bugs.
\item Consequently all variables {\tt must be
declared}.

\item  In algebraic mode $u:=(x+1)^\wedge 2;$
\footnote{In this text the caret symbol ``$^\wedge$" is
used for exponentiation instead of the older
FORTRAN like operator ``**".}
means  {it compute a formula by expanding (x+1)*(x+1);
if a value had been assigned to x, substitute the value for x}.
In symbolic mode an algebraic expression is interpreted 
as statement to compute a numeric value, just like in {\em C}
or {\em Pascal}.
So $u:=(x+1) ^\wedge 2;$ in symbolic mode means: ``take the
value of variable $x$ which is expected to be number,
add 1, square and assign the resulting number to
the variable $u$". 

\item If you want to refer to an
{\tt algebraic expression} as a data object, you
must code it as an {\tt algebraic form}\index{algebraic form}
(see below)
and mark it as {\tt constant}\ttindex{constant} 
by a {\tt quote}\ttindex{quote}
character. The only constants which don't need a
{\tt quote} are numbers and strings.
Example:
\begin{verbatim}
     u:='(expt (plus x 1) 2);
\end{verbatim}
assigns the (algebraic) expression $(x+1)^\wedge 2$ to the variable $u$.

\item algebraic mode implicitly supports standard arithmetic
and algebraic evaluation
for mathematical expressions of arbitrary complexity and
for numbers from various domains. In symbolic mode,
arithmetic with infix operators $+\,-\,*\, ^\wedge \,/$
is supported only for the basic LISP numbers
(mainly integers). All arithmetic for formulas, even
for domain elements such as rounded numbers, complex
numbers etc. has to be performed by calling explicitly
the functions which can do that job or by calling
explicitly the {\reduce} evaluator {\tt reval}\index{reval}
resp  {\tt aeval}\index{aeval} (see below). 
\end{itemize}

So symbolic mode programs are much more similar to
programs in conventional programming languages such as
Pascal or C. All algebraic functionality of {\reduce} is 
available only as an explicit subroutine interface.
In contrast to algebraic mode nothing is done implicitly
for you.

\subsection{Variable types}

RLISP supports in symbolic mode the following classes of variables:

\begin{itemize}
\item {\tt Local variables}\index{local variables} in a procedure are
 those 
  \begin{itemize}
   \item declared in the parameter list, 
   \item declared in a {\tt scalar}\ttindex{scalar}
      or {\tt integer}\ttindex{integer}
       statement in a begin-end block,
   \item bound in the right-hand side of a {\tt where statement}
   \item  implicitly introduced in a {\tt for statement}
      \ttindex{for statement}. 
  \end{itemize}
 These are
 valid only inside their local context. {\tt scalar} variables
 are initialized to $nil$, {\tt integer} to 0 unless they
 have a special initialization value (property {\tt initvalue*}
 \index{initvalue*}).
\begin{verbatim}
      symbolic procedure my_adder(x,y);
         begin integer r;
            r:=x+y;
            return r:
         end;
\end{verbatim}
In this routine $x$,$y$ and $r$ are local variables.

In algebraic mode the {\tt where} statment is used to activate
one more more rules locally. In symbolic mode mode only rules
of the form \verb+<name>=<value>+ are allowed the the right-hand
side of a {\tt where} statement.

\item {\tt Global}\index{global} variables have to be declared
 in a statement $global\, '(v_1\,v_2\,\cdots);$ These
 can be accessed from everywhere once they have been
 declared. Important: declare them before you use
 them for the first time in a session. They are
 initially set to $nil$.
\begin{verbatim}
      global '(my_support!* my_quotient!*);
\end{verbatim}
 It is a common practice to use a trailing asterisk in names
 for global and fluid variables such that they easily
 can be distinguished from locals. Names of global variables
 may not be used as local variables. 

\item {\tt Fluid}\ttindex{fluid} variables are similar to global
 variables as they can be accessed from everywhere. But
 in contrast to globals a {\tt fluid} variable can occur as a local 
 variable in a procedure. It then has temporarily the value
 assigned in the procedure (we say ``it is rebound");
 this value will be accessible globally by nested procedures
 as long as the rebinding procedure is not yet terminated.
 After termination the previous value is reassigned.
 Fluid variables are declared in a statement
 $fluid\,'(v_1\,v_2\,\cdots);$. Like global variables they
 must be declared before use and they are initially set to $nil$.
\begin{verbatim}
      fluid '(my_var!*);
      my_var!*:='x;
      procedure compute(ex,my_var!*);
         do_some_work(ex);
\end{verbatim}
 In this case the variable $my\_var*$ has been initialized
to the symbol $x$. If then a call $compute('a,'z)$ is
performed, the variable $my\_var*$ will be set to $z$
during the execution of the body of $compute$.

\item A {\tt switch}\ttindex{switch} can be declared using the
 {\tt switch} statement $switch\,s_1,s_2\cdots;$ where
 $s_1,s_2,\cdots$ are symbols. {\reduce}
 automatically connects a fluid variable with each switch
 by prefixing an asterisk to the symbol name. This variable
 represents the $value$ of the switch, which is either
 $nil$ or $t$ and is set by the statements {\tt on}\ttindex{on} and 
 {\tt off}\ttindex{off}.
 E.g.\ the switch $nat$ has the associated global variable
 $*nat$. 
\item {\tt share}\ttindex{share} variables are also global variables
 which are handled specially by the {\reduce} evaluator
 such that their symbolic mode values are identical with their 
 role as symbols in algebraic mode.
\end{itemize}

\subsection{Symbols, numbers and strings}

A {\tt symbol}\ttindex{symbol} or {\tt id}\ttindex{id} 
\footnote{``id" is an abbreviation for ``identifier".} 
in LISP is a unit which has a name, in the standard
case beginning with a letter and followed by letters and digits.
Special characters in a name or a leading digit have to be
flagged by a preceding exclamation mark. Symbols are the same
entities as in algebraic mode. Symbols 
are unique in the system. You may view a symbol as a data
structure with four slots, the {\it name} with is always a string, 
the {\it value cell} to store assigned values, 
the {\it function cell}
\footnote{There are LISP systems which support only one cell
for value and function -- in such systems a symbol can represent
only a variable or a function exclusively.} 
to point to an assigned program structure
and the {\it property cell} as anchor of the property list.
Initially all cells except the name cell are marked empty.


In contrast to algebraic mode in LISP, only integers and
floating point numbers (which don't exist in that form
in algebraic mode) are considered as numbers. All other
numeric quantities from algebraic mode such as rationals, rounded numbers,
Gaussian integers are composite data structures encoded in
the {\reduce} domain structure (see below).

Just as in algebraic mode, a sequence of characters enclosed
in string quotes is a string. Functions for manipulating symbols, 
numbers  and strings 
are\footnote{Read the trailing $p$ in names like ``idp", ``numberp" etc.
as ``predicate", e.g.\  ``numberp" meaning ``number predicate".}:
\begin{center}
\begin{tabular}{|r|l|} \hline
   idp(q)  & true if q is a symbol \\
   numberp(q) & true if q is a number \\
   stringp(q) & true if q is a string \\
   liter(q)   & true if q is a single alphabetic character symbol\\
   digit(q)   & true if a is a digit character symbol\\
   intern(s)  & make a symbol from a string \\
   explode(a) & decompose a symbol/number/string into its characters \\
   explode2(a)& explode, but without escape characters\\
   compress(l)& make a symbol/number/string from a list of characters \\
   gensym()   & generate a new symbol\\
\hline
\end{tabular}
\end{center}
\ttindex{idp}\ttindex{numberp}\ttindex{stringp}
\ttindex{intern}\ttindex{explode}\ttindex{compress}\ttindex{gensym}
\ttindex{liter}\ttindex{digit}

The functions {\tt gensym}  and {\tt compress} make symbols which are
not yet ``internal": they are unique, but there may be
other symbols with the same name. 
It makes debugging especially hard if you cannot
access the symbol by its name of course. The function {\tt intern}
can be used to convert such a symbol into an internal one -
then every reference to the symbol name guarantees 
access to the desired object.

\subsection{Boolean values}

There are two special symbols named $nil$ and $t$. $nil$ is
used at the same time for the empty list and the Boolean
value ``false" \ttindex{false} \ttindex{true} \ttindex{Boolean value}. 
Any value other than $nil$ is considered as
``true" in conditional statements. In order to facilitate programming,
the symbol $nil$ has the preassigned constant value $nil$ such
that a quote here is superfluous. And for the same reason
the symbol $t$ evaluates to itself
 - this symbol can be used if the Boolean value ``true"
is needed explicitly, e.g. as a return value. But please keep
in mind that any non-$nil$ value can play this role. In contrast to
algebraic mode the number zero does {\bf not} represent the
Boolean value ``false".

\subsubsection{Pairs, lists}

The central structure of LISP is the {\tt pair}\ttindex{pair}.
The external representation of a pair of two objects
is $( obj_1 . obj_2)$.  Here the infix dot is used
as print symbol as well as infix constructor.  
It is often useful to use a double box for
representing a pair graphically, e.g.\  for $(10\, . \,20)$:
{\nopagebreak[3]
\begin{verbatim}
            .---------.
            | 10 | 20 |
            `---------'
\end{verbatim}
}

Functions:
\begin{center}
\begin{tabular}{|r|l|} \hline
   o1.o2  & (infix) construct pair $(o_1 . o_2)$ \\
   cons(o1,o2) & same as 01.02 \\
   pairp(q) & true if q is a pair \\
   car(q)   & extract the first object from a pair \\
   cdr(q)   & extract the second object from a pair \\ \hline
\end{tabular}
\end{center}
\ttindex{cons}
\ttindex{.}\ttindex{pairp}\ttindex{car}\ttindex{cdr}

The pair is a very general construct - since its objects themselves
can be pairs, arbitrary trees \ttindex{tree} and data structures can be
built. E.g.\  $((a . b) . (c . d))$
{\nopagebreak[3]
\begin{verbatim}
            .--------.
            | *  | * |
            `/------\'
            /        \
           /          \
        .-------.    .-------.
        | a | b |    | c | d |
        `-------'    `-------'
\end{verbatim}
}


On the other hand, structures with many members
lead to a very complicated representation if printed using
the above notation. Therefore the {\tt list}\ttindex{list} concept has been
introduced which allows one to use a special class of pair trees in a 
simplified form:
\begin{itemize}
\item in a pair with second element $nil$ the dot and the nil
 are omitted: $(A.nil)$ = $(A)$.
\item for a pair with another pair as second element  one bracket
 pair and the dot are omitted: $(B.(A))$ = $(B\, A)$, $(C.(B\, A))$ =
  $(C\, B\, A)$.
\end{itemize}

So the {\tt list} is a linear sequence of pairs, where
the $car$s contain the ``data" items, while the $cdr$s contain
the reference to the successors. The last successor is $nil$, which
signals the end of the linear list. 
For the graphical representation of linear lists horizontally
aligned double boxes are useful, e.g.\ for $(C\, B\, A)$
{\nopagebreak[3]
\begin{verbatim}
        .-------.    .-------.    .-------.
        | c | *-+--->| b | *-+--->| a |nil|
        `-------'    `-------'    `-------'
\end{verbatim}
}
or with omitting the final $nil$
{\nopagebreak[3]
\begin{verbatim}
        .-------.    .-------.    .-------.
        | c | *-+--->| b | *-+--->| a | / |
        `-------'    `-------'    `-------'
\end{verbatim}
}
The {\tt list notation} \ttindex{list notation} is
much simpler than the {\tt dot notation}\ttindex{dot notation} - unfortunately
the dots cannot be avoided completely because pairs with
id's or numbers in the $cdr$ part occur and they play an
important role. 

Lists can be nested; an example is  the internal representation of
algebraic forms (see below) which uses linear lists; e.g.\  
$(x+1)^2$ would be represented by lists as
\begin{verbatim}
     (expt (plus x 1) 2)
\end{verbatim}
Here the top level list has three elements $expt$, $(plus \, x \, 1)$
and $2$ where the second element itself is a linear list of
length 3.
{\nopagebreak[3]
\begin{verbatim}
        .-------.    .-------.    .-------.
        |expt| *+--->| * | *-+--->| 2 | / |
        `-------'    `-|-----'    `-------'
                       |
                      .-------.    .-------.    .-------.
                      |plus| *+--->| x | *-+--->| 1 | / |
                      `-------'    `-------'    `-------'
\end{verbatim}
}
In this graph the {\tt car}s contain the symbols or numbers
or they point to substructures (downwards) while the {\tt cdr}s
point to the successors (to the right) until the end is reached.

As mentioned already there can
be mixtures of both forms if one $cdr$ is not $nil$:
$(A (B.C) D.E)$ - here the second element of the list is
a pair $(A.B)$, the third element is $D$ and there is
no fourth element - the list ends with a non $nil$ symbol.

{\nopagebreak[3]
\begin{verbatim}
        .-------.    .-------.    .-------.
        | a | *-+--->| * | *-+--->| d | e |
        `-------'    `-|-----'    `-------'
                       |
                      .-------. 
                      | b | c | 
                      `-------' 
\end{verbatim}
}
\noindent
The empty list $()$ is identical to the symbol ${\tt nil}$.
Note that the {\reduce} {\tt algebraic forms}\index{algebraic forms}  
are LISP lists where all lists are terminated by a $nil$. 

Functions on lists:
\begin{center}
\begin{tabular}{|r|l|} \hline
 \{o1,o2,..on\} & construct list $(o_1\, o_2 \cdots o_n)$ \\
 list(o1,o2,..on) & same as \{o1,o2,..on\} \\
   o1.l1  & (infix) put $o_1$ in front of list $l_1$ \\
   pairp(q) & true if $q$ is a pair (or a list) \\
   atom(q)  & true if $q$ is  NOT a pair/list  \\
   null(q) & true if $q$ = $nil$ (=empty list) \\
   car(q)   & extract the first object from a list \\
   cdr(q)   & extract the part of the list behind 1st element \\ 
   nth(q,n) & extract the n-th element from list $q$\\
   length(q) & count the number of elements of list $q$\\
   reverse(q) & reverse a list $q$\\
   append(a,b)& append $b$ to the end of a copy of $a$\\
   member(a,l) & test if $a$ is equal to one member of $l$\\
   memq(a,l) & test if $a$ is identical one member of $l$\\
   delete(a,l) & delete $a$ from list $l$\\
\hline
\end{tabular}
\end{center}
\ttindex{list}\ttindex{\{}\ttindex{\}}
\ttindex{.}\ttindex{pairp}\ttindex{atom}
\ttindex{null}\ttindex{car}\ttindex{cdr}\ttindex{nth}
\ttindex{length}\ttindex{reverse}\ttindex{append}
\ttindex{member}\ttindex{memq}

Remarks:
\begin{itemize}
\item All of the above functions are non destructive: an
  input list remains as it has been before - the value of each
  access function is a reference to the corresponding part of the
  structure. 
\item The access functions can operate only if the desired elements
  are really there; e.g.\  if $nth$ with 4 is applied to a 3 element
  list produces an error, and $car$ or $cdr$ applied to any symbol
  or number (including $nil$) will fail.
\item Nested calls of $car$ and $cdr$ can be abbreviated by
  combining the a's and d's between c and r up to four inner a/d letters.
  E.g. $cadr(u)$ = $car(cdr(u))$ \ttindex{c...r}
   - this is the direct access
  to the second element, $caddr$ returns the third element and
  $cadddr$ the fourth element.
\item Although the functions $first$, $second$, $third$, $rest$
  known from algebraic mode operate in some {\reduce}
  implementations in symbolic mode as synonyms of $car$,
  $cadr$, $caddr$, $cdr$, they should not be used here as
  this is the case in all {\reduce} versions.
\item The functions $member$ and $memq$ not only test the membership:
  if $a$ is member of the list $l$, $member$ (or $memq$) returns the (first)
  pair which contains $a$ as its $car$. E.g.\  $memq('b,'(a\ b\ c))$
  returns $(b\ c)$. This can be used either if you want to replace
  this item in the list, or if you want to process the part after
  the search item. 
  The function $memq$ looks for a member of the list which is {\bf eq}
  to the object, while $member$ uses {\bf equal} as test (see discussion
  of equality below).  If applicable, $memq$ is substantially faster than
  $member$.
\item $Delete$ produces a copy of the top level pair sequence leaving  
  out the first occurrence of the search item: the original list is
  not changed, and a possible second occurence of the search item
  in the list is not removed. $delete('a,'(0\,a\,b\,a))$ will result
  in $(0\,b\,a)$ where the part $(b\,a)$ is the original tail of the
  input list while the pairs in front of the original first $a$
  are new.
\end{itemize}

{\nopagebreak[3]
Examples: let $u$ be a variable with value $(A (B.C) NIL (D))$.  Then
\begin{center}
\begin{tabular}{rl}
  pairp(u)   &=t\\ 
  length(u)  &=4\\
  null(u)    &=nil\\
  car(u)     & = A\\
  cadr(u)    &=(B.C)\\
  caadr(u)   &=B\\
  cdadr(u)   &=C\\
  cdr(u)     &=((B.C) NIL (D))\\
  length cdr(u)&=3\\
  cddr(u)    &=(nil (D))\\
  null cddr(u) &=nil\\
  caddr(u)     &=nil\\
  null caddr(u)&=t\\
  cdddr(u)   &=((D))\\
  cddddr(u)  &=nil\\
  null cddddr(u)  &= t\\
\end{tabular}
\end{center}
}

All data which are not pairs (or lists) are classified
as {\tt atoms}\ttindex{atom}: symbols, numbers, strings, vectors(see below)
and, just as in real world, they are not very atomic -- there
are tools to crack them.

\subsection{Programming with lists}

There are several methods available to construct a list: 
\begin{verbatim}
     u := nil;
     u :=1 . u;
     u :=2 . u;
\end{verbatim}
Starting with an empty list, elements are pushed on its front
one after the other, here giving $(2\,\, 1)$. Please note the
blank between the number and the dot: this is necessary as otherwise
the dot would have been parsed as part of the number representing a 
floating point value.
\begin{verbatim}
     u := {9,10};
     u := 1 .u;
     u := 2 .u;
\end{verbatim}
The same, here not starting with the empty list giving $(2\,\, 1\,\, 9\,\, 10)$.
The {\tt for statement} \ttindex{for statement} has special forms
to  handle lists:
\begin{itemize}
\item {\tt for each}\ttindex{for each} $x$ {\tt in} $l$ $\cdots$ performs an
  action elementwise for the elements in list $l$; the elements
  are accessible by the automatically declared local variable $x$
  one after the other.
\item the iterator actions {\tt collect}\ttindex{collect} and {\tt join}
\ttindex{join}
  allow you to construct lists: 
  \begin{itemize}
  \item {\tt collect} generates a list where each
       iteration step produces exactly one element,
  \item {\tt join} expects that each iteration step produces a 
       complete list (or $nil$) and joins them to one long list.
  \end{itemize}
\end{itemize}
Examples: let e.g.\ $l$ be a list of numbers 
$(1\,\, -2\,\, 3\,\, -4\,\, 5\,\, -6)$:
\begin{verbatim}
     m:=for each x in l sum x;
\end{verbatim}
$m$ will be the sum of the elements,
\begin{verbatim}
     s:=for each x in l collect x*x;
\end{verbatim}
$s$ is computed as list with the numbers from $x$ squared,
\begin{verbatim}
     p:=for each x in l join 
         if x>0 then {x} else nil;
\end{verbatim}
in this loop the positive numbers produce one element
lists, while the negative numbers are mapped to nil; joined
together the result is a list with only the positive
elements,
\begin{verbatim}
     r:=for i:=1:10 collect -r;
\end{verbatim}
here the result is the list with numbers $(-1\,\, -2\,\, -3 \cdots -10)$.
The lists produced by {\em collect} and {\em join} are in ``correct" 
order.
{\bf Warning}: In symbolic mode  {\em join} links the lists 
{\bf in place} for maximum speed
\footnote{In algebraic mode {\em join} operates different: the operatend
are copied before linking them to one result list}:
The last {\em CDR} pointer  of the first list is modified to 
point to the head of the second list , and so on.
This is no problem if the joined objects are freshly built
structures, or if there are no other references to them.
If the in -- place operation is dangerous for your application,
you must copy the structure before, e.g. by a call to {\em append}
with {\em NIL} as second argument:
\begin{verbatim}
     r:=for i:=1:10 join append(my_extract(u,r),nil);
\end{verbatim}
In this example, the top level chain of the results of {\em my\_extract}
is copied such that the original structure is not modified when
joining.


Another very common style for list operations is to use 
iterative or recursive programs. The following programs each
sum the elements of the list:
\begin{verbatim}
   symbolic procedure sum_1(l);
    begin  integer n;
      while l do <<n:=n+car l; l:=cdr l>>;
      return n;
    end;
\end{verbatim}
This program picks in each step of the the while loop one
element from the list and sets the variable $l$ to
the successor position; the loop terminates as soon as the 
successor $nil$ is found .
\begin{verbatim}
   symbolic procedure sum_2(l);
     if null l then 0 else car l + sum_2(cdr l);
\end{verbatim}
This program uses recursion for the same task.
The program will go down the list until $nil$ is found,
then set the sum to $0$ and when coming back from the
recursion add the elements to the sum so far.
\begin{verbatim}
   symbolic procedure sum_3(l); sum_31(l,0); % initializing
   symbolic procedure sum_31(l,n);
     if null l then n else sum_31(cdr l,car l+n);
\end{verbatim}
The third example is a bit tricky: it initializes
the sum to 0, and when stepping down in the
recursion the sum is updated and passed to the next
recursion level as a parameter.

It is also very common to produce lists in recursive style, 
e.g.\  inverting the signs in a list of numbers could be written as
\begin{verbatim}
    symbolic procedure invertsgn(l);
       if null l then nil else -car l . invertsgn(cdr l);
\end{verbatim}
and with most LISP compilers this program would as  efficient 
as  the corresponding loop
\begin{verbatim}
    for each x in l collect -x;
\end{verbatim}
but produce less code.
Note that as with {\em for each - collect}, the elements in the resulting
list will be in the original order.
The recursive programming style is typical for LISP and it
is used for  hundreds of {\reduce} routines.

\subsection{In-place operations}

All list construction  described so far except {\bf join} 
is of a copying nature:
every construction by the dot or the curly brackets always creates
fresh pairs and an old list is untouched. However, there are operations
to modify elements or the structure of an existing list in place. These should
be used only with greatest care: if there is another reference to the list,
its structure will be modified too, a possibly unwanted side effect.
The functions 
are\footnote{Read ``reversip" as ``reverse-in-place" and ``nconc" as 
``concatenate"}:
\begin{center}
\begin{tabular}{|r|l|} \hline
 car l := u & replace first element of $l$ by $u$ \\
 rplaca(l,u) & old form for car l := u\\
 cdr l := u & replace cdr part of $l$ by $u$ \\
 rplacd(l,u) & old form for cdr l := u\\
 nth(l,j) := u & replace the j-th element of $l$ by $u$\\
 nconc(a,b)  & insert $b$ as cdr in the last pair of $a$\\
 reversip(l) & invert the sequence of $l$ using the same pairs again\\
\hline
\end{tabular}
\end{center}
\ttindex{nth}\ttindex{car}\ttindex{cdr}\ttindex{nconc}\ttindex{reversip}


\subsection{Equal tests: Equality vs.\  Identity}

There are two classes of equality in LISP, $eq$ and $equal$,
\ttindex{eq} \ttindex{equal}  \ttindex{=}
in {\reduce} written as the infix operators {\tt eq}, and   
equal sign $=$ respectively. As  symbols are unique,
both are equivalent for them: if symbols are {\tt equal}, 
they are {\tt eq}.

$u:='a; v:='a;$ $ u\, eq\, v \Rightarrow t$

\noindent
here the variables $u$ and $v$ represent the same symbol
and consequently they are {\tt eq}.

For lists, things are different. Here we must distinguish
between references to pairs which contain the same contents
(``equal") or references to the same identical pair
instance(``{\tt eq}").
A pair is {\tt eq}
to itself only, but it is {\tt equal} to a different pair if
their $car$s and $cdr$s are {\tt equal} - this is a
recursive definition. Example:
    
$u:=\{1,2\}; v:=\{1,2\};w:=u;$ 

$u=v \Rightarrow t$ ,$u=w \Rightarrow t$

 $u\,\,eq\,\,v\Rightarrow nil$, $u\,\,eq\,\,w\Rightarrow t$ 

Here $u$, $v$ and $w$ have as values pairs with same $car$s 
and {\tt equal} $cdr$s, but only $u$ and $w$ refer to
the identical pair because for $v$ fresh pairs have
been built. 
In some points {\reduce} relies on identity of some structures,
e.g. for composite kernels \index{kernel} of
{\tt standard forms}\index{standard forms}.

\subsection{Properties, Flags}

Another important concept of LISP is the ability to
assign {\tt properties}\ttindex{property} and 
{\tt flags}\ttindex{flag} to each
symbol. A {\tt property} is a symbol (a ``name") together
with a value. Both, property name and property value are
completely free. Properties are set by the function {\tt put}
\ttindex{put} and read by the function {\tt get}\ttindex{get}.
\begin{verbatim}
      put('hugo,'evaluator,'compute_hugo);
        . . .
      get('hugo,'evaluator) => compute_hugo
      get('otto,'evaluator) => nil
\end{verbatim}
A property-value pair remains assigned for a symbol until the value is 
replaced by a different value or until the property is removed
by {\tt remprop}\ttindex{remprop}.

A {\tt flag} is similar to a property, but
it represents only the values ``yes" or ``no" (= flag is
present or absent). The functions for setting and
testing flags are {\tt flag}\ttindex{flag} and {\tt flagp}\index{flagp}, 
{\tt remflag}\index{remflag} for removal.

As most other LISP programs, {\reduce} makes extensive use of properties. 
E.g.\  the  algebraic properties of an operator are encoded
in that way. So the symbol $plus$ has among others the {\reduce}
properties $prtch=+$ (the print character), 
$infix=26$\footnote{This number may change if additional operators
of lower precedence are introduced}
(the precedence among the infix operators), $simpfn=simpplus$
(the name of the routine responsible for its simplification)
and the flags $symmetric$, $realvalued$, $nary$ and some more.
These allow a data driven programming model which has been used
in the LISP world for decades already: if the simplifier
operates on an expression, it asks the leading operator for
its simplification function; if there is one (and there will be
one for all operators), this function is invoked to do the job.

There are various fields where this type of programming is used
inside {\reduce}, e.g.\ the complete domain arithmetic has been
implemented in that way: any non-integer domain element is encoded
as a list with a leading symbol (the domain mode) which contains
all knowledge about arithmetic, conversion, printing etc.\  for
this domain as properties and flags.  

If your {\reduce} system is based on PSL, you can inspect a
property list by calling the function $prop$\ttindex{prop} which  
has one argument (the symbol to be inspected) and which returns
the properties and flags as list. E.g. 
\begin{verbatim}
    a := mat((1,2),(3,4));
    prop 'a;
    prop 'mat;
    prop 'matrix;
\end{verbatim}


\subsection{Alists}

Another frequently used LISP structure is the 
{\tt association list}\ttindex{association list} 
({\tt alist}\ttindex{alist}), which is
an associative memory: a list where each element
is a pair with a search item and a value item.
The function {\tt assoc}\ttindex{assoc} picks the pair for
a given search item out of the {\tt alist}.
For example constructing a multiplier with memory:
{\nopagebreak[3]
\begin{verbatim}
      fluid '(product_database*);
      symbolic procedure multiply(u,v);
       begin scalar w,r;
        w:=u.v;
        r:=assoc(w,product_database*);
        if r then return cdr r; % found
        r:=u*v;
        product_database*:=(w.r).product_database*;
        return r;
      end;
\end{verbatim}
}
here the pair $u.v$ is used as search item. If an
entry is found in the database, its value part is
returned. Otherwise  the product is computed and
the new entry is added to the database. After calling
this function with arguments 2 and 3, and then with 2 and 6 the
structure would be
{\nopagebreak[3]
\begin{verbatim}
     .-------.    .-------.
     | * | *-+--->| * | / |
     `-|-----'    `-|-----'
       |            |
     .-------.    .-------.
     | * | 6 |    | * | 12|
     `-|-----'    `-|-----'
       |            |
     .-------.    .-------.
     | 2 | 3 |    | 2 | 6 |
     `-------'    `-------'
\end{verbatim} 
}
Here each element in the top level list points to one 
generalized name-value pair, where the name itself is a structure
(here a pair of two numbers). {\reduce} uses association lists
for many different purposes; e.g.\ if an expression is simplified, 
the expression and its simplified variant are stored in an 
association list such that for a second access to the same expression 
its simplified form is immediately accessible.


\subsection{Vectors}

Lists enable access of items of a linear collection by
position number using the function {\tt nth}; however, this
access is inefficient if used frequently because for
every access the sequence of pairs has to be traversed
until the desired object is reached. An alternative
structure for direct access by element number in 
symbolic mode is the {\tt vector}\ttindex{vector}. The first element
in a vector has the index $0$. \ttindex{putv}\ttindex{getv}\ttindex{upbv}
\begin{center}
\begin{tabular}{|r|l|} \hline
 v:=mkvect(n) & create a vector of length $n$ \\
 putv(v,m,u)  & put value $u$ into $m$th slot of vector $v$\\ 
 getv(v,m)    & extract element $m$ from vector $v$ \\
 upbv(v)      & returns upper bound for $v$ \\
\hline
\end{tabular}
\end{center}
\ttindex{mkvect}\ttindex{getv}\ttindex{putv}\ttindex{upbv}
Note that {\tt putv} is a ``destructive" operation.
\section{Algebraic data formats and evaluation}

\subsection{Prefix forms}

The basic format for representing mathematical
formulae is the {\tt algebraic form} \ttindex{ algebraic form}: an
expression is internally encoded as a list
with {\tt prefix}\index{ prefix} operators: 
\begin{itemize}
\item the first element of a list is an operator,
\item the following elements of a list are the arguments
  which can be symbols (unknowns), numbers or 
  {\tt algebraic forms} in prefix notation.
\end{itemize}

Examples:

$x+1 \Rightarrow (PLUS\,\, X\,\, 1)$

$x+y+1 \Rightarrow (PLUS\,\, X\,\, Y\,\, 1)$

$x+ y*z + 1 \Rightarrow (PLUS\,\, X\,\, (TIMES\,\, Y\,\, Z)\,\, 1)$

$x^\wedge (y+1) \Rightarrow (EXPT\,\, X\,\, (PLUS\,\, Y\,\, 1))$

Algebraic forms are used for many purposes, for data input,
for transferring data between various system components and
for output. To get a feel as to how algebraic forms look
like, you can make {\reduce} display them for you, e.g.\  
by the following sequence:

\begin{verbatim}
    u:=(x+y)^3/(log z-sqrt 2);
    symbolic reval 'u;
\end{verbatim}

The first statement assigns an algebraic expression to a
variable as usual in algebraic mode, and the second statement
forces {\reduce} to print the algebraic form corresponding
to this expression in symbolic mode.


\subsection{*SQ}

{\reduce} uses internally a different
representation for algebraic expressions, the
{\tt standard quotient}\index{standard quotient}({\tt SQ},
see below). As conversion
between algebraic forms and standard quotients is expensive,
{\reduce} tries to keep the data as long in {\tt SQ} 
form as possible. For this purpose there is a special
variant of the algebraic form $(*SQ \,\,  u \,\, ind)$: here the
internal operator $*SQ$ indicates that the parameter $u$
is not an algebraic form but a standard quotient. If such
expression is evaluated, {\reduce} detects
immediately that it will find the standard quotient
already as the second element of the list and no more
conversion is necessary. 

However, if since creation of this
form some values in the system have been changed, the
form $u$ might be no longer the ``correct" standard
quotient; e.g.\  one of its unknowns might have a value
now. For this purpose the third element $ind$ in the list
has been introduced: As long as the form $u$ is the correct
one, it will be $T$. As soon as a significant value in the
system changes which might affect the validity of $u$,
{\reduce} switches all $ind$ elements in the $*SQ$ forms
swimming around to $nil$. So the evaluator decides from the
second parameter whether to use $u$ unchanged or 
to re-evaluate.

By the way, the global replacement of the second parameter
in the $*SQ$ forms is a good example of an in-place
list operation with a wanted side effect:
{\reduce} maintains a unique  list $*sqvar*=(t)$ and all
$*SQ$ forms are built with this list as last pair.
If the environment changes it is sufficient to replace
this single {\bf $t$} in $*sqvar*$ by $nil$ and all $*SQ$ forms 
built with the actual $*sqvar*$-tail inherit this
automatically switching from $t$ to $nil$; 
{\reduce} then creates a fresh list $*sqvar*=(t)$
to be used for the next generation of $*SQ$ forms.

\subsection{Composite forms}

\subsubsection{Lists, equations}

For algebraic mode, lists are represented by the prefix
operator $LIST$. From the symbolic mode standpoint it may 
appear a bit curious that one has to insert
the symbol $LIST$ into a list in order to make it
a correct algebraic object, but this is necessary
in the prefix algebraic context.
E.g.\ try in algebraic mode

\begin{verbatim}
   u:={a,b,{c,d},e};
   lisp reval 'u;
\end{verbatim}
you will see the result $(LIST\,\, A\,\, B\,\, (LIST\,\, C\,\, D)\,\, E)$.

Correspondingly {\tt equations}\index{equation} are represented by
the prefix operator $EQUAL$ followed by the right-hand and
the left-hand sides as parameters; e.g.\  $x=sqrt(2)$
is represented as $(EQUAL\,\, X\,\,(SQRT\,\, 2))$;

The result of {\em solve (x$\wedge$ 2=2,x);} is an instructive
mixture of all these forms:
\begin{verbatim}
(LIST (EQUAL X (!*SQ (((((EXPT 2 (QUOTIENT 1 2)). 1). -1)). 1) T)) 
      (EQUAL X (!*SQ (((((EXPT 2 (QUOTIENT 1 2)). 1).  1)). 1) T)))
\end{verbatim}
this form is a list with two equations and the equations
have $*SQ$ forms as right-hand sides.

\subsubsection{Matrices}

A {\tt matrix}\ttindex{matrix} too is represented internally by
a prefix form with operator $MAT$ followed by the rows;
each row is a list containing the elements as algebraic
forms. The elements can be accessed easily using the
access function $nth$. Example: the matrix $mat((1,2),(3,4))$
is encoded as $(MAT\,\,(1\,\, 2)\,\, (3\,\, 4))$.
The form $nth(nth(m,3),2)$ gives access to the element (2,2).
Be careful if you use this form for an assignment:
$nth(nth(m,3),2):=17;$ is an in-place operation.
You can create a matrix easily by a symbolic program, e.g.
a n*n unit matrix
\begin{verbatim}
   symbolic procedure my_unit_matrix(n);
     'mat . for i:=1:n collect
       for j:=1:n collect if i=j then 1 else 0;
\end{verbatim}

\subsection{Algebraic evaluator: reval, aeval, aeval*}

The main entries to the algebraic evaluator are the 
symbolic mode functions {\tt reval}\ttindex{reval} and 
{\tt aeval}\ttindex{aeval}.
Both have as input an algebraic expression 
and return the fully evaluated result. Reval always returns 
an algebraic form in full prefix operator form, while
aeval tries to keep the result in $*SQ$-form whenever
possible. Almost all algebraic functionality of {\reduce} can be
invoked by calling these procedures, e.g.

\begin{verbatim}
  reval '(solve (plus (expt x 2) 3) x);
\end{verbatim} 

If you turn on the switch {\tt defn} in algebraic mode
you will see that {\reduce} translates most statements
into calls for {\tt aeval}.

Reval and aeval are also adequate tools for accessing
the values of algebraic quantities and for evaluation of
unevaluated procedure parameters; they do the complete
job for you.

 However, there is one exception: 
the left-hand side of an equation in general is not
evaluated by {\tt reval} and {\tt aeval}. E.g.\  

\begin{verbatim}
    l:= a=b; 
    a:=17;b:=4;
    symbolic reval 'l;
\end{verbatim}
produces the output $(EQUAL\,\, A\,\, 4)$. If you write
a program for equations you either should evaluate
the left-hand sides yourself by calling {\tt reval}
or rebind the internal switch  {\tt *evallhseqp}\ttindex{evallhseqp}
\footnote{Read ``evallhseqp" as ``evaluate left-hand side 
of equations predicate".}
locally(!) by $t$ - then {\reduce} will evaluate both sides
of an equation. 

{\reduce} maintains an internal buffer {\tt alglist*}\ttindex{aglist*}
where all evaluation results are stored such that a repeated request
for the same evaluation can be satisfied by a simple and fast table 
lookup\footnote{The use of {\tt alglist*} is controlled by an internal
switch {\tt *uncached}\ttindex{*uncached}\ttindex{uncached}: if this
variable has a non-nil value, the caching is suppressed.}. 
As a side effect, {\tt aeval} resets {\tt alglist*}
empty before starting a 
computation\footnote{Of course, there are many more places where {\tt alglist*}
needs to be reset; e.g.\  any assignment causes {\tt alglist*} to be
reset because the assigned value might make one of the stored values
invalid.}. There is a second function
{\tt aeval*}\ttindex{aeval*}: this function performs the same
operation, but in contrast to {\tt aeval} it does not reset
{\tt alglist*} and can profit from previous results. The {\reduce}
uses both, {\tt aeval} and {\tt aeval*} depending of the specific
environment of a translated statement.
\section{Standard Forms, Standard Quotients}

\subsection{Introduction}

On of the most important tasks performed by a computer algebra
system is the reduction of an algebraic expression to a 
canonical form. A reduction process is canonical if it
guarantees that two expressions are transformed to the
same form if and only if they are algebraically identical.
Unfortunately canonical forms are not known for all algebraic 
objects, e.g.\ for surds. The canonical form of {\reduce}
is the {\tt standard quotient}\ttindex{standard quotient} 
(``{\tt SQ}") which is
a quotient of two {\tt standard forms}\ttindex{standard form} (``{\tt SF}").
An {\tt SF} is a polynomial in internal representation and
an {\tt SQ} is a quotient of two polynomials, a 
rational function. At the same time {\tt SQ} and
{\tt SF} are the forms  used internally for most arithmetic.

In this section the structures of {\tt SF} and {\tt SQ}
are described, followed by a list of operations on these. For
programming and debugging in symbolic mode it is often
necessary to decipher an expression of this type.
However, the knowledge about this part of the internal
{\reduce} structure should never be used to access
{\tt SF} of {\tt SQ} directly by LISP primitives. Instead
use only ``official" primitives supplied by {\reduce}.

\subsection{Coefficients}

\ttindex{coefficient domains}
The coefficients in {\tt SF} and {\tt SQ} are the ``numbers"
from various algebraic contexts. They are called {\tt domain}\ttindex{domain}
elements  and uniformly represented either
as integer numbers or pairs where the {\tt car} part contains
an identifier describing the domain and its arithmetic
by its {\tt properties}\ttindex{property};
the {\tt cdr} part describes the value.

\begin{enumerate}
\item{Integers}: integer numbers $\in {\bf Z}$ represent integer 
coefficients directly,
\item{Gaussian integers}: complex numbers with integer real and
imaginary part {\em (:gi: re . im)} with $re,im \in {\bf Z}$,
\item{Rational numbers}: quotients of integers
{\em (:rn: nu . de)} with $nu,de \in {\bf Z}$,
\item{Modular numbers}: numbers modulo the current prime 
{\em (:mod: . j)} where $0 \leq j < currentmodulus*$
\footnote{The upper half of the modular numbers are printed as negative
numbers if the switch {\tt balanced\_mod}\ttindex{balanced\_mod} 
is on; hovever, that does not affect the internal representation
which uses always integers $\geq 0$}.
\item{Rounded numbers}: floating point numbers extended in
precision and magnitude {\em (:rd: . fp)} with LISP floating point number
$fp$ or {\em (:rd: mant . exp)} with integers $mant$ and $exp$
representing an extended floating point value,
\item{Rounded complex numbers}: numbers with rounded real
and imaginary parts {\em (:gr: fre . fim)} with LISP floating point numbers
$fre$ and $fim$ or {\em (:gr: (rmant . rexp) . (imant . iexp))} with
extended real and imaginary parts.
\end{enumerate}

The neutral elements {\em zero} and {\em one} for
addition and multiplication respectively are identical
for all domains: {\em zero} is represented as $nil$ and {\em one} by the
integer number 1.

The list of domains is open ended: if a new coefficient domain 
is required for an application field, this can be created and linked
to the system at any time. The necessary structure is described
in the source file for the module $dmodeop$\index{dmodeop module}.

The tag of the currently active domain is stored in the value of the
variable {\tt dmode*}\ttindex{dmode*}. 
$nil$ represents the default domain (integers).
This variable should not be modified directly - instead use 
the function {\tt setdmode}\ttindex{setdmode}:

\begin{verbatim}
     setdmode(DOM:domain,MODE:bool);
\end{verbatim}
where $domain$ is a domain name (e.g. $'rational$) and the Boolean
second parameter tells whether to activate ($t$) or to deactivate
($nil$) the domain. Note that $setdmode$ does not
set {\tt on} the corresponding switches. 
You {\em must} do that yourself.
E.g. for a proper activation and deactivation of complex rounded
call
\begin{verbatim}
    !*rounded := t;
    setdmode('rounded,t);
    !*complex := t;
    setdmode('complex,t);
    .... % do your computation
    setdmode('complex,nil);
    !*complex := nil;
    setdmode('rounded,nil);
    !*rounded := nil;
\end{verbatim}

In order to convert the domain tag into the domain name
use the property {\tt dname}\ttindex{dname}, e.g.
\begin{verbatim}
      get(dmode!*,'dname);
\end{verbatim}
will result in the name of the current domain.

\subsection{Generic arithmetic with domain elements}

There is a complete set of arithmetic functions for performing
arithmetic operations with domain elements:

\begin{center}
\begin{tabular}{|r|l|} \hline
    !:zerop(a)        & test $a=0$ \\
    !:plus(a,b)       & $a+b$ \\
    !:difference(a,b) & $a-b$ \\
    !:minus(a)        & $-a$  \\
    !:minusp(a)       & test $a<0$ \\
    !:times(a,b)      & $a*b$ \\
    !:quotient(a,b)   & $a/b$ \\
    !:recip(a)        & $1/b$ \\
    !:expt(a,n)       & $a^n$ \\
    !:gcd(a,b)  & $greatest\, common\, divisor\, of\,(a,b)$ \\
\hline
\end{tabular}
\end{center}

\subsection{Kernels, kernel order}

{\tt Kernels} are the ``unknowns" in a polynomial. 
A \ttindex{kernel}{\tt kernel} 
can be an
\begin{itemize}
\item{identifier}: e.g. $x$, $alpha$, represented as LISP symbols
$X$, $ALPHA$,
\item{operator form}: e.g. $sin(alpha)$,$sqrt(x+1)$,$bessel(j,x)$,
represented as algebraic expression $(SIN ALPHA)$,
$(EXPT\,\,(PLUS\,\, X\,\, 1)\,\, (QUOTIENT\,\, 1\,\, 2))$, $(BESSEL\,\, J\,\, X)$.
\item{A standard form}, e.g. $(((A\,\,.\,\,1)\,\,.\,\,1)\,\,.\,\,1)$  
This option is used only if the default mode of full expantions is
turned off (see below: non-expanded standard form).
\end{itemize}
It is essential for the operation of {\reduce} that kernels are unique.
This is no problem for id's as these are unique by LISP definition.
For composite kernels {\reduce} ensures that these are 
made unique before they enter a {\tt standard form}.
{\em Never construct a composite kernel by hand}. Instead use
the function {\tt *a2k}\ttindex{*a2k} for converting an algebraic expression
into a kernel or use $simp$ (see below). On the other hand,
the uniqueness of kernels is of big advantage for comparisons:
the equality can be tested with $eq$ instead of = and faster $memq$ and
$atsoc$ can be used instead of $member$ and $assoc$.

The set of all kernels is totally ordered by 
the {\tt kernel order}\ttindex{kernel order}.
{\reduce} has by default some ordering which is roughly based
on lexicographic comparison. The ordering can be changed internally
by calling {\tt setkorder}\ttindex{setkorder} 
(corresponding to the algebraic statement
{\tt korder}\ttindex{korder})  with a list of kernels as argument. 
These kernels 
precede all other kernels in the given sequence .{\tt Setkorder} returns
as result the previous order and it is a good policy to restore
the old order after you have changed it:
\begin{verbatim}
     begin scalar oldorder, ....;
       oldorder:=setkorder '(x y z);
        ....
       setkorder oldorder;
        ....
     end;
\end{verbatim}
The current kernel order list is the value of the variable 
{\tt kord*}\ttindex{kord*}.
The value $nil$ signals that the default ordering is active.

To check the  system order of two algebraic expressions use the 
function {\tt ordp}\ttindex{ordp} with two arguments, 
which returns true if the
first argument is lower in the ordering sequence as the second one.
This check does not consider the actual {\tt kord*}.
For kernels there is an extended order check function 
{\tt ordop}\ttindex{ordop} which does consider the actual {\tt kord*}.
However, you need to call these functions only rarely as the
algebraic engine of {\reduce} maintains a proper order in any
context.

\subsection{Standard form structure}

A {\tt standard form}\ttindex{standard form} is a polynomial in internal recursive
representation where the kernel order defines the recursive
structure. E.g.\  with kernel order $(x\,\,y)$ the polynomial 
$x^2y +x^2 + 2xy + y^2 + x + 3$ is represented as a polynomial
in $x$ with coefficients which are polynomials in $y$ and integer coefficients:
$x^2*(y*1+1) + x*(y*2+1) +(y^2*1 +3)$; for better correspondence
with the internal representation here the integer coefficients
are in the trailing position and the trivial coefficients $1$ are 
included.
A standard form is 

\begin{itemize}
\item $<$domain element$>$  
\item $<$mvar$>$\ .** $<$ldeg$>$\ .* $<$lc$>$\ .+ $<$red$>$, 
    internally represented as {\em (((mvar.ldeg).lc).red)} 
\end{itemize}
with the components
\begin{itemize}
\item {\em mvar}: the leading kernel,
\item {\em ldeg}: an integer representing the power of $mvar$ in the leading
 term,
\item{\em lc}: the leading coefficient is itself a standard form,
 not containing $mvar$ any more,
\item{\em red}: the reductum too is a standard form, containing $mvar$
  at most in powers below $ldeg$.
\end{itemize}
Note that any standard form ends with a domain element which 
is  $nil$ if there is no constant term.
 E.g.\  the above polynomial
will be represented internally by

\begin{verbatim}
  (((X .2) ((Y. 1). 1). 1) ((X. 1) ((Y. 1). 2)) ((Y. 2). 1). 3)
\end{verbatim}

with the components
\begin{itemize}
\item{mvar}: \verb+X+
\item{ldeg}: \verb+2+
\item{lc}: \verb+((Y. 1). 1). 1)+  = $(y+1)$
\item{red}: \verb+(((X. 1) ((Y. 1). 2)) ((Y. 2). 1). 3)+ = $xy^2+3$.
\end{itemize}

\subsection{Operations on standard forms}

Arithmetic with standard forms is performed by the
functions\footnote{Be careful not to intermix 
the names ``minusf" and ``negf".
And take into account that {\tt minusf} will return a positive result
only if the value is definitely known to be less that 0.}
\begin{center}
\begin{tabular}{|r|l|} \hline
    null(f)    & test $f=0$ \\
    minusf(f)  & test $f<0$ (formally) \\
    domainp(f) & test if $f$ is a domain element (e.g. number) \\
    addf(f1,f2)& f1 + f2 \\
    negf(f)    & -f \\
    addf(f1,negf f2) & f1-f2 \\
    multf(f1,f2) & f1*f2 \\
    mksp**(f1,n) & compute f1**n \\
    quotf(f1,f2) & $f1/f2$ if divisible, else $nil$ \\
    quotfx(f1,f2) & $f1/f2$ divisiblity assumend, no check\\
    qremf(f1,f2) & pair (quotient.remainder) \\
\hline
\end{tabular}
\end{center}
\ttindex{null}\ttindex{domainp}\ttindex{addf}
\ttindex{negf}\ttindex{multf}\ttindex{mksp}
\ttindex{quotf}\ttindex{qremf}
    
As standard forms are a superset of domain elements these
functions also can be used for arithmetic with domain elements
or for combining standard forms with integer numbers. They
can even be used to compute with integers as long as these
are regarded as ``degenerate" polynomials.


For access of the components of a standard form
there are functions

\begin{center}
\begin{tabular}{|r|l|} \hline
    mvar(f)    & extract leading kernel \\
    ldeg(f)    & extract leading degree \\
    lc(f)      & extract leading coefficient \\
    red(f)     & extract reductum f\\  \hline
\end{tabular}
\end{center}
\ttindex{mvar}\ttindex{ldeg}\ttindex{lc}\ttindex{red}

These functions can be applied only if the form
is not a domain element. The following example presents
a typical function evaluating a standard form; it looks for the
maximum power of a specified kernel $y$ in a standard form $f$:

\begin{verbatim}
  symbolic procedure my_maxpow(f,y);
    if domainp f then 0 else
    if mvar f eq y then ldeg f else
    max(my_maxpow(lc f,d),my_maxpow(red f,d));
\end{verbatim}

This function has a double recursion which is typically
for functions traversing standard forms: if the top node is
not trivial ({\tt domainp}) or not a power of $y$ 
the procedure branches to the
coefficient form and to the reductum form - both
can contain references to the kernel $y$. Kernels are
unique in the system and therefore the equality test is
performed by {\tt eq}.

For the initial construction of a standard form a safe and efficient
way is to build an algebraic expression first and convert
it using $simp$ (see below) or to use the arithmetic functions 
above too. For elementary conversions use

\begin{center}
\begin{tabular}{|r|l|} \hline
    *k2f(k) &  convert unique kernel $k$ to {\ SF} \\ 
    numr simp(a)& convert algebraic expression $a$ to {\tt SF}\\
    prepf(f)& convert a {\tt SF} $f$ to an algebraic expression\\ 
\hline
\end{tabular}
\end{center}
Note that integers are already an {\tt SF} and so need not be converted.
    
Although the are infix macros \fbox{.**} , \fbox{.*} , \fbox{.+} to
construct standard forms, they should be avoided unless
you are completely sure what you are doing: they construct
without any consistency checks, and wrong structures may lead
to inconsistent results; such cases are difficult to debug.

\subsection{Reordering}

If you change the {\tt kernel order}\ttindex{kernel order} it is 
necessary to reorder {\tt standard forms} if they
are processed under the new order by calling

\begin{center}
\begin{tabular}{|r|l|} \hline
    reorder(f)    & reorders $ f$ to the current order \\ \hline
\end{tabular}
\end{center}
\ttindex{reorder}

Please ensure that you never use a standard form which does
not correspond to the actual order - otherwise very
strange results can occur. Freshly created standard
forms will always have the current order.

\subsection{Standard Quotients}

A {\tt standard quotient}\ttindex{standard quotient} is a pair of 
{\tt standard forms}
$( numr\,\,./\,\,denr)$, in LISP represented as $(numr\,\,.\,\,denr)$.
If the denominator is trivial, the standard form $1$ is
used. The constructor, selector  and conversion functions are 

\begin{center}
\begin{tabular}{|r|l|} \hline
    numr(q)    & select the numerator part \\ 
    denr(q)    & select the denominator part \\
    f2q(f)      & convert a standard form f to {\tt SQ}\\
    simp(a)     & convert algebraic form $a$ to {\tt SQ}\\
    prepsq(q)   & convert {\tt SQ} to algebraic form\\
\hline
\end{tabular}
\end{center}
\ttindex{numr}\ttindex{denr}\ttindex{f2q}
\ttindex{simp}\ttindex{prepsq}


Arithmetic with standard quotients is performed by the
functions
\begin{center}
\begin{tabular}{|r|l|} \hline 
    addsq(q1,q2)& $q1 + q2 $\\
    negsq(q)    & -q \\
    subtrsq(q1,q2) & q1-q2 \\
    multsq(q1,q2) & q1*q2 \\
    quotsq(q1,q2) & $q1/q2$ \\
\hline
\end{tabular}
\end{center}
\ttindex{addsq}\ttindex{negsq}\ttindex{subtrsq}
\ttindex{multsq}\ttindex{quotsq}

Note that there is no zero test for standard quotients; instead
use the zero test for the numerator $null(numr(q))$.

When should you use {\tt standard quotients} and when {\tt standard forms}?
If you are sure that no denominator other
than one can occur you should use {\tt standard forms}, otherwise
{\tt standard quotients} are the only possible choice. You can of course
mix arithmetic with both, however that strategy is 
not recommended unless you keep track carefully;  
debugging is hard. Arithmetic functions have been coded
for maximum efficiency: none of them test the validity of their
arguments.

\subsection{Substitutions}

For {\tt substituting}\ttindex{substituting} kernels in {\tt standard forms} 
or  {\tt standard quotients} 
by other kernels or arbitrary algebraic expressions the
following functions are available \index{subf}\index{subsq}:
\begin{center}
\begin{tabular}{|r|l|} \hline 
    subf(f,a)& substitute $a$ in {\tt SF} $f$ \\
    subsq(q,a)& substitute $a$ in {\tt SQ} $q$ \\
\hline
\end{tabular}
\end{center}
where a is a list of pairs, each with a kernel to be replaced 
in the $car$ and its replacement as algebraic form in the $cdr$
\footnote{The second argument of {\tt subf} and {\tt subsq} 
is a typical association list structure.},
e.g.
\begin{verbatim}
    uhu := simp('(expt(plus x y) 10));
    otto:= subsq(uhu, '((x . 5) (y . (plus a b))));
\end{verbatim}
Here $x$ is replaced by 5 while y is replaced by $a+b$.

Note that
\begin{itemize}
\item both {\tt subf} and {\tt subsq} return a {\tt standard quotient}
(the substitution might have caused a non trivial denominator),
\item {\tt subf} and {\tt subsq} also introduce substitutions which have been
coded in rules - so it makes sense to call these routines with
$nil$ as the second argument if you have introduced a new rule and
some standard forms or standard quotients need to be re simplified
under these.
\end{itemize}

\subsection{Useful facilities for standard forms}

\subsubsection{Kernel list}

The function {\tt kernels}\ttindex{kernel} extracts all kernels
from a standard form and returns them as list.
\begin{verbatim}
    kernels numr simp '(plus (expt x 3)(expt y 4));
\end{verbatim}
\noindent
will result in $(x\,y)$.


\subsubsection{Greatest common divisor}

The greatest common divisor of two polynomials is computed
by the function \ttindex{gcdf}{\tt *gcdf}. It takes as arguments
two standard forms and returns a standard form or a 1
if there is no non trivial common divisor.

\subsubsection{Factorization}

The function {\tt fctrf}\ttindex{fctrf} is the interface for calling the 
{\tt factorizer}\ttindex{factorizer}. It takes a standard form as parameter
and returns a list which contains

\begin{itemize}
\item as first element the content of the polynomial - that is
the gcd of the coefficients or the ``common numeric factor",
\item as second and following elements the factors in the
form $({\tt SF} . m)$ where {\tt SF} is the factor as standard
form and $m$ is its multiplicity. 
\end{itemize}
So calling $fctrf$ for  $2x^2-2$ will result in
\begin{verbatim}
(2 ((((X . 1) . 1) . 1) . 1) ((((X . 1) . 1) . -1) . 1))
\end{verbatim}

Here you see the common domain factor 2 and the two factors
$x+1$ and $x-1$ as standard forms paired with their multiplicity
$1$.

\subsection{Variant: non-expanded standard form}

The standard form structure described so far is used by default
for the fully expanded expressions with common denominator.
By setting  {\tt on factor}\ttindex{factor}, 
{\tt off exp}\ttindex{exp}, or {\tt off mcd}\ttindex{mcd}
the slots $mvar$ and $ldeg$ may be used in an exended style:
\begin{itemize}
\item $ldeg$ can be a negative number to represent a reciprocal factor, 
    
\begin{verbatim}
off mcd; (x*a+y*b)/(a*b);

 -1      -1
a  *y + b  *x

lisp ws;
(!*sq ((((a . -1) ((y . 1) . 1)) ((b . -1) ((x . 1) . 1))) . 1) t)
\end{verbatim}
    Here the numerator form of the standard quotient is a sum where each
    term is a product of one kernel with a positive and one kernel with
    a neagative exponent.

\item $mvar$ may be a standard form to represent a factored polynomial,
\begin{verbatim}
on factor; (x^2+y)^2*(y+1);

  2     2
(x  + y) *(y + 1)

lisp ws;

(!*sq (((((((x . 2) . 1) ((y . 1) . 1)) . 2) (((((y . 1) . 1) . 1)
   . 1) . 1))) . 1) t)
\end{verbatim}
    Here the numerator is a standard form in factored form where the inner polynomials
    $(x+y)$ and $(y+1)$ are used as $mvar$s, the first one with the exponent 2
    and the second one with exponent 1. 
\end{itemize}

Special functions to work with composite standard forms:
\begin{center}
\begin{tabular}{|r|l|} \hline
    sfp(m)    & test if $m$ is a standard form (T) or a kernes(NIL) \\
    expnd(f)  & expand a standard form $f$ \\
\hline
\end{tabular}
\end{center}
\ttindex{sfp}\ttindex{expnd}

To distinguish between factored and expanded standard forms, use the
predicated {\tt sfp}: sfp applied to $mvar$ of a standard form returns T
if the main variable slot is occupied by a nested standard form.
To expand a factored standard form you may use the funtion {\tt expnd}; 
however, you need to turn the switch $exp$ on
during that call, e.g.     
\begin{verbatim}
            u:=expnd u where !*exp=t;
\end{verbatim}
Don't build non--expanded standard forms yourself -- otherwise you run
the risk to produce objects with a bad structure (e.g. a wrong term
order) resulting in wrong computational results. You better rely on the
standard routines for manipulating these objects -- as soon as
$*exp$ is off and $*factor$ is on they produce the product forms
anyway.
\section{Communication between algebraic and symbolic modes}

The main aim of programs written in symbolic mode
is to implement the evaluation steps to be invoked from
algebraic mode (top level {\reduce}). This section describes
how to link symbolic routines to algebraic mode 
and how to exchange data between the modes.


\subsection{Algebraic variables}

A variable which has been declared {\tt share}\ttindex{share} by the {\reduce}
$share$ command is accessible from both modes in the
same style and it represents the same algebraic value.
A share variable is the easiest way to transfer data
between the modes. The only restriction is that a
symbolic program should assign only a legal algebraic
form to it - otherwise a later access in algebraic mode
can lead to problems. Example:

\begin{verbatim}
   share hugo;
   hugo :=(x+y)**2$
   hugo;
   symbolic;
   hugo := reval {'sqrt,hugo};
   algebraic;
   hugo;
\end{verbatim}

Variables which have not been declared $share$ have
different values in symbolic and algebraic mode.
Nevertheless a symbolic mode program has access to the
algebraic value of a symbol:
\begin{itemize}
\item {\tt reval(x)}\ttindex{reval}: if the value of $x$ is a symbol 
or if the parameter of $reval$ is a directly quoted symbol
the function returns the algebraic value associated with this
symbol, e.g. $reval('y)$,
\item {\tt setk}\ttindex{setk}{\tt (x,y)} sets the algebraic value of the expression
 $x$ which then must be a symbol (or more general: 
a {\tt kernel}\ttindex{kernel})
to  $y$, e.g. $setk('z,reval'(plus\, u\, 17))$
\footnote{Read ``setk" as ``set kernel value", the {\reduce}
equivalent to the Standard LISP assignment function ``setq".}
\end{itemize}

Of course a clever LISP programmer easily sees how {\reduce}
organizes the assignment and storage of values to algebraic variables
in property lists, and he might be attempted
to use this knowledge for ``easier" access; please resist:
otherwise your program might not run in a future
version of {\reduce}.


\subsection{Calling symbolic routines from algebraic mode}

{\reduce} offers various ways to use symbolic routines
in algebraic mode such that they appear on the
algebraic level as genuine parts of {\reduce}. 
These mainly differ in their strategy of evaluating parameters 
(and results).

\subsubsection{Common protocol}

Some general protocol rules apply for {\reduce} symbolic mode
routines:
\begin{itemize}
\item A routine should check its parameters carefully; in
  the case of illegal parameters the {\reduce} error handlers
  should be invoked (see below).
\item If a routine cannot process its parameters for mathematical 
  reasons it best returns the expression which had caused the call
  unchanged.
\item Evaluators should operate silently; events such as messages needed
  for testing and program development should be carried out in
  dependency of a switch.
\item A routine should return its result as a value
  rather than printing it; for an isolated call in interactive
  mode a printed result would be sufficient, but for following evaluation
  steps the availability of an algebraic value is essential.
\end{itemize}

\subsubsection{Symbolic operator}

The simplest way to link a symbolic procedure to algebraic
mode is the {\tt symbolic operator}\ttindex{symbolic operator} declaration. 
In that case the routine can be called by its proper name in algebraic
mode with {\reduce} first evaluating its parameters to fully
simplified algebraic forms. Upon return the result
should also be an algebraic form (or NIL). Example:

\begin{verbatim}
   symbolic procedure my_plus(a,b);
     begin scalar r;
        print a; print b;
        r := reval{'plus,a,b};
        print r;
        return r;
     end;
   symbolic operator my_plus;
\end{verbatim}

This routine receives two algebraic expressions and computes
their sum by calling the algebraic evaluator.
The calls the the LISP function {\tt print} have been inserted here
and in the following examples to show you the forms passed to the routine.

\subsubsection{Polyfn}

If the symbolic routine is specialized for computations
with pure polynomials it can be declared {\tt polyfn}\ttindex{polyfn}.
{\reduce} will evaluate the arguments for such a routine
to {\tt standard forms}\ttindex{standard forms} and expects that
the result also will have that form. 
Example:

\begin{verbatim}
   symbolic procedure poly_plus(a,b);
     begin scalar r;
        print a; print b;
        r := addf(a,b);
        print r;
        return r;
     end;
   put('poly_plus,'polyfn,'poly_plus);
\end{verbatim}

This routine also adds its arguments
but it is specialized for polynomials. If called with
a non-polynomial form an error message will be generated.
In the {\tt put} statement the first argument is the algebraic
operator name and the third one is the name of the associated
evaluator function. These may differ.

\subsubsection{Psopfn}

The most general type of function is that of a {\tt psopfn}\ttindex{psopfn}.
{\reduce} will not evaluate the parameters for such a routine,
instead it passes the unevaluated list of parameters to the
function. The routine has to evaluate its parameters itself (of course
using the services of $reval$ and friends). So a {\tt psopfn}
can handle variable numbers of parameters. The result must
be an algebraic expression or $nil$. Example:

\begin{verbatim}
   symbolic procedure multi_plus0 u;
     begin scalar r;
        r:=0;
        for each x in u do 
        <<x:=reval x; print x;
          r:=reval{'plus,r,x}
        >>;
        print r;
        return r;
     end;

   put('multi_plus,'psopfn,'multi_plus0);
\end{verbatim}

This routine can be called with an arbitrary number of
parameters; it will evaluate them one after the other,
add them by calling $reval$ and return the sum as result.
Note that the name used in algebraic mode is $multi\_plus$
which is different from the name of the symbolic routine
doing the work. This is necessary because otherwise
the argument count check of {\reduce} would complain
as soon as the routine is called with more than one argument.

In the next example a {\tt psopfn} implements an operator with a
fixed number of arguments; it expects two numbers as
arguments and performs an extensive checking as any such routine
should do; again the name of the routine and the algebraic mode operator
are different:
\begin{verbatim}
   symbolic procedure bin_plus0 u;
    begin scalar p1,p2,r;
     if length u neq 2 then rederr "illegal number of arguments";
     p1:=reval car u; p2:=reval cadr u;
     if not numberp p1 then typerr(p1,"number");
     if not numberp p2 then typerr(p2,"number");
     r:=p1+p2;
     return r;
   end;

   put('bin_plus,'psopfn,'bin_plus0);
\end{verbatim}
The functions {\tt typerr}\ttindex{typerr} and {\tt rederr}\ttindex{rederr} are
explained in the section {\em error management}.

\subsubsection{Simpfn}

When you declare a new algebraic operator and want to assign a
special evaluation mode for it which cannot be written
in algebraic mode (e.g.\  as a rule set) you can create a
simplifier which has the task to convert each operator
expression to a {\tt standard quotient}\ttindex{standard quotient}.
A simplifier function is linked to the operator by
the property {\tt simpfn}\ttindex{simpfn}
and has one argument which is the unevaluated parameter
list of the operator expression. It will be called
every time the operator appears in an expression. It 
must return a standard quotient. Example:

\begin{verbatim}
   algebraic operator op_plus;
 
   symbolic procedure simp_op_plus u;
     begin scalar r;
        r:=simp 0;
        for each x in u do 
        <<x:=simp x; print x;
          r:=addsq(r,x)
        >>;
        return print r;
     end;

   put('op_plus,'simpfn,'simp_op_plus);
\end{verbatim}

In many cases the simpfn is the method of choice as it is
best integrated into the {\reduce} evaluation process: its
results can immediately be used for subsequent calculations
while algebraic form results need to be converted to
standard quotients before combining them with other
formulae.

Note that a simpfn which  cannot simplify its
argument must nevertheless return a {\tt standard quotient},
then with the unevaluable form as kernel. E.g.\  a function
for supporting the algebraic evaluation of the operator ``$<$":

\begin{verbatim}
   algebraic operator <;
 
   symbolic procedure simp_lessp u;
     begin scalar a1,a2,d;
        a1:=simp car u; a2:=simp cadr u;
        d:=subtrsq(a1,a2);
        if domainp denr d and domainp numr d then
          return if !:minusp numr d then simp 1 else simp 0;
        return mksq({'lessp,prepsq a1,prepsq a2},1);
     end;

   put('lessp,'simpfn,'simp_lessp);
\end{verbatim}

Here all comparisons with numeric items are reduced to zero or one
because of the equivalence of zero and ``false" in algebraic mode, while
in all other cases the non-evaluable expression is returned mapped by
the function {\tt mksq}\ttindex{mksq} which converts an algebraic
expression into a standard quotient, ensuring the
identity of composite kernels (see the section {\tt standard forms})
\ttindex{standard form}\ttindex{standard quotient}.

\subsection{Statements, forms}

An additional way to link symbolic entries to algebraic
mode is to invent a new syntactical unit. A {\reduce}
statement is introduced by assigning the
property {\tt stat}\ttindex{stat} = {\tt rlis}\ttindex{rlis} 
to the leading keyword 
of the new statement. The corresponding function will be called with
one parameter which is a list of the (unevaluated)
parameters of the statement. A statement normally
is called for its side effect rather than for its
value - so the result will be not interpreted as
an algebraic quantity ({\tt ws}\ttindex{WS} not set, printed
as symbolic list etc.). Example:

\begin{verbatim}
   put('my_add,'stat,'rlis);
 
   symbolic procedure my_add u;
     begin scalar r;
        r:= 0;
        for each x in u do 
        <<x:=reval x;
          r:=reval{'plus,x,r}
        >>;
        return r;
     end;

\end{verbatim}

to be called by a statement 
\begin{verbatim}
   my_add x,x^2,x^3;
\end{verbatim}

A statement without parameters is introduced by the
property - value pair 
{\tt stat}\ttindex{stat} = {\tt endstat}\ttindex{endstat}
because the statement ``end" is the most prominent
representative of this class. 

For a more complicated syntax you can write a function
which preprocesses a statement before evaluating
it. Such preprocessors are linked to the leading
keyword by the property {\tt formfn}\ttindex{formfn}. However
writing a formfn is rather difficult as you will
have to organize the reading of the rest of the
statement yourself. This approach is not recommended
as the {\tt formfn} interface is rather delicate
and might be subject to future changes.

\subsection{Printing and messages}

For printing in symbolic mode you can use the Standard LISP
functions 

\begin{center}
\begin{tabular}{|r|l|}\hline
   print(u) & print u with LISP syntax and following linefeed\\
   prin1(u) & print u with LISP syntax but without linefeed\\
   prin2(u) & print u without LISP syntax and without linefeed\\
   prin2t(u)& print u without LISP syntax but with linefeed\\
   terpri() & print a single linefeed\\ \hline
\end{tabular} 
\end{center}

{\em LISP syntax} here means that a string is surrounded by
string quotes and special characters in a symbol are
tagged by an exclamation mark.

However if you want to print an algebraic expression
in the same style as the {\reduce} {\tt write}\ttindex{write}
statement you should use the function {\tt writepri}\ttindex{writepri}.
This function needs two arguments, the first one
is the object to be printed which must be a string,
a number or an algebraic expresssion tagged with
a dynamic quote, e.g.\  using the function {\tt mkquote}\ttindex{mkquote}. 
The second argument is one of $only$,
$first$, $nil$, and $last$ and controls linefeeds.
Example:

\begin{verbatim}
  <<u:='(expt x 2);
    writepri(" u= ",'first);
    writepri(mkquote u,nil);
    writepri(" printed in algebraic style",'last);
  >>;
\end{verbatim}

For printing one single algebraic form you can also use the
function {\tt mathprint}\ttindex{mathrpint}:

\begin{verbatim}
    u := '(expt x 3);
    print u;
    mathprint u; 
\end{verbatim}
\section{Error management}

\subsection{Issue an error message}
When programming an application in symbolic mode, it may be
necessary to report an exception to the user, e.g.\  
if an algebraic prerequisite for applying the algorithm is
not fulfilled.
There are several routines for signalling an error in
symbolic mode:

\begin{itemize}
\item The easiest way to complain about a bad algebraic form is
  to call {\tt typerr}\ttindex{typerr}. This function has two parameters
  $typerr(exp,str)$: the first one is an algebraic expression,
  the bad object, and the second one is a string which describes
  what the object should have been. When printed {\reduce}
  will report $exp$ in mathematical notation and insert a string
  ``illegal as". The current program is terminated. 
\begin{verbatim}
    u := '(plus x 1);
      . . .
    if not numberp u then typerr(u,"number");
\end{verbatim}
\item A general error can be reported using the function 
  {\tt rederr}\ttindex{rederr}; it prints some asterisks 
  and then its argument in pure LISP style (no mathematical format);
  the argument can be a list - in that case the surrounding
  brackets are omitted in output. The current program is terminated.
\begin{verbatim}
    rederr "algorithm not applicable if denominator neq 1";
    rederr {"never call me with ",u," again"};
\end{verbatim}
\item Similar to {\em rederr} the routine {\tt rerror}\ttindex{rerror}
 terminates the run with an error message; it has
 been designed for usage in a {\reduce} package enabling
 the error handler to localize the error reason and to
 provide additional information to the end user (e.g.\ 
 via a {\tt help}\ttindex{help} system). It has 3 parameters:
 $rerror(pack,nbr,mess)$ where $pack$ is the name of
 the package (usually a quoted identifier), $nbr$ is
 an integer error number local to the package, beginning
 with 1 and $mess$  is the error message just like {\em rederr}.
\begin{verbatim}
    rerror('asys,13,"remainder not zero");
\end{verbatim}
 Wherever possible {\tt rerror} should be used instead of
 {\tt rederr}.
\end{itemize}

\subsection{Catching errors}

On the other hand a symbolic program might want to react
to an exception which occurred in a calculation branch.
{\reduce} supports this by encapsulation with the routines
{\tt errorset*}\ttindex{errorset*} and {\tt errorset}
\ttindex{errorset}. You have to
construct the action to be encapsulated as a subroutine
call in prefix LISP syntax which is identical to
algebraic form syntax: a list where the first element
is the name of the function (usually a
quoted symbol) followed by the arguments; these
must be tagged by quote, best using the function
{\tt mkquote}\ttindex{mkquote}.
The second parameter describes the error mode: if $t$, an eventual
error during the evaluation will be reported to the user,
if $nil$ the error message will be suppressed. {\tt errorset}
additionally has a third parameter which tells the
error manager whether to print a backtrace or not in an
error case. The result of these functions will be 
\begin{itemize}
\item in the case of an error an error code (its form depends
  on the underlying LISP system),
\item otherwise the result of the computed value as {\tt car} of a list.
\end{itemize} 
You should use the function {\tt errorp}\ttindex{errorp} for testing
whether the call failed or not; in the positive case
pick out the result by a $car$
access. Example:
\begin{verbatim}
  begin scalar u,v;
       .  .  .
   q := errorset!*({'quotsq,mkquote(u),mkquote(v)},nil);
   if errorp q then rederr "division failed" else
       q:=car q;
       .  .  .
\end{verbatim}
Note that you cannot write the expression to be
encapsulated in {\reduce} syntax here. And also $'(quotsq\,\, u\,\, v)$
would be incorrect because then $u$ and $v$ were
constants and not references to the local
variables $u$ and $v$. Only the $mkquote$ expressions
guarantee that at run time a list is generated where
the values of $u$ and $v$ are inserted as parameters for
the call.
\section{Compiling, modules, packages}

\subsection{Program modules}

It is a good practice to design a complex program
group in separate modules each residing in a
separate source file. You can use the {\reduce}
statement pair {\tt module}\ttindex{module} - 
{\tt endmodule}\ttindex{endmodule}
to reflect this structure. A module has a name 
which often will be identical with its source file
(without extension, of course). This correspondence
is not essential, but is very practical for the daily
work. As a side effect {\tt module} switches automatically
to symbolic mode during input and {\tt endmodule} switches
back to the original mode. A typical module in a file
would look like

\begin{verbatim}
module alggeo1; % Primitives for algebraic geometry.

 % Author: Hugo Nobody, Nirwanatown, Utopia.

 .......   the symbolic mode code

endmodule;
end;
\end{verbatim}

If this module resides in a file $alggeo1.red$ it can
either be read into a {\reduce} session by
\begin{verbatim}
  in "alggeo1.red"$
\end{verbatim}
or it can be compiled to a fast loadable binary by
\begin{verbatim}
  faslout "alggeo1";
  in "alggeo1.red"$
  faslend;
\end{verbatim}
You then can load this module by {load alggeo}.

Although {\reduce} currently does not yet rely exclusively on
declarations {\tt exports}\ttindex{exports} and {\tt imports}
\ttindex{imports}, these should be used in order to be
safe for the future. {\tt exports} should name all
entry points which the module offers for external access
and {\tt imports} is the list of modules
or packages which the module needs at run time or
during compilation.

\subsection{Packages}

If your application consists of several modules you can
design it as a {\tt package}\index{package}. A package is a
collection of modules which form a unit but allow you an individual
management, e.g.\  updating and recompilation. A package
is defined by a header module which should have the
name of the package as the module name. This module
then contains a declaration for the package by the
function {\tt create-package}\index{create-package}, which expects as
first parameter a list of the modules which belong to the package
(this list must begin with the header package) and
a second dummy parameter to be set to $nil$ (this
parameter is meaningful only for modules in the 
{\reduce} kernel). Additionally the header
module should contain all global declarations
for the complete package, e.g. {\tt fluid, global}
declarations and {\tt exports} and {\tt imports}
statements. 
Also, all macros, smacros used in more than one module
in the package should be defined in the header module.

As long as you develop your
application and frequently recompile single modules
it would be practical to replace the function
{\tt create-package} by a loop calling {\tt load\_package} (please
note the underscore: {\tt load-package} is a different 
function not usable here). You then can load the complete package
by one call to {\tt load\_package} for the header
module. Example:

\begin{verbatim}
module alggeo; % Algebraic geometry support.

 % Author: Hugo Nobody, Nirwanatown, Utopia.

% create_package('(alggeo alggeo1 alggeo2 alggeo3));
for each x in '(alggeo1 alggeo2 alggeo3) do load x;

fluid '(alggeo_vars!* ......);
exports ...
     ... and so on

endmodule;
end;
\end{verbatim}
Later you can replace the load loop {\tt load} by a correct 
{\tt create-package} expression - you will then have to compile the 
object as one big binary with the header module as the first
entry.

For compiling you use the function {\tt faslout} which creates
a fast loading file - see the corresponding section in the User's
manual.
\section{Coding suggestions}

In order to guarantee that your program is as portable
and version independent as possible 
the following rules may be useful. Note that these are recommendations
only - your program may do a good job for you even if
you ignore all of them.

\begin{itemize}
\item Don't use functions from the underlying LISP which are
  not explicitly part of {\reduce} or {\tt standard lisp}
  - they will probably be missing in other support systems.
\item Don't rely on character case. 
  Some implementations support upper case
  output, others prefer lower case. As the tendency goes
  towards lower case, the best policy is to use lower
  case characters only, except in strings or comments.
\item Use proper indentation. Take the {\reduce}
  sources as example. If the indentation corresponds
  to the block structure your programs will be better
  readable.
\item Don't produce lines with more than 72 characters.
  There are still computers and networks around which
  have problem if the size of the good old punched card
  is exceeded.
\item Write comments. The major entry point routines of your
  symbolic code should have one comment line following the
  procedure declaration; this comment should be one or more
  complete English sentences, beginning with an upper case
  character and ending with a dot.
\item Try to avoid name conflicts. It is a common policy to
  name global and fluid variables with a trailing asterisk
  and to use a common name prefix for the procedure names.
\item If you produce a large piece of software, organize it as
  a {\reduce} package.
\end{itemize}

 
\section{Appendix: Standard LISP extensions}

The following typical LISP functions are defined in the {\tt support}
module of the {\tt rlisp}  package.

\vspace{5mm}    
{\tt atsoc}($u$:any,$p$:alist)

\noindent
Same as {\tt assoc}, but using the operator $eq$ instead of $equal$
for the search. $atsoc$ can be applied only if the identity
of the object to be searched for is guaranteed, e.g.\  for symbols.
It is substantially faster than {\tt assoc}.

\vspace{5mm}    
{\tt copyd}($u$:id,$v$:id)

\noindent
Copies the procedure definition of $v$ to the symbol $u$. Afterwards
the procedure can be called by using the name $u$.  Frequently
used for embedding a function, e.g.\  for implementing a private
trace:
\begin{verbatim}
  if null getd 'myprog_old then copyd('myprog_old,'myprog);
  symbolic procedure myprog(u);
    <<print {"calling myprog with parameter ",u};
      myprog_old(u)>>
\end{verbatim}

\vspace{5mm}    
{\tt eqcar}($u$:any,$v$:any)

\noindent
Implements the frequently needed test {\em pairp u and car u eq v}.

\vspace{5mm}    
{\tt listp}($u$:any)

\noindent
Returns t if $u$ is a top level list ending with a $nil$.

\vspace{5mm}    
{\tt mkquote}($u$:any)

\noindent
tags $u$ with a quote. This function is useful for cases where
a LISP eval situation is given, e.g.\   for coding for an $errorset$
or for $writepri$.

\vspace{5mm}    
{\tt reversip}($u$:list)

\noindent
Reverses the elements of $l$ in place - faster than $reverse$, but
desctroys the input list.

\vspace{5mm}    
{\tt smember}($u$:any,$v$:any)

\noindent
Similar to $member$ this routine tests whether $u$ occurs in
the structure $v$, however in an arbitrarily deep nesting.
While $member$ tests only the top level
list, $smember$ descends down all branches of the tree.

\vspace{5mm}    
{\tt smemq}($u$:any,$v$:any)

\noindent
Same as $smember$, however using $eq$ as test instead of $equal$.

\vspace{5mm}    
\noindent
The following routines handle lists as sets (every element
at most once):

{\tt union}($u$:any,$v$:any)

{\tt intersection}($u$:any,$v$:any)

{\tt setdiff}($u$:any,$v$:any)  % set difference

\vspace{5mm}
\noindent
The Standard LISP {\tt apply} function requires that the parameters
for the function to be applied are encoded as a list. For standard
cases of 1,2 or 3 parameters the following variants allow an easier 
encoding:

{\tt apply1}($u$:id,$u$:any)

{\tt apply2}($u$:id,$u$:any,$v$:any)

{\tt apply3}($u$:id,$u$:any,$v$:any,$w$:any)
\printindex
\end{document}

