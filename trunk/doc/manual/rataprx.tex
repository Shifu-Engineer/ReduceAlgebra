\subsection{Periodic Decimal Representation}
\hyphenation{rat-ional}
\hyphenation{irrat-ional}
\hyphenation{par-am-eter}

\index{Periodic decimal representation}
The division of one integer by another often results in
a period in the decimal part. The \texttt{rational2periodic}
function in this package can recognise and represent
such an answer in a periodic representation. The inverse
function, \texttt{periodic2rational}, converts a
periodic representation back to a rational number.

\hypertarget{operator:RATIONAL2PERIODIC}{}
\hypertarget{operator:PERIODIC2RATIONAL}{}
\hypertarget{operator:PERIODIC}{}
\ttindextype{RATIONAL2PERIODIC}{operator}
\ttindextype{PERIODIC2RATIONAL}{operator}
\ttindextype{PERIODIC}{operator}
\begin{tabbing}
\textbf{\underline{Periodic Representation of a Rational Number}}\\[\baselineskip]

\textbf{SYNTAX:} \hspace{3mm} 
       \= \texttt{rational2periodic(n);}\\
\> {\tt rational2periodic(n, b);}\\[\baselineskip]

\textbf{INPUT:}\\  
\> \texttt{n} \hspace{3mm} is a rational number\\
\> {\tt b} \hspace{3mm} is the number base, if absent the default is 10.\\[\baselineskip]

\textbf{RESULT:}\\
     \> {\tt periodic(\{a1,...,an\},\{b1,...,bm\},\{c1,...,ck\},$\pm{\rm b}$)}\\
 \> where  {\tt\{a1,...,an\}} is a list of the digits in the integer part,\\
 \> {\tt\{b1,...,bm\}} is a list of the digits in the non-periodic part,\\
 \> {\tt\{c1,...,ck\}} is a list of the digits in the periodic part\\
 \> and $\pm${\tt b} where {\tt b} is the number base $2 \leq {\rm b} \leq 16$, \\
 \> a minus indicating the rational number {\tt n} was negative. \\[\baselineskip]

\textbf{EXAMPLES:}\\
    \> $-59/70$ written as $-0.8\overline{428571}$\\
    \>  \texttt{1: rational2periodic(-59/70);}\\
    \> \texttt{ periodic(\{0\}, \{8\}, \{4,2,8,5,7,1\}, -10)}\\[\baselineskip]
    \> $1/80$ written as a hexadecimal is $0.0\overline{3}$\\
    \>  \texttt{2: rational2periodic(1/80,16);}\\
    \> \texttt{ periodic(\{0\}, \{0\}, \{3\}, 16)}\\[\baselineskip]

    Normally the operator {\tt periodic} will not be seen as the output will
    be prettyprinted\\
    as $-0.8\overline{428571}$ and $0.0\overline{3}\ {\rm (base\ 16)}$
    respectively. Currently pretty-printed output \\
    looks better when the switch \texttt{FANCY} is \texttt{OFF}.\\
\end{tabbing}
\begin{tabbing}
  \textbf{\underline{Rational Number of a Periodic Representation}}\\[\baselineskip]
\textbf{SYNTAX:}\\
\hspace{3mm} \= \texttt{periodic2rational(periodic(\{a1,...,an\},\{b1...bm\},\{c1,...,ck\},$\pm{\rm b}$)}\\
   \> \texttt{periodic2rational(\{a1,...,an\},\{b1...bm\},\{c1,...,ck\},$\pm{\rm b}$)}\\[\baselineskip]

\textbf{INPUT:}\\
     \> \texttt{\{a1,...,an\}} is a list of the digits in the integer part,\\
     \> \texttt{\{b1,...,bm\}} is a list of the digits in the non-periodic part,\\
     \>\texttt{\{c1,...,ck\}} is a list of the digits in the periodic part\\
     \> and {\tt b} is the number base $2 \leq {\rm b} \leq 16$, a minus\\
     \> indicating the rational number result should be negative. \\
     \> If the base is omitted, 10 is assumed. \\[\baselineskip]

\textbf{RESULT:}\\
     \> A rational number.\\[\baselineskip]

\textbf{EXAMPLES:}\\[\baselineskip]
\hspace{10mm} \= $0.8\overline{428571}$ written as $59/70$ \\
    \> \texttt{3: periodic2rational(periodic(\{0\},\{8\},\{4,2,8,5,7,1\}));}
\\[\baselineskip]
    \> \hspace{1mm} {\tt 59}\\
    \> {\tt ----}\\
    \> \hspace{1mm} {\tt 70}\\[\baselineskip]
    \> \texttt{4: periodic2rational(\{0\},\{8\},\{4,2,8,5,7,1\}, -10);}
\\[\baselineskip]
    \> \hspace{4mm} \texttt{59}\\
    \> \texttt{- ----}\\
    \> \hspace{4mm} \texttt{70}
\end{tabbing}

Note that \texttt{periodic2rational} will produce the correct rational result
when passed a parameter for the periodic part which is not minimal.
Similarly, a parameter for the periodic part which consists of all 9's
(or in base $b$, all $(b-1)$'s) is treated correctly although such periodic
parts are not canonical and are never generated by calls to
\texttt{rational2periodic}.

\begin{tabbing}
For example,\\
\hspace{10mm}\= \texttt{periodic2rational(\{0\}, \{\}, \{1, 2, 1, 2\});} \\
    \> \texttt{periodic2rational(\{0\}, \{1\}, \{2, 1\});} \\
    \> \texttt{periodic2rational(\{0\}, \{1, 2\}, \{1, 2, 1, 2\});} \\[\baselineskip]
all produce the same rational result, namely $\frac{4}{33}$, as the canonical input\\
    \> \texttt{periodic2rational(\{0\}, \{\}, \{1, 2\});}\\[\baselineskip]
Similarly,\\
    \> \texttt{periodic2rational(\{0\}, \{\}, \{9\});} \\
    \> \texttt{periodic2rational(\{0\}, \{9\}, \{9\});} \\
    \> \texttt{periodic2rational(\{0\}, \{\}, \{9, 9, 9, 9\});} \\
all produce the same rational result, namely $1$, as the  canonical input\\
    \> \texttt{periodic2rational(\{1\}, \{\}, \{\});}
\end{tabbing}

Although the operators \texttt{periodic2rational} and
\texttt{rational2periodic} work even when \texttt{ROUNDED} is \texttt{ON},
they are best used when \texttt{ROUNDED} is \texttt{OFF}. The input
to \texttt{rational2periodic} should not be a rounded number, otherwise
an error results.

For example, the input \texttt{rational2periodic(1/7);} will produce the
intended periodic representation even with \texttt{ROUNDED ON}. However,
 the input

\texttt{a := 1/7; rational2periodic(a);}

will result in an error as the simplifier is applied in the assignment and
rounds the rational number.

Similarly, although the result of \texttt{periodic2rational} will always be
a rational number (represented by a \texttt{QUOTIENT} prefix form), if the
simplifier is applied to the result a rounded value will be produced.

%
%
%
%\newpage
\subsection{Continued Fractions}

\index{Continued fractions}
A continued fraction (see ~\cite{JonesThron:80}) has the general form
{\Large
\[a_0 + \frac{a_1}{b_1 +
         \frac{a_2}{b_2+
          \frac{a_3}{b_3 + \ldots
        }}}
\;.\]
}

A more compact way of writing this is as
\[a_0 + \frac{a_1|}{|b_1} + \frac{a_2|}{|b_2} + \frac{a_3|}{|b_3} + \ldots\,.\]
\\
Even more succinctly:
\[\{a_0,\ \{a_1, b_1\},\ \{a_2, b_2\},\ \ldots\}\]

%
\ttindextype{CONTFRAC}{operator}
This is represented in {\small REDUCE} as
\[{\tt
  contfrac(\textsl{Expression},
    \textsl{Rational approximant},
                \{a0, \{a1,b1\}, \{a2,b2\},.....\})
}\]

\hypertarget{CFRAC:operator}{}
\ttindextype{CFRAC}{operator}
The operator \texttt{CFRAC} is used to generate a generalised continued
fraction expansion of an algebraic expression.\\
\begin{syntaxtable}
  \texttt{cfrac(}\meta{num}\texttt{)} \\
  \texttt{cfrac(}\meta{num}\texttt{,}\meta{length}\texttt{)}\\
  \texttt{cfrac(}\meta{func}\texttt{,}\meta{var}\texttt{)}\\
  \texttt{cfrac(}\meta{func}\texttt{,}\meta{var}\texttt{,}
  \meta{length}\texttt{)}
\end{syntaxtable}


\textbf{INPUT:}\\
\meta{num} \ is any real number\\
\meta{func} \ is a function\\
\meta{var} \ is the function main variable\\
\meta{length} \ is the maximum number of terms (continuents) to be
generated and is \textbf{optional}.

For non-rational function or irrational number input the \meta{length}
argument specifies the number of continuents (ordered pairs, $\{a_i,b_i\}$),
to be returned. Its default value is five.
For rational function or rational number input the
\texttt{length} argument can only truncate the answer, it cannot
return additional pairs even if the precision is increased.
The default for rational function or rational number input is the
complete continued fraction.

\hypertarget{switch:CF_TAYLOR}{}
\ttindextype{CF\_TAYLOR}{switch}

For a non-rational function, power series expansion is necessary. The new
switch \texttt{cf\_taylor} controls whether the TAYLOR or the TPS package is 
used to produce the power series required. By default this switch is OFF and
so the TPS package is normally employed.
In most cases the choice is not important, but the TPS option is somewhat
better at handling cases where the series expansion is rather sparse.
In a few cases TPS may fail to produce a series expansion when TAYLOR succeeds
and vice-versa.

For numerical input the default value is exact for
rational number arguments whilst for irrational or rounded input it is
dependent on the precision of the session. The
\texttt{length} argument will only take effect if is smaller
than the number of ordered pairs which the default value would return.

If the number of continuent pairs returned does not exceed
twelve, the result will usually be pretty-printed as a two element list
consisting of the convergent followed by a rendering of the traditional
continued fraction expansion. For a larger number of pairs the output is
of the second element is printed as a list of pairs. Thus, usually the
operator \texttt{contfrac} is not seen in the output.

%\newpage
\large{\textbf{EXAMPLES}}
\begin{verbatim}
cfrac(pi, 4);

     355             1
{pi,-----,3 + ----------------}
     113               1
               7 + ----------
                          1
                    15 + ---
                          1

cfrac(sqrt 2, 5);

          41                1
{sqrt(2),----,1 + ---------------------}
          29                  1
                   2 + ---------------
                                1
                        2 + ---------
                                  1
                             2 + ---
                                  2

cfrac(23.696, 4);

  2962   237              1
{------,-----,23 + ---------------}
  125    10                 1
                    1 + ---------
                              1
                         2 + ---
                              3

cfrac((x+2/3)^2/(6*x-5), x, 10);

     2
  9*x  + 12*x + 4
{-----------------,   exact,
     54*x - 45

   6*x + 13          1
 ---------- + -------------}
     36         24*x - 20
               -----------
                    9

cfrac(e^x, x);


       3      2
  x   x  + 9*x  + 36*x + 60
{e , -----------------------,
           2
        3*x  - 24*x + 60

                  x
 1 + ---------------------------}
                    x
      1 - ---------------------
                      x
           2 + ---------------
                        x
                3 - ---------
                          x
                     2 + ---
                          5
\end{verbatim}

\hypertarget{CF:operator}{}

The operator \texttt{CF} \ttindextype{CF}{operator} is a 
synonym for the operator \texttt{CONTINUED\_FRACTION}.
\begin{syntaxtable}
  \texttt{cf(}\meta{num}\texttt{)} \hspace{5mm}\\
  \texttt{cf(}\meta{num}\texttt{,}\meta{size}\texttt{)} \\
  \texttt{cf(}\meta{num}\texttt{,}\meta{size}
  \texttt{,}\meta{numterms}\texttt{)}
\end{syntaxtable}

The meaning of the arguments is the same as for the operator
\texttt{CONTINUED\_FRACTION}: the original number to be expanded
\meta{num}, an optional maximum size \meta{size} permitted for the
denominator of the convergent and an optional maximum number of continuents
\meta{numterms} to be generated.

The output is in the same format as that of \texttt{CFRAC} described above.
As with the operator \texttt{CFRAC} output of \texttt{CF} is normally
pretty-printed so the operator \texttt{confract} will not be seen.

\hypertarget{CF_EXPRESSION:operator}{}
\hypertarget{CF_cONTINUENTS:operator}{}
\hypertarget{CF_CONVERGENT:operator}{}
\hypertarget{CF_CONVERGENTS:operator}{}

The accessor operators \texttt{CF\_EXPRESSION},
\ttindextype{CF\_EXPRESSION}{operator}
\texttt{CF\_CONVERGENT} \ttindextype{CF\_CONVERGENT}{operator} and \\ 
\texttt{CF\_CONTINUENTS} \ttindextype{CF\_CONTINUENTS}{operator} allow
the various parts of a continued fraction object \meta{cf\_object}
(as returned by any of the operators \texttt{cf}, \texttt{cfrac},
\texttt{continued\_fraction} and \texttt{cf\_euler}) to be extracted.
\ttindextype{CF\_EULER}{operator}

These three operators return, respectively, the originating
expression of the continued fraction object, the last convergent of the
continued fraction, a list of its continuents
(that is a list of pairs of partial numerators and denominators).

The operator \texttt{CF\_CONVERGENTS} \ttindextype{CF\_CONVERGENTS}{operator}
returns a list of all the convergents of the expansion.
\begin{syntaxtable}
  \texttt{cf\_expression(}\meta{cf\_object}\texttt{)} \\
  \texttt{cf\_convergent(}\meta{cf\_object}\texttt{)} \\
  \texttt{cf\_continuents(}\meta{cf\_object}\texttt{)} \\
  \texttt{cf\_convergents(}\meta{cf\_object}\texttt{)} 
\end{syntaxtable}

\large{\textbf{EXAMPLES}}
\begin{verbatim}
2: cf(6/11);

  6    6          1
{----,----,---------------}
  11   11           1
            1 + ---------
                      1
                 1 + ---
                      5

3: a := cf(pi,1000);

          355             1
a := {pi,-----,3 + ----------------}
          113               1
                    7 + ----------
                               1
                         15 + ---
                               1

4: cf_convergents a;

    22   333   355
{3,----,-----,-----}
    7    106   113

5: cf_continuents a;

{3,7,15,1}

6: precision 20;

12

7: cf pi;

     21053343141
{pi,-------------,{3,7,15,1,292,1,1,1,2,1,3,1,14,2,1,1,2,2,2,2,1}}
     6701487259
\end{verbatim}

\ttindextype{CF\_EULER}{operator}
\hypertarget{CF_EULER:operator}{}
The operator \texttt{CF\_EULER} is used to generate a generalised
continued fraction expansion of an algebraic expression using a formula due to
Leonhard Euler (~\cite{Euler:1748}).
\begin{syntaxtable}
 \texttt{cf\_euler(}\meta{func}\texttt{,}\meta{var}\texttt{)}\\
 \texttt{cf\_euler(}\meta{func}\texttt{,}\meta{var}\texttt{,}
  \meta{length}\texttt{)}
\end{syntaxtable}


\textbf{INPUT:}\\
\meta{func}\ is a function\\
\meta{var} \ is the function main variable\\
\meta{length} \ is the maximum number of continuents to be generated
and is \textbf{optional}.

The meaning of the parameters is similar to those of \texttt{CFRAC}, but the
continued fraction expansion generated will usually be different. Note that
unlike \texttt{CFRAC}, \texttt{CF\_EULER} cannot currently generate
continued fraction expansion of numbers and for a rational function argument
(with a non-constant denominator) the expansion will not be exact.

\ttindextype{CF\_UNIT\_NUMERATORS}{operator}
\hypertarget{CF_UNIT_NUMERATORS:operator}{}
\ttindextype{CF\_UNIT\_DENOMINATORS}{operator}
\hypertarget{CF_UNIT_DENOMINATORS:operator}{}
\ttindextype{CF\_REMOVE\_FRACTIONS}{operator}
\hypertarget{CF_REMOVE_FRACTIONS:operator}{}
\ttindextype{CF\_REMOVE\_CONSTANT}{operator}
\hypertarget{CF_REMOVE_CONSTANT:operator}{}
A number of operators are provided for transforming their continued
fraction argument \meta{cf\_object} into an equivalent expansion,
that is one with exactly the same convergents.
They all accept as their single argument any continued fraction
object \meta{cf\_object}.
These are:\\

\texttt{cf\_unit\_denominators}\\
converts all partial denominators to 1. \\\\
\texttt{cf\_unit\_numerators}\\
converts all partial numerators to 1. \\\\
\texttt{cf\_remove\_fractions}\\
converts the denominators of the partial numerators and partial
denominators in the continuents to 1.\\\\
\texttt{cf\_remove\_constant} \\
removes the zeroth continuent (if non-zero) absorbing it into the
first continuent pair.\\

\ttindextype{CF\_TRANSFORM}{operator}
\hypertarget{CF_TRANSFORM:operator}{}
The operator \texttt{CF\_TRANSFORM} is a general purpose function
for transforming its continued fraction argument \meta{cf\_object} into
an equivalent expansion. Unlike the four preceding operators it requires
a second argument: a list of multipliers used to modify the partial
numerators and denominators of the original expansion.
\begin{syntaxtable}
  \texttt{cf\_transform(}\meta{cf\_object}\texttt{,}
  \meta{multiplier-list}\texttt{)}
\end{syntaxtable}
To understand the operation of \texttt{cf\_transform} consider first the
special case where \meta{multiplier-list} is a list of the
form $\{1, 1, \ldots, 1, l_n, 1, \ldots, 1\}$ whose nth element is $l_n$.
Only the nth continuent pair $\{a_n,\ b_n\}$ and (n+1)th partial numerator
$a_{n+1}$ are altered and become $\{l_na_n,\ l_nb_n\}$ and $l_na_{n+1}$
respectively. For a \meta{multiplier-list} that has more than one non-unit
element, the above transformations are applied sequentially from left to
right.

If the number of continuent pairs in the \meta{cf\_object}
is greater than the length of the \meta{multiplier-list}, the latter is
(in effect) padded with 1's. Conversely if it is shorter, the surplus
elements of \meta{multiplier-list} are ignored.

\ttindextype{CF\_EVEN\_ODD}{operator}
\hypertarget{CF_EVEN_ODD:operator}{}
The operator \texttt{CF\_EVEN\_ODD} splits its continued fraction argument
\meta{cf\_object} into two continued fraction objects: namely its even and
odd parts (in that order) which are returned as a two-element list.
\begin{syntaxtable}
  \texttt{cf\_even\_odd(}\meta{cf\_object}\texttt{)}
\end{syntaxtable}
The convergents of the even part are the even-numbered convergents of the
original expansion and those of the odd part are the odd-numbered ones
(except the zeroth convergent which is necessarily zero).  For the continued
fraction expansions generated by the operators \texttt{cf} and
\texttt{cfrac} with a numerical first argument \meta{num}. The convergents of
the even part form a monotonically increasing sequence whilst those of the
odd part (after the zeroth) form a monotonically decreasing sequence.

\large{\textbf{EXAMPLES}}
\begin{verbatim}
cf_remove_fractions(cf_euler(e^x, x, 4));

        3      2
  x    x  + 3*x  + 6*x + 6
{e , ---------------------,
              6

                   1
 -------------------------------------}
                     x
  1 - -------------------------------
                          x
       (x + 1) - -------------------
                              2*x
                  (x + 2) - -------
                             x + 3

a := cf_remove_fractions(cf_euler(4*atan x, x, 4));

a := {4*atan(x),

              7       5        3
        - 60*x  + 84*x  - 140*x  + 420*x
      -----------------------------------,
                      105

                               4*x
      -----------------------------------------------------}
                                  2
                                 x
       1 + -----------------------------------------------
                                          2
                 2                     9*x
            ( - x  + 3) + -------------------------------
                                                   2
                                  2            25*x
                           ( - 3*x  + 5) + -------------
                                                  2
                                             - 5*x  + 7

b := (a where x => 1);

          304            4
b := {pi,-----,----------------------}
          105              1
                1 + ----------------
                             9
                     2 + ----------
                               25
                          2 + ----
                               2

c := cf(pi, 0, 6);

          104348                    1
c := {pi,--------,3 + ------------------------------}
          33215                       1
                       7 + ------------------------
                                         1
                            15 + -----------------
                                           1
                                  1 + -----------
                                              1
                                       292 + ---
                                              1

cf_remove_constant c;

     104348                22
{pi,--------,-------------------------------}
     33215                    1
              7 + -------------------------
                                22
                   333 + -----------------
                                   1
                          1 + -----------
                                      1
                               292 + ---
                                      1

c:= cf(pi, 0, 8)$
d := cf_even_odd c;

           208341                15
d := {{pi,--------,3 + ----------------------},
           66317                   292
                        106 - --------------
                                       15
                               4687 - -----
                                       585

           312689             22
      {pi,--------,-------------------------}}
           99532                 1
                    7 + -------------------
                                   22
                         355 - -----------
                                       1
                                294 - ---
                                       3
cf_convergents c;

    22   333   355   103993   104348   208341   312689
{3,----,-----,-----,--------,--------,--------,--------}
    7    106   113   33102    33215    66317    99532

cf_convergents first d;

    333   103993   208341
{3,-----,--------,--------}
    106   33102    66317

cf_convergents second d;

    22   355   104348   312689
{0,----,-----,--------,--------}
    7    113   33215    99532
\end{verbatim}

%\newpage
\subsection{Pad\'{e} Approximation}
\index{Pad\'{e} Approximation}

The Pad\'{e} approximant represents a function by the ratio of two 
polynomials. The coefficients of the powers occuring in the polynomials 
are determined by the coefficients in the Taylor series
expansion of the function (see ~\cite{BakerGravesMorris:81}). Given a power series
\[ f(x) = c_0 + c_1 (x-h) + c_2 (x-h)^2 \ldots \]
and the degree of numerator, $n$, and of the denominator, $d$,
the \texttt{pade} function finds the unique coefficients 
$a_i,\, b_i$ in the Pad\'{e} approximant 
\[ \frac{a_0+a_1 x+ \cdots + a_n x^n}{b_0+b_1 x+ \cdots + b_d x^d} \; .\]

\hypertarget{PADE:operator}{}
\ttindextype{PADE}{operator}

\textbf{SYNTAX:} \hspace{5mm} \texttt{pade(f, x, h, n, d);}\\[\baselineskip]

\textbf{INPUT:}\\\\
\texttt{f} \hspace{6mm} the funtion to be approximated\\
\texttt{x} \hspace{6mm} the function variable\\
\texttt{h} \hspace{6mm} the point at which the approximation is evaluated\\
\texttt{n} \hspace{6mm} the (specified) degree of the numerator\\
\texttt{d} \hspace{6mm} the (specified) degree of the denominator\\[\baselineskip]
\textbf{RESULT:} \\
Pad\a'{e} Approximant, ie. a rational function.\\[\baselineskip]

\textbf{ERROR MESSAGES:}\\

 \hspace{6mm} \texttt{***** not yet implemented}\\

The Taylor series expansion for the function, f, has not yet
been implemented in the {\small REDUCE} Taylor Package.\\[\baselineskip]

 \hspace{6mm} \texttt{***** no Pade Approximation exists}\\

A Pad\'{e} Approximant of this function does not exist.\\[\baselineskip]

%\newpage
 \hspace{6mm} \texttt{***** Pade Approximation of this order does not exist}\\

A Pad\'{e} Approximant of this order (ie. the specified
numerator and denominator orders) does not exist but one
of a different order may exist.\\[\baselineskip]

\large{\textbf{EXAMPLES}}

\begin{verbatim}

23: pade(sin(x),x,0,3,3);

          2
 x*( - 7*x  + 60)
------------------
       2
   3*(x  + 20)

24: pade(tanh(x),x,0,5,5);

     4        2
 x*(x  + 105*x  + 945)
-----------------------
      4       2
 15*(x  + 28*x  + 63)

25: pade(atan(x),x,0,5,5);

        4        2
 x*(64*x  + 735*x  + 945)
--------------------------
         4       2
 15*(15*x  + 70*x  + 63)

26: pade(exp(1/x),x,0,5,5);

***** no Pade Approximation exists

27: pade(factorial(x),x,1,3,3);

***** not yet implemented

28: pade(asech(x),x,0,3,3);

            2                        2                 2
- 3*log(x)*x  + 8*log(x) + 3*log(2)*x  - 8*log(2) + 2*x
--------------------------------------------------------
                          2
                       3*x  - 8

29: taylor(ws-asech(x),x,0,10);

               11  
log(x)*(0 + O(x  ))

     13    6     43    8    1611    10      11
 + (-----*x  + ------*x  + -------*x   + O(x  ))
     768        2048        81920

30: pade(sin(x)/x^2,x,0,10,0);

***** Pade Approximation of this order does not exist

31:  pade(sin(x)/x^2,x,0,10,2);

     10        8         6           4            2
( - x   + 110*x  - 7920*x  + 332640*x  - 6652800*x

  + 39916800)/(39916800*x)

32: pade(exp(x),x,0,10,10);



  10        9         8           7            6
(x   + 110*x  + 5940*x  + 205920*x  + 5045040*x

              5               4                3
  + 90810720*x  + 1210809600*x  + 11762150400*x

                 2
  + 79394515200*x  + 335221286400*x + 670442572800)/

      10        9         8           7            6
    (x   - 110*x  + 5940*x  - 205920*x  + 5045040*x

                     5               4
         - 90810720*x  + 1210809600*x 

                        3               2
         - 11762150400*x + 79394515200*x  

         - 335221286400*x + 670442572800)

33: pade(sin(sqrt(x)),x,0,3,3);
        
(sqrt(x)*
            3            2
  (56447*x  - 4851504*x  + 132113520*x - 885487680))\

              3         2
    (7*(179*x  - 7200*x  - 2209680*x - 126498240))
\end{verbatim}

