
%\begin{abstract}
Program \char`\"{}xCOLOR\char`\"{} is intended for calculation the colour
factor in non-abelian gauge field theories. It is 
realized Cvitanovich algorithm {[}1{]}. In comparision with
\char`\"{}COLOR\char`\"{} program {[}2{]} it was made many improvements.
The package was writen by symbolic mode. This version is
faster then {[}2{]} more then 10 times.
%\end{abstract} 


After load the program by the following command \quad {\tt load xcolor}; \\
user can be able to use the next additional commands and operators. 

\subsection*{Command SUdim.}

Format: {\tt SUdim <any expression>}; \\
%
Set the order of SU group. \\
%
The default value is 3, i.e. SU(3). 

\subsection*{Command SpTT.}

Format: {\tt SpTT <any expression>}; \\
%
Set the normalization coefficient A: Sp(TiTj) = A{*}Delta(i,j).
Default value is 1/2. 

\subsection*{Operator QG.}

Format: {\tt QG(inQuark,outQuark,Gluon)} \\
%
Describe the quark-gluon vertex. Parameters may be any identifiers.
First and second of then must be in- and out- quarks correspondently.
Third one is a gluon. 

\subsection*{Operator G3.}

Format: {\tt G3(Gluon1,Gluon2,Gluon3)} \\
%
Describe the three-gluon vertex. Parameters may be any identifiers.
The order of gluons must be clock. \\
%
In terms of QG and G3 operators you input diagram in \char`\"{}color\char`\"{}
space as a product of these operators. For example. 

\begin{verbatim}


 Diagram:                       REDUCE expression:

            e1

         ---->---
        /        \
       |    e2    |
     v1*..........*v2  <===>  QG(e3,e1,e2)*QG(e1,e3,e2)
       |          |
        \   e3   /
         ----<---


Here: --->--- quark

      ....... gluon

\end{verbatim}

More detail see {[}2{]}. \\
%\ \\
%\ \\
\underline{References.} \\ 
%\ \\
{[}1{]} P.Cvitanovic, Phys. Rev. D14(1976), p.1536.

{[}2{]} A.Kryukov \& A.Rodionov, Comp. Phys. Comm., 48(1988), pp.327-334.\\
%\ \\
%\ \\
%\ \\
Please send any remarks to my address above! \\
%\ \\
%\ \\
Good luck!
