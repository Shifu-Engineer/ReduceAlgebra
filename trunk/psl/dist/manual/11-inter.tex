%\begin{verbatim}
\chapter{The Interpreter}

\section{Evaluator Functions Eval and Apply}

  The   evaluator  is  responsible  for  the  execution  of  PSL
programs. The evaluator is available to  the  user  through  the
function  eval.   This function gives the operational semantics,
or meaning, to the programming constructs found in  PSL.    With
eval  we  have  the ability to evaluate constructed expressions.
The ability to evaluate  lists  which  have  the  appearance  of
expressions highlights the program data duality of LISP.

  Apply   represents   the  piece  of  the  evaluator  which  is
responsible for invoking functions.    Any  list  which  is  not
quoted  is  taken to be a function application by the evaluator.
Apply  provides  the  user  with  a  tool  to  explicity  invoke
functions.

  In  eval, apply and the support functions which follow, errors
are continuable.  When an error occurs the user is permitted  to
correct  the  offending  expression  or perhaps define a missing
function (see Chapter 16 for more information).  If  the  number
of  arguments  to  a  function  is  not  equal  to the number of
parameters specified by the definition then the error

\begin{verbatim}
***** Argument number mismatch
\end{verbatim}
occurs.  If there is an id in the function position of  a  form,
and there is no function definition associated with it, an error
occurs and the message

\begin{verbatim}
***** `NAME' is an undefined function
\end{verbatim}
is  printed.  If the function position is not occupied by an id,
lambda expression or code-pointer then it is an error and one of
the following is printed.

\begin{verbatim}
***** Ill-formed expression in EVAL `FORM'
\end{verbatim}
\begin{verbatim}
***** Ill-formed expression `FORM'
\end{verbatim}
The first message is displayed when the error is detected within
eval.

\de{eval}{(eval U:form): any}{expr}
{    The value of the form U  is  computed.    Since  eval  is  a
    function  of  type expr it will receive its argument already
    evaluated. Since it will then evaluate the argument  itself,
    it  may  seem  as  though the evaluation is done twice.  The
    argument U cannot access local variables.
}
    
As a basis for discussion, an approximation to the actual
definition of eval is given.

\begin{verbatim}
    (de eval (e)
      (cond ((is-constant e) (denote e))
            ((idp e) (valuecell e))
            (t (let ((fun (first e))
                     (args (rest e)))
                 (cond ((and (idp fun)
                             (not (funboundp fun)))
                        (selectq (first (getd fun))
                          (expr (apply fun (evlis args)))
                          (fexpr (apply fun (list args)))
                          (nexpr   
                            (apply fun 
                                   (list (evlis  args))))
                          (macro 
                            (eval (apply fun (list e))))))
                       ((or (codep fun)
                            (is-lambda fun))
                        (apply fun (evlis args)))
                       (t (error)))))))
\end{verbatim}
    Eval  first  determines  if  its  argument  is  a  constant.
    Examples  of  constants  are  numbers,  strings, vectors and
    quoted expressions.  For the most part, the function  denote
    will  simply  return its argument.  However, if the constant
    is a quoted expression then the expression  is  returned  as
    value.

\begin{verbatim}
    1 lisp> (eval "STRING")
    "STRING"
    2 lisp> (eval (mkquote '(one)))
    (ONE)
\end{verbatim}
    An  identifier  is  considered  a variable.  In such a case,
    eval  returns  the  contents  of  the  value  cell  of   the
    identifier.  Of course things are not quite all that simple.
    This  raises  a  conceptual  issue  about  when  to find the
    values.  The issue is one of scoping rules.   Scoping  rules
    come  into  play when functions are defined.  In particular,
    this involves free variables (those variables which are  not
    parameters  of  the  definition).    One  approach is static
    scoping.  This strategy locates the values of free variables
    at the time the function is defined.    PSL  uses  a  second
    approach  called  dynamic scoping.  Using this approach, the
    values of  free  variables  are  not  determined  until  the
    function is applied.

\begin{verbatim}
    1 lisp> (setq number 0)
    0
    2 lisp> (de bar () (foo) (add1 number))
    BAR
    3 lisp> (de foo () (setq number 1))
    FOO
    % When bar was defined the value of number was 0,
    %	during the evaluation
    % of bar the application of foo changed the value 
    %	of number to 1. If
    % static scoping were being used then the result
    %	would be 1 instead of 2.
    4 lisp> (bar)
    2
\end{verbatim}
    The  PSL  environment (the values associated with variables)
    will typically change during evaluation.  When a function is
    applied, the  variables  specified  by  its  definition  are
    associated  with  the  arguments  of  the application.  As a
    result, the value of each variable has changed.    As  noted
    above,  the  value of a variable is found in its value cell.
    However, the definition and access of  values  depends  upon
    the  binding strategy being used by the implementation.  Two
    common techniques are shallow binding (used in  the  current
    implementation  of  PSL), and deep binding.  In a deep bound
    system the environment is defined by a table  of  names  and
    values.  Each time a variable is set to a value, an entry is
    added  to the table.  To locate the value of a variable this
    table must be searched.  When there are multiple definitions
    for a given variable within this table,  the  convention  is
    adopted  that  it  is  always  the  latest  value  which  is
    retrieved.  With this strategy we can easily restore an  old
    value,  by simply removing the current value from the table.
    With shallow binding, the value of a variable is  positioned
    with  the  id  which  represents the variable.  In this case
    there is no need to search for a value.    It  is  known  to
    always  be  in  a  fixed location, the value cell of the id.
    This  second  technique  makes  finding  values  much   more
    efficient,  but  it is much more difficult now to keep track
    of previous values.\\

    If the expression being evaluated is neither a constant or a
    variable it is assumed to be a  function  application  form.
    When  we  apply  the function definition to the arguments we
    associate  the  parameters  of  the  definition   with   the
    arguments.    This  process of association is called binding
    and simulates substitution.  Within the new environment  the
    body  of  the  function  is  evaluated.    Notice  we do not
    explicitly substitute the values for  the  variables  in  an
    expression.

    % Assume that THIS does not have a value, during the evaluation of the
    % body of the lambda expression the value of THIS will be defined to be 0.
\begin{verbatim}
    1 lisp> (eval '((lambda (this) (add1 this)) 0))
    1
    % If an expression is neither a constant or
    %	a variable then it is assumed
    % to be a function application form.
    2 lisp> (eval '(one))
    ***** `ONE' is an undefined function
\end{verbatim}
    Eval  first  looks  at  the function position, making sure a
    function definition is available.  The manner in  which  the
    function  is  applied depends upon the type of the function.
    For both exprs and nexprs the arguments are evaluated  in  a
    left  to  right  order  by the function evlis.  Since nexprs
    expect a single argument, the arguments are gathered into  a
    list.   The arguments to an fexpr are not evaluated but they
    are gathered into a list.  The remaining  function  type  to
    consider  is macro.  The entire function application form is
    passed to macros.  The result of applying the macro is  then
    evaluated.


\de{apply}{(apply FUN:{id, function} ARGS:any-list): any}{expr}
{    Applies the function FUN to the list of arguments ARGS.  The
    following is an approximation of the real code.
}
\begin{verbatim}
    (de apply (fun args)
      (cond ((and (idp fun) (not (funboundp fun)))
             (if (fcodep fun)
               (codeapply (getfcodepointer fun) args)
               (lambdaapply (get fun '*lambdalink) args)))
            ((codep fun) (codeapply fun args))
            ((is-lambda fun) (lambdaapply fun args))
            (t (error))))
\end{verbatim}
    Apply   uses   the   additional   functions   codeapply  and
    lambdaapply to actually evaluate the body of  the  function.
    The  arguments  in ARGS are not evaluated (for example, by a
    call on evlis).  This may seem odd since  the  type  of  the
    function  may  be  expr  or  nexpr.   Apply assumes that the
    arguments are in the form required for binding to the formal
    parameters of FUN.  Getcodepointer returns the  code-pointer
    associated  with an id.  The body of a function which is not
    compiled is found on the property list of  its  name,  under
    the *lambdalink indicator.

\begin{verbatim}
    1 lisp> (setq fn 'add1)
    ADD1
    2 lisp> (de fn (a) (sub1 a))
    FN
    3 lisp> (apply fn '(1))
    2
    4 lisp> (apply 'fn '(1))
    0
    5 lisp> (apply 'cons '((add1 2) 3))
    ((add1 2) . 3)     % NOT (3 . 3)
\end{verbatim}

\de{funcall}{(funcall FUN:{id, function} [ARGUMENT:any]): any}{macro}
{    Equivalent  to  (apply  FUN (list ARGUMENT1 ... ARGUMENTN)).
    This function is defined in the USEFUL module.
}
\section{Support Functions for Eval and Apply}

\de{evlis}{(evlis U:any-list): any-list}{expr}
{    Evlis evaluates each element of U  The list  of  results  is
    returned.
}
\begin{verbatim}
    (de evlis (p)
      (if (not (pairp p))
        ()
        (cons (eval (first u))
              (evlis (rest u)))))
\end{verbatim}

\de{idapply}{(idapply FN:id U:any-list): any}{expr}
{    Applies  the  function referenced by FN to the argument list
    U.  An equivalent form would be (apply FN U).   The  use  of
    idapply  is  more  efficient.    If FN is not an id it is an
    error and the message
}
\begin{verbatim}
    ***** Ill-formed function expression
\end{verbatim}
    is printed.

\de{codeapply}{(codeapply FN:code-pointer U:any-list): any}{expr}
{    The body of compiled function referenced by FN  is  executed
    with arguments in U.  }

\de{codeevalapply}{(codeevalapply FN:code-pointer U:any-list): any}{expr}
{    (codeevalapply  FN  U)  is  essentially (codeapply FN (evlis
    U)).  The difference between the two is that  the  first  is
    more efficient than the second.  }

\de{evprogn}{(evprogn U:form-list): any}{expr}
{    The  forms in U are evaluated in a left to right order.  The
    value of the last is returned.  This  function  is  used  in
    situations  where  an  application of progn is implied.  For
    example, the definition of  many  functions  consists  of  a
    sequence  of  expressions.  Each expression is evaluated and
    the value of the last is returned, without  having  to  wrap
    the definition inside a progn.}
\begin{verbatim}
    (de evprogn (u)
      (if (pairp u)
        (progn (while (pairp (cdr u))
                 (eval (car u))
                 (setq u (cdr u)))
               (eval (car u)))
        nil))
\end{verbatim}
\section{Special Evaluator Functions, Quote and Function}

\de{quote}{(quote U:any): any}{fexpr}
{    Quote  is  used  to distinguish a constant s-expression from
    one which is to be evaluated.  The return value is U.  Since
    quote is an fexpr U is never evaluated.  }

\de{mkquote}{(mkquote U:any): list}{expr}
{    Returns the result of (list 'quote U).  }

\de{function}{(function FN:function): function}{fexpr}
{    Function is similar to quote, except that its argument is a
    reference to a function. FN is either the name of a
    function, a lambda expression, or a code-pointer. It is not
    correct to use  quote  to  suppress  the  evaluation  of  an
    expression  which  represents  a  function. The misuse of
    function can effect the way in which code is compiled, in
    particular, application of the mapping functions  (see
    Chapter ??? for more information).
}
    Use of the read macro \#' will result in  an  application  of
    function.  For example, if the reader encounters  \#'fun  the
    expression (FUNCTION FUN) will be returned. This read macro
    is part of the {\tt useful} package.

\section{Support Functions for Macro Evaluation}

\de{expand}{(expand L:list FN:function): list}{expr}
{    Expand is a convenient way  to  expand  a  certain  type  of
    macro.   PSL  supports  functions like plus which accept any
    number of arguments.  The macro plus is actually defined  in
    terms of the binary operator plus2.  For example,
}
\begin{verbatim}
    (plus 1 2 3)
\end{verbatim}
    expands into

\begin{verbatim}
    (plus2 1 (plus2 2 3))
\end{verbatim}
    FN  should  be a function which accepts two arguments, and L
    should be a list of values which are  acceptable  to  FN  as
    arguments.

\begin{verbatim}
    (de expand (l fn)
      (cond ((not (pairp l)) l)
            ((not (pairp (cdr l))) (car l))
            (t (list fn (car l) (expand (cdr l) fn)))))

    1 lisp> (dm plus (a) (expand (rest a) 'plus2))
    plus
    2 lisp> (macroexpand '(plus 1 2 3))
    (plus2 1 (plus2 2 3))
\end{verbatim}

\de{robustexpand}{(robustexpand L: list  FN: function
 EMPTYCASE: form): list}{expr}      
{    If the list L is empty then EMPTYCASE is evaluated,
    otherwise the result of (expand L FN) is returned.
}
