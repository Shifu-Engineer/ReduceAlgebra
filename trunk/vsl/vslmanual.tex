\documentclass[a4paper,12pt]{article}
\title{Visible Standard Lisp User Manual}
\author{Arthur Norman}
\usepackage{epsfig}
\usepackage{times,latexsym}
\usepackage{wrapfig}
\makeindex
\newcommand{\vsl}{{\ttfamily vsl}}
\newcommand{\nil}{{\ttfamily nil}}
\newcommand{\tx}{\ttfamily}
\newcommand{\sverb}{\small\begin{verbatim}}
\begin{document}
\maketitle


There is a separate larger document that explaims what \vsl is
and how it is implemented. This merely contains an explanation of how
to fetch the sources and compile them, and a list of the functions
that \vsl then provides.


\section{Fetching and building vsl}
You can fetch \vsl sources using {\tx subversion}. If you do not have that
installed on your computer already you need to discover how to fetch it.
On Linux this will be easy using a package manager (typically {\tx yum}
on Fedora or {\tx apt-get} on Debian or Ubuntu). On Windows if you visit
{\tx www.cygwin.com} you can fetch their setup program and install their
environment -- ensuring that you install {\tx make}, {\tx gcc} and
{\tx subversion}, and probably {\tx tex}.

Then you can go
{\sverb
U=https://reduce-algebra.svn.sourceforge.net
V=$U/svnroot/reduce-algebra/trunk
svn co $V/vsl
\end{verbatim}}
which should create a directory called {\tx vsl} and put a collection of
files in it. The above instructions build up the path at {\tx sourceforge}
to fetch this from in parts so you only have to type a modest
amount on each line. The files you will fetch amount to about a megabyte,
and so should not be a severe strain on anything.

If you also want to use \vsl{} to build a version of the Reduce algebra
system you should follow up the subversion calls with another call (relying
on the variable {\tx \$V} you set up to point to the subversion repository).
{\sverb
svn co $V/packages
\end{verbatim}}
This time you will end up with around 60 Mbytes in a directory called
{\tx packages}. That is the full set of sources for Reduce. Even though you
are not liable to be able to make use of all of them the easiest recipe
involves you fetching everything.

To build \vsl{} and then try it you can then go
{\sverb
   cd vsl
   make
   ./vsl
   (oblist)
   (stop 0)
\end{verbatim}}

For the build to succeed you will need to have the {\tx gcc} C compiler
installed, and to rebuild the manual the {\tx pdflatex}\footnote{typically
instaled for you as part of some broader \LaTeX package such as texlive.}
command is required. Again you may need to use a package manager to ensure
that they are available.

In the above small example you verify that \vsl{} will run by calling the
function {\tx (oblist)} to display a list of all \vsl's built-in
symbols, then you use {\tx (stop 0)} to exit from the system.

To try Reduce (see {\tx http://reduce-algebra/sourceforge.net} for
full information. In particular there is a manual hidden there) you go
{\sverb
  make reduce
  ./vsl
\end{verbatim}}

The ``{\tx make reduce}'' step builds (much of) Reduce and saves
the result in {\tx vsl.img}. When you next start \vsl it reloads that
image file. Then you can try various algebraic examples. One of my
favourites is
{\sverb
  int(1/(x^6-1), x);
  in "../packages/alg/alg.tst";
\end{verbatim}}

The \vsl-hosted Reduce will be significantly slower than other versions
and it certainly has severe limitations because \vsl{} does not provide
arbitrary prevision arithmetic. It is expected that there will be cases
where some Reduce facilities try to use functions that \vsl{} does not
provide -- and then crashes. It would be helpful if such cases can be
collected and reported at least so that a section can be inserted here
documenting Reduce limitations under \vsl.


\section{Summary of functions in \vsl}

\input vslfunctions.tex

\input vslmanual.ind

\end{document}
