\chapter{Commands for Interactive Use}\index{Interactive use}

{\REDUCE} is designed as an interactive system, but naturally it can also
operate in a batch processing or background mode by taking its input
command by command from the relevant input stream. There is a basic
difference, however, between interactive and batch use of the system. In
the former case, whenever the system discovers an ambiguity at some point
in a calculation, such as a forgotten type assignment for instance, it asks
the user for the correct interpretation. In batch operation, it is not
practical to terminate the calculation at such points and require
resubmission of the job, so the system makes the most obvious guess of the
user's intentions and continues the calculation.

There is also a difference in the handling of errors.  In the former case,
the computation can continue since the user has the opportunity to correct
the mistake.  In batch mode, the error may lead to consequent erroneous
(and possibly time consuming) computations.  So in the default case, no
further evaluation occurs, although the remainder of the input is checked
for syntax errors.  A message {\tt "Continuing with parsing only"}
informs the user that this is happening.  On the other hand, the switch
{\tt ERRCONT},\ttindex{ERRCONT} if on, will cause the system to continue
evaluating expressions after such errors occur.

When a syntactical error occurs, the place where the system detected the
error is marked with three dollar signs (\$\$\$). In interactive mode, the
user can then use {\tt ED}\ttindex{ED} to correct the error, or retype the
command.  When a non-syntactical error occurs in interactive mode, the
command being evaluated at the time the last error occurred is saved, and
may later be reevaluated by the command {\tt RETRY}.\ttindex{RETRY}

\section{Referencing Previous Results}

It is often useful to be able to reference results of previous
computations during a {\REDUCE} session.  For this purpose, {\REDUCE}
maintains a history\index{History} of all interactive inputs and the
results of all interactive computations during a given session.  These
results are referenced by the command number that {\REDUCE} prints
automatically in interactive mode.  To use an input expression in a new
computation, one writes {\tt input(}$n${\tt )},\ttindex{INPUT} where
$n$ is the command number.  To use an output expression, one writes {\tt
WS(}$n${\tt )}.\ttindex{WS} {\tt WS} references the previous command.
E.g., if command number 1 was {\tt INT(X-1,X)}; and the result of command
number 7 was {\tt X-1}, then
\begin{verbatim}
        2*input(1)-ws(7)^2;
\end{verbatim}
would give the result {\tt -1}, whereas
\begin{verbatim}
        2*ws(1)-ws(7)^2;
\end{verbatim}
would yield the same result, but {\em without\/} a recomputation of the
integral.

The operator {\tt DISPLAY}\ttindex{DISPLAY} is available to display previous
inputs.  If its argument is a positive integer, {\it n} say, then the
previous n inputs are displayed.  If its argument is {\tt ALL} (or in fact
any non-numerical expression), then all previous inputs are displayed.

\section{Interactive Editing}
It is possible when working interactively to edit any {\REDUCE} input that
comes from the user's terminal, and also some user-defined procedure
definitions.  At the top level, one can access any previous command string
by the command {\tt ed(}$n${\tt )},\ttindex{ED} where n is the desired
command number as prompted by the system in interactive mode. {\tt ED};
(i.e. no argument) accesses the previous command.

After {\tt ED} has been called, you can now edit the displayed string using a
string editor with the following commands:

\begin{tabular}{lp{\rboxwidth}}
{\tt~~~~~  B} & move pointer to beginning \\
{\tt~~~~~  C<character>} & replace next character by
{\em character} \\
{\tt~~~~~  D} & delete next character \\
{\tt~~~~~  E} & end editing and reread text \\
{\tt~~~~~  F<character>} & move pointer to next
occurrence of {\em character} \\[1.7pt]
{\tt~~~~~  I<string><escape>} &
 insert {\em string\/} in front of pointer \\
{\tt~~~~~  K<character>} & delete all characters
 until {\em character} \\
{\tt~~~~~  P} & print string from current pointer \\
{\tt~~~~~  Q} & give up with error exit \\
{\tt~~~~~  S<string><escape>} &
 search for first occurrence of {\em string},
                             positioning pointer just before it \\
{\tt~~~~~  space} or {\tt X} & move pointer right
one character.
\end{tabular}

The above table can be displayed online by typing a question mark followed
by a carriage return to the editor. The editor prompts with an angle
bracket. Commands can be combined on a single line, and all command
sequences must be followed by a carriage return to become effective.

Thus, to change the command {\tt x := a+1;} to {\tt x := a+2}; and cause
it to be executed, the following edit command sequence could be used:
\begin{verbatim}
        f1c2e<return>.
\end{verbatim}
The interactive editor may also be used to edit a user-defined procedure that
has not been compiled.  To do this, one says:
\ttindex{EDITDEF}
\begin{verbatim}
        editdef <id>;
\end{verbatim}
where {\tt <id>} is the name of the procedure.  The procedure definition
will then be displayed in editing mode, and may then be edited and
redefined on exiting from the editor.

Some versions of {\REDUCE} now include input editing that uses the
capabilities of modern window systems.  Please consult your system
dependent documentation to see if this is possible.  Such editing
techniques are usually much easier to use then {\tt ED} or {\tt EDITDEF}.

\section{Interactive File Control}
If input is coming from an external file, the system treats it as a batch
processed calculation.  If the user desires interactive
\index{Interactive use} response in this case, he can include the command
{\tt on int};\ttindex{INT} in the file.  Likewise, he can issue the
command {\tt off int}; in the main program if he does not desire continual
questioning from the system.  Regardless of the setting of {\tt INT},
input commands from a file are not kept in the system, and so cannot be
edited using {\tt ED}.  However, many implementations of {\REDUCE} provide
a link to an external system editor that can be used for such editing.
The specific instructions for the particular implementation should be
consulted for information on this.

Two commands are available in {\REDUCE} for interactive use of files. {\tt
PAUSE};\ttindex{PAUSE} may be inserted at any point in an input file.  When
this command is encountered on input, the system prints the message {\tt
CONT?} on the user's terminal and halts.  If the user responds {\tt Y}
(for yes), the calculation continues from that point in the file.  If the
user responds {\tt N} (for no), control is returned to the terminal, and
the user can input further statements and commands.  Later on he can use
the command {\tt cont;}\ttindex{CONT} to transfer control back to the
point in the file following the last {\tt PAUSE} encountered.  A top-level
{\tt pause;}\ttindex{PAUSE} from the user's terminal has no effect.

