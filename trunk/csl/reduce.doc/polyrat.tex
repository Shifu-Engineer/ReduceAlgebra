\chapter{Polynomials and Rationals}

Many operations in computer algebra are concerned with polynomials
\index{Polynomial} and rational functions\index{Rational function}.  In
this section, we review some of the switches and operators available for
this purpose.  These are in addition to those that work on general
expressions (such as {\tt DF} and {\tt INT}) described elsewhere.  In the
case of operators, the arguments are first simplified before the
operations are applied.  In addition, they operate only on arguments of
prescribed types, and produce a type mismatch error if given arguments
which cannot be interpreted in the required mode with the current switch
settings.  For example, if an argument is required to be a kernel and
{\tt a/2} is used (with no other rules for {\tt A}), an error
\begin{verbatim}
        A/2 invalid as kernel
\end{verbatim}
will result.

With the exception of those that select various parts of a polynomial or
rational function, these operations have potentially significant effects on
the space and time associated with a given calculation. The user should
therefore experiment with their use in a given calculation in order to
determine the optimum set for a given problem.

One such operation provided by the system is an operator {\tt LENGTH}
\ttindex{LENGTH} which returns the number of top level terms in the
numerator of its argument.  For example,
\begin{verbatim}
        length ((a+b+c)^3/(c+d));
\end{verbatim}
has the value 10.  To get the number of terms in the denominator, one
would first select the denominator by the operator {\tt DEN}\ttindex{DEN}
and then call {\tt LENGTH}, as in
\begin{verbatim}
        length den ((a+b+c)^3/(c+d));
\end{verbatim}
Other operations currently supported, the relevant switches and operators,
and the required argument and value modes of the latter, follow.

\section{Controlling the Expansion of Expressions}

The switch {\tt EXP}\ttindex{EXP} controls the expansion of expressions.  If
it is off, no expansion of powers or products of expressions occurs.
Users should note however that in this case results come out in a normal
but not necessarily canonical form.  This means that zero expressions
simplify to zero, but that two equivalent expressions need not necessarily
simplify to the same form.

{\it Example:} With {\tt EXP} on, the two expressions
\begin{verbatim}
        (a+b)*(a+2*b)
\end{verbatim}
and
\begin{verbatim}
        a^2+3*a*b+2*b^2
\end{verbatim}
will both simplify to the latter form.  With {\tt EXP}
off, they would remain unchanged, unless the complete factoring {\tt
(ALLFAC)} option were in force. {\tt EXP} is normally on.

Several operators that expect a polynomial as an argument behave
differently when {\tt EXP} is off, since there is often only one term at
the top level.  For example, with {\tt EXP} off
\begin{verbatim}
        length((a+b+c)^3/(c+d));
\end{verbatim}
returns the value 1.

\section{Factorization of Polynomials}\index{Factorization}
{\REDUCE} is capable of factorizing univariate and multivariate polynomials
that have integer coefficients, finding all factors that also have integer
coefficients. The package for doing this was written by Dr. Arthur C.
Norman and Ms. P. Mary Ann Moore at The University of Cambridge. It is
described in P. M. A. Moore and A. C. Norman, ``Implementing a Polynomial
Factorization and GCD Package'', Proc. SYMSAC '81, ACM (New York) (1981),
109-116.

The easiest way to use this facility is to turn on the switch
{\tt FACTOR},\ttindex{FACTOR} which causes all expressions to be output in
a factored form.  For example, with {\tt FACTOR} on, the expression
{\tt A\verb|^|2-B\verb|^|2} is returned as {\tt (A+B)*(A-B)}.

It is also possible to factorize a given expression explicitly.  The
operator {\tt FACTORIZE}\ttindex{FACTORIZE} that invokes this facility is
used with the syntax
\begin{verbatim}
     FACTORIZE(EXPRN:polynomial[,INTEXP:prime integer]):list,
\end{verbatim}
the optional argument of which will be described later. Thus to find and
display all factors of the cyclotomic polynomial $x^{105}-1$, one could
write:
\begin{verbatim}
        factorize(x^105-1);
\end{verbatim}
The result is a list of factor,exponent pairs.
In the above example, there is no overall numerical factor in the result,
so the results will consist only of polynomials in x.  The number of such
polynomials can be found by using the operator {\tt LENGTH}.\ttindex{LENGTH}
If there is a numerical factor, as in factorizing $12x^{2}-12$,
that factor will appear as the first member of the result.
It will however not be factored further.  Prime factors of such numbers
can be found, using a probabilistic algorithm, by turning on the switch
{\tt IFACTOR}.\ttindex{IFACTOR}  For example,
\begin{verbatim}
        on ifactor; factorize(12x^2-12);
\end{verbatim}
would result in the output
\begin{verbatim}
	{{2,2},{3,1},{X + 1,1},{X - 1,1}}.
\end{verbatim}

If the first argument of {\tt FACTORIZE} is an integer, it will be
decomposed into its prime components, whether or not {\tt IFACTOR} is on.

Note that the {\tt IFACTOR} switch only affects the result of {\tt FACTORIZE}.
It has no effect if the {\tt FACTOR}\ttindex{FACTOR} switch is also on.

The order in which the factors occur in the result (with the exception of
a possible overall numerical coefficient which comes first) can be system
dependent and should not be relied on. Similarly it should be noted that
any pair of individual factors can be negated without altering their
product, and that {\REDUCE} may sometimes do that.

The factorizer works by first reducing multivariate problems to univariate
ones and then solving the univariate ones modulo small primes. It normally
selects both evaluation points and primes using a random number generator
that should lead to different detailed behavior each time any particular
problem is tackled. If, for some reason, it is known that a certain
(probably univariate) factorization can be performed effectively with a
known prime, {\tt P} say, this value of {\tt P} can be handed to
{\tt FACTORIZE}\ttindex{FACTORIZE} as a second
argument. An error will occur if a non-prime is provided to {\tt FACTORIZE} in
this manner. It is also an error to specify a prime that divides the
discriminant of the polynomial being factored, but users should note that
this condition is not checked by the program, so this capability should be
used with care.

Factorization can be performed over a number of polynomial coefficient
domains in addition to integers. The particular description of the relevant
domain should be consulted to see if factorization is supported. For
example, the following statements will factorize $x^{4}+1$ modulo 7:
\begin{verbatim}
        setmod 7;
        on modular;
        factorize(x^4+1);
\end{verbatim}
The factorization module is provided with a trace facility that may be useful
as a way of monitoring progress on large problems, and of satisfying
curiosity about the internal workings of the package. The most simple use
of this is enabled by issuing the {\REDUCE} command\ttindex{TRFAC}
{\tt on trfac;} .
Following this, all calls to the factorizer will generate informative
messages reporting on such things as the reduction of multivariate to
univariate cases, the choice of a prime and the reconstruction of full
factors from their images.  Further levels of detail in the trace are
intended mainly for system tuners and for the investigation of suspected
bugs.  For example, {\tt TRALLFAC} gives tracing information at all levels
of detail.  The switch that can be set by {\tt on timings;} makes it
possible for one who is familiar with the algorithms used to determine
what part of the factorization code is consuming the most resources.
{\tt on overview}; reduces the amount of detail presented in other forms of
trace.  Other forms of trace output are enabled by directives of the form
\begin{verbatim}
        symbolic set!-trace!-factor(<number>,<filename>);
\end{verbatim}
where useful numbers are 1, 2, 3 and 100, 101, ... .  This facility is
intended to make it possible to discover in fairly great detail what just
some small part of the code has been doing --- the numbers refer mainly to
depths of recursion when the factorizer calls itself, and to the split
between its work forming and factorizing images and reconstructing full
factors from these.  If {\tt NIL} is used in place of a filename the trace
output requested is directed to the standard output stream.  After use of
this trace facility the generated trace files should be closed by calling
\begin{verbatim}
        symbolic close!-trace!-files();
\end{verbatim}
{\it NOTE:} Using the factorizer with {\tt MCD}\ttindex{MCD} off will
result in an error.

\section{Cancellation of Common Factors}
Facilities are available in {\REDUCE} for cancelling common factors in the
numerators and denominators of expressions, at the option of the user. The
system will perform this greatest common divisor computation if the switch
{\tt GCD}\ttindex{GCD} is on. ({\tt GCD} is normally off.)

A check is automatically made, however, for common variable and numerical
products in the numerators and denominators of expressions, and the
appropriate cancellations made.

When {\tt GCD} is on, and {\tt EXP} is off, a check is made for square
free factors in an expression.  This includes separating out and
independently checking the content of a given polynomial where
appropriate. (For an explanation of these terms, see Anthony C. Hearn,
``Non-Modular Computation of Polynomial GCDs Using Trial Division'', Proc.
EUROSAM 79, published as Lecture Notes on Comp.  Science, Springer-Verlag,
Berlin, No 72 (1979) 227-239.)

{\it Example:} With {\tt EXP}\ttindex{EXP} off and {\tt GCD}\ttindex{GCD}
on,
the polynomial {\tt a*c+a*d+b*c+b*d} would be returned as {\tt (A+B)*(C+D)}.

Under normal circumstances, GCDs are computed using an algorithm described
in the above paper. It is also possible in {\REDUCE} to compute GCDs using
an alternative algorithm, called the EZGCD Algorithm, which uses modular
arithmetic.  The switch {\tt EZGCD}\ttindex{EZGCD}, if on in addition to
{\tt GCD}, makes this happen.

In non-trivial cases, the EZGCD algorithm is almost always better
than the basic algorithm, often by orders of magnitude.  We therefore
{\em strongly\/} advise users to use the {\tt EZGCD} switch where they have the
resources available for supporting the package.

For a description of the EZGCD algorithm, see J. Moses and D.Y.Y. Yun,
``The EZ GCD Algorithm'', Proc. ACM 1973, ACM, New York (1973) 159-166.

{\it NOTE:}
This package shares code with the factorizer, so a certain amount of trace
information can be produced using the factorizer trace switches.

\subsection{Determining the GCD of Two Polynomials}
This operator, used with the syntax
\begin{verbatim}
        GCD(EXPRN1:polynomial,EXPRN2:polynomial):polynomial,
\end{verbatim}
returns the greatest common divisor of the two polynomials {\tt EXPRN1} and
{\tt EXPRN2}.

{\it Examples:}
\begin{verbatim}
        gcd(x^2+2*x+1,x^2+3*x+2) ->  X+1
        gcd(2*x^2-2*y^2,4*x+4*y) ->  2*X+2*Y
        gcd(x^2+y^2,x-y)         ->  1.
\end{verbatim}

\section{Working with Least Common Multiples}

Greatest common divisor calculations can often become expensive if
extensive work with large rational expressions is required. However, in
many cases, the only significant cancellations arise from the fact that
there are often common factors in the various denominators which are
combined when two rationals are added. Since these denominators tend to be
smaller and more regular in structure than the numerators, considerable
savings in both time and space can occur if a full GCD check is made when
the denominators are combined and only a partial check when numerators are
constructed. In other words, the true least common multiple of the
denominators is computed at each step. The switch {\tt LCM}\ttindex{LCM}
is available for this purpose, and is normally on.

In addition, the operator {\tt LCM},\ttindex{LCM} used with the syntax
\begin{verbatim}
        LCM(EXPRN1:polynomial,EXPRN2:polynomial):polynomial,
\end{verbatim}
returns the least common multiple of the two polynomials {\tt EXPRN1} and
{\tt EXPRN2}.

{\it Examples:}
\begin{verbatim}
        lcm(x^2+2*x+1,x^2+3*x+2) ->  X**3 + 4*X**2 + 5*X + 2
        lcm(2*x^2-2*y^2,4*x+4*y) ->  4*(X**2 - Y**2)
\end{verbatim}

\section{Controlling Use of Common Denominators}

When two rational functions are added, {\REDUCE} normally produces an
expression over a common denominator. However, if the user does not want
denominators combined, he or she can turn off the switch {\tt MCD}
\ttindex{MCD} which controls this process.  The latter switch is
particularly useful if no greatest common divisor calculations are
desired, or excessive differentiation of rational functions is required.

{\it CAUTION:}  With {\tt MCD} off, results are not guaranteed to come out in
either normal or canonical form.  In other words, an expression equivalent
to zero may in fact not be simplified to zero.  This option is therefore
most useful for avoiding expression swell during intermediate parts of a
calculation.

{\tt MCD}\ttindex{MCD} is normally on.

\section{REMAINDER Operator}\ttindex{REMAINDER}

This operator is used with the syntax
\begin{verbatim}
     REMAINDER(EXPRN1:polynomial,EXPRN2:polynomial):polynomial.
\end{verbatim}
It returns the remainder when {\tt EXPRN1} is divided by {\tt EXPRN2}.  This
is the true remainder based on the internal ordering of the variables, and
not the pseudo-remainder. The pseudo-remainder \ttindex{PSEUDO\_REMAINDER}
and in general pseudo-division \ttindex{PSEUDO\_DIVIDE} of polynomials
can be calculated after loading  the {\tt polydiv} package.
Please refer to the documentation of this package for details.

{\it Examples:}
\begin{verbatim}
        remainder((x+y)*(x+2*y),x+3*y) ->  2*Y**2
        remainder(2*x+y,2)             ->  Y.
\end{verbatim}

{\it CAUTION:} In the default case, remainders are calculated over the
integers.  If you need the remainder with respect to another domain, it
must be declared explicitly.

{\it Example:}
\begin{verbatim}
        remainder(x^2-2,x+sqrt(2)); -> X^2 - 2
        load_package arnum;
        defpoly sqrt2**2-2;
        remainder(x^2-2,x+sqrt2); -> 0
\end{verbatim}

\section{RESULTANT Operator}\ttindex{RESULTANT}

This is used with the syntax
\begin{verbatim}
     RESULTANT(EXPRN1:polynomial,EXPRN2:polynomial,VAR:kernel):
        polynomial.
\end{verbatim}
It computes the resultant of the two given polynomials with respect to the
given variable, the coefficients of the polynomials can be taken from any
domain. The result can be identified as the determinant of a
Sylvester matrix, but can often also be thought of informally as the
result obtained when the given variable is eliminated between the two input
polynomials. If the two input polynomials have a non-trivial GCD their
resultant vanishes.

The switch {\tt Bezout}\ttindex{Bezout} controls the computation of the
resultants. It is off by default. In this case a subresultant algorithm
is used. If the switch Bezout is turned on, the resultant is computed via
the Bezout Matrix. However, in the latter case, only polynomial coefficients
are permitted.

\begin{samepage}
The sign conventions used by the resultant function follow those in R.
Loos, ``Computing in Algebraic Extensions'' in ``Computer Algebra --- Symbolic
and Algebraic Computation'', Second Ed., Edited by B. Buchberger, G.E.
Collins and R. Loos, Springer-Verlag, 1983. Namely, with {\tt A} and {\tt B}
not dependent on {\tt X}:

\begin{verbatim}
                               deg(p)*deg(q)
   resultant(p(x),q(x),x)= (-1)             *resultant(q,p,x)

                            deg(p)
   resultant(a,p(x),x)   = a

   resultant(a,b,x)      = 1
\end{verbatim}
\end{samepage}

{\it Examples:}

\begin{samepage}
\begin{verbatim}
                                     2
   resultant(x/r*u+y,u*y,u)   ->  - y
\end{verbatim}
\end{samepage}

{\it calculation in an algebraic extension:}

\begin{samepage}
\begin{verbatim}
   load arnum;
   defpoly sqrt2**2 - 2;

   resultant(x + sqrt2,sqrt2 * x +1,x) -> -1
\end{verbatim}
\end{samepage}

{\it or in a modular domain:}

\begin{samepage}
\begin{verbatim}
   setmod 17;
   on modular;

   resultant(2x+1,3x+4,x)    -> 5
\end{verbatim}
\end{samepage}
\section{DECOMPOSE Operator}\ttindex{DECOMPOSE}

The {\tt DECOMPOSE} operator takes a multivariate polynomial as argument,
and returns an expression and a list of equations from which the
original polynomial can be found by composition.  Its syntax is:
\begin{verbatim}
     DECOMPOSE(EXPRN:polynomial):list.
\end{verbatim}
For example:
\begin{verbatim}
     decompose(x^8-88*x^7+2924*x^6-43912*x^5+263431*x^4-
                    218900*x^3+65690*x^2-7700*x+234)
                   2                  2            2
              -> {U  + 35*U + 234, U=V  + 10*V, V=X  - 22*X}
                                     2
     decompose(u^2+v^2+2u*v+1)  -> {W  + 1, W=U + V}
\end{verbatim}
Users should note however that, unlike factorization, this decomposition
is not unique.

\section{INTERPOL operator}\ttindex{INTERPOL}

Syntax:
\begin{verbatim}
        INTERPOL(<values>,<variable>,<points>);
\end{verbatim}

where {\tt <values>} and {\tt <points>} are lists of equal length and
{\tt <variable>} is an algebraic expression (preferably a kernel).

{\tt INTERPOL} generates an interpolation polynomial {\em f\/} in the given
variable of degree length({\tt <values>})-1.  The unique polynomial {\em f\/}
is defined by the property that for corresponding elements {\em v\/} of
{\tt <values>} and {\em p\/} of {\tt <points>} the relation $f(p)=v$ holds.

The Aitken-Neville interpolation algorithm is used which guarantees a
stable result even with rounded numbers and an ill-conditioned problem.

\section{Obtaining Parts of Polynomials and Rationals}

These operators select various parts of a polynomial or rational function
structure. Except for the cost of rearrangement of the structure, these
operations take very little time to perform.

For those operators in this section that take a kernel {\tt VAR} as their
second argument, an error results if the first expression is not a
polynomial in {\tt VAR}, although the coefficients themselves can be
rational as long as they do not depend on {\tt VAR}.  However, if the
switch {\tt RATARG}\ttindex{RATARG} is on, denominators are not checked
for dependence on {\tt VAR}, and are taken to be part of the coefficients.

\subsection{DEG Operator}\ttindex{DEG}

This operator is used with the syntax
\begin{verbatim}
        DEG(EXPRN:polynomial,VAR:kernel):integer.
\end{verbatim}
It returns the leading degree\index{Degree} of the polynomial {\tt EXPRN}
in the variable {\tt VAR}.  If {\tt VAR} does not occur as a variable in
{\tt EXPRN}, 0 is returned.

{\it Examples:}
\begin{verbatim}
        deg((a+b)*(c+2*d)^2,a) ->  1
        deg((a+b)*(c+2*d)^2,d) ->  2
        deg((a+b)*(c+2*d)^2,e) ->  0.
\end{verbatim}
Note also that if {\tt RATARG} is on,
\begin{verbatim}
        deg((a+b)^3/a,a)       ->  3
\end{verbatim}
since in this case, the denominator {\tt A} is considered part of the
coefficients of the numerator in {\tt A}.  With {\tt RATARG} off, however,
an error would result in this case.

\subsection{DEN Operator}\ttindex{DEN}

This is used with the syntax:
\begin{verbatim}
        DEN(EXPRN:rational):polynomial.
\end{verbatim}
It returns the denominator of the rational expression {\tt EXPRN}.  If
{\tt EXPRN} is a polynomial, 1 is returned.

{\it Examples:}
\begin{verbatim}
        den(x/y^2)   ->  Y**2
        den(100/6)   ->  3
                [since 100/6 is first simplified to 50/3]
        den(a/4+b/6) ->  12
        den(a+b)     ->  1
\end{verbatim}

\subsection{LCOF Operator}\ttindex{LCOF}

LCOF is used with the syntax
\begin{verbatim}
        LCOF(EXPRN:polynomial,VAR:kernel):polynomial.
\end{verbatim}
It returns the leading coefficient\index{Leading coefficient} of the
polynomial {\tt EXPRN} in the variable {\tt VAR}.  If {\tt VAR} does not
occur as a variable in {\tt EXPRN}, {\tt EXPRN} is returned.
\extendedmanual{\newpage}
{\it Examples:}
\begin{verbatim}
        lcof((a+b)*(c+2*d)^2,a) ->  C**2+4*C*D+4*D**2
        lcof((a+b)*(c+2*d)^2,d) ->  4*(A+B)
        lcof((a+b)*(c+2*d),e)   ->  A*C+2*A*D+B*C+2*B*D
\end{verbatim}

\subsection{LPOWER Operator}\ttindex{LPOWER}

\begin{samepage}
Syntax:
\begin{verbatim}
        LPOWER(EXPRN:polynomial,VAR:kernel):polynomial.
\end{verbatim}
LPOWER returns the leading power of {\tt EXPRN} with respect to {\tt VAR}.
If {\tt EXPRN} does not depend on {\tt VAR}, 1 is returned.
\end{samepage}

{\it Examples:}
\begin{verbatim}
        lpower((a+b)*(c+2*d)^2,a) ->  A
        lpower((a+b)*(c+2*d)^2,d) ->  D**2
        lpower((a+b)*(c+2*d),e)   ->  1
\end{verbatim}

\subsection{LTERM Operator}\ttindex{LTERM}

\begin{samepage}
Syntax:
\begin{verbatim}
        LTERM(EXPRN:polynomial,VAR:kernel):polynomial.
\end{verbatim}
LTERM returns the leading term of {\tt EXPRN} with respect to {\tt VAR}.
If {\tt EXPRN} does not depend on {\tt VAR}, {\tt EXPRN} is returned.
\end{samepage}

{\it Examples:}
\begin{verbatim}
        lterm((a+b)*(c+2*d)^2,a) ->  A*(C**2+4*C*D+4*D**2)
        lterm((a+b)*(c+2*d)^2,d) ->  4*D**2*(A+B)
        lterm((a+b)*(c+2*d),e)   ->  A*C+2*A*D+B*C+2*B*D
\end{verbatim}

{\COMPATNOTE} In some earlier versions of REDUCE, {\tt LTERM} returned
{\tt 0} if the {\tt EXPRN} did not depend on {\tt VAR}.  In the present
version, {\tt EXPRN} is always equal to {\tt LTERM(EXPRN,VAR)} $+$ {\tt
REDUCT(EXPRN,VAR)}.

\subsection{MAINVAR Operator}\ttindex{MAINVAR}

Syntax:
\begin{verbatim}
        MAINVAR(EXPRN:polynomial):expression.
\end{verbatim}
Returns the main variable (based on the internal polynomial representation)
of {\tt EXPRN}. If {\tt EXPRN} is a domain element, 0 is returned.

{\it Examples:}

Assuming {\tt A} has higher kernel order than {\tt B}, {\tt C}, or {\tt D}:
\begin{verbatim}
        mainvar((a+b)*(c+2*d)^2) ->  A
        mainvar(2)               ->  0
\end{verbatim}

\subsection{NUM Operator}\ttindex{NUM}

Syntax:
\begin{verbatim}
        NUM(EXPRN:rational):polynomial.
\end{verbatim}
Returns the numerator of the rational expression {\tt EXPRN}.  If {\tt EXPRN}
is a polynomial, that polynomial is returned.

{\it Examples:}
\begin{verbatim}
        num(x/y^2)  ->  X
        num(100/6)   ->  50
        num(a/4+b/6) ->  3*A+2*B
        num(a+b)     ->  A+B
\end{verbatim}

\subsection{REDUCT Operator}\ttindex{REDUCT}

Syntax:
\begin{verbatim}
        REDUCT(EXPRN:polynomial,VAR:kernel):polynomial.
\end{verbatim}
Returns the reductum of {\tt EXPRN} with respect to {\tt VAR} (i.e., the
part of {\tt EXPRN} left after the leading term is removed).  If {\tt
EXPRN} does not depend on the variable {\tt VAR}, 0 is returned.

{\it Examples:}
\begin{verbatim}
     reduct((a+b)*(c+2*d),a) ->  B*(C + 2*D)
     reduct((a+b)*(c+2*d),d) ->  C*(A + B)
     reduct((a+b)*(c+2*d),e) ->  0
\end{verbatim}

{\COMPATNOTE} In some earlier versions of REDUCE, {\tt REDUCT} returned
{\tt EXPRN} if it did not depend on {\tt VAR}.  In the present version, {\tt
EXPRN} is always equal to {\tt LTERM(EXPRN,VAR)} $+$ {\tt
REDUCT(EXPRN,VAR)}.

\section{Polynomial Coefficient Arithmetic}\index{Coefficient}
{\REDUCE} allows for a variety of numerical domains for the numerical
coefficients of polynomials used in calculations.  The default mode is
integer arithmetic, although the possibility of using real coefficients
\index{Real coefficient} has been discussed elsewhere.  Rational
coefficients have also been available by using integer coefficients in
both the numerator and denominator of an expression, using the {\tt ON
DIV}\ttindex{DIV} option to print the coefficients as rationals.
However, {\REDUCE} includes several other coefficient options in its basic
version which we shall describe in this section.  All such coefficient
modes are supported in a table-driven manner so that it is
straightforward to extend the range of possibilities.  A description of
how to do this is given in R.J. Bradford, A.C. Hearn, J.A. Padget and
E. Schr\"ufer, ``Enlarging the {\REDUCE} Domain of Computation,'' Proc. of
SYMSAC '86, ACM, New York (1986), 100--106.

\subsection{Rational Coefficients in Polynomials}\index{Coefficient}
\index{Rational coefficient}
Instead of treating rational numbers as the numerator and denominator of a
rational expression, it is also possible to use them as polynomial
coefficients directly. This is accomplished by turning on the switch
{\tt RATIONAL}.\ttindex{RATIONAL}

{\it Example:} With {\tt RATIONAL} off, the input expression {\tt a/2}
would be converted into a rational expression, whose numerator was {\tt A}
and denominator 2.  With {\tt RATIONAL} on, the same input would become a
rational expression with numerator {\tt 1/2*A} and denominator {\tt 1}.
Thus the latter can be used in operations that require polynomial input
whereas the former could not.

\subsection{Real Coefficients in Polynomials}\index{Coefficient}
\index{Real coefficient}
The switch {\tt ROUNDED}\ttindex{ROUNDED} permits the use of arbitrary
sized real coefficients in polynomial expressions.  The actual precision
of these coefficients can be set by the operator {\tt PRECISION}.
\ttindex{PRECISION} For example, {\tt precision 50;} sets the precision to
fifty decimal digits.  The default precision is system dependent and can
be found by {\tt precision 0;}.  In this mode, denominators are
automatically made monic, and an appropriate adjustment is made to the
numerator.

{\it Example:} With {\tt ROUNDED} on, the input expression {\tt a/2} would
be converted into a rational expression whose numerator is {\tt 0.5*A} and
denominator {\tt 1}.

Internally, {\REDUCE} uses floating point numbers up to the precision
supported by the underlying machine hardware, and so-called {\em
bigfloats} for higher precision or whenever necessary to represent numbers
whose value cannot be represented in floating point.  The internal
precision is two decimal digits greater than the external precision to
guard against roundoff inaccuracies.  Bigfloats represent the fraction and
exponent parts of a floating-point number by means of (arbitrary
precision) integers, which is a more precise representation in many cases
than the machine floating point arithmetic, but not as efficient.  If a
case arises where use of the machine arithmetic leads to problems, a user
can force {\REDUCE} to use the bigfloat representation at all precisions by
turning on the switch {\tt ROUNDBF}.\ttindex{ROUNDBF}  In rare cases,
this switch is turned on by the system, and the user informed by the
message
\begin{verbatim}
        ROUNDBF turned on to increase accuracy
\end{verbatim}

Rounded numbers are normally printed to the specified precision.  However,
if the user wishes to print such numbers with less precision, the printing
precision can be set by the command {\tt PRINT\_PRECISION}.
\ttindex{PRINT\_PRECISION} For example, {\tt print\_precision 5;} will
cause such numbers to be printed with five digits maximum.

Under normal circumstances when {\tt ROUNDED} is on, {\REDUCE} converts the
number 1.0 to the integer 1.  If this is not desired, the switch
{\tt NOCONVERT}\ttindex{NOCONVERT} can be turned on.

Numbers that are stored internally as bigfloats are normally printed with
a space between every five digits to improve readability.  If this
feature is not required, it can be suppressed by turning off the switch
{\tt BFSPACE}.\ttindex{BFSPACE}

Further information on the bigfloat arithmetic may be found in T. Sasaki,
``Manual for Arbitrary Precision Real Arithmetic System in {\REDUCE}'',
Department of Computer Science, University of Utah, Technical Note No.
TR-8 (1979).

When a real number is input, it is normally truncated to the precision in
effect at the time the number is read.  If it is desired to keep the full
precision of all numbers input, the switch {\tt ADJPREC}\ttindex{ADJPREC}
(for {\em adjust precision\/}) can be turned on.  While on, {\tt ADJPREC}
will automatically increase the precision, when necessary, to match that
of any integer or real input, and a message printed to inform the user of
the precision increase.

When {\tt ROUNDED} is on, rational numbers are normally converted to
rounded representation.  However, if a user wishes to keep such numbers in
a rational form until used in an operation that returns a real number,
the switch {\tt ROUNDALL}\ttindex{ROUNDALL} can be turned off.  This
switch is normally on.

Results from rounded calculations are returned in rounded form with two
exceptions: if the result is recognized as {\tt 0} or {\tt 1} to the
current precision, the integer result is returned.

\subsection{Modular Number Coefficients in Polynomials}\index{Coefficient}
\index{Modular coefficient}
{\REDUCE} includes facilities for manipulating polynomials whose
coefficients are computed modulo a given base.  To use this option, two
commands must be used; {\tt SETMOD} {\tt <integer>},\ttindex{SETMOD} to set
the prime modulus, and {\tt ON MODULAR}\ttindex{MODULAR} to cause the
actual modular calculations to occur.
For example, with {\tt setmod 3;} and {\tt on modular;}, the polynomial
{\tt (a+2*b)\verb|^|3} would become {\tt A\verb|^|3+2*B\verb|^|3}.

The argument of {\tt SETMOD} is evaluated algebraically, except that
non-modular (integer) arithmetic is used.  Thus the sequence
\begin{verbatim}
        setmod 3; on modular; setmod 7;
\end{verbatim}
will correctly set the modulus to 7.

Modular numbers are by default represented by integers in the interval
[0,p-1] where p is the current modulus.  Sometimes it is more convenient
to use an equivalent symmetric representation in the interval
[-p/2+1,p/2], or more precisely
[-floor((p-1)/2), ceiling((p-1)/2)],
especially if the modular numbers map objects that include
negative quantities.  The switch {\tt BALANCED\_MOD}\ttindex{BALANCED\_MOD}
allows you to select the symmetric representation for output.

Users should note that the modular calculations are on the polynomial
coefficients only.  It is not currently possible to reduce the exponents
since no check for a prime modulus is made (which would allow
$x^{p-1}$ to be reduced to 1 mod p).  Note also that any division by a
number not co-prime with the modulus will result in the error ``Invalid
modular division''.

\subsection{Complex Number Coefficients in Polynomials}\index{Coefficient}
\index{Complex coefficient}
Although {\REDUCE} routinely treats the square of the variable {\em i\/} as
equivalent to $-1$, this is not sufficient to reduce expressions involving
{\em i\/} to lowest terms, or to factor such expressions over the complex
numbers.  For example, in the default case,
\begin{verbatim}
        factorize(a^2+1);
\end{verbatim}
gives the result
\begin{verbatim}
	{{A**2+1,1}}
\end{verbatim}
and
\begin{verbatim}
        (a^2+b^2)/(a+i*b)
\end{verbatim}
is not reduced further.  However, if the switch
{\tt COMPLEX}\ttindex{COMPLEX} is turned on, full complex arithmetic is then
carried out.  In other words, the above factorization will give the result
\begin{verbatim}
	{{A + I,1},{A - I,1}}
\end{verbatim}
and the quotient will be reduced to {\tt A-I*B}.

The switch {\tt COMPLEX} may be combined with {\tt ROUNDED} to give complex
real numbers; the appropriate arithmetic is performed in this case.

Complex conjugation is used to remove complex numbers from denominators of
expressions.  To do this if {\tt COMPLEX} is off, you must turn the switch
{\tt RATIONALIZE}\ttindex{RATIONALIZE} on.

