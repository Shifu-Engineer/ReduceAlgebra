\chapter{The Objects Module}

\section{Introduction}

The {\bf objects} module provides simple support for object-oriented
programming in PSL. It is based on the "flavors" facility of
the LISP machine, which is the source of its terminology. The
LISP  Machine  Manual contains a much longer introduction to the
idea of object oriented programming, generic operations, and the
flavors facility in particular.  This discussion briefly  covers
the  basics  of  using  objects  to  give you an idea of what is
involved; then it explains the details.

\subsection{Terminology}

  An object datatype is known as a flavor.  It can be thought of
in two parts:


\begin{enumerate}
\item  A template that defines the characteristics of the flavor.
\item  A set of operations that can be performed on the flavor.
\end{enumerate}
The template that defines the characteristics  of  a  flavor  is
known  as  the  flavor  definition.   It is created by the macro
defflavor.  Each operation that can be performed on an object is
known as a method definition.  They are  created  by  the  macro
defmethod.   Invoking  an  operation  on  an  object is known as
sending an object a message.  A created object of a given flavor
is known as an instance of that flavor.

  Example of a flavor definition is:

\begin{verbatim}
(defflavor complex-number
           (real-part
            (imaginary-part 0.0)
            )
           ()
           gettable-instance-variables
           initable-instance-variables
           )
\end{verbatim}

A flavor definition specifies the fields, or in our terminology,
the  instance variables, that each object of that flavor  is  to
have.    In  the  example  above,  the  instance  variables  are
REAL-PART and  IMAGINARY-PART.   The  mention  of  the  instance
variable  IMAGINARY-PART indicated that by default the imaginary
part of a complex number will be initialized to 0.0. There is no
default initialization for REAL-PART.

  Instance variables may be strictly part of the  implementation
of a flavor, totally invisible to users.  Typically though, some
of  the  instance  variables are directly visible in some way to
the user of the object. The flavor definition may specify
"initable-instance-variables", "gettable-instance-variables",
and "settable-instance-variables".  These options mean that  the
instance variables specified are able to be initialized
(initable), able to be accessed by name (gettable), and 
able to be assigned and accessed (settable). None, some of, 
or all of the instance variables may be specified in each  
option. In the example above, both REAL-PART and IMAGINARY-
PART may be initialized and may be accessed by name.

  With the objects package the  programmer  completely  controls
what  operations  are  to  be done on objects of each flavor, so
this is a  true  object-oriented  programming  facility.   Also,
operations on flavored objects are generic.  This means that the
operations  can  be  done on an object of any flavor, as long as
the operations are defined for that flavor of object.  The  same
operation  can  be  defined  for  many flavors; and whenever the
operation is invoked, what is actually done will depend  on  the
flavor of the object it is being done to.

  To see the power of generic operations, consider the following
example:

  Say  we    wish  to  write  a scanner that reads a sequence of
characters out of some object and processes them.  It  does  not
need  to  assume  that the characters are coming from a file, or
even from an I/O channel.

  Suppose the scanner gets a character by invoking the
GET-CHARACTER operation. In this case any object of a flavor with
a GET-CHARACTER operation can be passed to the scanner, and the
GET-CHARACTER operation defined for that object's flavor will
performed  to fetch the character. This means that the scanner can
get  characters from a string, or from a text editor's buffer, or
from  any object at all that provides a GET-CHARACTER operation.
The  scanner is automatically general.

\section{Creating Objects}

  The function make-instance provides a convenient way to create
objects of any flavor.  The flavor of the object to  be  created
and  the initializations to be done are given as parameters in a
way that is fully independent of the internal representation  of
the object.

\subsection{Methods}

  The  function  "=$>$", whose name is intended to suggest the
sending of a message to an object, is usually used to invoke  
a method.

Examples:

\begin{verbatim}
(=> my-object zap)
(=> thing1 set-location 2.0 3.4)
\end{verbatim}

The  first  argument  to  =$>$  is  the  object being operated on:
my-object and thing1 in the examples.  The  second  argument  is
the name of the method to be invoked: zap and set-location.  The
method  name  is  not  evaluated.   Any further arguments become
arguments to  the  method,  and  are  evaluated.   (There  is  a
function  send which is just like =$>$ except that the method name
argument is evaluated just like everything else.)

  Once an object is created, all operations on it are  performed
by  methods  defined for its flavor.  The flavor definition also
defines some methods.  For each "gettable" instance variable,  a
method  of  the  same name is defined which returns the value of
that instance variable.  For  "settable"  instance  variables  a
method  named  "set-$<$variable  name$>$" is defined. Given a new
value, this method sets the instance variable to have that value.

\subsection{Protection of Objects}

  Most LISP's, and PSL in particular, leave open the possibility
for  the  user  to  perform  illicit operations on LISP objects.
Instances of a flavor are represented as ordinary  LISP  objects
(evectors  at  present),  so  in  a sense it is quite easy to do
illicit  operations  on  them:  just  operate  directly  on  its
representation (do evector operations).

  On the other hand, there are major practical pitfalls in doing
this.    The  representation of a flavor of objects is generated
automatically, and there  is  no  guarantee  that  a  particular
flavor  definition will result in a particular representation of
the objects.  There is also no guarantee that the representation
of a flavor will remain the same over time.  It is  likely  that
at  some  point  evectors  will  no  longer  even be used as the
representation.

  In addition, using the objects package is quite convenient, so
the temptation to operate on the  underlying  representation  is
reduced.    For debugging, one can even define a couple of extra
methods "on the fly" if need be.

\section{Reference Information}

\subsection{Loading Objects}

{\bf Note: This file defines both macros and ordinary LISP 
functions. It must be loaded before any of these 
functions are used.} The recommended way of doing
this is to  put the expression: (BothTimes (load objects)) at the
beginning   of your source file. This will cause the package to be
loaded   at both compile and load time.

\subsection{Flavor Definition}

  Flavors  are  defined  by using the macro defflavor This macro
has the form:

\de{defflavor}{(defflavor flavor-name (var1 var2 ...) 
			(inheritance-flavor-list)\\ 
   option1 option2 ...):list}{ macro}
{}

    flavor-name is the name of the flavor being defined.
    var1 var2 ... are the  instance  variables  of  the  flavor.
    Each must be a symbol that is the name of the variable, or a
    list  of  two elements.  The list form must have the name of
    the instance variable  (id)  as  its  first  element  and  a
    default  initialization form as its second argument.  Do not
    use names like "IF" or "WHILE" for instance variables:  they
    are  translated freely within method bodies (see defmethod).
    The translation process is not very smart.  For example,  it
    doesn't  distinguish  between  formal  parameters to methods
    with the same name as instance variables.   For  safety,  if
    formal  parameters  are found that match instance variables,
    an  error  message  is  issued.   Although  the  translation
    process  isn't  smart, it at least understands the nature of
    quote.

    The inheritance-flavor-list must be a list of one element or
    the empty list.  When the empty list, it specifies  that  no
    inheritance   is   used  in  defining  this  flavor  (for  a
    description of inheritance, see section 10.3).  When  a  one
    element list, the element must be the name of another flavor
    to  inherit  from.  This other flavor must have already been
    compiled or loaded or an error message will be issued.

    option1 option2 ... are the options that effect the instance
    variables and methods of  the  flavor.   Recognized  options
    are:

\begin{verbatim}
   (INITABLE-INSTANCE-VARIABLES var1 var2 ...)
   (GETTABLE-INSTANCE-VARIABLES var1 var2 ...)
   (SETTABLE-INSTANCE-VARIABLES var1 var2 ...)
   (FAST-ACCESS-FOR-METHODS method1 method2 ...)
\end{verbatim}
%\begin{verbatim}
\begin{tabbing}
   \=INITABLE-INSTANCE-VARIABLES  \quad\=[make all instance
variables\\
																																\>\>INITABLE]\\
   \>GETTABLE-INSTANCE-VARIABLES  \>[make all instance variables\\
																																\>\>GETTABLE]\\
   \>SETTABLE-INSTANCE-VARIABLES  \>[make all instance variables\\
																																\>\>SETTABLE]\\
   \>FAST-ACCESS-FOR-METHODS      \>[access all methods fast]
\end{tabbing}
%\end{verbatim}
    An empty list of variables is taken as meaning all variables
    rather   than   none,  so  (GETTABLE-INSTANCE-VARIABLES)  is
    equivalent to GETTABLE-INSTANCE-VARIABLES.

    Initable instance variables may be initialized using options
    to make-instance or instantiate-flavor. See below.

    For each gettable instance variable a  method  of  the  same
    name  is  generated  to  access  the  instance variable.  If
    instance variable LOCATION is gettable, one can  invoke  (=$>$
    $<$object$>$ LOCATION).

    For  each  settable instance variable a method with the name
    SET-$<$name$>$ is generated. If instance variable LOCATION is
    settable,   one   can   invoke   (=$>$  $<$object$>$  SET-LOCATION
    $<$expression$>$).  Settable instance variables are always  also
    gettable  and  initable  by implication.  If this feature is
    not desired, define a method such as  SET-LOCATION  directly
    rather than declaring the instance variable to be settable.

    FAST-ACCESS-FOR-METHODS  is used to speed up message sending
    for a select set of  methods.   Methods  specified  in  this
    option  are  accessed  by  a  different  means then "normal"
    methods.  This option should only be used when:

\begin{itemize}
\item It is known that message sending speed is critical.
\item A select set of methods of a flavor  are  known  to  be
         called  very  frequently  compared  to  the rest of the
         methods of that flavor.
\end{itemize}

    The order methods are specified with this option  determines
    the order they are accessed (speed of method lookup).  Thus,
    the first method specified will be accessed the quickest and
    the  last  method  specified the slowest.  If no methods are
    given (the second form  above),  the  objects  package  will
    decide  on  its  own order.  Specifying no methods should be
    used only if a flavor has a few  methods  (i.e.,  less  then
    10).  If any of the methods specified are not methods of the
    flavor, a warning message will be issued and fast access for
    it will be ignored.

    As  an  example,  consider  the flavor TUNAFISH which has 70
    methods.  It is determined that much time is  spent  calling
    five  low level TUNAFISH methods whose names are A, B, C, D,
    and E.  Specifying:

\begin{verbatim}
         (FAST-ACCESS-FOR-METHODS A B C D E)
\end{verbatim}

    in the flavor definition of TUNAFISH will cause method A  to
    be  accessed  faster then any other method, B to be accessed
    slower then A, C slower then B, D slower then C, and finally
    E the slowest of the five methods.  The remaining 65 methods
    of TUNAFISH will be accessed as slow or slower then E.

    As its value, defflavor returns a list of the form:

\begin{verbatim}
         (flavor <flavor-name>)
\end{verbatim}
    Where $<$flavor-name$>$ is the name of the flavor being defined.
  Examples of Flavor Definition

\begin{verbatim}
  (defflavor complex-number
           (real-part imaginary-part)
           ()
           gettable-instance-variables
           initable-instance-variables
           )

  (defflavor complex-number
           ((real-part 0.0)
            (imaginary-part 0.0)
            )
           (number)
           gettable-instance-variables
           (settable-instance-variables real-part)
           )
\end{verbatim}
\subsection{Method Definition}

  As you  may  recall,  methods  are  defined  using  the  macro
defmethod.  It has the form:

\begin{verbatim}
(defmethod (flavor-name method-name) ([arg1 arg2 ...])
<expression1> <expression2>...): list                  macro
\end{verbatim}
    flavor-name is the name of the flavor a method is begin
    defined for. It must be an id.

    method-name is the name of this method. It must also be an
    id.

    arg1 arg2 ... are the arguments of the method. There may be
    zero or more and they must all be ids.

    $<$expression1$>$  $<$expression2$>$... make up the body of the
    method. This body can refer to any instance variable of the
    flavor by using the name just  like  an  ordinary  variable.
    They  can  also  set  them  using  setf.  All occurrences of
    instance variables (except within evectors or quoted  lists)
    are translated to an invocation of the form (EGETV SELF n).

    The  body  of a method can also freely use SELF like it were
    another instance variable.  SELF is bound to the object that
    the method applies to.  SELF may not be setqed or setfed.

    The value returned by defmethod is a list of the form:

\begin{verbatim}
         (method <flavor-name> <method-name>)
\end{verbatim}

    Where $<$flavor-name$>$ and $<$method-name$>$ are simply the names
				of the flavor being defined and method being defined
    respectively.\\

  Examples of Method Definition:

\begin{verbatim}
  (defmethod (complex-number real-part) () real-part)

  (defmethod (complex-number set-real-part) (new-real-part)
  (setf real-part new-real-part))

  (defmethod (toaster plug-into) (socket)
  (setf plugged-into socket)
  (=> socket assert-as-plugged-in self))
\end{verbatim}
\subsection{Object Creation}

  The most common way instances of a flavor are  created  is  by
using the function make-instance.  It has the form:


\de{make-instance}{(make-instance flavor-name:id
 [initialization-sequence]):\\
flavored object}{expr}
{
    make-instance  takes  as  arguments  a  flavor  name  and an
    optional initialization sequence.

    Flavor-name must  be  an  id  that  represents  an  existing
    flavor.

    The   optional   initialization-sequence,   when  specified,
    consists of alternating pairs of instance variable names and
    corresponding  initial  values.   Note  that  the   instance
    variable names are quoted, in the example above, because all
    the arguments are evaluated. make-instance returns an object
    that is "initialized."
}

  Examples of make-instance

\begin{verbatim}
            (make-instance 'complex-number)
            (make-instance 'complex-number 'real-part 0.0 
	                                   'imaginary-part 1.0)
\end{verbatim}
  An object returned by make-instance is initialized as follows:

  First, all instance variables with initialization specified in
the  call  to  make-instance are initialized to the value given.
After this the objects DEFAULT-INIT  method  is  envoked. This
method  is constructed by the flavor definition (defflavor).
It initializes any instance variables not specified in the 
call to make-instance but having default initializations 
specified in the flavor definition. It then sets all remaining
uninitialized instance variables to the value *UNBOUND*. The 
default initialization and initialization to *UNBOUND* is 
performed by the DEFAULT-INIT method.

  If a method named INIT is defined for this flavor  of  object,
it  is  invoked  automatically  after  the  initializations just
discussed.  The INIT method is passed, as its  one  argument,  a
list  of  alternating  variable  names and initial values.  This
list is the result of evaluating the  initializations  given  to
make-instance.  For example, if we call:

\begin{verbatim}
(make-instance 'complex-number 'real-part (sin 30)
                                'imaginary-part (cos 30))
\end{verbatim}
then the argument to the INIT method (if any) would be

\begin{verbatim}
  (real-part .5 imaginary-part .866).
\end{verbatim}
  The  INIT method may do anything desired to set up the desired
initial state of the object.

  At present, this  value  passed  to  the  INIT  method  is  of
virtually  no  use to the INIT method since the values have been
stored into the instance variables  already.    In  the  future,
though,  the  objects package may be extended to permit keywords
other  than  names  of  instance  variables   to   be   in   the
initialization part of calls to make-instance.  If this is done,
INIT methods will be able to use the information by scanning the
argument.

  For  details on how initialization is done when inheritance is
in effect, see Using Inheritance, section 10.3.

  Another less common way to create an object is through the use
of the function instantiate-flavor.  It has the form:

\DE{instantiate-flavor}{(instantiate-flavor flavor name:any
 [U:list]):\\flavored object}{expr}
{   
    This  is  the  same  as  make-instance,  except   that   the
    initialization  list  is  provided  as  a  single,  required
    argument. The list must be a  list  of  alternating  keyword
    names and initial values.
}

    Example:

\begin{verbatim}
  (instantiate-flavor 'complex-number
          (list  'real-part  (sin  30) 'imaginary-part (cos 30)))
\end{verbatim}

\subsection{Message Sending}

  There are several ways to invoke a method:

\de{=$>$}{(=$>$ object:object method:id): any}{expr}
{}
    Examples:

\begin{verbatim}
    (=> r real-part)
\end{verbatim}
\begin{verbatim}
    (=> r set-real-part 1.0)
\end{verbatim}
    The message name is not quoted. Arguments to the method are
    supplied as arguments to =$>$. In these examples, r is the
    object, real-part and set-real-part are the methods, and 1.0
    is the argument to the set-real-part method.

\de{send}{(send object:object method:id): any}{expr}
{}
    Examples:
\begin{verbatim}
    (send r 'real-part)
\end{verbatim}
\begin{verbatim}
    (send r 'set-real-part 1.0)
\end{verbatim}
   
 The  meanings  of  these  two  examples  are the same as the
    meanings of the previous two.  Only the syntax is different:
    the message name is quoted.

\de{send-if-handles}{(send-if-handles object:object method:id 
[U:any]): any}{expr}
{    Conditionally Send a Message
}

    Examples:
\begin{verbatim}
    (send-if-handles r 'real-part)
\end{verbatim}
\begin{verbatim}
    (send-if-handles r 'set-real-part 1.0)
\end{verbatim}
    
SEND-IF-HANDLES is like SEND,  except  that  if  the  object
    defines  no  method  to  handle  the  message,  no  error is
    reported and NIL is returned.

\DE{lexpr-send}{(lexpr-send object:object method:id [U:any] V:list):any}{expr}
{    Send a message (explicit "Rest" Argument List)
}
    
				Examples:
\begin{verbatim}
    (lexpr-send foo 'bar a b c (list x y))
\end{verbatim}
    
The last argument to LEXPR-SEND is a list of  the  remaining
    arguments.

\DE{lexpr-send-if-handles}{(lexpr-send-if-handles object:any\\ 
method:id [U:any] V:list):any}{expr}
{                                                      
    This  is  the  same  as  LEXPR-SEND, except that no error is
    reported if the object fails to handle the message.
}

\DE{lexpr-send-1}{(lexpr-send-1 object:object method:id V:list):any}{expr}
{    Send a Message (Explicit Argument List)
}
    Examples:
\begin{verbatim}
    (lexpr-send-1 r 'real-part nil)
\end{verbatim}
\begin{verbatim}
    (lexpr-send-1 r 'set-real-part (list 1.0))
\end{verbatim}
    
Note that the message name is quoted and that  the  argument
    list is passed as a single argument to LEXPR-SEND-1.

\de{lexpr-send-1-if-handles}{(lexpr-send-1-if-handles object:any\\ 
method:any V:list): any}{expr}
{   This  is  the  same as LEXPR-SEND-1, except that no error is
    reported if the object fails to handle the message.
}

\de{ev-send}{(ev-send object:any method:any V:list): any}{expr}
{    EV-SEND is just like LEXPR-SEND-1, except that it is an EXPR
    instead of a MACRO.  Its sole purpose is to  be  used  as  a
    run-time   function  object,  for  example,  as  a  function
    argument to a function.
}
\subsection{Printing Objects}

  Objects can print out according to their flavor.  If a  flavor
has  a  channelprin  method,  that method will be used by prin1,
prin2, etc. to print objects of that flavor.

\de{=$>$}{(=$>$ any CHANNELPRIN channel level prin1?): any}{method}
{    A channelprin method must accept three arguments: a  channel
    (currently  a  small  integer); either nil or a number which
    will indicate the current level  of  nesting  within  lists,
    vectors, or other objects; and a boolean which will indicate
    whether prin1- or prin2-style printing is being done.
}

    Ordinarily, an object should print in the form \#$<$flavor info
    ...   $>$. If the level of printing is not nil, printing of
    lists, vectors, objects, etc. within the channelprin  method
    should be done with prinlevel bound to 1 less than its value
    at the time of entry to the method.

\subsection{Useful Functions on Objects}

\de{object-type}{(object-type U:any): {id, NIL}}{expr}
{
    The  object-type  function  returns  the type (an id) of the
    specified object, or nil, if the argument is not an  object.
    At present this function cannot be guaranteed to distinguish
    between  objects  created  by  the objects package and other
    LISP entities, but  the  only  possible  confusion  is  with
    vectors or evectors.
}


\de{flavor-defined?}{(flavor-defined? flavor-name:id): {t, nil}}{expr}
{
    Returns  t  if  flavor-name  is  a currently defined flavor.
    Returns nil otherwise.
}

\de{declare-flavor}{(declare-flavor $<flavor name> <var1> <var2>$...):\\ 
undefined}{macro}
{ 
}

\de{undeclare-flavor}{(undeclare-flavor $<var1> <var2>$...): undefined}{macro}
{
{\tt  *** Read these warnings carefully! ***}

  This facility can reduce the overhead of invoking  methods  on
particular  variables,  but  it should be used sparingly.  It is
not well integrated with the rest of  the  language.    At  some
point a proper declaration facility is expected and then it will
be  possible  to  make  declarations  about  objects,  integers,
vectors, etc., all in a uniform and clean way.
}

The declare-flavor macro allows you to declare that a specific
symbol is bound to an object of a specific flavor.  This  allows
the  flavors  implementation  to  eliminate  the run-time method
lookup normally  associated  with  sending  a  message  to  that
variable,  which  can  result  in  an appreciable improvement in
execution speed.  This feature is motivated solely by efficiency
considerations and should be used  ONLY  where  the  performance
improvement is critical.

  Details:  if  you  declare  the  variable  X to be bound to an
object of flavor FOO, then {\bf within the context of the declaration}
(see below), expressions of the form ($=>$ X GORP ...) or
(SEND X 'GORP ...) will be replaced by function invocations of the form
(FOO GORP X ...). Note that there is no  check made that the
flavor FOO actually contains a method GORP.If it does not, then a 
run-time error "Invocation of undefined function FOO"GORP" will be
reported.
  
WARNING:  The  declare-flavor  feature  is  not presently well
integrated with the compiler. Currently,  the  declare-flavor
macro  may  be used only as a top-level form, like the PSL FLUID
declaration.  It takes effect for all code evaluated or compiled
henceforth.  Thus, if you should later compile a different  file
in  the  same compiler, the declaration will still be in effect!
{\bf this is a dangerous crock, so be careful!}  To
avoid problems, I recommend  that  declare-flavor  be used only for
uniquely-named variables. The effect of a declare-flavor can be
undone  by an undeclare-flavor, which  also  may  be used only as a
top-level form. Therefore, it is good practice to bracket  your code in
the  source  file  with  a  declare-flavor  and  a corresponding
undeclare-flavor.

\begin{verbatim}
  *** Did you read the above warnings??? ***
\end{verbatim}

\de{object-get-handler}{(object-get-handler object:object method-name:id): id}{expr}
{}
    Note that the use of object-get-handler is not  recommended.
    It should only be used in the most critical places where the
    time  to  perform  a normal message send is too slow and the
    declare-flavor facility cannot be used.

    Given  an  object  and  a  method-name,   object-get-handler
    returns  the  name  of  the  function  that  implements this
    method.  It is used to repetitively send the  same  message.
    For example:\\

\begin{verbatim}
       (let
           ((func-name (object-get-handler obj 'banana)))
       ...
           (for
           ...
           (do (idapply func-name (list arg1 arg2 ... argn)))
           )
       )
\end{verbatim}
    This is equivalent to sending the message:

\begin{verbatim}
         (=> obj banana arg1 arg2 ... argn)
\end{verbatim}
    inside  the  for but the lookup of the function name for the
    message send is only performed once.

\subsubsection{Miscellaneous Functions}

\de{objects-version):}{(objects-version): NIL}{expr}
{
    This function prints a date used to check the version of the
    objects package in use.  When bugs are reported, the version
    date should be given.  Older versions of the objects package
    did not have this function defined.}

\section{Using Inheritance}

  Inheritance is a mechanism that allows a flavor to be  defined
in  terms  of  other flavors.  This allows objects to be created
that are specializations or generalizations  of  other  objects.
The object package supports what is called single inheritance --
only  one  flavor  can be inherited from by another flavor.  The
flavor being inherited, however, could have inherited  from  one
other  flavor,  and  so on.  Note that the LISP machine supports
multiple  inheritance  allowing  more  than  one  flavor  to  be
inherited from at a time.\\

  The best way to explain single inheritance is through an example:\\
Suppose we have defined the flavor mother and then define the flavor daughter in
which we specified the inheritance-flavor-list to be mother. The inheriting
process causes daughter to contain all the instance variables as well as
the methods that were defined for mother. That is,a daughter object can
deal with any of the instance variables and receive any of the messages
defined for flavor mother without having to recreate them.

We refer to flavor mother as the parent of flavor daughter,  and
daughter  as  a  child of mother.  daughter, in turn, can be the
parent of other flavors.    Each  child  inherits  the  instance
variables and methods of its parent, its parent's parent, and so
on.

  When  inheriting instance variables, a child flavor has as its
instance variables the  union  of  all  those  defined  for  its
parents and those specifically defined for the new flavor.  As a
result,  it  would  not  make  sense  to  give a child flavor an
instance variable with the same  name  as  one  in  one  of  its
parent's.   For  this  reason,  duplicate instance variables are
flagged as an error.

  Inheriting methods, however, is different.  It is possible  to
define a method in a child flavor with the same name as a method
in  one  of the parents.  The effect is to override the parent's
method with the child's method, as far as the  child  flavor  is
concerned.  In other words, when sending a message to an object,
the method executed is the first one found by searching backward
from  it through its parents.  First the methods defined for the
child flavor are searched, then the  child's  parent's  methods,
and so on until one is found.

  As  an  example  of  overriding  methods,  the  following code
defines a flavor geometric-object.  Then, a new flavor square is
created with geometric-object as its parent.  Finally, a  flavor
colored-square is created with square as its parent.

\begin{verbatim}
(defflavor geometric-object
  (name
   size)
  ()                    % A flavor with no parents
  )

(defmethod (geometric-object display) ()
  (printf "%w is %w units big" name size))

(defmethod (geometric-object set-name) (new-name)
  (setf name new-name))

(defflavor square
  (side-length)
  (geometric-object)    % A flavor with one parent, geometric-object
  )

% Overriding for square the inherited method display

(defmethod (square display) ()
  (printf "%w is a square, %w units long on all sides" name
          side-length))

(defmethod (square new-length) (length)
  (setf side-length length)
  (setf size (* length length)))

(defflavor colored-square
  (color)
  (square)
  )

(defmethod (colored-square get-color) ()
  color)
\end{verbatim}
After  the  definitions,  flavor colored-square has the instance
variables:

\begin{verbatim}
        color
        side-length
        name
        size
\end{verbatim}
and has the methods:
\begin {itemize}
%\begin{verbatim}
\item   get-color (as defined for flavor colored-square)
\item        new-length (as defined for flavor square)
\item        display (as defined for flavor square)
\item        set-name (as defined for flavor geometric-object)
%\end{verbatim}
\end {itemize}

\subsection{Warning on Inheritance Usage}

  The Common lisp objects package will be incompatible with  the
existing objects package as far as inheritance is concerned.  In
Common  lisp,  an inherited instance variable cannot be accessed
by simply referring to  it  within  a  method.   Also,  the  way
variables are initialized is quite different.

  To  access  inherited  instance  variables,  you  must  do the
equivalent of sending the object a message to get  the  instance
variable's  value.   For  example,  referring  to  the  instance
variable banana in a method for the flavor fruit,   a  reference
to  banana  in  PSL  becomes  something like (=$>$ self banana) in
Common lisp.  When you reference inherited instance variables in
PSL and performance is important, you should clearly mark  them.
If  performance  is  not  an issue,  make the instance variables
gettable and refer to them by their method.  This  also  applies
to settable instance variables.

  Uses  of  the  INIT method may also have to be changed because
the new objects package  performs  object  initialization  in  a
different way.  Avoid  the INIT method if you can,  if you can't
mark it for change for Common lisp objects.

  In this way, transition to Common lisp will be much easier.

\subsection{Using SELF and MYSELF with Inheritance}

  Suppose  an  additional  method   was   defined   for   flavor
geometric-object:

\begin{verbatim}
  (defmethod (geometric-object display-me) ()
  (=> self display))
\end{verbatim}
  Now  suppose a display-me message is sent to an object of type
square.  Which method for the message 'display' should  be  used
by  display-me?   Should it look up the method in the context of
the flavor where the method was  defined  (geometric-object)  or
the  context  of  the  flavor that received the original message
(square)? The answer is to provide a new symbol, MYSELF. MYSELF means
to look up messages in the context of the object that received
the original message. SELF looks up methods in the context of
the flavor it is defined in. If display-me was rewritten as:

\begin{verbatim}
(defmethod (geometric-object display-me) ()
  (=> self display)
  (=> myself display))
\end{verbatim}
and  sent  to an object of flavor square, then the first display
would be that defined in geometric-object, and the second  would
be  the  one  defined for square.  If the display-me message was
sent to an object of flavor geometric-object,  self  and  myself
would be identical.

\subsection{Inheritance and Initialization}

  You  may  have  noticed that there is ambiguity in the order a
newly  created  object  is  initialized  when  the  object  uses
inheritance.    When  inheritance  is  used,  initialization  is
performed as describe above with the following additions:

  When  the  son  flavor's  (i.e.,  the  flavor  given  in   the
make-instance or instantiate-flavor call) DEFAULT-INIT method is
executed,   the  first  thing  it  does  is  call  its  father's
DEFAULT-INIT method.  The second thing it does  is  execute  its
father's INIT method, if it exists.

  Notice  that  if  the  father  flavor  used  inheritance,  its
DEFAULT-INIT  method  would   first   call   its   grandfather's
DEFAULT-INIT  and INIT methods, and so on.

\subsection{Making Changes to Inherited Code}

  The  inheritance scheme used in this facility is static -- all
of the work is done at compile-time.  This has some implications
for compiling object definitions that use inheritance:

\begin{itemize}
\item All of the parents of a flavor must be loaded  or  compiled
     before  compiling  the new child's definitions.  If this is
     not the case, an error message will be issued.
\item If a change is made  to  a  parent  flavor,  including  the
     addition  of  a  new method or changing instance variables,
     all of the children will have to be recompiled to  use  it.
     You  do  not have to recompile the children, however, if an
     existing method of a parent is redefined.
\end{itemize}

\section{Debugging Information}

  Any object may  be  displayed  symbolically  by  invoking  the
method  DESCRIBE,  e.g. ($=>$ x describe). This method prints the
name of each instance variable and its value, using the ordinary
LISP printing routines.   Flavored  objects  are  liable  to  be
complex and deeply nested or even circular.  This makes it often
a  good idea to set PRINLEVEL to a small integer before printing
structures containing objects to control the amount of output.

  When printed by the standard LISP printing routines, "flavored
objects" appear as evectors whose zeroth element is the name  of
the flavor.

  For  each  method  defined,  there  is  a  corresponding  LISP
function named\\
 \verb+<flavor-name>$<method-name>+. Such function names
show up in backtrace printouts.

  It is permissible to define new methods on the fly for debugging
purposes.
