
\chapter*{List Structure}

\section{Introduction to Lists and Pairs}

  The  pair  is  a  fundamental PSL data type, and is one of the
major attractions of LISP programming.  A  pair  consists  of  a
two-item  structure.  In PSL the first element is called the car
and  the  second  the  cdr;  in  other   LISPs,   the   physical
relationship  of the parts may be different.  An illustration of
the tree structure is given below as a box diagram; the car  and
the cdr are each represented as a portion of the box.

\begin{verbatim}
                -----------------
                || Car  | Cdr  ||
                -----------------
\end{verbatim}
As  an  example,  a  tree    written  as  ((A . B) . (C . D)) in
dot-notation is drawn below as a box diagram.

\begin{verbatim}
                -----------------
                ||   /  |  \   ||
                ----/-------\----
                   /         \
     -----------------       -----------------
     ||  A   |   B  ||       ||  C   |   D  ||
     -----------------       -----------------
\end{verbatim}
  The box diagrams are  tedious  to  draw,  so  dot-notation  is
normally  used.  Note that a space is left on each side of the .
to ensure that pairs are not confused with floats.

  A list is a special case of a dotted pair structure.   A  list
is either


\begin{enumerate}
\item  NIL
\item  A dotted pair whose car is an expression and whose cdr is a
     list.
\end{enumerate}
List  notation  in  general  is  a  lot  easier to read than the
equivalent dotted pair notation.

\begin{verbatim}
(A . NIL)               = (A)
(A . B)                 = (A . B)
(NIL . NIL)             = (NIL)
(A . (B . NIL))         = (A B)
((A . NIL) . NIL)       = ((A))
((A . NIL) . (B . NIL)) = ((A) B)
(A . (B . C))           = (A B . C)
\end{verbatim}
  The following is  an  algorithm  for  writing  a  dotted  pair
structure in list notation.


\begin{enumerate}
\item  Set  a  pointer  Q  to  the  beginning  of  the dotted pair
     structure and write a left parenthesis.
\item  Write the list notation for the data structure  pointed  to
     by the car of Q and reset Q to the cdr of Q.
\item  If   Q  is  now  the  null  pointer,  then  write  a  right
     parenthesis; otherwise, write a space, and if Q is an atom,
     write a period, a space, Q's name, and a right parenthesis;
     otherwise write a space and go to step 2.
\end{enumerate}

\section{Basic Functions on Pairs}

The  following  are  elementary  functions  on  pairs. All
functions  in  this  Chapter  which  require pairs as parameters
signal a type mismatch error if the parameter  given  is  not  a
pair.


\de{cons}{(cons U:any V:any): pair}{expr}
{    Returns a pair which is not eq to anything else and has U as
    its car part and  V as its cdr part.
}

\de{car}{(car U:pair): any}{open-compiled expr}
{    The  left  part  of  the  pair  U  is  returned.   Note that
    applications  of  car  are   open   compiled,   a   compiled
    application  of  car  will not verify that its argument is a
    pair.  The function SafeCar may be used  in  place  of  car.
    The  definitions  of  these two functions are identical, the
    difference  between  them  is  that  safecar  is  not   open
    compiled.    For interpreted applications, the car of nil is
    nil and if U is something other  that  a  pair  or  nil  the
    following error will result.
}
\begin{verbatim}
*****  An attempt was made to do CAR on `U', which is not a pair
\end{verbatim}

\de{cdr}{(cdr U:pair): any}{open-compiled expr}
{    The left part  of  the  pair  U  is  returned.    Note  that
    applications   of   cdr   are   open  compiled,  a  compiled
    application of cdr will not verify that its  argument  is  a
    pair.    The  function  SafeCdr may be used in place of cdr.
    The definitions of these two functions  are  identical,  the
    difference   between  them  is  that  safecdr  is  not  open
    compiled.  For interpreted applications, the cdr of  nil  is
    nil  and  if  U  is  something  other that a pair or nil the
    following error will result.
}
\begin{verbatim}
***** An attempt was made to do CDR on `U', which is not a pair
\end{verbatim}
  The composites of car and cdr are supported up to four levels.

\begin{verbatim}
              Car                             Cdr
      Caar            Cdar            Cadr            Cddr
  Caaar  Cdaar    Cadar  Cddar    Caadr  Cdadr    Caddr  Cdddr
 Caaaar  Cadaar  Caadar  Caddar  Caaadr  Cadadr  Caaddr  Cadddr
 Cdaaar  Cddaar  Cdadar  Cdddar  Cdaadr  Cddadr  Cdaddr  Cddddr
\end{verbatim}
    These  are all exprs of one argument.  Applications of these
    functions are generally open compiled.  An example of  their
    use  is that (cddar p) is equivalent to (cdr (cdr (car p))).
    For interpreted applications, a type mismatch  error  occurs
    if the argument does not possess the specified component.

As  an  alternative  to  employing  chains  of  car and cdr to
obscure depths, particularly in extracting elements of  a  list,
consider  the use of the functions first, second, third, fourth,
rest, nth, and pnth.


\de{ncons}{(ncons U:any): pair}{expr}
{    Equivalent to (cons u nil).
}

\de{xcons}{(xcons LEFT:any RIGHT:any): pair}{expr}
{    Equivalent to (cons RIGHT LEFT), this function is useful for
    generating efficient list building code for the compiler.
}

\de{copy}{(copy X:any): any}{expr}
{    This function returns a copy of  X.    While  each  pair  is
    copied,  atomic  elements  (for  example  ids,  strings, and
    vectors) are not.  See totalcopy in section 7.5.  Note  that
    copy  is recursive and will not terminate if its argument is
    a circular list.
}
\begin{verbatim}
    (de copy (u)
      (if (pairp u)
        (cons (copy (car u)) (copy (cdr u)))
        u))
\end{verbatim}
\begin{verbatim}
    1 lisp> (setq p '("AKU" (charlie)))
    ("AKU" (CHARLIE))
    2 lisp> (setq q (copy p))
    ("AKU" (CHARLIE))
    3 lisp> (eq p q)
    NIL
    4 lisp> (eq (first p) (first q))
    T
    5 lisp> (eq (third p) (third q))
    NIL
\end{verbatim}
  See Chapter 6 for other relevant functions.\\

\noindent
The following functions are known as "destructive"  functions,
because  they  change  the  structure of the pair given as their
argument, and consequently change the structure  of  any  object
containing the pair.  They are frequently used to make code more
efficient.    For  example  append  will copy its first argument
whereas nconc will not.  These functions are also used to  build
structures  which share sub-structure.  It is possible to create
self referential structures with  these  functions.    This  can
create havoc with normal printing and list traversal functions.


\de{rplaca}{(rplaca U:pair V:any): pair}{open-compiled expr}
{    The  car of the pair U is replaced by V, the modified pair U
    is returned.  A type mismatch error occurs if  U  is  not  a
    pair
}
\begin{verbatim}
    1 lisp> (setq fruit '(orange apple))
    (ORANGE APPLE)
    2 lisp> (setq food (cons 'cheese fruit))
    (CHEESE ORANGE APPLE)
    3 lisp> (rplaca fruit 'peach)
    (PEACH APPLE)
    4 lisp> food
    (CHEESE PEACH APPLE)
\end{verbatim}

\de{rplacd}{(rplacd U:pair V:any): pair}{open-compiled expr}
{    The  cdr of the pair U is replaced by V, the modified pair U
    is returned.  A type mismatch error occurs if  U  is  not  a
    pair.
}
\begin{verbatim}
    1 lisp> (setq pair '(left))
    (LEFT)
    2 lisp> (progn (rplacd pair 'right) pair)
    (LEFT . RIGHT)
\end{verbatim}

\de{rplacw}{(rplacw A:pair B:pair): pair}{expr}
{    Replaces  the whole pair:  the car of A is replaced with the
    car of B, and the cdr of A with the cdr of B.  The  modified
    pair A is returned.
}
\section{Functions for Manipulating Lists}

The  following functions are meant for the special pairs which
are lists, as described in Section 5.1.  An  argument  which  is
not  a  list could give unexpected results.  For example, length
is used to determine the number of top level elements in a list.

\begin{verbatim}
1 lisp> (length '(a b c))
3
2 lisp> (length '(a b . c))
2
\end{verbatim}
\subsubsection{Selecting List Elements}


\de{first}{(first L:pair): any}{macro}
{    A synonym for car.
}

\de{second}{(second L:pair): any}{macro}
{    A synonym for cadr.
}

\de{third}{(third L:pair): any}{macro}
{    A synonym for caddr.
}

\de{fourth}{(fourth L:pair): any}{macro}
{    A synonym for cadddr.
}

\de{rest}{(rest L:pair): any}{macro}
{    A synonym for cdr.
}

\de{lastpair}{(lastpair L:pair): any}{expr}
{    Returns the last pair of a L.   It is often useful to  think
    of  this  as  a  pointer  to  the  last element for use with
    destructive functions such as rplaca.  If L is  not  a  pair
    then a type mismatch error occurs.
}
\begin{verbatim}
    (de lastpair (l)
      (if (or (atom l) (atom (cdr l)))
        l
        (lastpair (cdr l))))
\end{verbatim}

\de{lastcar}{(lastcar L:pair): any}{expr}
{    Returns  the  last  element  of the pair L.  A type mismatch
    error results if L is not a pair.
}
\begin{verbatim}
    (de lastcar (l)
      (if (atom l) l (car (lastpair l))))
\end{verbatim}

\de{nth}{(nth L:pair N:integer): any}{expr}
{    Returns the Nth element of the list L.  If L  is  atomic  or
    contains  fewer  than  N  elements,  an  out  of range error
    occurs.
}
\begin{verbatim}
    (de nth (l n)
      (cond ((null l) (range-error))
            ((onep n) (first l))
            (t (nth (rest l) (sub1 n)))))
\end{verbatim}
    Note that this definition  is  not  compatible  with  Common
    LISP.  The Common LISP definition reverses the arguments and
    defines the car of a list to be the "zeroth" element.


\de{pnth}{(pnth L:list N:integer): any}{expr}
{    Returns  a list starting with the nth element of the list L.
    Note that the result is a pointer to the nth element of L, a
    destructive function like rplaca can be used to  modify  the
    structure  of  L.    If L is atomic or contains fewer than N
    elements,  an out of range error occurs.
}
\begin{verbatim}
    (de pnth (l n)
      (cond ((onep n) l)
            ((not (pairp l)) (range-error))
            (t (pnth (rest l) (sub1 n)))))
\end{verbatim}
\subsection{Membership and Length of Lists}


\de{member}{(member A:any L:list): extra-boolean}{expr}
{    Returns nil if A is not equal to some top level  element  of
    the  list  L;  otherwise it returns the remainder of L whose
    first element is equal to A.
}
\begin{verbatim}
    (de member (a l)
      (cond ((not (pairp l)) nil)
            ((equal a (car l)) l)
            (t (member a (cdr l)))))
\end{verbatim}

\de{memq}{(memq A:any B:list): extra-boolean}{expr}
{    The same as member except that eq  is  used  for  comparison
    instead of equal. Note that the value returned by either
				member or memq is eq to the  portion  of  the list which begins
				with A. Thus a function like rplaca may be used to alter A.
}
\begin{verbatim}
1 lisp> (setq sequence '(1 3 3))
(1 3 3)
2 lisp> (rplaca (memq 3 sequence) 2)
(2 3)
3 lisp> sequence
(1 2 3)
\end{verbatim}

\de{length}{(length X:any): integer}{expr}
{    The top level length of the list X is returned.
}
\begin{verbatim}
    (de length (l)
      (if (atom l) 0 (add1 (length (cdr l)))))
\end{verbatim}
\subsubsection{Constructing, Appending, and Concatenating Lists}


\de{list}{(list [U:any]): list}{fexpr}
{    Construct a list of the arguments.
}
\begin{verbatim}
    1 lisp> (list (car '(left . right)) (list 'next))
    (LEFT (NEXT))
\end{verbatim}

\de{append}{(append U:list V:list): list}{expr}
{    Returns a constructed  list in which  the last element  of U
    is followed by the first element  of  V.    The  list  U  is
    copied, but V is not.
}
\begin{verbatim}
    (de append (u v)
      (cond ((not (pairp u)) v)
            (t (cons (first u) (append (rest u) v)))))
\end{verbatim}

\de{nconc}{(nconc U:list V:list): list}{expr}
{    Destructive version of append, the cdr of the last pair of U
    is  modified to reference V.  While append creates a copy of
    U, nconc uses U itself in constructing the result.
}
\begin{verbatim}
    (de nconc (u v)
      (if (not (pairp u))
        v
        (rplacd (lastpair u) v)))
\end{verbatim}
\begin{verbatim}
    1 lisp> (setq a '(swan))
    (SWAN)
    3 lisp> (nconc a '(giles))
    (SWAN GILES)
    4 lisp> a
    (SWAN GILES)
\end{verbatim}

\de{aconc}{(aconc U:list V:any): list}{expr}
{    Destructively adds element V to the tail of list U.
}
\begin{verbatim}
    1 lisp> (setq a '(phillips))
    (PHILLIPS)
    2 lisp> (progn (aconc a 'posner) a)
    (PHILLIPS POSNER)
\end{verbatim}

\de{lconc}{(lconc PTR:list LST:list): list}{expr}
{    Effectively nconc, but avoids scanning from the front to the
    end of the first list by maintaining a pointer.   PTR  is  a
    pair  whose  car is a list L and whose cdr is a reference to
    the last pair of L.  The value returned is the updated  pair
    PTR.
}
\begin{verbatim}
    (progn (rplacd (cdr ptr) lst)
           (rplacd ptr (lastpair lst)))
\end{verbatim}
    This  function  is  useful  for  building lists from left to
    right, PTR should be initialized to (nil . nil)  before  the
    first call on lconc.


\de{tconc}{(tconc PTR:list ELM:any): list}{expr}
{    Effectively aconc, but avoids scanning from the front to the
    end  of  the  first list by maintaining a pointer.  PTR is a
    pair whose car is a list L and whose cdr is a  reference  to
    the  last pair of L.  The value returned is the updated pair
    PTR.
}
\begin{verbatim}
    (progn (setq elm (ncons elm))
           (rplacd (cdr ptr) elm)
           (rplacd ptr elm))
\end{verbatim}
    This function is useful for  building  lists  from  left  to
    right.   PTR should be initialized to (nil . nil) before the
    first call on tconc.

\subsubsection{Lists as Sets}

A set is a list in which each element occurs only once.  Since
the order of elements in a set does not matter, these  functions
may not preserve order.


\de{adjoin}{(adjoin ELEMENT:any SET:list): list}{expr}
{    Add Element to SET if it is not already a member.
}
\begin{verbatim}
    (de adjoin (elm set)
      (if (member elm set) set (cons elm set)))
\end{verbatim}
    Recall that member uses equal to test for equality.


\de{adjoinq}{(adjoinq ELEMENT:any SET:list): list}{expr}
{    Similar  to  adjoin  except  that eq is used to test for set
    membership.
}

\de{union}{(union X:list Y:list): list}{expr}
{    Returns the union of sets X and Y.
}
\begin{verbatim}
    (de union (x y)
      (if (not (pairp x))
        y
        (union (rest x)
               (if (member (first x) y)
                 y
                 (cons (first x) y)))))
\end{verbatim}
    Notice that the two arguments to union  are  assumed  to  be
    sets,  if  either  contains  duplicates  then the result may
    contain duplicates as well.

\begin{verbatim}
    1 lisp> (union '(1 2 2) '(3))
    (2 1 3)
    2 lisp> (union '(3) '(1 2 2))
    (3 1 2 2)
\end{verbatim}

\de{unionq}{(unionq X:list Y:list): list}{expr}
{    Similar to union except that eq is  used  to  test  for  set
    membership.
}

\de{intersection}{(intersection U:list V:list): list}{expr}
{    Returns the intersection of sets U and V.
}
\begin{verbatim}
    (de intersection (u v)
      (cond ((not (pairp u)) nil)
            ((member (car u) v)
             (cons (car u)
                   (intersection (cdr u) (delete (car u) v))))
            (t (intersection (cdr u) v))))
\end{verbatim}
    Notice that the two arguments to intersection are assumed to
    be  sets,  if either contains duplicates then the result may
    contain duplicates as well.

\begin{verbatim}
    1 lisp> (intersection '(1 2) '(2))
    (2)
    2 lisp> (intersection '(2 2) '(1 2 2))
    (2 2)
\end{verbatim}

\de{intersectionq}{(intersectionq U:list V:list): list}{expr}
{    Similar to intersection except that eq is used to  test  for
    set membership.
}

\de{list2set}{(list2set SET:list): list}{expr}
{    Remove  redundant  elements  from the top level of SET using
    equal.
}

\de{list2setq}{(list2setq SET:list): list}{expr}
{    Remove redundant elements from the top level  of  SET  using
    eq.
}
\subsection{Deleting Elements of Lists}

Note  that  the  functions  suffixed  by IP will destructively
modify the list from which elements are being deleted.   If  you
use  such  a  function do not rely on the result being eq to the
argument.  The value returned will  have  all  of  the  elements
removed,    but  the  modifications  which have been made to the
argument may not reflect  this.    In  particular,  the  leading
elements  which  are equal to the element being deleted will not
be spliced out of the list.

\begin{verbatim}
1 lisp> (setq this '(a b c))
(a b c)
2 lisp> (deletip 'a this)
(b c)
3 lisp> this
(a b c)
2 lisp> (reversip this)
(c b a)
3 lisp> this
(a)
\end{verbatim}

\de{delete}{(delete U:any V:list): list}{expr}
{    Returns V with the first top level occurrence of  U  removed
    from  it.  Equal is used for comparing elements.  The result
    consists of a copy of the portion of U  which  comes  before
    the  deleted element, and the portion of U which follows the
    deleted element.
}
\begin{verbatim}
    (de delete (u v)
      (cond ((not (pairp v)) v)
            ((equal (car v) u) (cdr v))
            (t (cons (car v) (delete u (cdr v))))))
\end{verbatim}

\de{del}{(del F:function U:any V:list): list}{expr}
{    Generalized  delete  function  with  F  as  the   comparison
    function.
}
\begin{verbatim}
    1 lisp> (del '(lambda (i e) (> i e)) 0 '(-2 3 -1))
    (-2 -1)
\end{verbatim}

\de{deletip}{(deletip U:any V:list): list}{expr}
{    The destructive version of delete,  V may be modified.
}

\de{delq}{(delq U:any V:list): list}{expr}
{    Returns  V  with the first top level occurrence of U removed
    from it.  Eq is used for comparing  elements.    The  result
    consists  of  a  copy of the portion of U which comes before
    the deleted element, and the portion of U which follows  the
    deleted element.
}

\de{delqip}{(delqip U:any V:list): list}{expr}
{    The destructive version of delq,  V may be modified.
}

\de{delasc}{(delasc U:any V:a-list): a-list}{expr}
{    Returns  V  with the first top level occurrence of (U . ANY)
    removed from it.  Equal is used for comparisons.  The result
    consists of a copy of the portion of U  which  comes  before
    the  deleted element, and the portion of U which follows the
    deleted element.
}

\de{delascip}{(delascip U:any V:a-list): a-list}{expr}
{    The destructive version of delasc,  V may be modified.
}

\de{delatq}{(delatq U:any V:a-list): a-list}{expr}
{    Returns V with the first top level occurrence of (U  .  ANY)
    removed  from  it.   Eq is used for comparisons.  The result
    consists of a copy of the portion of U  which  comes  before
    the  deleted element, and the portion of U which follows the
    deleted element.
}

\de{delatqip}{(delatqip U:any V:a-list): a-list}{expr}
{    The destructive version of delatq,  V may be modified.
}
\subsection{List Reversal}


\de{reverse}{(reverse U:list): list}{expr}
{    Returns a copy of the top level of U in reverse order.
}

\begin{verbatim}
    (de reverse (l)
      (do ((result nil (cons (car pointer) result))
           (pointer l (cdr pointer)))
          ((not (pairp pointer)) result)))
\end{verbatim}

\de{reversip}{(reversip U:list): list}{expr}
{    The destructive version of reverse.   The  argument  may  be
    destructively  modified  to  produce  the result.  Note also
    that the result may not be eq to the argument.
}

\begin{verbatim}
    (de reversip (l)
      (prog (next result)
            (while (pairp l)
              (setq next (cdr l))
              (setq result (rplacd l result))
              (setq l next))
            (return result)))
\end{verbatim}
\subsection{Functions for Sorting}

The {\bf gsort} module provides functions for sorting lists.  Some
of the functions take a comparison function as an argument.  The
comparison function takes two arguments and should return nil if
the  second  argument should come before the first in the sorted
result.  A lambda  expression  is  acceptable  as  a  comparison
function.    Note  that since sorting requires many comparisons,
and thus many calls on the comparison function, the sort will be
much faster if the comparison function is compiled.


\de{gsort}{(gsort TABLE:list LEQ-FN:{id,function}): list}{expr}
{    Returns a sorted list.  LEQ-FN is  the  comparison  function
    used  to determine the sorting order.  The original TABLE is
    unchanged.  Gsort uses a stable sorting algorithm.  In other
    words, if X appears before Y in TABLE  then  X  will  appear
    before Y in the result unless X and Y are out of order.
}

\de{gmergesort}{(gmergesort TABLE:list LEQ-FN:{id,function}): list}{expr}
{    The  destructive version of gsort, this function is somewhat
    faster than gsort.  Note  that  you  should  use  the  value
    returned  by  the  function,  don't  depend  on the modified
    argument to give the right answer.
}

\de{idsort}{(idsort TABLE:list): list}{expr}
{    Returns a list of the ids in TABLE, sorted into alphabetical
    order.  The  original  list  is  unchanged.    Case  is  not
    significant in determining the alphabetical order.
}
\begin{verbatim}
1 lisp> (setq x '(3 8 -7 2 1 5))
(3 8 -7 2 1 5)
2 lisp>     % sort from smallest to largest
2 lisp> (gsort x 'leq)
(-7 1 2 3 5 8)
3 lisp>     % sort from largest to smallest
3 lisp> (gmergesort x 'geq)
(8 5 3 2 1 -7)
4 lisp>     % note that the value of x has been modified
4 lisp> x
(3 2 1 -7)
5 lisp> (idsort '(the quick brown fox jumped over the lazy dog))
(brown dog fox jumped lazy over quick the the)
\end{verbatim}
\section{Functions for Building and Searching A-Lists}

\de{assoc}{(assoc U:any V:a-list): {pair, nil}}{expr}
{    If  U  occurs as the car portion of an element of the a-list
    V, the pair in which  U occurred is returned, otherwise  nil
    is  returned.    The  function  equal  is  used  to test for
    equality.  As illustrated below, it is  possible  to  update
    the table that was the second argument to assoc by using the
    function rplacd on the result.
}
\begin{verbatim}
    (de assoc (u v)
      (cond ((not (pairp v)) nil)
            ((and (pairp (car v)) (equal u (caar v))) (car v))
            (t (assoc u (cdr v)))))

    1 lisp> (setq table '((oranges . 4) (apples . 2)))
    ((oranges . 4) (apples . 2))

    2 lisp> (rplacd (assoc 'apples table) 0)
    (apples . 0)

    3 lisp> table
    ((oranges . 4) (apples . 0))
\end{verbatim}

\de{atsoc}{(atsoc R1:any R2:any): any}{expr}
{    Similar to assoc except that eq is used to make comparisons.
}

\de{ass}{(ass F:function U:any V:a-list): {pair, nil}}{expr}
{    Ass  is  a  generalized assoc function.  F is the comparison
    function.
}

\de{sassoc}{(sassoc U:any V:a-list FN:function): any}{expr}
{    Searches the a-list V for  an occurrence of U.  If  U is not
    in the a-list, the evaluation of function  FN  is  returned.
    Note that FN should be a function with no formal parameters.
}
\begin{verbatim}
    (de sassoc (u v fn)
      (cond ((not (pairp v)) (apply fn nil))
            ((and (pairp (car v)) (equal u (caar v))) (car v))
            (t (sassoc u (cdr v) fn))))
\end{verbatim}

\de{pair}{(pair U:list V:list): a-list}{expr}
{    U  and  V are lists which  must have an identical number  of
    elements.  If not, an error occurs.  Returned is a  list  in
    which    each  element is a pair, the  car of the pair being
    from U  and  the  cdr  being  the   corresponding    element
    from V.
}
\begin{verbatim}
    (de pair (u v)
      (cond ((and (pairp u) (pairp v))
             (cons (cons (car u) (car v)) (pair (cdr u) (cdr v))))
            ((or (pairp u) (pairp v))
             (length-error))
            (t nil)))
\end{verbatim}
\section{Substitutions}


\de{subst}{(subst U:any V:any W:any): any}{expr}
{    Returns the  result of  substituting U  for all  occurrences
    of  V  in W.   Copies all of  W which is not replaced  by U.
    The  test  used   is  equal.
}
\begin{verbatim}
    (de subst (u v w)
      (cond ((null w) nil)
            ((equal v w) u)
            ((not (pairp w)) w)
            (t (cons (subst u v (car w)) (subst u v (cdr w))))))
\end{verbatim}

\de{substip}{(substip U:any V:any W:any): any}{expr}
{    Destructive subst.
}

\de{sublis}{(sublis X:a-list Y:any): any}{expr}
{    This performs  a  series  of  substs  in  parallel.      The
    value  returned  is  the  result of  substituting the cdr of
    each  element of the a-list  X for every  occurrence of  the
    car  part    of  that element in Y.  Sublis is not quite the
    correct function  for  arbitrary  code  substitutions.    As
    illustrated  below, substitutions may enter places you might
    wish they did not.
}
\begin{verbatim}
    (de sublis (x y)
      (if (not (pairp x))
        y
        (let ((u (assoc y x)))
          (cond ((pairp u) (cdr u))
                ((not (pairp y)) y)
                (t (cons (sublis x (car y)) (sublis x (cdr y))))))))

    1 lisp> (sublis '((x . 100)) '(list 'x 'is x))
    (list (quote 100) (quote is) 100)
\end{verbatim}

\de{subla}{(subla U:a-list V:any): any}{expr}
{    Eq version of sublis;  replaces atoms only.
}
