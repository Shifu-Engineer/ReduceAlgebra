\section{Numbers and Arithmetic Functions}

Most  of  the  arithmetic  functions  in PSL expect numbers as
arguments.    In all  cases an error occurs  if the parameter to
an arithmetic function is not a number:

\begin{verbatim}
*****  Non-numeric argument in arithmetic
\end{verbatim}
Exceptions to the rule are noted.\\

\noindent
The underlying machine arithmetic requires  parameters  to  be
either all integers or all floats.  If a function receives mixed
types  of  arguments,  integers  are  converted to floats before
arithmetic operations are performed.  The range of numbers which
can  be  represented  by  an  integer  is  different  than  that
represented   by  a  float.    Because  of  this  difference,  a
conversion is not always possible; an  unsuccessful  attempt  to
convert may cause an error to be signalled.

\noindent
The {\bf mathlib} package contains some useful mathematical functions. 

\subsection{Big Integers}

  Loading  the  ZBIG \footnote{ZIB version of big integers, the
optimal version for the particular architecture }
  module  redefines  the   basic   arithmetic
operations,   including   the   logical  operations,  to  permit
arbitrary precision (or "bignum") integer operations.

\subsection{Conversion Between Integers and Floats}

  The conversions mentioned above can be done explicitly by  the
following  functions.   Other functions which alter types can be
found in Section 2.3.


\de{fix}{(fix U:number): integer}{expr}
{    Returns the integer which corresponds to the truncated value
    of U.  The result of conversion must retain all  significant
    portions of U.  If U is an integer it is returned unchanged.
}
\begin{verbatim}
    1 lisp> (fix 2.1)
    2
    2 lisp> (fix -2.1)
    -2
\end{verbatim}

\de{float}{(float U:number): float}{expr}
{    The  float  corresponding  to the value of the argument U is
    returned.  Some  of  the  least  significant  digits  of  an
    integer  may be lost due to the implementation of float.  If
    U a float then it will be returned unchanged.  If U  is  too
    large to represent in float an error occurs.
}
\begin{verbatim}
    ***** Argument to FLOAT is too large
\end{verbatim}
\subsection{Arithmetic Operators}

 This section describes arithmetic functions in the library
module {\bf numeric-ops}. The names of these functions are
based upon mathematical notation.

\noindent
There  is  a  switch  called  {\bf fast-integers} 
\index{*fast-integers}whose value has an
effect on the compilation of forms which contain applications of
the functions described here.  The  documentation  assumes  that
the  switch  fast-integers  is nil.  When this switch is non-nil
the  compiler  will  generate  code  which  is  very  efficient.
However,    it  is  assumed  that  arguments and results will be
integers in the inum range.  If this assumption is violated then
at best your code will not generate  correct  results,  you  may
actually  damage  the  PSL system. 

\subsubsection{Common LISP operators}


\de{=}{(= X:number Y:number): boolean}{expr}
{    Numeric Equal.  True if the two arguments are numbers of the
    same type and same value.  Unlike the Common LISP  operator,
    no  type  coercion  is done, no error is signalled if one or
    both arguments are non-numeric, and only two  arguments  are
    permitted.    Instead,  it  is  merely incorrect to supply a
    non-numeric argument.
}

\de{/=}{(/= X:number Y:number): boolean}{expr}
{    Numeric Not Equal.  Nil if X and Y are numbers of equal type
    and value; t if X and Y  are  numbers  of  unequal  type  or
    value.  It is incorrect to supply a non-numeric argument.
}

\de{$<$}{($<$ X:number Y:number): boolean}{expr}
{    Numeric  Less Than.  True if X is less than Y, regardless of
    type.  An error is  signalled  if  either  argument  is  not
    numeric.  }

\de{$>$}{($>$ X:number Y:number): boolean}{expr}
{    Numeric  Greater  Than.    True  if  X  is  greater  than Y,
    regardless of  type.    An  error  is  signalled  if  either
    argument is not numeric.
}

\de{$<=$}{($<=$ X:number Y:number): boolean}{expr}
{    Numeric Less Than or Equal.  True if X is less than or equal
    to  Y, regardless of numeric type.  An error is signalled if
    either argument is not numeric.
}

\de{$>=$}{($>=$ X:number Y:number): boolean}{expr}
{    Numeric Greater Than or Equal.  True if X is greater than or
    equal to Y,  regardless  of  numeric  type.    An  error  is
    signalled if either argument is not numeric.
}

\de{+}{(+ [N:number]): number}{macro}
{    Numeric  Addition.  The value returned is the sum of all the
    arguments.  The arguments may be of any numeric  type.    An
    error  is  signalled  if  any  argument  is not numeric.  If
    supplied no arguments, the value is 0.
}

\de{--}{(-- N:number [N:number]): number}{macro}
{    Numeric Minus  or  Subtraction.    If  given  one  argument,
    returns  the  negative of that argument.  If given more than
    one argument, returns the result of successively subtracting
    succeeding arguments from the first argument.    Signals  an
    error  if  no  arguments  are supplied or if any argument is
    non-numeric.
}

\de{$\ast$}{($\ast$ [N:number]): number}{macro}
{    Numeric Multiplication.  The value returned is  the  product
    of  all  the arguments.  The arguments may be of any numeric
    type.  An error is signalled if any argument is not numeric.
    If supplied no arguments, the value is 1.
}

\de{/}{(/ N:number [N:number]): number}{macro}
{    Numeric Reciprocal or Division.    If  given  one  argument,
    returns the reciprocal of that argument.  If given more than
    one  argument,  returns  the result of successively dividing
    succeeding arguments from the first argument.    Signals  an
    error  if  no  arguments  are supplied or if any argument is
    non-numeric.
}
\subsubsection{Additional Operators}


\de{$\sim =$}{($\sim =$ X:number Y:number): number}{expr}
{    Numeric Not Equal.  Same as /=.
}

\de{//}{(// X:integer Y:integer): integer}{expr}
{    Integer Remainder.  Same as remainder.
}

\de{$\sim$}{($\sim$ X:integer): integer}{expr}
{    Integer Bitwise Logical Not.  Same as lnot.
}

\de{\&}{(\& X:integer Y:integer): integer}{expr}
{    Integer Bitwise Logical And.  Same as land.
}

\de{$\mid$}{($|$ X:integer Y:integer): integer}{expr}
{    Integer Bitwise Logical Or.  Same as lor.
}

\de{$\wedge$}{($\wedge$ X:integer Y:integer): integer}{expr}
{    Integer Bitwise Logical Xor.  Same as lxor.
}

\de{$<<$}{($<<$ X:integer Y:integer): integer}{expr}
{    Integer Bitwise Logical Left Shift.  Same as lshift.
}

\de{$>>$}{($>>$ X:integer Y:integer): integer}{expr}
{    Integer Bitwise Logical Right Shift. Same as (lshift  X
    (minus Y)).
}
\subsection{Arithmetic Functions}

  The  functions  described  below handle arithmetic operations.
Please note  the  remarks  at  the  beginning  of  this  Chapter
regarding the mixing of argument types.


\de{abs}{(abs U:number): number}{expr}
{    Returns the absolute value of its argument.
}

\de{add1}{(add1 U:number): number}{expr}
{    Returns  the value of U plus 1; the returned value is of the
    same type as U (integer or float).
}

\de{decr}{(decr U:form [Xi:number]): number}{macro}
{    This function is defined in the {\bf useful} module. With only
    one argument, this is equivalent to
}
\begin{verbatim}
    (setf u (sub1 u))
\end{verbatim}
    With multiple arguments, it is equivalent to

\begin{verbatim}
    (setf u (difference u (plus x1 ... xn)))
\end{verbatim}
\begin{verbatim}
    1 lisp> (setq y '(1 5 7))
    (1 5 7)
    2 lisp> (decr (car y))
    0
    3 lisp> y
    (0 5 7)
    4 lisp> (decr (cadr y) 3 4)
    -2
    5 lisp> y
    (0 -2 7)
\end{verbatim}
\de{difference}{(difference U:number V:number): number}{expr}
{    The value of U - V is returned.
}

\de{divide}{(divide U:number V:number): pair}{expr}
{    The  pair  (quotient  .  remainder)  is  returned, as if the
    quotient part was computed by the quotient function and  the
    remainder  by  the  remainder  function.  An error occurs if
    division by zero is attempted:
}
\begin{verbatim}
    ***** Attempt to divide by 0 in Divide
\end{verbatim}

\de{expt}{(expt U:number V:integer): number}{expr}
{    Returns U raised to the V power.  A float U  to  an  integer
    power   V  does  not  have  V  changed  to  a  float  before
    exponentiation.
}

\de{incr}{(incr U:form [Xi:number]): number}{macro}
{    Part of the {\bf useful} package. With only one argument,
	this is equivalent to
}
\begin{verbatim}
    (setf u (add1 u))
\end{verbatim}
    With multiple arguments it is equivalent to
\begin{verbatim}
    (setf u (plus u x1 ... xn))
\end{verbatim}
\de{minus}{(minus U:number): number}{expr}
{    Returns -U.
}

\de{plus}{(plus [U:number]): number}{macro}
{    Forms the sum of all its arguments.  Plus may be called with
    only  one  argument.   In this case it returns its argument.
    If plus is called with no arguments, it returns zero.
}

\de{plus2}{(plus2 U:number V:number): number}{expr}
{    Returns the sum of U and V.  }

\de{quotient}{(quotient U:number V:number): number}{expr}
{    The quotient of U divided by V is returned.  Division of two
    positive or two negative integers is conventional.  If  both
    U  and  V  are integers and exactly one of them is negative,
    the value  returned  is  truncated  toward  0.    If  either
    argument  is  a  float,  a  float is returned which is exact
    within the implemented precision of floats.  An error occurs
    if division by zero is attempted:
}
\begin{verbatim}
    ***** Attempt to divide by 0 in QUOTIENT
\end{verbatim}
\de{recip}{(recip U:number): float}{expr}
{    Recip converts U to a float if necessary, and then finds the
    inverse using the function quotient.  }

\de{remainder}{(remainder U:integer V:integer): integer}{expr}
{    (- U (* V (fix (/ U (float V))))) }
\begin{verbatim}
    (remainder 13 4)    =  1
    (remainder -13 4)   = -1
    (remainder 13 -4)   =  1
    (remainder -13 -4)  = -1
\end{verbatim}
\de{sub1}{(sub1 U:number): number}{expr}
{    Returns the value of U minus 1.  If U is a float, the  value
    returned is U minus 1.0.  }

\de{times}{(times [U:number]): number}{macro}
{    Returns  the  product  of  all  its arguments.  Times may be
    called with only one argument.  In this case it returns  the
    value  of  its  argument.    If  times  is  called  with  no
    arguments, it returns 1.  }

\de{times2}{(times2 U:number V:number): number}{expr}
{    Returns the product of U and V.
}
\subsection{Functions for Numeric Comparison}

  The following functions compare the values of their arguments.
Functions which test for equality and inequality are  documented
in Section \ref{sec:testing-equality}.


\de{geq}{(geq U:any V:any): boolean}{expr}
{    Equivalent to ($>=$ U V).  }

\de{greaterp}{(greaterp U:number V:number): boolean}{expr}
{    Equivalent to ($>$ U V).  }

\de{leq}{(leq U:number V:number): boolean}{expr}
{    Equivalent to ($<=$ U V).  }

\de{lessp}{(lessp U:number V:number): boolean}{expr}
{    Equivalent to ($<$ U V).  }

\de{max}{(max [U:number]): number}{macro}
{    Returns  the  largest  of the values in U (numeric maximum).
    If two or more  values are the same, the first is returned.  }

\de{max2}{(max2 U:number V:number): number}{expr}
{    Returns the larger of U and V.  If U and V are of  the  same
    value, U is returned (U and  V might be of different types).  }

\de{min}{(min [U:number]): number}{macro}
{    Returns  the smallest (numeric minimum), of the values in U.
    If two or more  values are the same, the first of  these  is
    returned.  }

\de{min2}{(min2 U:number V:number): number}{expr}
{    Returns  the  smaller of its arguments.  If U and V are  the
    same value,  U is  returned  (U  and    V  might    be    of
    different types).  }

\de{minusp}{(minusp U:any): boolean}{expr}
{    Returns  t  if  U  is  a number and less than 0.  The return
    value is nil if U is not a number, or if  U  is  a  positive
    number.  }

\de{onep}{(onep U:any): boolean}{expr}
{    Returns  t if  U is a  number and  has the value  1 or  1.0.
    Returns nil otherwise.  }

\de{zerop}{(zerop U:any): boolean}{expr}
{    Returns t if  U is a  number and  has the value  0 or   0.0.
    Returns nil otherwise.  }

\subsection{Bit Operations}

  The  functions described in this section operate on the binary
representation of the integers given as arguments.  The returned
value is an integer.


\de{land}{(land U:integer V:integer): integer}{expr}
{    Bitwise  or  logical  and.    Each  bit  of  the  result  is
    independently  determined from the corresponding bits of the
    operands.  }

\de{lor}{(lor U:integer V:integer): integer}{expr}
{    Bitwise  or  logical  or.    Each  bit  of  the  result   is
    independently  determined  from  corresponding  bits  of the
    operands.  This is an inclusive or, the value of (lor  1  1)
    is 1. }

\de{lnot}{(lnot U:integer): integer}{expr}
{    Logical  not.    Defined  as  (-  -U 1) so that it works for
    bignums as if they were 2's complement.  }

\de{lxor}{(lxor U:integer V:integer): integer}{expr}
{    Bitwise or logical exclusive or, the value of (lxor 1 1)  is
    0.   Each bit of the result is independently determined from
    the corresponding bits of the operands.  }

\de{lshift}{(lshift N:integer K:integer): integer}{expr}
{    Shifts N to the left by K bits.  The effect  is  similar  to
    multiplying  by  2  to  the  K  power.   Negative values are
    acceptable for K, and cause a  right  shift  (in  the  usual
    manner).   Lshift is a logical shift, so right shifts do not
    resemble division by a power of 2.  }

\subsection{Various Mathematical Functions}

The optionally loadable {\bf mathlib} module  defines  several
commonly used mathematical functions.  Some effort has been made
to  be  compatible with Common Lisp.  When reading the examples,
note that the precision of the results  depend  on  the  machine
being used.


\de{ceiling}{(ceiling X:number): integer}{expr}
{    Returns  the  smallest  integer  greater than or equal to X.
    For example: }
\begin{verbatim}
    1 lisp> (ceiling 2.1)
    3
    2 lisp> (ceiling -2.1)
    -2
\end{verbatim}
\de{floor}{(floor X:number): integer}{expr}
{    Returns the largest integer less than or equal to X.   (Note
    that this differs from the fix function.)
}
\begin{verbatim}
    1 lisp> (floor 2.1)
    2
    2 lisp> (floor -2.1)
    -3
    3 lisp> (fix -2.1)
    -2
\end{verbatim}
\de{round}{(round X:number): integer}{expr}
{    Returns the nearest integer to X.  If the fractional part of
    X is 0.5 then the smaller integer is returned.
}
\begin{verbatim}
    (de round (x)
      (if (fixp x) x (floor (plus x 0.5))))
\end{verbatim}
\begin{verbatim}
    1 lisp> (round 2.5)
    3
    2 lisp> (round -2.5)
    -2
\end{verbatim}
\de{transfersign}{(transfersign S:number VAL:number): number}{expr}
{    Transfers the sign of S to VAL.
}
\begin{verbatim}
    (de transfersign (s val)
      (if (>= s 0) (abs val) (minus (abs val))))
\end{verbatim}
\de{mod}{(mod M:integer N:integer): integer}{expr}
{    (- U (* V (floor (/ U (float V)))))
}
\begin{verbatim}
    (mod 13 4)    =  1
    (mod -13 4)   =  3
    (mod 13 -4)   = -3
    (mod -13 -4)  = -1
\end{verbatim}
\de{degreestoradians}{(degreestoradians X:number): number}{expr}
{    Returns an angle in radians given an angle in degrees.
}
\begin{verbatim}
    1 lisp> (degreestoradians 180)
    3.1415926
\end{verbatim}
\de{radianstodegrees}{(radianstodegrees X:number): number}{expr}
{    Returns an angle in degrees given an angle in radians.
}
\begin{verbatim}
    1 lisp> (radianstodegrees 3.1415926)
    180.0
\end{verbatim}
\de{radianstodms}{(radianstodms X:number): list}{expr}
{    Given  an  angle  X  in  radians,  returns  a  list of three
    integers, which represent the degrees, minutes, and seconds.  }
\begin{verbatim}
    1 lisp> (radianstodms 1.0)
    (57 17 45)
\end{verbatim}
\de{dmstoradians}{(dmstoradians Degs:number Mins:number Secs:number): number}{expr}
{    Returns  an  angle  in  radians,   given   three   arguments
    representing an angle in degrees minutes and seconds.  }
\begin{verbatim}
    1 lisp> (dmstoradians 57 17 45)
    1.0000009
    2 lisp> (dmstoradians 180 0 0)
    3.1415926
\end{verbatim}
\de{degreestodms}{(degreestodms X:number): list}{expr}
{    Given  an  angle  X  in  degrees,  returns  a  list of three
    integers giving the angle in (Degrees  Minutes  Seconds).  }

\de{dmstodegrees}{(dmstodegrees Degs:number Mins:number Secs:number): number}{expr}
{    Returns  an  angle  in  degrees,   given   three   arguments
    representing an angle in degrees minutes and seconds.  }

\de{sin}{(sin X:number): number}{expr}
{    Returns the sine of X, an angle in radians.  }

\de{sind}{(sind X:number): number}{expr}
{    Returns the sine of X, an angle in degrees.  }

\de{cos}{(cos X:number): number}{expr}
{    Returns the cosine of X, an angle in radians.  }

\de{cosd}{(cosd X:number): number}{expr}
{    Returns the cosine of X, an angle in degrees.  }

\de{tan}{(tan X:number): number}{expr}
{    Returns the tangent of X, an angle in radians.  }

\de{tand}{(tand X:number): number}{expr}
{    Returns the tangent of X, an angle in degrees.  }

\de{cot}{(cot X:number): number}{expr}
{    Returns the cotangent of X, an angle in radians.  }

\de{cotd}{(cotd X:number): number}{expr}
{    Returns the cotangent of X, an angle in degrees.  }

\de{sec}{(sec X:number): number}{expr}
{    Returns the secant of X, an angle in radians.  }

\de{secd}{(secd X:number): number}{expr}
{    Returns the secant of X, an angle in degrees.  }

\de{csc}{(csc X:number): number}{expr}
{    Returns the cosecant of X, an angle in radians.  }

\de{cscd}{(cscd X:number): number}{expr}
{    Returns the cosecant of X, an angle in degrees.  }

\de{asin}{(asin X:number): number}{expr}
{    Returns the arc sine, as an angle in radians, of X.  }
\begin{verbatim}
    (eqn (sin (asin X)) X)
\end{verbatim}
\de{asind}{(asind X:number): number}{expr}
{    Returns the arc sine, as an angle in degrees, of X.  }

\de{acos}{(acos X:number): number}{expr}
{    Returns the arc cosine, as an angle in radians, of X.  }
\begin{verbatim}
    (eqn (cos (acos X)) X)
\end{verbatim}
\de{acosd}{(acosd X:number): number}{expr}
{    Returns the arc cosine, as an angle in degrees, of X.  }

\de{atan}{(atan X:number): number}{expr}
{    Returns the arc tangent, as an angle in radians, of X.  }
\begin{verbatim}
    (eqn (tan (atan X)) X)
\end{verbatim}
\de{atand}{(atand X:number): number}{expr}
{    Returns the arc tangent, as an angle in degrees, of X.  }

\de{atan2}{(atan2 Y:number X:number): number}{expr}
{    Returns  an  angle  in  radians  corresponding  to the angle
    between the X axis and the vector [X Y].  Note that Y is the
    first argument.
}
\begin{verbatim}
    1 lisp> (atan2 0 -1)
    3.1415927
\end{verbatim}
\de{atan2d}{(atan2d Y:number X:number): number}{expr}
{    Returns an angle  in  degrees  corresponding  to  the  angle
    between the X axis and the vector [X Y].
}
\begin{verbatim}
    1 lisp> (atan2d -1 1)
    315.0
\end{verbatim}
\de{acot}{(acot X:number): number}{expr}
{    Returns the arc cotangent, as an angle in radians, of X.  }
\begin{verbatim}
    (eqn (cot (acot X)) X)
\end{verbatim}

\de{acotd}{(acotd X:number): number}{expr}
{    Returns the arc cotangent, as an angle in degrees, of X.  }

\de{asec}{(asec X:number): number}{expr}
{    Returns the arc secant, as an angle in radians, of X.  }
\begin{verbatim}
    (eqn (sec (asec X)) X)
\end{verbatim}
\de{asecd}{(asecd X:number): number}{expr}
{    Returns the arc secant, as an angle in degrees, of X.  }

\de{acsc}{(acsc X:number): number}{expr}
{    Returns the arc cosecant, as an angle in radians, of X.  }
\begin{verbatim}
    (eqn (csc (acsc X)) X)
\end{verbatim}
\de{acscd}{(acscd X:number): number}{expr}
{    Returns the arc cosecant, as an angle in degrees, of X.  }

\de{sqrt}{(sqrt X:number): number}{expr}
{    Returns the square root of X.  }

\de{exp}{(exp X:number): number}{expr}
{    Returns the exponential of X.  }

\de{log}{(log X:number): number}{expr}
{    Returns the natural (base e) logarithm of X.  note that (log
    (log (exp X)) is equal to X.  }

\de{log2}{(log2 X:number): number}{expr}
{    Returns the base two logarithm of X.  }

\de{log10}{(log10 X:number): number}{expr}
{    Returns the base ten logarithm of X.  }

\de{random}{(random N:integer): integer}{expr}
{    Returns  a  pseudo-random number uniformly selected from the
    range 0 ... (sub1 N).  }

The random number generator uses a linear congruential method.    

\variable{randomseed}{Initially: set from time}{global}
{
To  get a reproducible sequence of random numbers
    you should assign one (or some other small  number)  to  the
    fluid variable randomseed.  }

\de{factorial}{(factorial N:integer): integer}{expr}
{    The factorial function is defined as follows
}
\begin{verbatim}
      1. The factorial of 0 is 1.
\end{verbatim}
\begin{verbatim}
      2. The factorial of N is N times the factorial of N-1.
\end{verbatim}
\begin{verbatim}
    (de factorial (n)
      (if (zerop n) 1 (* n (factorial (sub1 n)))))
\end{verbatim}
