\chapter{Miscellaneous Utilities}

\section{Simulating a Stack}

  The  following  macros  are  in  the USEFUL package.  They are
convenient for adding and deleting things from  the  head  of  a
list.


\de{push}{(push ITM:any STK:list): any}{macro}
{
(push item stack) is equivalent to\\
(setf stack (cons item stack))
}

\de{pop}{(pop STK:list): any}{macro}
{
(pop stack) does (setf stack (cdr stack)) and returns the item 
popped off stack. An additional argument may be supplied to Pop, in which 
case it is a variable which is SetQ'd to the popped value.
}

\section{Ring Buffers}

  The code-address-to-symbol function takes an integer argument,
and attempts to find out what function that address is  in.   If
the  address  does not map to any function, the name of the most
recently loaded function is returned.

\de{code-address-to-symbol}{(code-address-to-symbol ADDRESS:integer): id}{expr}
{}
  for example:

\begin{verbatim}
(code-address-to-symbol 16#38393)
\end{verbatim}
  This function is available by evaluating (load addr2id).

\section{Word Vector Operations}

  These functions are defined in \$pu/vector-fix.sl
  resp the vector-fix module.


\de{mkwords}{(mkwords N:integer): vector}{expr}
{    Allocates a vector of N words with all elements  initialized
    to zero.
}

\de{truncatevector}{(truncatevector V:vector I:integer): vector}{expr}
{    Truncates V to I elements.
}

\de{truncatewords}{(truncatewords V:words I:integer): vector}{expr}
{    Truncates V to I elements.
}

\de{getwords}{(getwords WRD:words I:integer): any}{expr}
{    Retrieves the I'th entry of WRD.
}

\de{putwords}{(putwords WRD:words I:integer VAL:any): any}{expr}
{    Store VAL at I'th position of WRD.
}

\de{upbw}{(upbw V:words): integer}{expr}
{    Returns the upper limit of words V.
}
