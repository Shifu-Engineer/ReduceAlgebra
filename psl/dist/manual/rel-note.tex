\documentstyle[11pt]{report}

\begin{document}

\title{PSL 3.4 Release Notes}

\author{Herbert Melenk, Winfried Neun}

\section{Introduction}

This document was derived to a large extent from 
"Unix PSL 3.4 Release Notes" by Shebs and ? 
(Opnote # 1987) Utah PASS Project, University of Utah.

Portable Standard Lisp (PSL) 3.4 runs on a wide variety of platform,e.g.

DEC Vax (under Ultrix or VMS), 
HP Series 9000 (300/400 and 700/800) HP-UX, 
SPARC and SUN 3 systems (Sun OS or Solaris), 
MIPS based systems (e.g. DECstation, CDC 4000 series, SGI Iris, Stardent), 
IBM RS/6000, 
IBM (and compatible) mainframe computers under MVS/XA and VM/XA, 
Fujitsu (Siemens) S400 systems under UXP/M
Intel 386 or 486 based systems under MS-DOS, OS/2, Windows/NT and various
Unixes
MC 88000 systems (e.g. Data General AViiON)
Cray X/Y computers running UNICOS,
NeXT systems
DEC Alpha systems under OSF/1 and OpenVMS
Convex systems  
Intel i860 based systems

PSL as a dialect of LISP is mainly used as a delivery vehicle for 
applications like e.g. the computer algebra system REDUCE.

The COMMON LISP subset implementation 'PCLS' from the University of Utah 
is currently not supported.

Although this document is written for a Unix environment, it applies to 
the VMS, MVS and DOS versions too, since the latter PSL versions
use the same conventions for the operating system interface routines.
The handling of the distribution tape and starting PSL is of course 
system specific and different for the various platforms 
and distribution media too. Please read the
note "Installation of PSL 3.4" which accompanies these Release notes
for the necessary steps to undump the distribution tapes (diskettes).

Most people running PSL will not be interested in all of
the files, and probably will not want to have them all on line.  Only
the executable programs, {\tt psl-names} and the files residing on the
sub-directory {\tt dist/lap} (described below) must be on line to use
PSL.  Any other files may be deleted after the system is installed.
The full distribution tape should be retained if you plan to delete
files after installation.  

\section{DISCLAIMER}

Please be aware that this is a research system, not a program product, 
and some of the files and documentation are not quite complete; 
we may also have forgotten some files, or sent incorrect versions.  
We are releasing this version to you at this time to enhance our 
collaborative research.

For these reasons please:

\begin{itemize}

\item Make a note of ANY problems, concerns, suggestions you have, 
and send this information to us to aid in improving the system and 
this distribution mechanism.

\item Please do not REDISTRIBUTE any of these files, listings or 
machine readable form to anyone.

\end{itemize}

\section{CONTENTS OF THE TAPE}

Attached to this note is a listing of the files on the tape, produced
as the tape was built.  There is no list of files for diskette
distributions though.
The tape (or diskettes) contains the following files and directories
(directories are indicated by {\tt */}). Some of the directories
might be empty, but are included for compatibility with the full
system. The c-shell variables (\$ variables) referencing these
directories are indicated by {\tt [\$*]}.  Some of the directories
mention {\tt MACHINE}, which refers to the specific machine.  

\vskip 12pt
\noindent
EXECUTABLES: {\tt \$proot/bin/} (the bin directory is {\tt \$psys})

\begin{description}

\item[{\tt bare-psl}:] The smallest executable used mostly for
building other executables. 

\item[{\tt psl}:] {\tt bare-psl} with several modules loaded (see your
installer or look at the {\tt psl} variable {\tt options*} to see which
modules are loaded.

\item[{\tt pslcomp}:] {\tt psl} with the compiler loaded.

\end{description}

\noindent
SHELL SCRIPTS: {\tt \$proot/dist/} 

\begin{description}


\item[{\tt psl-names}:]
A file to be {\tt source}ed to define environment variables (such as
{\tt \$pu} for {\tt util/}, {\tt \$pl} for {\tt lap/}, etc.) to aid in
accessing the various PSL sub-directories.
On a VMS system the equivalent file pslnames.com must be run a procedure 
e.g. started in your login.ini file.
\end{description}

\item[{\tt oload}:]
A C shell script to to permit {\tt .o} files to be dynamically loaded.  
(This does not exist on all systems. We strongly recommend NOT to use
oload.)

\noindent
DIRECTORIES: {\tt \$proot/dist/} 

\begin{description}

\item[{\tt lap/}:]
PSL binary ({\tt *.b}) files.  Contains binary files needed to
rebuild the executables plus binary files for optional loadable
modules.  PSL binary files are sometimes referred to as FASL (fast
loading) files.  [{\tt \$pl}]

\item[{\tt kernel/}:]
Machine-independent kernel sources. [{\tt \$pk}]

\item[{\tt kernel/MACHINE/}:]
Machine-specific kernel sources and object files. [{\tt \$pxk}]

\item[{\tt nonkernel/}:]
Files which are part of a standard {\tt bare-psl}, but not
compiled in the kernel build. [{\tt \$pnk}]

\item[{\tt nonkernel/MACHINE/}:]
Machine-specific files which are part of a standard {\tt bare-psl}, but
not compiled in the kernel build. [{\tt \$pxnk}]

\item[{\tt comp/}:]
Machine-independent compiler sources. [{\tt \$pc}]

\item[{\tt comp/MACHINE/}:]
Machine-specific compiler sources. [{\tt \$pxc}]

\item[{\tt util/}:]
Sources for machine-independent utilities. [{\tt \$pu}]

\item[{\tt util/MACHINE/}:]
Sources for machine-specific utilities. [{\tt \$pxu}]

\item[{\tt distrib/}:]
Scripts to help build and maintain the machine-independent parts of
PSL. [{\tt \$pdist}]

\item[{\tt distrib/MACHINE/}:]
Scripts to help build and maintain the machine-specific parts of PSL.
[{\tt \$pdist}]

\item[{\tt lpt/}:]
The PSL manual in printable form (has overprinting and underlining).

\item[{\tt doc/}:]
Versions of this document and the addendum to the PSL manual.  We keep
the source texts and printable versions in postscript ({\tt .ps}) and
laserjet format ({\tt .jep}).

\end{description}

\bigskip


\section{A SUGGESTION TO SYSTEM MANAGERS}

PSL has a large heap (of which only half may be used at any
given time if the copying garbage collector is being used, 
rather than the compacting collector).  
The default size of a running PSL core image is roughly 4.5 megabytes.  
Because of this, it is recommended
that your system be configured with at least 32 megabytes of swap
space.  If this is too great a burden on your system, see the section
on rebuilding the kernel for instructions on creating a version with 
a smaller heap.

If a large number of people wish to run PSL simultaneously, either
the swap space will have to be increased, or a smaller PSL should be built.

\section{READING THE TAPE}

This information depends very much on your specific combination of
operating system and distribution media. Please find a separate note 
on undumping the distirbution tape attached to this note.

Unless absolutely essential, we recommend that you use exactly the same
directory structure as we do (relative to the root of the PSL tree),
even though the system is set up to be installed according to local
conventions.

\section{INSTALLING PSL}

\subsection{Update the Root Location}

After reading the tape, you need to customize the files to reflect the
root location of PSL.  This can be accomplished with
{\tt newroot.csh}.  It takes one argument which is the fully rooted
path of the PSL root directory.  {\tt newroot.csh} will replace the root
in {\tt psl-names} and all of the makefiles.  For example,

\begin{verbatim}
        ./dist/distrib/newroot.csh /v/misc/psl
\end{verbatim}

\subsection{Adding the PSL Environment}

Your {\tt .cshrc} file should be edited to {\tt source} the new
{\tt psl-names}, and set up your search path.  If you have previously
installed a version of PSL, you should at this time make sure that
your {\tt .cshrc} refers to this NEW {\tt psl-names}.

To do this, insert the lines 
\begin{verbatim}
        source $~psl/dist/psl-names          # or equivalent
        set path=(. $psys /usr/bin /bin ...) # plus any others
\end{verbatim}
into your {\tt .cshrc} file at an appropriate point.

Now do
\begin{verbatim}
        source ~/.cshrc
\end{verbatim}
to set all paths to that of the new PSL.

\subsection{Make links to executables}

Do:

\begin{verbatim}
        make install
\end{verbatim}

to make symbolic links from your standard system directory to the
executables in {\tt \$psys}.  The default is {\tt /usr/site/bin/}.
This makes it unnecessary for users of PSL to put {\tt \$psys} in the
search path variables of their {\tt .login} files before they can use
PSL.  (For users on System V Unix or Apollos, this step does not work.
Users must put {\tt \$psys} on their search path.)  This default is
defined in {\tt \$psl/Makefile} as the {\tt DESTDIR} variable.  PSL
users must still {\tt source psl-names} to use PSL if they intend to
load modules from the {\tt lap/} directory or use {\tt oload} to load
{\tt .o files}.

\subsection{Delete unneeded directories}

Only the {\tt lap/} subdirectory and the {\tt \$psl/psl-names} and
{\tt \$xxx-names} files are  necessary for running PSL
programs.  If you do not wish to keep the source files on-line, you
may delete any other files after successfully testing PSL, and
rebuilding any modules or binaries that you need to change.  But DO
retain the distribution tape!  Many users of PSL find that on-line
sources are useful tutorial and reference material, and we once again
recommend that you retain the file structure as distributed.

\subsection{Announce the system}

You should only announce the system if there are users who are going
to use PSL directly. Here is how to do it:

Send out a message to all those interested in using PSL.  The file
{\tt \$psl/doc/bboard.msg} is a suggested start.  In particular, make
sure that you inform people of the exact path needed to place a
{\tt source .../psl-names} in their {\tt .login} file, if they desire, since
{\tt psl-names} facilitates reference to the PSL source directories.

Edit as you see fit, but please REMIND people not to re-distribute the PSL
system and sources. Note the [.....] blocks in which you are to insert
the name of some person responsible to answer questions.

You may also want to set the directory protection and limit access only to
those that you feel should have access at this time.

\section{REBUILDING PSL LOADABLE MODULES}

All of the utilities, are kept as binary FASL files (with extensions
{\tt .b}) on the {\tt lap/} subdirectory.  After modifying a utility source
file just connect to {\tt \$pu} and do {\tt make}.  This will
automatically detect which files need to be recompiled.  Once
recompiled it will place the binary version(s) on {\tt \$pl}, the
directory from which the executables obtain loadable modules.

\subsection{Using Pslcomp}

{\tt pslcomp} can be used to compile a file, regardless of whether it is
listed in one of the makefiles.  Connect to the directory where the
sources for file {\tt XXX} reside.  Do
\begin{verbatim}
       pslcomp  XXX
\end{verbatim}
This will compile the file {\tt XXX}, placing the binary (.b) file is
placed in the directory to which you are connected.  {\tt pslcomp}
knows about {\tt .sl} and {\tt .l} extensions.

\section{REBUILDING THE SYSTEM}

It is not necessary to rebuild the system after reading the tape.  The
executables should run on your system.  PSL determines the location of
its tree (necessary for finding the {\tt lap/} directory, the {\tt
oload} script, etc.) by checking the environment variables.  That is
why a user must source {\tt psl-names} to use PSL.  If you do need to
rebuild the system then this section explains the necessary steps.

To rebuild the system, you must have read the entire contents of the
tape.  If you have then change your
working directory to {\tt \$psl}, and do:
\begin{verbatim}
        make
\end{verbatim}

This will connect to several subdirectories of {\tt \$psl} and run
make in those directories.  These ``recursive'' makes will:

\begin{itemize}

\item Recompile C system interface routines used in the kernel.

\item Recompile any out-of-date PSL system source files, putting their
binary versions on {\tt \$pl} (none of the PSL source files should be
out-of-date after first reading a tape).

\item Rebuild {\tt bare-psl}, {\tt psl}, and {\tt pslcomp}, placing
them on {\tt \$psys}.

\end{itemize}

\section{REBUILDING THE PSL KERNEL}

A running {\tt xxx-cross} and {\tt pslcomp} are required to rebuild
the kernel, since the entire system is written in PSL itself.  The
kernel modules are compiled to assembly code ({\tt as}) using {\tt
xxx-cross}.  They are then linked using the system loader {\tt ld}.
After that, the rest of the system is compiled to {\tt .b} files,
which are loaded into the kernel to produce the executables.  You can
modify any file in the PSL tree.  Then connect to {\tt \$psl} and do
{\tt make}.  This will recompile any modified files and rebuild the
executables.

\subsection{Cleanup}

{\tt cd \$psl; make cleanup} will remove log files, editor backup files, and
backup copies of the executables.

\subsection{Building PSL with different static sizes}

PSL uses static sizes for Binary Program Space (BPS), whereas the heap
size is adjustable.
The BPS holds compiled code (both {\tt oload}ed C code and Lisp code),
while the heap contains dynamic Lisp data.

To determine how much bps space is left call {\tt (free-bps)}.  The
result is in words, so multiply by 4 to get total bytes left.  The
amount of available heap space can be determined by first running the
garbage collector to compact memory and then calling {\tt gtheap}
with a {\tt nil} argument: {\tt (gtheap nil)}.  This also returns the
number of words, so multiply by 4 to get bytes.  The following shows
an example of this:

\begin{verbatim}
PSL 3.4,  8-Sep-91
1 lisp> (free-bps)
216013
2 lisp> (reclaim)
*** Garbage collection starting
+++++++++++++++++++++++++++++++++++++++++++++++++++++++++++++++++
NIL
3 lisp> (gtheap nil)
396538
4 lisp> 
\end{verbatim}

In order to change the amount of BPS space, do the following:

\begin{enumerate}

\item Connect to the {\tt \$pxk} directory.

\item Edit {\tt Makefile} and change the value of the variable
{\tt BPSSIZE}.  This is the number of bytes. {\it It must be a
multiple of four.}

\item Connect to the {\tt \$psl} directory.

\item Run ``{\tt make resize}''.

\end{enumerate}

\noindent For example:

\begin{verbatim}
% cd $pxk
% vi Makefile ... make changes to this file
% cd $psl
% make resize
\end{verbatim}

In order to change the amount of Heap space, you must do the
following:

\begin{enumerate}

\item If your system uses ``images'', i.e. all non UNIX systems and
Unix versions where a file {\tt \$psys/.imago} exists, you can simply add
or modify the memory size uses in the startup file {\tt \$psys/psl}
e.g. if you want to run PSL with 800000 Bytes total system size:

\begin{verbatim}
#! /bin/csh -f
$psys/bpsl -f $psys/psl.img -td 8000000
\end{verbatim}

Under VMS, you can simply start PSL with  \$psl -td 8000000  .


\item If your system does not use images, you can set the heap size
by using the command set-heap-size, e.g.

\begin{verbatim}
(set-heap-size 8000000)
\end{verbatim}

If set-heap-size returns its argument the additional memory was allocated. 

\section{FUTURE UPDATES}

It is currently envisioned that future updates will be complete
releases.  It is therefore suggested that you

\begin{itemize}

\item Retain this distribution tape in case you may have to compare files.

\item Do not make any changes on these distributed directories. If you must
make your own bug fixes, it is suggested that you put the changed
files on some other directories, such as {\tt \$proot/local}, etc.  We
will not send any files on these directories, so that you can compare
your local versions with our newer ones.

\end{itemize}

\section{BUG REPORTS}

If you wish, you can send bug reports, suggestions, fixes, etc. to us.
If you have Internet access, you can send email to {\tt
neun@sc.ZIB-Berlin.de}.  For those with BITNET access, use {\tt
neun@sc.ZIB-Berlin.dbp.de}.

\section{PSL init files}

When {\tt psl} starts up it will attempt to read and evaluate a file named
{\tt .pslrc} ({\tt PSL.INI} under VMS) from your home directory.  
Similarly, {\tt pslcomp} will use a
{\tt .pslcomprc} ({\tt PSLCOMP.INI) init file.

\section{UNIX specific PSL procedures}

Following are some PSL procedures available in all UNIX versions of PSL.
Most of them are available in all versions of PSL.

\def \thing #1#2#3#4 {{\noindent {\tt(#1): #2} \hfill {\it #3}}
                      \begin{itemize}\item[] #4
                      \end{itemize}}

\thing{cd S:string}{boolean}{expr}{Sets the current working directory to be
the value of string.  Returns {\tt t} if it is successful, otherwise
{\tt nil}.}

\thing{channelflush CHNL:inum}{(inum,0)}{expr}{Flush the existing channel
CHNL so that all characters that have not actually been written
to the channel are done so.}

\thing{exitlisp}{Undefined}{expr}{Equivalent to {\tt quit}
(Refer to {\tt quit}).}

\thing{exit-with-status N:number}{Undefined}{expr}{Exit from PSL,
returning the integer {\tt N} to the caller. This operation
eventually calls the C library routine {\tt exit}}.

\thing{filestatus FILENAME:string DOSTRINGS:string}{(list, nil)}{expr}{
Return system dependent information about the file FILENAME. The values
returned are user, group, mode, size, creation time, access time, and
modification time. The DOSTRINGS argument is a flag that turns
on conversion of the information to string representation. If {\tt nil},
the information is returned in a system dependent manner.}

\thing{flushstdoutputbuffer}{any}{expr}{Flush the standard output buffer
so that any characters that have not been written, are done so. The
buffer flushed is the one usually associated with {\tt stdout}. This
operation only works when the terminal is in {\tt cooked} mode (see
{\tt echoon}).}

\thing{getenv S:string}{(string, nil)}{expr}{Getenv searches the environment
for a string of the form S=value and returns the value if such a
string is present, otherwise {\tt nil} is returned. This is equivalent, but
not identical to the Unix operation of the same name.}

\thing{getstartupname}{string}{expr}{Return the absolute path of the file
that PSL was started from.}

\thing{getunixargs}{(vector, nil)}{expr}{The arguments supplied on the
command line to the PSL executable are placed in the vector stored in
the PSL variable {\tt unixargs*}. The vector is a vector of strings.}

\thing{init-file-string NAME:string}{string}{expr}{Construct a string
containing the name of an initialization file in the user's home
directory. The name of the file is constructed by prepending a ``.''
to NAME, and appending ``rc'' to NAME. For example, if the value of
NAME was ``psl'', the initialization filename constructed would be
``.pslrc''.}

\thing{named-user-homedir NAME:string}{string}{expr}{Return the home
directory path of the user named NAME.}

\thing{pwd}{string}{expr}{Determine the current working directory and
return the value as a string.}
\section{Abstract}

\thing{quit}{Undefined}{expr}{Exit from PSL, returning to the caller.
If PSL is in a break loop, a non-zero status is returned.}

\thing{setenv VAR:string VAL:string}{any}{expr}{Setenv searches the
environment for a string of the form VAR=value and if it is not found,
a string of the form VAR=VAL is added to the environment.  If it is
found, VAR=value is overwritten so that it is equal to VAR=VAL. This
is equivalent, but not identical to the Unix operation of the same
name.}

\thing{system S:string}{string}{expr}{Executes S in a sub (Bourne) shell.
Returns the exit status of the shell.}

\thing{unix-time}{integer}{expr}{Returns the number of seconds since
some arbitrary point. For most Unix systems, this is January 1, 1970.}

\thing{unlink FILENAME:string}{any}{expr}{Remove the file FILENAME (works
similar to {\tt rm}).}

\thing{user-homedir-string}{string}{expr}{Return the home directory
path of the user running PSL.}


         SUN 3               16 MByte
         SUN 386i           128 MByte
         SUN 4              128 MByte 
         Mips Systems       256 MByte

Since all systems use a two space copying garbage collector, one LISP 
item costs 8 byte of memory and the maximum amount of heap items varies 
between approx 1,5 M (Sun 3) and approx 30 M (Mips).   


In some rare cases a "Binding stack overflow" can occur. 
Normally this is due to a problem formulation which involves infinte
recursion. For very big problems, a correct problem may exhaust the binding
stack space, in this case the user may increase the size of the 
Binding stack by: (default is 2000)

        PSL                               REDUCE
 
   (set-bndstk-size nnnn)       symbolic set!-bndstk!-size nnnn;
 


The arithmetic and specially the infinite precision arithmetic for PSL
and REDUCE was dramatically improved. This software benefits from the
usage of hardware specific instructions for double precision arithmetic
(if such instructions are provided by the processor).
The load module is called nbig30, which is loaded into REDUCE by the 
usual build procedure and should be used in PSL instead of nbig.
  
The floating point coprocessors of the systems are used for all
floating point calculations. 



At the moment the overflow of the system stack is not handled gently.
An overflow causes a 'Illegal Instruction' trap and a core dump (if
not prevented by setting a limit to core files). We hope that later
versions of PSL (and partially of Operating systems) 
will have an improved handling for that.


For issueing commands to the operating system the "sytem" interface
has been defined previously, e.g.


   (system "ls")              or              system "ls";


will list the names of the files in the current working directory.
The system command uses the system call "system" which creates a new
(Bourne) shell to process the command enclosed in the string.
Some operating system command are useful only is they are issued
in the same shell as the running REDUCE or PSL run, e.g.


    (system "cd hugo")        or              system "cd hugo";


will NOT change the current directory for the next REDUCE or PSL
commands. To provide the user with such "built-in" system calls a
number of Unix commands are available in REDUCE or PSL, e.g.


    (cd "hugo")               or              symbolic cd "hugo";


will change the current directory for the next REDUCE or PSL commands,
if "hugo" is a valid directory name. 
The most widely used "built-in" commands available in this version are:
 
   (cd  "dirname")                         symbolic cd  "dirname";           
   (getenv "string")             or           symbolic getenv "string";            (setenv "string" "string")           symbolic setenv("string","string");
 

The code generated by the PSL compiler can be inspected by loading
the utility disassemble (not yet for SUN386i) and calling the
function 

\end{document}

