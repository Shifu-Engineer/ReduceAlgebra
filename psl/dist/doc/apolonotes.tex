\documentstyle[11pt]{article}
% Pass group opnote title page and macros.
% Global definitions 
\def\opnumber#1{\number\yearonly -#1}
\def\opnote#1{\gdef\opheader
     {\xiipt Utah PASS Project OpNote \opnumber{#1}\hfill\today
      \markright{\today\hfill PASS Opnote \opnumber{#1}\hfill}}}
\def\mpr#1{\gdef\opheader
     {\xiipt Utah PASS Project Progress Report \opnumber{#1}\hfill\today
      \markright{\today\hfill PASS Progress \opnumber{#1}\hfill}}}
\def\title#1{\gdef\Xtitle{#1}}
\def\author#1{\gdef\Xauthor{#1}}
\gdef\opheader{}
\gdef\Xtitle{}
\gdef\Xauthor{}

\def\endtitlepage{\eject}

\def\maketitle{
\begin{titlepage}
%
\newcount\yearonly
\yearonly=\year
\advance\yearonly by-1900

% reset certain absurd settings for the title page.
\topmargin -12pt \headsep 0pt \headheight 0pt

\setcounter{page}{0}
\noindent\opheader
\vskip 1.25in
\begin{center}
{\xivpt\bf \Xtitle \par}
\vfil
{\xivpt\bf \Xauthor \par}
\vfil


% \begin{picture}(52,80)
% \put(0,0){\line(0,1){80}}        % left side
% \put(50,0){\line(0,1){65}}       % right side
% \put(0,0){\line(1,0){50}}        % bottom
% \put(0,80){\line(1,0){30}}       % top
% \put(30,65){\line(0,1){15}}      % left dent
% \put(30,65){\line(1,0){20}}       % bottom dent
% \put(10,45){\xpt(\( \textstyle\rm pass\))}  % pass  name
% \put(20,55){\(\textstyle \star\)}       % star
% \end{picture}

\setlength{\unitlength}{0.01in}
\begin{picture}(100,100)

\put (0,0){\line(1,2){60}}
\put (0,0){\line(1,0){120}}
\put (60,120){\line(1,-2){60}}
\put (18,0){\makebox(84,36){P A S S}}
\put (18,36){\line(1,0){84}}
\put (36,36){\makebox(48,36){( :- )}}
\put (36,72){\line(1,0){48}}

\put (56,72){\framebox(9,10){}}
\put (60,86){\line(2,1){14.2}}
\put (60,86){\line(-2,1){14.2}}
\put (60,99){\line(-2,-1){14.2}}
\put (60,99){\line(2,-1){14.2}}
\put (60,93){\circle*{4}}

\put (20,0){\oval(20,20)[b]}
\put (100,0){\oval(20,20)[b]}

\end{picture}
\vfil
Utah Portable Artificial Intelligence Support Systems Project\\
Computer Science Department\\
University of Utah\\
Salt Lake City, Utah 84112\\
\vfil
Copyright \copyright \number\year, Utah Pass Project\\
\end{center}
\bigskip
{\xpt Work supported in part by the Hewlett Packard
Corporation, the Schlumberger Palo Alto Laboratory for A.I. Research,
the National
Science foundation Under Grant Number MCS81-21750 and the
Defense Advanced Research Projects Agency under contract number
DAAK11-84-K-0017.}

\end{titlepage}
% \protect\topmargin 0pt \headsep 35pt \headheight 12pt 
\setcounter{footnote}{0}}

\headheight 14pt
\textwidth 5.75in
\oddsidemargin 0.50in
\pagestyle{myheadings}
\title{Apollo Addendum\\
to the\\
UNIX PSL 3.4/PCLS 2.1 Release Notes}

\author{Robert Kessler}

\opnote{07}

\begin{document}

\maketitle
\nonstopmode

\section{Introduction}

The tape is in ``wbak'' format, written with:
\begin{verbatim}
wbak ./dist ./bin ./pclsdist -f 1 -full -l -vid 1 | tee backup.log
\end{verbatim}

\section{Section 5.2.  Read the tape}

If you want to read the entire tape, use the following command:
\begin{verbatim}
rbak -f 1 -pdt -all
\end{verbatim}

If you do not wish to retrieve all the files on the tape,
but only a subset (listed below), then do:
\begin{verbatim}
rbak ./dist/distrib/read-tape.csh ./dist/xxx-names -f 1 -vid 1 -pdt
\end{verbatim}

\section{Section 6.2.  Adding the PSL Environment}

On the Apollo the executable file is actually two files.  One is a
script.  The other is a saved image map file.  The script loads and
starts the map file.  The map file is highly dependent on the exact
configuratoin of the system that it was built on.  This is because the
PSL image uses absolute addressing and the Apollo may load an
executable image in different addresses.  Therefore, you need to
consistently run executables on the same mahine with the exact same
environment.  For instance, a saved image on one DN3000 may or may not
run on another one.  Also, two different users may or may not be able
to run the same image.  Therefore, you are advised to keep local
copies of the map files.  To facilitate this, you should define a {\tt
\$map} variable which points to a directory that you own.  Then when
an executable image is created the map file will be stored in that
directory.  The script will than load inyour own version of the image.
The {\tt setenv} to create the {\tt \$map} variable must follow the
source of the {\tt psl-names} file.

\begin{verbatim}
        source $~psl/dist/psl-names          # or equivalent
        setenv map //a/login/foo/map         # or equivalend
        set path=(. $psys /usr/bin /bin ...) # plus any others
\end{verbatim}

\section{Section 6.3.  Make links to executables}

Ignore this section.

\section{Section 9. Rebuilding the System}

You must rebuild the executables on the Apollo after reading the tape.

\section{Section 10. Rebuilding the PSL Kernel}

You need to rebuild {\tt apollo-cross} for the machine you are running
it on (see discussing about section 6.2 above).  To rebuild the cross
compiler run the {\tt \$pxdist/make-apollo-cross} script.

\section{Section 11. Difference between Berkeley Unix Versions}

PSL runs on Apollo under BSD using the C shell.  PSL does not run
on Apollo under the BSD Bourne shell, Apollo's System V, or Aegis.

\end{document}
