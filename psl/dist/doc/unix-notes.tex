\documentstyle[11pt]{article}
% Pass group opnote title page and macros.  % Global definitions
\def\opnumber#1{\number\yearonly -#1}
\def\opnote#1{\gdef\opheader
     {\xiipt Utah PASS Project OpNote \opnumber{#1}\hfill\today
      \markright{\today\hfill PASS Opnote \opnumber{#1}\hfill}}}
\def\mpr#1{\gdef\opheader
     {\xiipt Utah PASS Project Progress Report \opnumber{#1}\hfill\today
      \markright{\today\hfill PASS Progress \opnumber{#1}\hfill}}}
\def\title#1{\gdef\Xtitle{#1}}
\def\author#1{\gdef\Xauthor{#1}}
\gdef\opheader{}
\gdef\Xtitle{}
\gdef\Xauthor{}

\def\endtitlepage{\eject}

\def\maketitle{
\begin{titlepage}
%
\newcount\yearonly
\yearonly=\year
\advance\yearonly by-1900

% reset certain absurd settings for the title page.
\topmargin -12pt \headsep 0pt \headheight 0pt

\setcounter{page}{0}
\noindent\opheader
\vskip 1.25in
\begin{center}
{\xivpt\bf \Xtitle \par}
\vfil
{\xivpt\bf \Xauthor \par}
\vfil


% \begin{picture}(52,80)
% \put(0,0){\line(0,1){80}}        % left side
% \put(50,0){\line(0,1){65}}       % right side
% \put(0,0){\line(1,0){50}}        % bottom
% \put(0,80){\line(1,0){30}}       % top
% \put(30,65){\line(0,1){15}}      % left dent
% \put(30,65){\line(1,0){20}}       % bottom dent
% \put(10,45){\xpt(\( \textstyle\rm pass\))}  % pass  name
% \put(20,55){\(\textstyle \star\)}       % star
% \end{picture}

\setlength{\unitlength}{0.01in}
\begin{picture}(100,100)

\put (0,0){\line(1,2){60}}
\put (0,0){\line(1,0){120}}
\put (60,120){\line(1,-2){60}}
\put (18,0){\makebox(84,36){P A S S}}
\put (18,36){\line(1,0){84}}
\put (36,36){\makebox(48,36){( :- )}}
\put (36,72){\line(1,0){48}}

\put (56,72){\framebox(9,10){}}
\put (60,86){\line(2,1){14.2}}
\put (60,86){\line(-2,1){14.2}}
\put (60,99){\line(-2,-1){14.2}}
\put (60,99){\line(2,-1){14.2}}
\put (60,93){\circle*{4}}

\put (20,0){\oval(20,20)[b]}
\put (100,0){\oval(20,20)[b]}

\end{picture}
\vfil
Utah Portable Artificial Intelligence Support Systems Project\\
Computer Science Department\\
University of Utah\\
Salt Lake City, Utah 84112\\
\vfil
Copyright \copyright \number\year, Utah Pass Project\\
\end{center}
\bigskip
{\xpt Work supported in part by the Hewlett Packard
Corporation, the Schlumberger Palo Alto Laboratory for A.I. Research,
the National
Science foundation Under Grant Number MCS81-21750 and the
Defense Advanced Research Projects Agency under contract number
DAAK11-84-K-0017.}

\end{titlepage}
% \protect\topmargin 0pt \headsep 35pt \headheight 12pt 
\setcounter{footnote}{0}}

\headheight 14pt
\textwidth 5.75in
\oddsidemargin 0.50in
\pagestyle{myheadings}
\title{PSL 3.4 - PCLS 2.1 Under Unix}

\author{Harold Carr and Leigh Stoller}

\opnote{07}

\begin{document}

\maketitle
\nonstopmode

\section{Abstract}

This document is for users of PSL/PCLS under Unix (System V or
Berkeley).  All or none of the features described here may or may not
exist in any particular implementation of PSL under UNIX.

\section{PSL init files}

When {\tt psl} starts up it will attempt to read and evaluate a file named
{\tt .pslrc} from your home directory.  Similarly, {\tt pslcomp} will use a
{\tt .pslcomprc} init file, and {\tt pcls} will use a {\tt .pclsrc} init file.

\section{UNIX specifc PSL procedures}

Following are some PSL procedures available in the Unix version of PSL.

\def \thing #1#2#3#4 {{\noindent {\tt(#1): #2} \hfill {\it #3}}
                      \begin{itemize}\item[] #4
                      \end{itemize}}

\thing{cd S:string}{boolean}{expr}{Sets the current working directory to be
the value of string.  Returns {\tt t} if it is successful, otherwise
{\tt nil}.}

\thing{channelflush CHNL:inum}{(inum,0)}{expr}{Flush the existing channel
CHNL so that all characters that have not actually been written
to the channel are done so.}

\thing{charsininputbuffer}{integer}{expr}{Returns the number of input
characters waiting in the input buffer. The input buffer is that
channel normally associated with {\tt stdin}.}

\thing{echooff}{any}{expr}{Enters tty character input mode. This is the
equivalent of {\em raw} mode. No more characters will be displayed on
the terminal until the procedure {\tt echoon} is called.}

\thing{echoon}{any}{expr}{Resumes tty line input mode. This is the
equivalent of {\em cooked} mode. Characters will be displayed normally
on the terminal.}

\thing{exitlisp}{Undefined}{expr}{Equivalent to {\tt quit}
(Refer to {\tt quit}).}

\thing{exit-with-status N:number}{Undefined}{expr}{Exit from PSL,
returning the integer {\tt N} to the caller. This operation
eventually calls the C library routine {\tt exit}}.

\thing{filestatus FILENAME:string DOSTRINGS:string}{(list, nil)}{expr}{
Return system dependent information about the file FILENAME. The values
returned are user, group, mode, size, creation time, access time, and
modification time. The DOSTRINGS argument is a flag that turns
on conversion of the information to string representation. If {\tt nil},
the information is returned in a system dependent manner.}

\thing{flushstdoutputbuffer}{any}{expr}{Flush the standard output buffer
so that any characters that have not been written, are done so. The
buffer flushed is the one usually associated with {\tt stdout}. This
operation only works when the terminal is in {\tt cooked} mode (see
{\tt echoon}).}

\thing{getenv S:string}{(string, nil)}{expr}{Getenv searches the environment
for a string of the form S=value and returns the value if such a
string is present, otherwise {\tt nil} is returned. This is equivalent, but
not identical to the Unix operation of the same name.}

\thing{getstartupname}{string}{expr}{Return the absolute path of the file
that PSL was started from.}

\thing{getunixargs}{(vector, nil)}{expr}{The arguments supplied on the
command line to the PSL executable are placed in the vector stored in
the PSL variable {\tt unixargs*}. The vector is a vector of strings.}

\thing{importforeignstring S:string}{string}{expr}{This procedure is used
convert a C string to a Lisp string. Given the address of C string S,
the string is copied into the PSL heap, tagged appropriately, and
returned.}

\thing{init-file-string NAME:string}{string}{expr}{Construct a string
containing the name of an initialization file in the user's home
directory. The name of the file is constructed by prepending a ``.''
to NAME, and appending ``rc'' to NAME. For example, if the value of
NAME was ``psl'', the initialization filename constructed would be
``.pslrc''.}

\thing{named-user-homedir NAME:string}{string}{expr}{Return the home
directory path of the user named NAME.}

\thing{pwd}{string}{expr}{Determine the current working directory and
return the value as a string.}

\thing{quit}{Undefined}{expr}{Exit from PSL, returning to the caller.
If PSL is in a break loop, a non-zero status is returned.}

\thing{setenv VAR:string VAL:string}{any}{expr}{Setenv searches the
environment for a string of the form VAR=value and if it is not found,
a string of the form VAR=VAL is added to the environment.  If it is
found, VAR=value is overwritten so that it is equal to VAR=VAL. This
is equivalent, but not identical to the Unix operation of the same
name.}

\thing{system S:string}{string}{expr}{Executes S in a sub shell.
Returns the exit status of the shell.}

\thing{unix-time}{integer}{expr}{Returns the number of seconds since
some arbitrary point. For most Unix systems, this is January 1, 1970.}

\thing{unlink FILENAME:string}{any}{expr}{Remove the file FILENAME (works
similar to {\tt rm}).}

\thing{user-homedir-string}{string}{expr}{Return the home directory
path of the user running PSL.}


\section{Loading C code into PSL}

\thing{oload LOADSPECS:string}{boolean}{expr}{\ }

{\tt oload} provides a mechanism for linking Unix {\tt .o} and {\tt
.a} files into a running PSL.  The single argument to {\tt oload} is a
string containing arguments for the Unix {\tt ld} loader, separated by
blanks.  File names ending in {\tt .o} are compiled relocatable code
files.  {\tt .a} files are {\tt ar} load libraries, which are assumed
to contain a set of {\tt .o} files, all of which are to be loaded.
Other loader arguments should follow, specifying whatever libraries
are necessary to satisfy all external references from the {\tt .o} and
{\tt .a} files mentioned.  Library specs are in the form {\tt -lfoo}
to search {\tt libfoo.a} on {\tt /lib}, {\tt /usr/lib}, {\tt
/usr/local/lib}, etc.  {\tt -lc} searchs the C library.

{\tt oload} performs an ``incremental'' ({\tt -A} flag) load.  Symbols
which are already known in the running PSL will be linked to the
existing addresses.

If LOADSPECS is {\tt nil}, an attempt is made to re-load from an
existing {\tt .oload.out} file.  This can only be done if the {\tt
bps} and {\tt warray} base addresses are exactly the same as they were
on the previously done, full OLOAD.  An error message results if these
addresses are different.  This is meant to facilitate rapidly
repeating an OLOAD at startup time, and is probably useful only to a
PSL system expert.

Alternately, a customized version of PSL may be saved by the {\tt
savesystem} procedure, after first performing OLOADs and loading PSL
code which interfaces to the OLOADed code.

{\tt oload} returns a status code of {\tt t} if it succeeds, or {\tt
nil} if not.

\subsection{Calling oloaded procedures}

All entry points and global data objects loaded by {\tt oload} are
made known to the PSL system.  C procedures may be called from
compiled code only, and are flagged {\tt foreignfunction} by {\tt
oload}.  Data areas are flagged {\tt foreigndata}, with a property
containing a pair of the data location and size in bytes for use by
interface code.

Foreign procedure calls may be compiled into FASL files, in which case
the C procedure names must be flagged {\tt foreignfunction} at compile
time in order to get the proper calling sequence.

The names of OLOADed C procedures within PSL are the ``true'' names,
which have an underscore ({\tt \_}) prefixed to the C name.  This makes
it easy to make a compiled ``pass through'' interface procedure which
gives the same name within PSL as the C names.  For example, your PSL
definition might look like this:
\begin{verbatim}
        (flag '(_foo) 'foreignfunction)

        (de foo (x)
          (_foo x))
\end{verbatim}
while your C program might read:
\begin{verbatim}
        foo(x)
        int x;
        {
            printf("%d\n%", x);
            return(30);
        }
\end{verbatim}
Note that the C language version of foo must be all lower case, and
does not start with an underscore (the underscore is added by the C
compiler).

Procedures which take integer arguments can be called directly, due to
PSL's ``invisible tagging'' of integers.  These ``invisibly tagged''
integers are known as the {\tt inum} datatype in PSL.  They are a
maximum of 27 bits in the Vax implementation of PSL, 24 bits on 68K
versions.  Similarly, integer return values in the same range may be
passed directly back from the C procedures.

It is possible to pass other data types (such as floating point
numbers and structs) between C and PSL.  The file {\tt
\$pxu/l2cdatacon.sl} contains the conversion procedures that make this
possible.

Note that, currently, foreign procedure calls may have no more than
five arguments.  This restrictions may be circumvented by putting
arguments in work areas and passing the address of the work area as an
argument to an intermediate C ``kludge procedure'' which unpacks the
real arguments and passes them on to the target C procedure.

If work areas are needed in interface code, as when arrays must be
passed to the C code, use a PSL variable to hold the address of a word
block acquired via {\tt gtwarray} (for static arrays) or {\tt gtwrds}
(for dynamic blocks in the heap).  Pass the array's base address (the
number returned by {\tt gtwarray} or {\tt gtwrds}) to the C procedure
as the pointer argument.  When the C procedure returns the PSL array
will contain the data filled in by the C procedure.

\subsection{C calling PSL}

It is possible for C to call PSL.  First {\tt oload} the object file
{\tt \$pl/c2l.o}.  Then see the source file {\tt \$pxu/c2l.c} for
details on C procedures calling PSL procedures.  This file also allows
C to access and set the values of PSL {\tt global} and {\tt fluid} variables.

\subsection{Oload Internals}

{\tt oload} invokes the Unix {\tt ld} loader through a C shell script to
convert the relocatable code in {\tt .o} files into absolute form, then
reads the absolute form into space allocated within the {\tt bps} area of
PSL.  The text segment goes at the low end of {\tt bps}, and the data and
bss segments go at the high end, following the {\tt bps} storage allocation
conventions of the PSL compiler.

Since an incremental ({\tt -A}) load is done, {\tt oload} needs a
filename path to the executable file containing the loader symbol
table of the previous load.  The variable {\tt symbolfilename*} tells
both {\tt oload} and the {\tt savesystem} and {\tt dumplisp}
procedures the filename string to use.  (Since {\tt oload} reads the
executable file, it should be publicly readable.)

When PSL is built, {\tt symbolfilename*} is set to the
name of the PSL kernel upon which the PSL is built.

{\tt symbolfilename*} is set to {\tt .oload.out} by {\tt oload}, so
that successive loads will accumulate a loader symbol table, and so
that the C procedure {\tt unexec}, called by {\tt dumpsystem}, will
get the right symbol table in the saved PSL.  (It may be useful to
know that the initial value of {\tt symbolfilename*} is saved in
the PSL variable {\tt startupname*}.)

A number of work files are created on the current directory by the
oload script, with file names that begin {\tt .oload}.  Some of these may
be quite large.  It is a good idea to remove those files if you do not
intend to repeat the load exactly.

\subsection{Using Oload in PCLS}

You must be in the {\tt psl} package to use {\tt oload} within PCLS.
Switch to the {\tt psl} package and use {\tt oload} there:
\begin{verbatim}
        (in-package 'psl :use nil}
        (load oload)
        (oload "foo -lc")
\end{verbatim}
The OLOADed procedures will live in the {\tt psl} package.

\end{document}

