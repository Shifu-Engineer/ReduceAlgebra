\chapter{Symbolic Mode}\index{Symbolic mode}

\hypertarget{target:modes}{}
\hypertarget{reserved:LISP}{}

At the system level, {\REDUCE} is based on a version of the programming
language Lisp\index{Lisp} known as {\em Standard Lisp\/} which is described
in J. Marti, Hearn, A. C., Griss, M. L. and Griss, C., ``Standard LISP
Report" SIGPLAN Notices, ACM, New York, 14, No 10 (1979) 48-68.  We shall
assume in this section that the reader is familiar with the material in
that paper.  This also assumes implicitly that the reader has a reasonable
knowledge about Lisp in general, say at the level of the LISP 1.5
Programmer's Manual (McCarthy, J., Abrahams, P. W., Edwards, D. J., Hart,
T. P. and Levin, M. I., ``LISP 1.5 Programmer's Manual'', M.I.T.  Press,
1965) or any of the books mentioned at the end of this section.  Persons
unfamiliar with this material will have some difficulty understanding this
section.

Although {\REDUCE} is designed primarily for algebraic calculations, its
source language is general enough to allow for a full range of Lisp-like
symbolic calculations.  To achieve this generality, however, it is
necessary to provide the user with two modes of evaluation, namely an
algebraic mode\index{Algebraic mode} and a symbolic mode.\index{Symbolic
mode} To enter symbolic mode, the user types {\tt symbolic;}
\ttindex{SYMBOLIC} (or {\tt lisp;})\ttindex{LISP} and to return to
algebraic mode one types {\tt algebraic;}.\ttindex{ALGEBRAIC}
Evaluations proceed differently in each mode so the user is advised to
check what mode he is in if a puzzling error arises.  He can find his mode
by typing\ttindex{EVAL\_MODE}

\begin{verbatim}
        eval_mode;
\end{verbatim}
The current mode will then be printed as {\tt ALGEBRAIC} or {\tt SYMBOLIC}.

Expression evaluation may proceed in either mode at any level of a
calculation, provided the results are passed from mode to mode in a
compatible manner. One simply prefixes the relevant expression by the
appropriate mode. If the mode name prefixes an expression at the top
level, it will then be handled as if the global system mode had been
changed for the scope of that particular calculation.

For example, if the current mode is {\tt ALGEBRAIC}, then the commands
%\extendedmanual{\newpage}
\begin{verbatim}
        symbolic car '(a);
        x+y;
\end{verbatim}
will cause the first expression to be evaluated and printed in symbolic
mode and the second in algebraic mode. Only the second evaluation will
thus affect the expression workspace. On the other hand, the statement
\begin{verbatim}
        x + symbolic car '(12);
\end{verbatim}
will result in the algebraic value {\tt X+12}.

The use of {\tt SYMBOLIC} (and equivalently {\tt ALGEBRAIC}) in this
manner is the same as any operator.  That means that parentheses could be
omitted in the above examples since the meaning is obvious.  In other
cases, parentheses must be used, as in

\begin{verbatim}
        symbolic(x := 'a);
\end{verbatim}
Omitting the parentheses, as in
\begin{verbatim}
        symbolic x := a;
\end{verbatim}
would be wrong, since it would parse as
\begin{verbatim}
        symbolic(x) := a;
\end{verbatim}
For convenience, it is assumed that any operator whose {\em first\/} argument is
quoted is being evaluated in symbolic mode, regardless of the mode in
effect at that time. Thus, the first example above could be equally well
written:
\begin{verbatim}
        car '(a);
\end{verbatim}
Except where explicit limitations have been made, most {\REDUCE} algebraic
constructions carry over into symbolic mode.\index{Symbolic mode}
However, there are some differences.  First, expression evaluation now
becomes Lisp evaluation.  Secondly, assignment statements are handled
differently, as we shall discuss shortly.  Thirdly, local variables and array
elements are initialized to {\tt NIL} rather than {\tt 0}. (In fact, any
variables not explicitly declared {\tt INTEGER} are also initialized to
{\tt NIL} in algebraic mode, but the algebraic evaluator recognizes {\tt
NIL} as {\tt 0}.) Finally, function definitions follow the conventions of
Standard Lisp.

To begin with, we mention a few extensions to our basic syntax which are
designed primarily if not exclusively for symbolic mode.

\section{Symbolic Infix Operators}

There are three binary infix operators in {\REDUCE} intended for use in
symbolic mode, namely . {\tt (CONS), EQ and MEMQ}. The precedence of
these operators was given in another section.

\section{Symbolic Expressions}

These consist of scalar variables and operators and follow the normal
rules of the Lisp meta language.

{\it Examples:}
\begin{verbatim}
        x
        car u . reverse v
        simp (u+v^2)
\end{verbatim}

\section{Quoted Expressions}\ttindex{QUOTE}

Because symbolic evaluation requires that each variable or expression has a
value, it is necessary to add to {\REDUCE} the concept of a quoted expression
by analogy with the Lisp {\tt QUOTE} function. This is provided by the single
quote mark {\tt '}.  For example,
\begin{quote}
\begin{tabbing}
{\tt '(a b c)} \= represents the Lisp S-expression \= {\tt (quote (a b
c))}\kill
{\tt 'a} \> represents the Lisp S-expression \>
{\tt (quote a)} \\
{\tt '(a b c)} \> represents the Lisp S-expression \> {\tt (quote (a b c))}
\end{tabbing}
\end{quote}
Note, however, that strings are constants and therefore evaluate to
themselves in symbolic mode. Thus, to print the string {\tt "A String"}, one
would write
\begin{verbatim}
        prin2 "A String";
\end{verbatim}
Within a quoted expression, identifier syntax rules are those of {\REDUCE}.
Thus {\tt  (A~!.~~B)} is the list consisting of the three elements {\tt A},
{\tt .}, and {\tt B}, whereas {\tt (A .  B)} is the dotted pair of {\tt A}
and {\tt B}.

\section{Lambda Expressions}\ttindex{LAMBDA}
\hypertarget{reserved:LAMBDA}{}
\label{sec-lambda}

{\tt LAMBDA} expressions provide the means for constructing Lisp {\tt LAMBDA}
expressions in symbolic mode. They may not be used in algebraic mode.

Syntax:
\begin{verbatim}
        <LAMBDA expression> ::=
                LAMBDA <varlist><terminator><statement>
\end{verbatim}
 where
\begin{verbatim}
        <varlist> ::= (<variable>,...,<variable>)
\end{verbatim}
e.g.,
\begin{verbatim}
        lambda (x,y); car x . cdr y;
\end{verbatim}
is equivalent to the Lisp {\tt LAMBDA} expression
\begin{verbatim}
        (lambda (x y) (cons (car x) (cdr y)))
\end{verbatim}
The parentheses may be omitted in specifying the variable list if desired.

{\tt LAMBDA} expressions may be used in symbolic mode in place of prefix
operators, or as an argument of the reserved word {\tt FUNCTION}.

In those cases where a {\tt LAMBDA} expression is used to introduce local
variables to avoid recomputation, a {\tt WHERE} statement can also be
used.  For example, the expression
\begin{verbatim}
        (lambda (x,y); list(car x,cdr x,car y,cdr y))
            (reverse u,reverse v)
\end{verbatim}
can also be written
\begin{verbatim}
      {car x,cdr x,car y,cdr y} where x=reverse u,y=reverse v
\end{verbatim}
Where possible, {\tt WHERE} syntax is preferred to {\tt LAMBDA} syntax,
since it is more natural.

\section{Symbolic Assignment Statements}\index{Assignment}
\hypertarget{command:SETQ}{}

In symbolic mode, if the left side of an assignment statement is a
variable, a {\tt SETQ} of the right-hand side to that variable occurs.  If
the left-hand side is an expression, it must be of the form of an array
element, otherwise an error will result.  For example, {\tt x:=y}
translates into {\tt (SETQ X Y)} whereas {\tt a(3) := 3} will be valid if
{\tt A} has been previously declared a single dimensioned array of at
least four elements.

\section{FOR EACH Statement}\ttindex{FOR EACH}

The {\tt FOR EACH} form of the {\tt FOR} statement, designed for iteration
down a list, is more general in symbolic mode.  Its syntax is:

\begin{verbatim}
        FOR EACH ID:identifier {IN|ON} LST:list
            {DO|COLLECT|JOIN|PRODUCT|SUM} EXPRN:S-expr
\end{verbatim}

As in algebraic mode, if the keyword {\tt IN} is used, iteration is on
each element of the list.  With {\tt ON}, iteration is on the whole list
remaining at each point in the iteration.  As a result, we have the
following equivalence between each form of {\tt FOR EACH} and the various
mapping functions in Lisp:
\begin{center}
{\tt
\begin{tabular}{|l|lr r|} \hline
& DO & COLLECT & JOIN \\ \hline
        IN &   MAPC & MAPCAR & MAPCAN \\
        ON &   MAP &  MAPLIST & MAPCON \\ \hline
\end{tabular}}
\end{center}
{\it Example:} To list each element of the list {\tt (a b c)}:
\begin{verbatim}
        for each x in '(a b c) collect list x;
\end{verbatim}

\section{Symbolic Procedures}\index{Symbolic procedure}

All the functions described in the Standard Lisp Report are available to
users in symbolic mode. Additional functions may also be defined as
symbolic procedures. For example, to define the Lisp function {\tt ASSOC},
the following could be used:
\begin{verbatim}
        symbolic procedure assoc(u,v);
           if null v then nil
            else if u = caar v then car v
            else assoc(u, cdr v);
\end{verbatim}
\hypertarget{reserved:MACRO}{}
\hypertarget{reserved:EXPR}{}
If the default mode were symbolic, then {\tt SYMBOLIC} could be omitted in
the above definition. {\tt MACRO}s\ttindex{MACRO} may be defined by
prefixing the keyword {\tt PROCEDURE} by the word {\tt MACRO}.
(In fact, ordinary functions may be defined with the keyword {\tt EXPR}
\ttindex{EXPR} prefixing {\tt PROCEDURE} as was used in the Standard Lisp
Report.) For example, we could define a {\tt MACRO CONSCONS} by
\begin{verbatim}
        symbolic macro procedure conscons l;
           expand(cdr l,'cons);
\end{verbatim}

\hypertarget{reserved:SMACRO}{}
\hypertarget{reserved:FEXPR}{}
Another form of macro, the {\tt SMACRO}\ttindex{SMACRO} is also available.
These are described in the Standard Lisp Report.  The Report also defines
a function type {\tt FEXPR}.\ttindex{FEXPR}
However, its use is discouraged since it is hard to implement efficiently,
and most uses can be replaced by macros.  At the present time, there are
no {\tt FEXPR}s in the core REDUCE system.

\section{Standard Lisp Equivalent of Reduce Input}
\hypertarget{switch:DEFN}{}

A user can obtain the Standard Lisp equivalent of his {\REDUCE} input by
turning on the switch {\tt DEFN}\ttindex{DEFN} (for definition).  The
system then prints the Lisp translation of his input but does not evaluate
it.  Normal operation is resumed when {\tt DEFN} is turned off.

\section{Communicating with Algebraic Mode}\index{Mode communication}
\hypertarget{command:SHARE}{}

One of the principal motivations for a user of the algebraic facilities of
{\REDUCE} to learn about symbolic mode\index{Symbolic mode} is that it
gives one access to a wider range of techniques than is possible in
algebraic mode\index{Algebraic mode} alone.  For example, if a user
wishes to use parts of the system defined in the basic system source code,
or refine their algebraic code definitions to make them more efficient,
then it is necessary to understand the source language in fairly complete
detail.  Moreover, it is also necessary to know a little more about the
way {\REDUCE} operates internally.  Basically, {\REDUCE} considers
expressions in two forms: prefix form, which follow the normal Lisp rules
of function composition, and so-called canonical form, which uses a
completely different syntax.

Once these details are understood, the most critical problem faced by a
user is how to make expressions and procedures communicate between symbolic
and algebraic mode. The purpose of this section is to teach a user the
basic principles for this.

If one wants to evaluate an expression in algebraic mode, and then use
that expression in symbolic mode calculations, or vice versa, the easiest
way to do this is to assign a variable to that expression whose value is
easily obtainable in both modes.  To facilitate this, a declaration {\tt
SHARE}\ttindex{SHARE} is available. {\tt SHARE} takes a list of
identifiers as argument, and marks these variables as having recognizable
values in both modes.  The declaration may be used in either mode.

E.g.,
\begin{verbatim}
        share x,y;
\end{verbatim}
says that {\tt X} and {\tt Y} will receive values to be used in both modes.

If a {\tt SHARE} declaration is made for a variable with a previously
assigned algebraic value, that value is also made available in symbolic
mode.

\subsection{Passing Algebraic Mode Values to Symbolic Mode}

If one wishes to work with parts of an algebraic mode
\index{Algebraic mode} expression in symbolic mode,\index{Symbolic mode}
one simply makes an assignment\index{Assignment} of a shared variable to
the relevant expression in algebraic mode.  For example, if one wishes to
work with {\tt (a+b)\verb|^|2}, one would say, in algebraic mode:
\begin{verbatim}
        x := (a+b)^2;
\end{verbatim}
assuming that {\tt X} was declared shared as above.  If we now change to
symbolic mode and say
\begin{verbatim}
        x;
\end{verbatim}
its value will be printed as a prefix form with the syntax:
\begin{verbatim}
        (*SQ <standard quotient> T)
\end{verbatim}
This particular format reflects the fact that the algebraic mode processor
currently likes to transfer prefix forms from command to command, but
doesn't like to reconvert standard forms\index{Standard form} (which
represent polynomials) and standard quotients back to a true Lisp prefix
form for the expression (which would result in excessive computation).  So
{\tt *SQ} is used to tell the algebraic processor that it is dealing with
a prefix form which is really a standard quotient\index{Standard
quotient} and the second argument ({\tt T} or {\tt NIL}) tells it whether
it needs further processing (essentially, an {\em already simplified\/}
flag).

So to get the true standard quotient form in symbolic mode, one needs
{\tt CADR} of the variable. E.g.,
\begin{verbatim}
        z := cadr x;
\end{verbatim}
would store in {\tt Z} the standard quotient form for {\tt (a+b)\verb|^|2}.

Once you have this expression, you can now manipulate it as you wish.  To
facilitate this, a standard set of selectors\index{Selector} and
constructors\index{Constructor} are available for getting at parts of the
form.  Those presently defined are as follows:
\extendedmanual{\newpage}
\begin{center}
\vspace{10pt}
{\large  REDUCE Selectors\par}
%\end{center}
%\begin{center}
\renewcommand{\arraystretch}{1.5}
\begin{tabular}{lp{\rboxwidth}}
{\tt DENR} & denominator of standard quotient \\
%
{\tt LC} & leading coefficient of polynomial \\
%
{\tt LDEG} & leading degree of polynomial \\
%
{\tt LPOW} & leading power of polynomial \\
%
{\tt LT} & leading term of polynomial \\
%
{\tt MVAR} & main variable of polynomial \\
%
{\tt NUMR} & numerator (of standard quotient) \\
%
{\tt PDEG} & degree of a power \\
%
{\tt RED} & reductum of polynomial \\
%
{\tt TC} & coefficient of a term \\
%
{\tt TDEG} & degree of a term \\
%
{\tt TPOW} & power of a term
\end{tabular}
\end{center}

\begin{center}
\vspace{10pt}
{\large REDUCE Constructors \par}
%\end{center}
%\begin{center}
\renewcommand{\arraystretch}{1.5}
\begin{tabular}{lp{\redboxwidth}}
\verb|.+| & add a term to a polynomial \\
%
\verb|./| & divide (two polynomials to get quotient) \\
\verb|.*| & multiply power by coefficient to produce term \\
%
\verb|.^| & raise a variable to a power
\end{tabular}
\end{center}

For example, to find the numerator of the standard quotient above, one
could say:
\begin{verbatim}
        numr z;
\end{verbatim}
or to find the leading term of the numerator:
\begin{verbatim}
        lt numr z;
\end{verbatim}
Conversion between various data structures is facilitated by the use of a
set of functions defined for this purpose. Those currently implemented
include:

{\renewcommand{\arraystretch}{1.5}
\begin{tabular}{lp{\reduceboxwidth}}
{\tt !*A2F} & convert an algebraic expression to
a standard form.  If result is rational, an error results; \\
%
{\tt !*A2K} & converts an algebraic expression to
a kernel.  If this is not possible, an error results; \\
%
{\tt !*F2A} & converts a standard form to an
algebraic expression; \\
%
{\tt !*F2Q} & convert a standard form to a
standard quotient; \\
%
{\tt !*K2F} & convert a kernel to a standard form; \\
{\tt !*K2Q} & convert a kernel to a standard quotient; \\
%
{\tt !*P2F} & convert a standard power to a
standard form; \\
%
{\tt !*P2Q} & convert a standard power to a standard quotient; \\
%
{\tt !*Q2F} & convert a standard quotient to a
standard form.  If the quotient denominator is not 1, an error results; \\
%
{\tt !*Q2K} & convert a standard quotient to a
kernel.  If this is not possible, an error results; \\
%
{\tt !*T2F} & convert a standard term to a standard form \\
%
{\tt !*T2Q} & convert a standard term to a standard quotient.
\end{tabular}}

\subsection{Passing Symbolic Mode Values to Algebraic Mode}

In order to pass the value of a shared variable from symbolic mode to
algebraic mode, the only thing to do is make sure that the value in
symbolic mode is a prefix expression. E.g., one uses
{\tt (expt (plus a b) 2)} for {\tt (a+b)\verb|^|2}, or the format ({\tt *sq
<standard quotient> t}) as described above.  However, if you have
been working with parts of a standard form they will probably not be in
this form.  In that case, you can do the following:
\begin{enumerate}
\item If it is a standard quotient, call {\tt PREPSQ} on it.  This takes a
standard quotient as argument, and returns a prefix expression.
Alternatively, you can call {\tt MK!*SQ} on it, which returns a prefix
form like ({\tt *SQ <standard quotient> T)} and avoids translation of
the expression into a true prefix form.

\item If it is a standard form, call {\tt PREPF} on it.  This takes a
standard form as argument, and returns the equivalent prefix expression.
Alternatively, you can convert it to a standard quotient and then call
{\tt MK!*SQ}.

\item If it is a part of a standard form, you must usually first build up a
standard form out of it, and then go to step 2. The conversion functions
described earlier may be used for this purpose. For example,
\begin{enumerate}
\item If {\tt Z} is an expression which is a term, {\tt !*T2F Z} is a
standard form.
\item If {\tt Z} is a standard power, {\tt !*P2F Z} is a standard form.
\item If {\tt Z} is a variable, you can pass it direct to algebraic mode.
\end{enumerate}
\end{enumerate}
For example, to pass the leading term of {\tt (a+b)\verb|^|2} back to
algebraic mode, one could say:
\begin{verbatim}
        y:= mk!*sq !*t2q lt numr z;
\end{verbatim}
where {\tt Y} has been declared shared as above.  If you now go back to
algebraic mode, you can work with {\tt Y} in the usual way.


\subsection{Complete Example}

The following is the complete code for doing the above steps. The end
result will be that the square of the leading term of $(a+b)^{2}$ is
calculated.

%%\begin{tabular}{lp{\rboxwidth}}
%%{\tt share x,y;} & {\tt \% declare {\tt X} and
%%{\tt Y} as shared} \\
%%{\tt x := (a+b)\verb|^|2;} & {\tt \% store (a+b)\verb|^|2 in X} \\
%%{\tt symbolic;} & {\tt \% transfer to symbolic mode} \\
%%{\tt z := cadr x;} & {\tt \% store a true standard quotient \newline
%%                          \% in Z} \\[1.7pt]
%%{\tt lt numr z;} & {\tt \% print the leading term of the \newline
%%                        \% numerator of Z} \\
%%{\tt y := mk!*sq !*t2q lt numr z;} & {\tt \% store the
%%                       prefix form of this \newline
%%                     \% leading term in Y} \\
%%{\tt algebraic;} & {\tt \% return to algebraic mode} \\
%%{\tt y\verb|^|2;} & {\tt \% evaluate square of the leading \newline
%%\% term of (a+b)\verb|^|2}
%%\end{tabular}
\begin{verbatim}
share x,y;                   % declare X and Y as shared
x := (a+b)^2;                % store (a+b)^2 in X
symbolic;                    % transfer to symbolic mode
z := cadr x;                 % store a true standard quotient
                             % in Z
lt numr z;                   % print the leading term of the
                             % numerataor of Z
y := mk!*sq !*t2q lt numr z; % store the prefix form of this
                             % leading term in Y
algebraic;                   % return to algebraic mode
y^2;                         % evaluate square of the
                             % leading term of (a+b)^2
\end{verbatim}

\subsection{Defining Procedures for Intermode Communication}

If one wishes to define a procedure in symbolic mode for use as an
operator in algebraic mode, it is necessary to declare this fact to the
system by using the declaration {\tt OPERATOR}\ttindex{OPERATOR} in
symbolic mode. Thus
\begin{verbatim}
        symbolic operator leadterm;
\end{verbatim}
would declare the procedure {\tt LEADTERM} as an algebraic operator. This
declaration {\em must\/} be made in symbolic mode as the effect in algebraic
mode is different.  The value of such a procedure must be a prefix form.

The algebraic processor will pass arguments to such procedures in prefix
form. Therefore if you want to work with the arguments as standard
quotients you must first convert them to that form by using the function
{\tt SIMP!*}. This function takes a prefix form as argument and returns the
evaluated standard quotient.

For example, if you want to define a procedure {\tt LEADTERM} which gives the
leading term of an algebraic expression, one could do this as follows:
\begin{samepage}
\begin{verbatim}
symbolic operator leadterm; % Declare LEADTERM as a symbolic
                            % mode procedure to be used in
                            % algebraic mode.

symbolic procedure leadterm u; % Define LEADTERM.
   mk!*sq !*t2q lt numr simp!* u;
\end{verbatim}
\end{samepage}
Note that this operator has a different effect than the operator {\tt LTERM}
\ttindex{LTERM}.  In the latter case, the calculation is done
with respect to the second argument of the operator.  In the example here,
we simply extract the leading term with respect to the system's choice of
main variable.

Finally, if you wish to use the algebraic evaluator on an argument in a
symbolic mode definition, the function {\tt REVAL} can be used.  The one
argument of {\tt REVAL} must be the prefix form of an expression. {\tt
REVAL} returns the evaluated expression as a true Lisp prefix form.

