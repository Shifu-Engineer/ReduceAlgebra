
\index{EXCALC package}

\subsection*{Acknowledgments}

This program was developed over several years. I would like to express
my deep gratitude to Dr.\ Anthony Hearn for his continuous interest in
this work, and especially for his hospitality and support during a
visit in 1984/85 at the RAND Corporation, where substantial progress
on this package could be achieved. The Heinrich Hertz-Stiftung
supported this visit. Many thanks are also due to Drs. F.W. Hehl,
University of Cologne, and J.D. McCrea, University College Dublin, for
their suggestions and work on testing this program.

\subsection{Introduction}

\index{Differential geometry}
{\bf EXCALC} is designed for easy use by all who are familiar with the
calculus of Modern Differential Geometry.  Its syntax is kept as close
as possible to standard textbook notations.  Therefore, no great
experience in writing computer algebra programs is required.  It is
almost possible to input to the computer the same as what would have
been written down for a hand-calculation.  For example, the statement
\begin{verbatim}
                       f*x^y + u _| (y^z^x)
\end{verbatim}
\index{Exterior calculus}
would be recognized by the program as a formula involving exterior
products and an inner product.  The program is currently able to
handle scalar-valued exterior forms, vectors and operations between
them, as well as non-scalar valued forms (indexed forms).  With this,
it should be an ideal tool for studying differential equations,
doing calculations in general relativity and field theories, or doing
such simple things as calculating the Laplacian of a tensor field for
an arbitrary given frame.  With the increasing popularity of this
calculus, this program should have an application in almost any field
of physics and mathematics.

Since the program is completely embedded in {\REDUCE}, all features and
facilities of {\REDUCE} are available in a calculation.  Even for those
who are not quite comfortable in this calculus, there is a good chance
of learning it by just playing with the program.

This is the last release of version 2. A much extended differential
geometry package (which includes complete symbolic index simplification,
tensors, mappings, bundles and others) is under development.

Complaints and comments are appreciated and should be send to the author.
If the use of this program leads to a publication, this document should
be cited, and a copy of the article to the above address would be
welcome.

\subsection{Declarations}

Geometrical objects like exterior forms or vectors are introduced to the
system by declaration commands.  The declarations can appear anywhere in
a program, but must, of course, be made prior to the use of the object.
Everything that has no declaration is treated as a constant; therefore
zero-forms must also be declared.

An exterior form is introduced by\label{PFORM} \ttindextype{PFORM}{statement}
\index{Exterior form!declaration}

\hspace*{2em} \k{PFORM} \s{declaration$_1$}, \s{declaration$_2$}, \ldots;

where
\begin{tabbing}
\s{declaration} ::= \s{name} $\mid$ \s{list of names}=\s{number} $\mid$  \s{identifier} $\mid$ \\
\s{expression} \\
\s{name} ::= \s{identifier} $\mid$ \s{identifier}(\s{arguments})
\end{tabbing}

For example
\begin{verbatim}
     pform u=k,v=4,f=0,w=dim-1;
\end{verbatim}
declares {\tt U} to be an exterior form of degree {\tt K}, {\tt V} to be a
form of degree 4, {\tt F} to be a form of degree 0 (a function), and {\tt W}
to be a form of degree {\tt DIM}-1.

If the exterior form should have indices, the declaration would be
\index{Exterior form!with indices}
\begin{verbatim}
     pform curv(a,b)=2,chris(a,b)=1;
\end{verbatim}
The names of the indices are arbitrary.

Exterior forms of the same degree can be grouped into lists to save typing.
\begin{verbatim}
     pform {x,y,z}=0,{rho(k,l),u,v(k)}=1;
\end{verbatim}
The declaration of vectors is similar. The command {\tt TVECTOR}\label{TVECTOR}
takes a list of names. \ttindextype{TVECTOR}{command} \index{Exterior form!vector}

\hspace*{2em} \k{TVECTOR} \s{name$_1$}, \s{name$_2$}, \ldots;

For example, to declare {\tt X} as a vector and {\tt COMM} as a vector with 
two indices, one would say
\begin{verbatim}
     tvector x,comm(a,b);
\end{verbatim}
If a declaration of an already existing name is made, the old
declaration is removed, and the new one is taken.

The exterior degree of a symbol or a general expression can be obtained
with the function \label{EXDEGREE} \ttindextype{EXDEGREE}{command}

\hspace*{2em} \k{EXDEGREE} \s{expression};

Example:
\begin{verbatim}
     exdegree(u + 3*chris(k,-k));

     1
\end{verbatim}


\subsection{Exterior Multiplication}

\ttindextype{"\^}{! exterior multiplication} \index{Exterior product}
Exterior multiplication between exterior forms is carried out with the
nary infix operator \^{ } (wedge)\label{wedge}.  Factors are ordered
according to the usual ordering in {\REDUCE} using the commutation
rule for exterior products.

\example\index{EXCALC package!example}

\begin{verbatim}
     pform u=1,v=1,w=k;

     u^v;

     U^V

     v^u;

     - U^V

     u^u;

     0

     w^u^v;

           K
     ( - 1) *U^V^W

     (3*u-a*w)^(w+5*v)^u;

     A*(5*U^V^W - U^W^W)
\end{verbatim}

It is possible to declare the dimension of the underlying space
by\label{SPACEDIM} \ttindextype{SPACEDIM}{command} \index{Dimension}

\hspace*{2em} \k{SPACEDIM} \s{number} $\mid$ \s{identifier};

If an exterior product has a degree higher than the dimension of the
space, it is replaced by 0:

\begin{verbatim}
     spacedim 4;

     pform u=2,v=3;

     u^v;

     0
\end{verbatim}


\subsection{Partial Differentiation}

Partial differentiation is denoted by the operator {\tt @}\label{at}.  Its
capability is the same as the {\REDUCE} {\tt DF} operator.
\ttindextype{"@}{operator} \index{Partial differentiation}
\index{Differentiation!partial}

\example\index{EXCALC package!example}

\begin{verbatim}
     @(sin x,x);

     COS(X)

     @(f,x);

     0
\end{verbatim}

An identifier can be declared to be a function of certain variables.
\ttindextype{FDOMAIN}{command}
This is done with the command {\tt FDOMAIN}\label{FDOMAIN}.  The
following would tell the partial differentiation operator that {\tt F}
is a function of the variables {\tt X} and {\tt Y} and that {\tt H} is
a function of {\tt X}.
\begin{verbatim}
     fdomain f=f(x,y),h=h(x);
\end{verbatim}
Applying {\tt @} to {\tt F} and {\tt H} would result in
\begin{verbatim}
     @(x*f,x);

     F + X*@  F
            X

     @(h,y);

     0
\end{verbatim}

\index{Tangent vector}
The partial derivative symbol can also be an operator with a single
argument.  It then represents a natural base element of a tangent
vector\label{at1}.

\example\index{EXCALC package!example}

\begin{verbatim}
     a*@ x + b*@ y;

     A*@  + B*@
        X      Y
\end{verbatim}

\subsection{Exterior Differentiation}
\index{Exterior differentiation}
Exterior differentiation of exterior forms is carried out by the
operator {\tt d}\label{d}.  Products are normally differentiated out,
{\em i.e.}

\begin{verbatim}
     pform x=0,y=k,z=m;

     d(x * y);

     X*d Y + d X^Y

     d(r*y);

     R*d Y

     d(x*y^z);

           K
     ( - 1) *X*Y^d Z  + X*d Y^Z + d X^Y^Z
\end{verbatim}

This expansion can be suppressed by the command {\tt NOXPND D}\label{NOXPNDD}.
\ttindextype{NOXPND}{! D}

\begin{verbatim}
     noxpnd d;

     d(y^z);

     d(Y^Z)
\end{verbatim}

To obtain a canonical form for an exterior product when the expansion
is switched off, the operator {\tt D} is shifted to the right if it
appears in the leftmost place.

\begin{verbatim}
     d y ^ z;

             K
     - ( - 1) *Y^d Z + d(Y^Z)
\end{verbatim}

Expansion is performed again when the command {\tt XPND D}\label{XPNDD}
is executed. \ttindextype{XPND}{! \texttt{D}}

Functions which are implicitly defined by the {\tt FDOMAIN} command are
expanded into partial derivatives:

\begin{verbatim}
     pform x=0,y=0,z=0,f=0;

     fdomain f=f(x,y);

     d f;

     @  F*d X + @  F*d Y
      X          Y
\end{verbatim}

If an argument of an implicitly defined function has further
dependencies the chain rule will be applied {\em e.g.} \index{Chain rule}


\begin{verbatim}
     fdomain y=y(z);

     d f;

     @  F*d X + @  F*@  Y*d Z
      X          Y    Z
\end{verbatim}

Expansion into partial derivatives can be inhibited by
{\tt NOXPND @}\label{NOXPNDA}
and enabled again by {\tt XPND @}\label{XPNDA}.
\ttindextype{NOXPND}{! "@} \ttindextype{XPND}{! "@}

The operator is of course aware of the rules that a repeated
application always leads to zero and that there is no exterior form of
higher degree than the dimension of the space.

\begin{verbatim}
     d d x;

     0

     pform u=k;
     spacedim k;

     d u;

     0
\end{verbatim}
\subsection{Inner Product}
\index{Inner product!exterior form}
The inner product between a vector and an exterior form is represented
by the diphthong \_$|$ \label{innerp} (underscore or-bar), which is the
notation of many textbooks.  If the exterior form is an exterior
product, the inner product is carried through any factor.
\ttindextype{\_\textbar}{operator}

\example\index{EXCALC package!example}

\begin{verbatim}
     pform x=0,y=k,z=m;

     tvector u,v;

     u _| (x*y^z);

              K
     X*(( - 1) *Y^U _| Z + U _| Y^Z)
\end{verbatim}

In repeated applications of the inner product to the same exterior
form the vector arguments are ordered {\em e.g.}

\begin{verbatim}
     (u+x*v) _| (u _| (3*z));

     - 3*U _| V _| Z
\end{verbatim}

The duality of natural base elements is also known by the system, {\em i.e.}

\begin{verbatim}
     pform {x,y}=0;

     (a*@ x+b*@(y)) _| (3*d x-d y);

     3*A - B
\end{verbatim}

\subsection{Lie Derivative}

\index{Lie Derivative}
The Lie derivative can be taken between a vector and an exterior form
or between two vectors.  It is represented by the infix operator $|$\_
\label{lie}.  In the case of Lie differentiating, an exterior form by
a vector, the Lie derivative is expressed through inner products and
exterior differentiations, {\em i.e.} \ttindextype{\textbar\_}{operator}

\begin{verbatim}
     pform z=k;

     tvector u;

     u |_ z;

     U _| d Z + d(U _| Z)
\end{verbatim}

If the arguments of the Lie derivative are vectors, the vectors are
ordered using the anticommutivity property, and functions (zero forms)
are differentiated out.

\example\index{EXCALC package!example}

\begin{verbatim}
     tvector u,v;

     v |_ u;

     - U |_ V 

     pform x=0,y=0;

     (x*u) |_ (y*v);

     - U*Y*V _| d X + V*X*U _| d Y + X*Y*U |_ V
\end{verbatim}

\subsection{Hodge-* Duality Operator}

\index{Hodge-* duality operator} \ttindextype{"\#}{! Hodge-* operator}
The Hodge-*\label{hodge} duality operator maps an exterior form of degree
{\tt K} to an exterior form of degree {\tt N-K}, where {\tt N} is the
dimension of the space.  The double application of the operator must
lead back to the original exterior form up to a factor. The following
example shows how the factor is chosen here

\begin{verbatim}
     spacedim n;
     pform x=k;

     # # x;

             2
           (K  + K*N)
     ( - 1)          *X*SGN
\end{verbatim}
%\pagebreak
\ttindextype{SGN}{! indeterminate sign} \index{Coframe}
The indeterminate SGN in the above example denotes the sign of the
determinant of the metric. It can be assigned a value or will be
automatically set if more of the metric structure is specified (via
COFRAME), {\em i.e.} it is then set to $g/|g|$, where $g$ is the
determinant of the metric.  If the Hodge-* operator appears in an
exterior product of maximal degree as the leftmost factor, the Hodge-*
is shifted to the right according to

\begin{verbatim}
     pform {x,y}=k;

     # x ^ y;

             2
           (K  + K*N)
     ( - 1)          *X^# Y
\end{verbatim}

More simplifications are performed if a coframe is defined.



\subsection{Variational Derivative}

\index{Derivative!variational} \index{Variational derivative}
\ttindex{VARDF}
The function {\tt VARDF}\label{VARDF} returns as its value the
variation of a given Lagrangian n-form with respect to a specified
exterior form (a field of the Lagrangian).  In the shared variable
\ttindex{BNDEQ"!*}
{\tt BNDEQ!*}, the expression is stored that has to yield zero if
integrated over the boundary.

Syntax:

\hspace*{2em} \k{VARDF}(\s{Lagrangian n-form},\s{exterior form})

\example\index{EXCALC package!example}
\begin{verbatim}
  spacedim 4;

  pform l=4,a=1,j=3;

  l:=-1/2*d a ^ # d a - a^# j$  %Lagrangian of the e.m. field

  vardf(l,a);

  - (# J + d # d A)             %Maxwell's equations

  bndeq!*;

  - 'A^# d A                    %Equation at the boundary
\end{verbatim}
Restrictions:

In the current implementation, the Lagrangian must be built up by the
fields and the operations {\tt d}, {\tt \#}, and {\tt @}. Variation
with respect to indexed quantities is currently not allowed.

For the calculation of the conserved currents induced by symmetry
operators (vector fields), the function {\tt NOETHER}\label{NOETHER}
\ttindextype{NOETHER}{function}
is provided.  It has the syntax:

\hspace*{2em}
\k{NOETHER}(\s{Lagrangian n-form},\s{field},\s{symmetry generator})

\example\index{EXCALC package!example}
\begin{verbatim}
  pform l=4,a=1,f=2;

  spacedim 4;

  l:= -1/2*d a^#d a;   %Free Maxwell field;

  tvector x(k);        %An unspecified generator;

  noether(l,a,x(-k));

  ( - 2*d(X _|A)^# d A - (X _|d A)^# d A + d A^(X _|# d A))/2
           K               K                     K
\end{verbatim}

The above expression would be the canonical energy
momentum 3-forms of the Maxwell field, if X is interpreted
as a translation;



\subsection{Handling of Indices}
\index{Exterior form!with indices}
Exterior forms and vectors may have indices.  On input, the indices
are given as arguments of the object.  A positive argument denotes a
superscript and a negative argument a subscript.  On output, the
indexed quantity is displayed two dimensionally if \texttt{NAT} is on.
\ttindextype{NAT}{switch}
Indices may be identifiers or numbers.  

\example\index{EXCALC package!example}

\begin{verbatim}
     pform om(k,l)=m,e(k)=1;

     e(k)^e(-l);

      K
     E ^E
         L

     om(4,-2);

       4
     OM
        2
\end{verbatim}

In the current release, full simplification is performed only if an
index range is specified.  It is hoped that this restriction can be
removed soon.  If the index range (the values that the indices can
obtain) is specified, the given expression is evaluated for all
possible index values, and the summation convention is understood.

\example\label{INDEXRANGE}\index{EXCALC package!example}

\begin{verbatim}
     indexrange t,r,ph,z;

     pform e(k)=1,s(k,l)=2;

     w := e(k)*e(-k);

              T       R        PH       Z
     W := E *E  + E *E  + E  *E   + E *E
           T       R       PH        Z


     s(k,l):=e(k)^e(l);

      T T
     S    := 0

      R T       T  R
     S    := - E ^E

      PH T       T  PH
     S     := - E ^E

       .
       .
       .

\end{verbatim}

If the expression to be evaluated is not an assignment, the values of
the expression are displayed as an assignment to an indexed variable
with name {\tt NS}.  This is done only on output, {\em i.e.} no actual
binding to the variable NS occurs.
\ttindextype{NS}{dummy variable}

\begin{verbatim}
     e(k)^e(l);

       T T
     NS    := 0

       R T       T  R
     NS    := - E ^E
       .
       .
       .
\end{verbatim}

It should be noted, however, that the index positions on the variable
NS can sometimes not be uniquely determined by the system (because of
possible reorderings in the expression). Generally it is advisable to
use assignments to display complicated expressions.

A range can also be assigned to individual index-names. For example,
the declaration

\begin{verbatim}
    indexrange {k,l}={x,y,z},{u,v,w}={1,2};
\end{verbatim}

would assign to the index identifiers k,l the range values x,y,z and
to the index identifiers u,v,w the range values 1,2. The use of an
index identifier not listed in previous indexrange statements has the
range of the union of all given index ranges.

With the above example of an indexrange statement, the following
index evaluations would take place

\begin{verbatim}
     pform w n=0;
    
     w(k)*w(-k);

         X       Y       Z
     W *W  + W *W  + W *W
      X       Y       Z

     w(u)*w(-u);

         1       2
     W *W  + W *W
      1       2

     w(r)*w(-r);

         1       2       X       Y       Z
     W *W  + W *W  + W *W  + W *W  + W *W
      1       2       X       Y       Z
\end{verbatim}

In certain cases, one would like to inhibit the summation over
specified index names, or at all.  For this the command

\ttindextype{NOSUM}{command}
\hspace*{2em} \k{NOSUM} \s{indexname$_1$}, \ldots;\label{NOSUM}

and the switch \texttt{NOSUM} are\ttindexswitch{NOSUM}{EXCALC}
available.  The command \texttt{NOSUM} has the effect that summation is
not performed over those indices which had been listed.  The command
{\tt RENOSUM}\label{RENOSUM} enables summation again.  The switch \sw{NOSUM}, 
if on, inhibits any summation. \ttindextype{RENOSUM}{command}

\label{INDEXSYMMETRIES} \ttindextype{INDEX\_SYMMETRIES}{command}
It is possible to declare symmetry properties for an indexed quantity by 
the command {\tt INDEX\_SYMMETRIES}. A prototypical example is as
follows

\begin{verbatim}

    index_symmetries u(k,l,m,n): symmetric     in {k,l},{m,n}
                                 antisymmetric in {{k,l},{m,n}},
                     g(k,l),h(k,l): symmetric;

\end{verbatim}

It declares the object {\tt u} symmetric in the first two and last 
two indices and antisymmetric with respect to commutation of the given
index pairs. If an object is completely symmetric or antisymmetric,
the indices need not to be given after the corresponding keyword as
shown above for {\tt g} and {\tt h}.

If applicable, this command should
be issued, since great savings in memory and execution time result.
Only strict components are printed.

The commands symmetric and antisymmetric of earlier releases have no
effect.


\subsection{Metric Structures}

\index{Metric structure} \index{Coframe}
A metric structure is defined in {\bf EXCALC} by specifying a set of
basis one-forms (the coframe) together with the metric.

Syntax:\label{COFRAME}

\begin{tabbing}
\hspace*{2em} \k{COFRAME} \=
\s{identifier}\s{(index$_1$)}=\s{expression$_1$}, \\
\> \s{identifier}\s{(index$_2$)}=\s{expression$_2$}, \\
\> . \\
\> . \\
\> . \\
\> \s{identifier}\s{(index$_n$)}=\s{expression$_n$} \\
\> \hspace{1em} \k{WITH} \k{METRIC} \s{name}=\s{expression}; \\
\end{tabbing}

\index{Euclidean metric} \index{COFRAME!WITH METRIC}
This statement automatically sets the dimension of the space and the
index range. The clause {\tt WITH METRIC} can be omitted if the metric
\index{COFRAME!WITH SIGNATURE}
is Euclidean and the shorthand {\tt WITH SIGNATURE \s{diagonal elements}}
\label{SIGNATURE} can be used in the case of a pseudo-Euclidean metric. The
splitting of a metric structure in its metric tensor coefficients and
basis one-forms is completely arbitrary including the extremes of an
orthonormal frame and a coordinate frame.

\example\index{EXCALC package!example}

\begin{verbatim}
 coframe e r=d r, e(ph)=r*d ph
   with metric g=e(r)*e(r)+e(ph)*e(ph);    %Polar coframe

 coframe e(r)=d r,e(ph)=r*d(ph);           %Same as before

 coframe o(t)=d t, o x=d x
   with signature -1,1;                    %A Lorentz coframe

 coframe b(xi)=d xi, b(eta)=d eta          %A lightcone coframe
   with metric w=-1/2*(b(xi)*b(eta)+b(eta)*b(xi));

 coframe e r=d r, e ph=d ph                %Polar coordinate
   with metric g=e r*e r+r**2*e ph*e ph;   %basis

\end{verbatim}

Individual elements of the metric can be accessed just by calling them
with the desired indices. The value of the determinant of the
\index{Determinant!in DETM"!*} \ttindex{DETM"!*}
covariant metric is stored in the variable {\tt DETM!*}.  The metric
is not needed for lowering or raising of indices as the system
performs this automatically, {\em i.e.} no matter in what index
position values were assigned to an indexed quantity, the values can
be retrieved for any index position just by writing the indexed
quantity with the desired indices.

\example\index{EXCALC package!example}

\begin{verbatim}
     coframe e t=d t,e x=d x,e y=d y
      with signature -1,1,1;

     pform f(k,l)=0;

     antisymmetric f;

     f(-t,-x):=ex$ f(-x,-y):=b$  f(-t,-y):=0$
     on nero;

     f(k,-l):=f(k,-l);

      X
     F    := - EX
        T

      T
     F    := - EX
        X

      Y
     F    := - B
        X

      X
     F    := B
        Y
\end{verbatim}

Any expression containing differentials of the coordinate functions will
be transformed into an expression of the basis one-forms.The system also
knows how to take the exterior derivative of the basis one-forms.

\index{Spherical coordinates}
\example (Spherical coordinates)\index{EXCALC package!example}

\begin{verbatim}
     coframe e(r)=d(r), e(th)=r*d(th), e(ph)=r*sin(th)*d(ph);

     d r^d th;

       R  TH
     (E ^E  )/R

     d(e(th));

       R  TH
     (E ^E  )/R


     pform f=0;

     fdomain f=f(r,th,ph);

     factor e;

     on rat;

     d f;       %The "gradient" of F in spherical coordinates;

      R          TH              PH
     E *@  F + (E  *@   F)/R + (E  *@   F)/(R*SIN(TH))
         R           TH              PH
\end{verbatim}

The frame dual to the frame defined by the {\tt COFRAME} command can
be introduced by \k{FRAME} command. \ttindextype{FRAME}{command}

\hspace*{2em} \k{FRAME} \s{identifier};\label{FRAME}

This command causes the
dual property to be recognized, and the tangent vectors of the
coordinate functions are replaced by the frame basis vectors.

\example\index{EXCALC package!example}

\begin{verbatim}
   coframe b r=d r,b ph=r*d ph,e z=d z; %Cylindrical coframe;

   frame x;

   on nero;

   x(-k) _| b(l);

       R
   NS    := 1
     R

        PH
   NS      := 1
     PH

       Z
   NS    := 1
     Z

   x(-k) |_ x(-l);       %The commutator of the dual frame;


   NS     := X  /R
     PH R     PH


   NS     := ( - X  )/R  %i.e. it is not a coordinate base;
     R PH         PH

\end{verbatim}

\ttindextype{DISPLAYFRAME}{command} \index{Tracing!EXCALC package}\index{EXCALC package!tracing}
As a convenience, the frames can be displayed at any point in a program
by the command {\tt DISPLAYFRAME;}\label{DISPLAYFRAME}.

\index{Hodge-* duality operator}
The Hodge-* duality operator returns the explicitly constructed dual
element if applied to coframe base elements. The metric is properly
taken into account.

\index{Levi-Cevita tensor} \ttindex{EPS}
The total antisymmetric Levi-Cevita tensor {\tt EPS}\label{EPS} is
also available.  The value of {\tt EPS} with an even permutation of the
indices in a covariant position is taken to be +1.


\subsection{Riemannian Connections}

\index{Riemannian Connections}
The command {\tt RIEMANNCONX} is provided for calculating the
\ttindextype{RIEMANNCONX}{command} \label{RIEMANNCONX}
connection 1 forms.  The values are stored on the name given to {\tt
RIEMANNCONX}.  This command is far more efficient than calculating the
connection from the differential of the basis one-forms and using
inner products.

\example (Calculate the connection 1-form and curvature 2-form on S(2))
\index{EXCALC package!example}

\begin{verbatim}
   coframe e th=r*d th,e ph=r*sin(th)*d ph;

   riemannconx om;

   om(k,-l);                 %Display the connection forms;

     TH
   NS      := 0
        TH

     PH         PH
   NS      := (E  *COS(TH))/(SIN(TH)*R)
        TH

     TH            PH
   NS      := ( - E  *COS(TH))/(SIN(TH)*R)
        PH

     PH
   NS      := 0
        PH

   pform curv(k,l)=2;


   curv(k,-l):=d om(k,-l) + om(k,-m)^om(m-l);
                 %The curvature forms

       TH
   CURV      := 0
          TH

       PH            TH  PH   2
   CURV      := ( - E  ^E  )/R
          TH               %Of course it was a sphere with
                           %radius R.

       TH         TH  PH   2
   CURV      := (E  ^E  )/R
          PH

       PH
   CURV      := 0
          PH
\end{verbatim}

\subsection{Ordering and Structuring}

\index{Ordering!exterior form} \index{Exterior form!ordering}
\ttindextype{FORDER}{command}
The ordering of an exterior form or vector can be changed by the
command {\tt FORDER}.\label{FORDER}  In an expression, the first
identifier or kernel in the arguments of {\tt FORDER} is ordered ahead
of the second, and so on, and ordered ahead of all not appearing as
arguments.  This ordering is done on the internal level and not only
on output.  The execution of this statement can therefore have
tremendous effects on computation time and memory requirements.  {\tt
REMFORDER}\label{REMFORDER} brings back standard ordering for those
elements that are listed as arguments. \ttindextype{REMFORDER}{command}

An expression can be put in a more structured form by renaming a
subexpression.  This is done with the command {\tt KEEP} which
has the syntax \ttindextype{KEEP}{command}\label{KEEP}

\hspace*{2em} \k{KEEP}
\s{name$_1$}=\s{expression$_1$},\s{name$_2$}=\s{expression$_2$}, \ldots

The effect is that rules are set up for simplifying \s{name} without
introducing its definition in an expression. In an expression the system
also tries by reordering to generate as many instances of \s{name} as
possible.

\example\index{EXCALC package!example}

\begin{verbatim}
     pform x=0,y=0,z=0,f=0,j=3;

     keep j=d x^d y^d z;

     j;

     J

     d j;

     0

     j^d x;

     0

     fdomain f=f(x);

     d f^d y^d z;

     @  F*J
      X
\end{verbatim}

\index{Exterior product}
The capabilities of {\tt KEEP} are currently very limited.  Only exterior
products should occur as righthand sides in {\tt KEEP}.


\subsection{Summary of Operators and Commands}
Table~\ref{EXCALC:sum} summarizes EXCALC commands and the page number they are
defined on.

\begin{table}
\begin{tabular}{l l r}
\ttindextype{"\^}{! exterior multiplication} \index{Wedge}
\^{ }  &  Exterior Multiplication & \pageref{wedge} \\
\ttindextype{"@}{! partial differentiation}
@  & Partial Differentiation & \pageref{at}  \\
\ttindextype{"@}{! tangent vector}
@  & Tangent Vector  & \pageref{at1}  \\
\ttindextype{"\#}{! Hodge-* operator}
\#  & Hodge-* Operator & \pageref{hodge} \\
\ttindextype{\_\textbar}{operator}
\_$|$  & Inner Product  & \pageref{innerp} \\
\ttindextype{\textbar\_}{operator}
$|$\_  & Lie Derivative  & \pageref{lie}  \\
\ttindextype{COFRAME}{command}
COFRAME & Declaration of a coframe & \pageref{COFRAME} \\
\ttindextype{d}{! exterior differentiation}
d &  Exterior differentiation & \pageref{d} \\
\ttindextype{DISPLAYFRAME}{command}
DISPLAYFRAME & Displays the frame & \pageref{DISPLAYFRAME}\\
\ttindextype{EPS}{! Levi-Civita tensor}
EPS & Levi-Civita tensor  & \pageref{EPS}  \\
\ttindex{EXDEGREE}
EXDEGREE & Calculates the exterior degree of an expression & \pageref{EXDEGREE}  \\
\ttindextype{FDOMAIN}{command}
FDOMAIN & Declaration of implicit dependencies &\pageref{FDOMAIN} \\
\ttindextype{FORDER}{command}
FORDER & Ordering command  & \pageref{FORDER} \\
\ttindextype{FRAME}{command}
FRAME & Declares the frame dual to the coframe & \pageref{FRAME} \\
\ttindextype{INDEXRANGE}{command}
INDEXRANGE & Declaration of indices & \pageref{INDEXRANGE} \\
\ttindextype{INDEX\_SYMMETRIES}{command}
INDEX\_SYMMETRIES & Declares arbitrary index symmetry properties  & \pageref{INDEXSYMMETRIES} \\
\ttindextype{KEEP}{command}
KEEP & Structuring command  & \pageref{KEEP} \\
\ttindextype{METRIC}{command}
METRIC & Clause of COFRAME to specify a metric & \pageref{COFRAME} \\
\ttindextype{NOETHER}{function}
NOETHER & Calculates the Noether current & \pageref{NOETHER} \\
\ttindextype{NOSUM}{command}
NOSUM & Inhibits summation convention & \pageref{NOSUM} \\
\ttindextype{NOXPND}{command}
NOXPND d & Inhibits the use of product rule for d &
\pageref{NOXPNDD} \\
\ttindextype{NOXPND "@}{command}
NOXPND @ & Inhibits expansion into partial derivatives &
\pageref{NOXPNDA} \\
\ttindextype{PFORM}{command}
PFORM & Declaration of exterior forms & \pageref{PFORM} \\
\ttindextype{REMFORDER}{command}
REMFORDER & Clears ordering  & \pageref{REMFORDER} \\
\ttindextype{RENOSUM}{command}
RENOSUM & Enables summation convention & \pageref{RENOSUM} \\
\ttindextype{RIEMANNCONX}{command}
RIEMANNCONX & Calculation of a Riemannian Connection &
\pageref{RIEMANNCONX} \\
\ttindextype{SIGNATURE}{command}
SIGNATURE & Clause of COFRAME to specify a pseudo- & \pageref{SIGNATURE} \\
  & Euclidean metric &   \\
\ttindextype{SPACEDIM}{command}
SPACEDIM & Command to set the dimension of a space &
\pageref{SPACEDIM} \\
\ttindextype{TVECTOR}{command}
TVECTOR & Declaration of vectors  & \pageref{TVECTOR} \\
\ttindex{VARDF}
VARDF & Variational derivative  & \pageref{VARDF} \\
\ttindextype{XPND}{command}
XPND d & Enables the use of product rule for d & \pageref{XPNDD} \\
  & (default)  &   \\
\ttindex{XPND!"@}
XPND @ & Enables expansion into partial derivatives & \pageref{XPNDA} \\
  & (default)
\end{tabular}
\caption{EXCALC Command Summary}\label{EXCALC:sum}
\end{table}
\newpage
\subsection{Examples}

The following examples should illustrate the use of {\bf EXCALC}. It is not
intended to show the most efficient or most elegant way of stating the
problems; rather the variety of syntactic constructs are exemplified.
The examples are on a test file distributed with {\bf EXCALC}.
\index{EXCALC package!example}
{\small
\begin{verbatim}

% Problem: Calculate the PDE's for the isovector of the heat equation.
% --------
%         (c.f. B.K. Harrison, f.B. Estabrook, "Geometric Approach...",
%          J. Math. Phys. 12, 653, 1971)

% The heat equation @   psi = @  psi is equivalent to the set of exterior
%                    xx        t

% equations (with u=@ psi, y=@ psi):
%                    T        x


pform {psi,u,x,y,t}=0,a=1,{da,b}=2;

a := d psi - u*d t - y*d x;

da := - d u^d t - d y^d x;

b := u*d x^d t - d y^d t;


% Now calculate the PDE's for the isovector.

tvector v;

pform {vpsi,vt,vu,vx,vy}=0;
fdomain vpsi=vpsi(psi,t,u,x,y),vt=vt(psi,t,u,x,y),vu=vu(psi,t,u,x,y),
                               vx=vx(psi,t,u,x,y),vy=vy(psi,t,u,x,y);

v := vpsi*@ psi + vt*@ t + vu*@ u + vx*@ x + vy*@ y;


factor d;
on rat;

i1 := v |_ a - l*a;

pform o=1;

o := ot*d t + ox*d x + ou*d u + oy*d y;

fdomain f=f(psi,t,u,x,y);

i11 := v _| d a - l*a + d f;

let vx=-@(f,y),vt=-@(f,u),vu=@(f,t)+u*@(f,psi),vy=@(f,x)+y*@(f,psi),
    vpsi=f-u*@(f,u)-y*@(f,y);

factor ^;

i2 := v |_ b - xi*b - o^a + zeta*da;

let ou=0,oy=@(f,u,psi),ox=-u*@(f,u,psi),
    ot=@(f,x,psi)+u*@(f,y,psi)+y*@(f,psi,psi);

i2;

let zeta=-@(f,u,x)-@(f,u,y)*u-@(f,u,psi)*y;

i2;

let xi=-@(f,t,u)-u*@(f,u,psi)+@(f,x,y)+u*@(f,y,y)+y*@(f,y,psi)+@(f,psi);

i2;

let @(f,u,u)=0;

i2;      % These PDE's have to be solved.


clear a,da,b,v,i1,i11,o,i2,xi,t;
remfdomain f,vpsi,vt,vu,vx,vy;
clear @(f,u,u);


% Problem:
% --------
% Calculate the integrability conditions for the system of PDE's:
% (c.f. B.F. Schutz, "Geometrical Methods of Mathematical Physics"
% Cambridge University Press, 1984, p. 156)


% @ z /@ x + a1*z  + b1*z  = c1
%    1           1       2

% @ z /@ y + a2*z  + b2*z  = c2
%    1           1       2

% @ z /@ x + f1*z  + g1*z  = h1
%    2           1       2

% @ z /@ y + f2*z  + g2*z  = h2
%    2           1       2      ;


pform w(k)=1,integ(k)=4,{z(k),x,y}=0,{a,b,c,f,g,h}=1,
      {a1,a2,b1,b2,c1,c2,f1,f2,g1,g2,h1,h2}=0;

fdomain  a1=a1(x,y),a2=a2(x,y),b1=b1(x,y),b2=b2(x,y),
         c1=c1(x,y),c2=c2(x,y),f1=f1(x,y),f2=f2(x,y),
         g1=g1(x,y),g2=g2(x,y),h1=h1(x,y),h2=h2(x,y);


a:=a1*d x+a2*d y$
b:=b1*d x+b2*d y$
c:=c1*d x+c2*d y$
f:=f1*d x+f2*d y$
g:=g1*d x+g2*d y$
h:=h1*d x+h2*d y$

% The equivalent exterior system:
factor d;
w(1) := d z(-1) + z(-1)*a + z(-2)*b - c;
w(2) := d z(-2) + z(-1)*f + z(-2)*g - h;
indexrange 1,2;
factor z;
% The integrability conditions:

integ(k) := d w(k) ^ w(1) ^ w(2);

clear a,b,c,f,g,h,x,y,w(k),integ(k),z(k);
remfdomain a1,a2,b1,c1,c2,f1,f2,g1,g2,h1,h2;

% Problem:
% --------
% Calculate the PDE's for the generators of the d-theta symmetries of
% the Lagrangian system of the planar Kepler problem.
% c.f. W.Sarlet, F.Cantrijn, Siam Review 23, 467, 1981
% Verify that time translation is a d-theta symmetry and calculate the
% corresponding integral.

pform {t,q(k),v(k),lam(k),tau,xi(k),eta(k)}=0,theta=1,f=0,
      {l,glq(k),glv(k),glt}=0;

tvector gam,y;

indexrange 1,2;

fdomain tau=tau(t,q(k),v(k)),xi=xi(t,q(k),v(k)),f=f(t,q(k),v(k));

l := 1/2*(v(1)**2 + v(2)**2) + m/r$      % The Lagrangian.

pform r=0;
fdomain r=r(q(k));
let @(r,q 1)=q(1)/r,@(r,q 2)=q(2)/r,q(1)**2+q(2)**2=r**2;

lam(k) := -m*q(k)/r;                                % The force.

gam := @ t + v(k)*@(q(k)) + lam(k)*@(v(k))$

eta(k) := gam _| d xi(k) - v(k)*gam _| d tau$

y  := tau*@ t + xi(k)*@(q(k)) + eta(k)*@(v(k))$     % Symmetry generator.

theta := l*d t + @(l,v(k))*(d q(k) - v(k)*d t)$

factor @;

s := y |_ theta - d f$

glq(k) := @(q k) _| s;
glv(k) := @(v k) _| s;
glt := @(t) _| s;

% Translation in time must generate a symmetry.
xi(k) := 0;
tau := 1;

glq k := glq k;
glv k := glv k;
glt;

% The corresponding integral is of course the energy.
integ := - y _| theta;


clear l,lam k,gam,eta k,y,theta,s,glq k,glv k,glt,t,q k,v k,tau,xi k;
remfdomain r,f,tau,xi;

% Problem:
% --------
% Calculate the "gradient" and "Laplacian" of a function and the "curl"
% and "divergence" of a one-form in elliptic coordinates.


coframe e u = sqrt(cosh(v)**2 - sin(u)**2)*d u,
        e v = sqrt(cosh(v)**2 - sin(u)**2)*d v,
        e phi = cos u*sinh v*d phi;

pform f=0;

fdomain f=f(u,v,phi);

factor e,^;
on rat,gcd;
order cosh v, sin u;
% The gradient:
d f;

factor @;
% The Laplacian:
# d # d f;

% Another way of calculating the Laplacian:
-#vardf(1/2*d f^#d f,f);

remfac @;

% Now calculate the "curl" and the "divergence" of a one-form.

pform w=1,a(k)=0;

fdomain a=a(u,v,phi);

w := a(-k)*e k;
% The curl:
x := # d w;

factor @;
% The divergence:
y := # d # w;


remfac @;
clear x,y,w,u,v,phi,e k,a k;
remfdomain a,f;


% Problem:
% --------
% Calculate in a spherical coordinate system the Navier Stokes equations.

coframe e r=d r, e theta =r*d theta, e phi = r*sin theta *d phi;
frame x;

fdomain v=v(t,r,theta,phi),p=p(r,theta,phi);

pform v(k)=0,p=0,w=1;

% We first calculate the convective derivative.

w := v(-k)*e(k)$

factor e; on rat;

cdv := @(w,t) + (v(k)*x(-k)) |_ w - 1/2*d(v(k)*v(-k));

%next we calculate the viscous terms;

visc := nu*(d#d# w - #d#d w) + mu*d#d# w;

% Finally we add the pressure term and print the components of the
% whole equation.

pform nasteq=1,nast(k)=0;

nasteq := cdv - visc + 1/rho*d p$

factor @;

nast(-k) := x(-k) _| nasteq;

remfac @,e;

clear v k,x k,nast k,cdv,visc,p,w,nasteq,e k;
remfdomain p,v;


% Problem:
% --------
% Calculate from the Lagrangian of a vibrating rod the equation of
% motion and show that the invariance under time translation leads
% to a conserved current.

pform {y,x,t,q,j}=0,lagr=2;

fdomain y=y(x,t),q=q(x),j=j(x);

factor ^;

lagr := 1/2*(rho*q*@(y,t)**2 - e*j*@(y,x,x)**2)*d x^d t;

vardf(lagr,y);

% The Lagrangian does not explicitly depend on time; therefore the
% vector field @ t generates a symmetry. The conserved current is

pform c=1;
factor d;

c := noether(lagr,y,@ t);

% The exterior derivative of this must be zero or a multiple of the
% equation of motion (weak conservation law) to be a conserved current.

remfac d;

d c;

% i.e. it is a multiple of the equation of motion.

clear lagr,c,j,y,q;
remfdomain y,q,j;

% Problem:
% --------
% Show that the metric structure given by Eguchi and Hanson induces a
% self-dual curvature.
% c.f. T. Eguchi, P.B. Gilkey, A.J. Hanson, "Gravitation, Gauge Theories
% and Differential Geometry", Physics Reports 66, 213, 1980

for all x let cos(x)**2=1-sin(x)**2;

pform f=0,g=0;
fdomain f=f(r), g=g(r);

coframe   o(r) = f*d r,
      o(theta) = (r/2)*(sin(psi)*d theta - sin(theta)*cos(psi)*d phi),
        o(phi) = (r/2)*(-cos(psi)*d theta - sin(theta)*sin(psi)*d phi),
        o(psi) = (r/2)*g*(d psi + cos(theta)*d phi);

frame e;


pform gamma(a,b)=1,curv2(a,b)=2;
index_symmetries gamma(a,b),curv2(a,b): antisymmetric;

factor o;

gamma(-a,-b) := -(1/2)*( e(-a) _| (e(-c) _| (d o(-b)))
		        -e(-b) _| (e(-a) _| (d o(-c)))
		        +e(-c) _| (e(-b) _| (d o(-a))) )*o(c)$


curv2(-a,b) := d gamma(-a,b) + gamma(-c,b)^gamma(-a,c)$

let f=1/g,g=sqrt(1-(a/r)**4);

pform chck(k,l)=2;
index_symmetries chck(k,l): antisymmetric;

% The following has to be zero for a self-dual curvature.

chck(k,l) := 1/2*eps(k,l,m,n)*curv2(-m,-n) + curv2(k,l);

clear gamma(a,b),curv2(a,b),f,g,chck(a,b),o(k),e(k),r,phi,psi;
remfdomain f,g;

% Example: 6-dimensional FRW model with quadratic curvature terms in
% -------
% the Lagrangian (Lanczos and Gauss-Bonnet terms).
% cf. Henriques, Nuclear Physics, B277, 621 (1986)

for all x let cos(x)**2+sin(x)**2=1;

pform {r,s}=0;
fdomain r=r(t),s=s(t);

coframe o(t) = d t,
        o(1) = r*d u/(1 + k*(u**2)/4),
        o(2) = r*u*d theta/(1 + k*(u**2)/4),
        o(3) = r*u*sin(theta)*d phi/(1 + k*(u**2)/4),
        o(4) = s*d v1,
        o(5) = s*sin(v1)*d v2
 with metric g =-o(t)*o(t)+o(1)*o(1)+o(2)*o(2)+o(3)*o(3)
                +o(4)*o(4)+o(5)*o(5);

frame e;

on nero; factor o,^;

riemannconx om;

pform curv(k,l)=2,{riemann(a,b,c,d),ricci(a,b),riccisc}=0;

index_symmetries curv(k,l): antisymmetric,
                 riemann(k,l,m,n): antisymmetric in {k,l},{m,n}
                                   symmetric in {{k,l},{m,n}},
                 ricci(k,l): symmetric;

curv(k,l) := d om(k,l) + om(k,-m)^om(m,l);

riemann(a,b,c,d) := e(d) _| (e (c) _| curv(a,b));

% The rest is done in the Ricci calculus language,

ricci(-a,-b) := riemann(c,-a,-d,-b)*g(-c,d);

riccisc := ricci(-a,-b)*g(a,b);

pform {laglanc,inv1,inv2} = 0;

index_symmetries riemc3(k,l),riemri(k,l),
                 hlang(k,l),einst(k,l): symmetric;

pform {riemc3(i,j),riemri(i,j)}=0;

riemc3(-i,-j) := riemann(-i,-k,-l,-m)*riemann(-j,k,l,m)$
inv1 := riemc3(-i,-j)*g(i,j);
riemri(-i,-j) := 2*riemann(-i,-k,-j,-l)*ricci(k,l)$
inv2 := ricci(-a,-b)*ricci(a,b);
laglanc := (1/2)*(inv1 - 4*inv2 + riccisc**2);


pform {einst(a,b),hlang(a,b)}=0;

hlang(-i,-j) := 2*(riemc3(-i,-j) - riemri(-i,-j) -
		   2*ricci(-i,-k)*ricci(-j,K) +
		   riccisc*ricci(-i,-j) - (1/2)*laglanc*g(-i,-j));

% The complete Einstein tensor: 

einst(-i,-j) := (ricci(-i,-j) - (1/2)*riccisc*g(-i,-j))*alp1 +
		hlang(-i,-j)*alp2$

alp1 := 1$
factor alp2;

einst(-i,-j) := einst(-i,-j);

clear o(k),e(k),riemc3(i,j),riemri(i,j),curv(k,l),riemann(a,b,c,d),
      ricci(a,b),riccisc,t,u,v1,v2,theta,phi,r,om(k,l),einst(a,b),
      hlang(a,b);

remfdomain r,s; 

% Problem:
% --------
% Calculate for a given coframe and given torsion the Riemannian part and
% the torsion induced part of the connection. Calculate the curvature.

% For a more elaborate example see E.Schruefer, F.W. Hehl, J.D. McCrea,
% "Application of the REDUCE package EXCALC to the Poincare gauge field
% theory of gravity", GRG Journal, vol. 19, (1988)  197--218

pform {ff, gg}=0;

fdomain ff=ff(r), gg=gg(r);

coframe o(4) = d u + 2*b0*cos(theta)*d phi,
        o(1) = ff*(d u + 2*b0*cos(theta)*d phi) + d r,
        o(2) = gg*d theta,
        o(3) = gg*sin(theta)*d phi
 with metric g = -o(4)*o(1)-o(4)*o(1)+o(2)*o(2)+o(3)*o(3);

frame e;

pform {tor(a),gwt(a)}=2,gamma(a,b)=1,
      {u1,u3,u5}=0;

index_symmetries gamma(a,b): antisymmetric;

fdomain u1=u1(r),u3=u3(r),u5=u5(r);

tor(4) := 0$

tor(1) := -u5*o(4)^o(1) - 2*u3*o(2)^o(3)$

tor(2) := u1*o(4)^o(2) + u3*o(4)^o(3)$

tor(3) := u1*o(4)^o(3) - u3*o(4)^o(2)$

gwt(-a) := d o(-a) - tor(-a)$

% The following is the combined connection.
% The Riemannian part could have equally well been calculated by the
% RIEMANNCONX statement.

gamma(-a,-b) := (1/2)*( e(-b) _| (e(-c) _| gwt(-a))
	               +e(-c) _| (e(-a) _| gwt(-b))
                       -e(-a) _| (e(-b) _| gwt(-c)) )*o(c);

pform curv(a,b)=2;
index_symmetries curv(a,b): antisymmetric;
factor ^;

curv(-a,b) := d gamma(-a,b) + gamma(-c,b)^gamma(-a,c);

clear o(k),e(k),curv(a,b),gamma(a,b),theta,phi,x,y,z,r,s,t,u,v,p,q,c,cs;
remfdomain u1,u3,u5,ff,gg;

showtime;
end;

\end{verbatim}
}
