
\subsection{Introduction}

The package {\bf Boolean} supports the computation with
boolean expressions in the propositional calculus.
The data objects are composed from algebraic expressions (``atomic parts'', ``leafs'')
connected by the infix boolean operators {\bf and}, {\bf or}, 
{\bf implies}, {\bf equiv}, and the unary prefix operator
{\bf not}. {\bf Boolean} allows you to simplify expressions
built from these operators, and to test properties like
equivalence, subset property etc. Also the reduction of
a boolean expression by a partial evaluation and combination
of its atomic parts is supported.

\subsection{Entering boolean expressions}

\ttindextype{BOOLEAN}{operator}
In order to distinguish boolean data expressions from 
boolean expressions in the \REDUCE programming
language (e.g. in an {\bf if} statement), each expression
must be tagged explicitly by an operator \f{boolean}.
Otherwise the boolean operators are not accepted in the
\REDUCE  algebraic mode input.
The first argument of \f{boolean} can be any boolean expression,
which may contain references to other boolean values.
\begin{verbatim}
    boolean (a and b or c);
    q := boolean(a and b implies c);
    boolean(q or not c);
\end{verbatim}
Brackets are used to override the operator precedence as usual.
The leafs or atoms of a boolean expression are those parts which
do not contain a leading boolean operator. These are
considered as constants during the boolean evaluation. There
are two pre-defined values:
\begin{itemize}
\item {\bf true}, {\bf t} or {\bf 1}
\item {\bf false}, {\bf nil} or {\bf 0}
\end{itemize}
These represent the boolean constants. In a result
form they are used only as {\bf 1} and {\bf 0}.

By default, a {\bf boolean} expression is converted  to a
disjunctive normal form, that is a form where terms are connected
by {\bf or} on the top level and each term is set of leaf
expressions, eventually preceded by {\bf not} and connected
by  {\bf and}. An operators {\bf or} or {\bf and} is omitted
if it would have only one single operand. The result of
the transformation is again an expression with leading 
operator {\bf boolean} such that the boolean expressions
remain separated from other algebraic data. Only the boolean
constants {\bf 0} and {\bf 1} are returned untagged.

On output, the
operators {\bf and} and {\bf or} are represented as
\verb+/\+ and \verb+\/+, respectively.
\begin{verbatim}
boolean(true and false);    ->   0
boolean(a or not(b and c)); -> boolean(not(b) \/ not(c) \/ a)
boolean(a equiv not c);     -> boolean(not(a)/\c \/ a/\not(c))
\end{verbatim}

\subsection{Normal forms}

The {\bf disjunctive} normal form is used by default. It
represents the ``natural'' view and allows us to represent
any form free or parentheses.
Alternatively a {\bf conjunctive} normal form can be
selected as simplification target, which is a form with
leading operator {\bf and}. To produce that form add the keyword  {\bf and}
as an additional argument to a call of {\bf boolean}.
\begin{verbatim}
boolean (a or b implies c); 
                    -> 
     boolean(not(a)/\not(b) \/ c)

boolean (a or b implies c, and); 
                    ->
     boolean((not(a) \/ c)/\(not(b) \/ c))
\end{verbatim}

Usually the result is a fully reduced disjunctive or conjuntive normal
form, where all redundant elements have been eliminated following the
rules

$ a \wedge b \vee \neg a \wedge b \longleftrightarrow b$

$ a \vee b \wedge \neg a \vee b \longleftrightarrow b$
 

Internally the full normal forms are computed
as intermediate result; in these forms each term contains
all leaf expressions, each one exactly once. This unreduced form is returned 
when you set the additional keyword {\bf full}:
\begin{verbatim}
boolean (a or b implies c, full);
                   ->
boolean(a/\b/\c \/ a/\not(b)/\c \/ not(a)/\b/\c \/ not(a)/\not(b)/\c

         \/ not(a)/\not(b)/\not(c))
\end{verbatim}

The keywords {\bf full} and {\bf and} may be combined.

\subsection{Evaluation of a boolean expression}

If the leafs of the boolean expression are algebraic expressions
which may evaluate to logical values because the environment
has changed (e.g. variables have been bound), you can re--investigate
the expression using the operator {\tt testbool} with the boolean
expression as argument. This operator tries to evaluate all
leaf expressions in \REDUCE boolean style. As many
terms as possible are replaced by their boolean values; the others
remain unchanged. The resulting expression is contracted to a
minimal form. The result {\bf 1} (= true) or {\bf 0} (=false)
signals that the complete expression could be evaluated. 

In the following example the leafs are built as numeric greater test.
For using ${\bf >}$ in the expressions the greater sign must
be declared operator first. The error messages are meaningless.
\begin{verbatim}
operator >;
fm:=boolean(x>v or not (u>v));
        ->
    fm := boolean(not(u>v) \/ x>v)

v:=10$

testbool fm;

   ***** u - 10 invalid as number
   ***** x - 10 invalid as number

        ->
   boolean(not(u>10) \/ x>10)

x:=3$
testbool fm;

   ***** u - 10 invalid as number

        ->
   boolean(not(u>10))

x:=17$

testbool fm;

   ***** u - 10 invalid as number
      
        ->
    1
 
\end{verbatim}


