\chapter{Lists}

A list\index{List} is an object consisting of a sequence of other objects
(including lists themselves), separated by commas and surrounded by
braces.  Examples of lists are:
\begin{verbatim}
        {a,b,c}

        {1,a-b,c=d}

        {{a},{{b,c},d},e}.
\end{verbatim}
The empty list is represented as
\begin{verbatim}
        {}.
\end{verbatim}

\section{Operations on Lists}\index{List operation}

Several operators in the system return their results as lists, and a user
can create new lists using braces and commas.  Alternatively, one can use
the operator LIST to construct a list.  An important class of operations
on lists are MAP and SELECT operations.  For details, please refer to the
chapters on MAP, SELECT and the FOR command.  See also the documentation
on the \hyperlink{ASSIST}{ASSIST}\extendedmanual{ (chapter~\ref{ASSIST})} package.

To facilitate the use of
lists, a number of operators are also available for manipulating
them. \texttt{PART(\meta{list},n)}\ttindextype{PART}{operator} for example will return the
$n^{th}$ element of a list. \texttt{LENGTH}\ttindextype{LENGTH}{operator} will return the
length of a list.  Several operators are also defined uniquely for lists.
For those familiar with them, these operators in fact mirror the
operations defined for Lisp lists.  These operators are as follows:

\subsection{LIST}
\hypertarget{operator:LIST}{}

The operator LIST is an alternative to the usage of curly brackets. LIST
accepts an arbitrary number of arguments and returns a list
of its arguments. This operator is useful in cases where operators
have to be passed as arguments. E.g.,
\begin{verbatim}
list(a,list(list(b,c),d),e);       ->  {{a},{{b,c},d},e}
\end{verbatim}

\subsection{FIRST}
\hypertarget{operator:FIRST}{}

This operator\ttindextype{FIRST}{operator} returns the first member of a list.  An error
occurs if the argument is not a list, or the list is empty.

\subsection{SECOND}
\hypertarget{operator:SECOND}{}

\texttt{SECOND}\ttindextype{SECOND}{operator} returns the second member of a list.  An error
occurs if the argument is not a list or has no second element.

\subsection{THIRD}
\hypertarget{operator:THIRD}{}

This operator\ttindextype{THIRD}{operator} returns the third member of a list.  An error
occurs if the argument is not a list or has no third element.

\subsection{REST}
\hypertarget{operator:REST}{}

\texttt{REST}\ttindextype{REST}{operator} returns its argument with the first element
removed.  An error occurs if the argument is not a list, or is empty.

\subsection{\textbf{.} (Cons) Operator}
\hypertarget{operator:CONS}{}
\hypertarget{reserved:consop}{}

This operator\ttindextype{. (CONS)}{operator} adds (``conses'') an expression to the
front of a list.  For example:
\begin{verbatim}
        a . {b,c}     ->   {a,b,c}.
\end{verbatim}

\subsection{APPEND}
\hypertarget{operator:APPEND}{}

This operator\ttindextype{APPEND}{operator} appends its first argument to its second to
form a new list.
\textit{Examples:}
\begin{verbatim}
        append({a,b},{c,d})     ->     {a,b,c,d}
        append({{a,b}},{c,d})   ->     {{a,b},c,d}.
\end{verbatim}

\subsection{REVERSE}
\hypertarget{operator:REVERSE}{}

The operator \texttt{REVERSE}\ttindextype{REVERSE}{operator} returns its argument with the
elements in the reverse order.  It only applies to the top level list, not
any lower level lists that may occur.  Examples are:\index{List operation}
\begin{verbatim}
        reverse({a,b,c})        ->     {c,b,a}
        reverse({{a,b,c},d})    ->     {d,{a,b,c}}.
\end{verbatim}

\subsection{List Arguments of Other Operators}
\hypertarget{switch:LISTARGS}{}
\hypertarget{command:LISTARGP}{}

If an operator other than those specifically defined for lists is given a
single argument that is a list, then the result of this operation will be
a list in which that operator is applied to each element of the list.  For
example, the result of evaluating \texttt{log\{a,b,c\}} is the expression
\texttt{\{LOG(A),LOG(B),LOG(C)\}}.

There are two ways to inhibit this operator distribution.  Firstly, the
switch \sw{LISTARGS},\ttindexswitch{LISTARGS} if on, will globally inhibit
such distribution.  Secondly, one can inhibit this distribution for a
specific operator by the declaration \texttt{LISTARGP}.\ttindextype{LISTARGP}{declaration} For
example, with the declaration \texttt{listargp log}, \texttt{log\{a,b,c\}}
 would evaluate to \texttt{LOG(\{A,B,C\})}.

If an operator has more than one argument, no such distribution occurs.

\subsection{Caveats and Examples}

Some of the natural list operations such as \textit{member} or
\textit{delete} are available only after loading the package
\textit{\hyperlink{ASSIST}{ASSIST}}\extendedmanual{ (chapter~\ref{ASSIST})}.

Please note that a non-list as second argument to CONS
(a "dotted pair" in LISP terms) is not allowed
and causes an "invalid as list" error.
\begin{verbatim}
 a := 17 . 4;

***** 17 4 invalid as list
\end{verbatim}
Also, the initialization of a scalar variable is not the empty list --
one has to set list type variables explicitly, as in the following
example:
\begin{verbatim}
 load_package assist;

 procedure lotto (n,m);
  begin scalar list_1_n, luckies, hit;
     list_1_n := {};
     luckies := {};
     for k:=1:n do list_1_n := k . list_1_n;
     for k:=1:m do
       << hit := part(list_1_n,random(n-k+1) + 1);
          list_1_n := delete(hit,list_1_n);
          luckies := hit . luckies >>;
     return luckies;
  end;                         % In Germany, try lotto (49,6);
\end{verbatim}

\textit{Another example:} Find all coefficients of a multivariate
polynomial with respect to a list of variables:

\begin{verbatim}
procedure allcoeffs(q,lis); % q : polynomial, lis: list of vars
   allcoeffs1 (list q,lis);

procedure allcoeffs1(q,lis);
  if lis={} then q else
    allcoeffs1(foreach qq in q join coeff(qq,first lis),
               rest lis);
\end{verbatim}

