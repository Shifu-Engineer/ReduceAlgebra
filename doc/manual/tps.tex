\documentstyle[11pt,reduce]{article}
\title{Truncated Power Series}
\date{}
\author{Alan Barnes \\
School of Engineering \& Applied Science  \\
Aston University, Aston Triangle, \\
Birmingham B4 7ET \\ GREAT BRITAIN  (now retired)\\
Email: Alan.Barnes45678@gmail.com \\[0.1in]
and \\[0.1in]
Julian Padget \\
School of Mathematics, University of Bath \\
Claverton Down, Bath, BA2 7AY \\
GREAT BRITAIN \\
Email: jap@maths.bath.ac.uk}
\begin{document}
\maketitle
\index{power series} \index{truncated power series}
\index{Barnes, Alan} \index{Padget, Julian}
\section{Introduction}
\index{Power series expansions}
This package implements formal power series expansions in one
variable using the domain mechanism of REDUCE. This means that power
series objects can be added, multiplied, differentiated etc. like other
first class objects in the system. A lazy evaluation scheme is used in
the package and thus terms of the series are not evaluated until they
are required for printing or for use in calculating terms in other
power series. The series are extendible giving the user the impression
that the full infinite series is being manipulated.  The errors that
can sometimes occur using series that are truncated at some fixed depth
(for example when a term in the required series depends on terms of an
intermediate series beyond the truncation depth) are thus avoided.

Below we give a brief description of the operators available in the
power series package together with some examples of their use.

\subsection{PS Operator}

Syntax:

\noindent{\tt PS}(EXPRN:{\em algebraic},DEPVAR:{\em kernel},
ABOUT:{\em algebraic}):{\em ps object}

\index{PS operator}
The {\tt PS} operator returns a  power series object
(a tagged domain element)
representing the univariate formal power series expansion of EXPRN with
respect to the dependent variable DEPVAR about the expansion point
ABOUT.  EXPRN may itself contain power series objects.

The algebraic expression ABOUT should simplify to an expression
which is independent of the dependent variable DEPVAR, otherwise
an error will result.  If ABOUT is the identifier {\tt INFINITY}
then the power series expansion about DEPVAR = $\infty$ is
obtained in ascending powers of 1/DEPVAR.

\index{PSEXPLIM operator}
If the command is terminated by a semi-colon, a power series object
representing EXPRN is compiled and then a number of terms of the
power series expansion are evaluated and printed.  The expansion is
carried out as far as the value specified by {\tt PSEXPLIM}.  If,
subsequently, the value of {\tt PSEXPLIM} is increased, sufficient
information is stored in the power series object to enable the
additional terms to be calculated without recalculating the terms
already obtained.

If the command is terminated by a dollar symbol, a power series object
is compiled, but at most one term is calculated at this stage.

If the function has a pole at the expansion point then the correct
Laurent series expansion will be produced.

\noindent The following examples are valid uses of {\tt PS}:
\begin{verbatim}
    psexplim 6;
    ps(log x,x,1);
    ps(e**(sin x),x,0);
    ps(x/(1+x),x,infinity);
    ps(sin x/(1-cos x),x,0);
\end{verbatim}

\index{power series ! of user defined function}
New user-defined functions may be expanded provided the user provides
LET rules giving

\begin{enumerate}
\item the value of the function at the expansion point
\item a differentiation rule for the new function.
\end{enumerate}

\noindent For example
\begin{verbatim}
    operator sech;
    forall x let df(sech x,x)= - sech x * tanh x;
    let sech 0 = 1;
    ps(sech(x**2),x,0);
\end{verbatim}
 
\index{power series ! of integral}
The power series expansion of an integral may also be obtained (even if
REDUCE cannot evaluate the integral in closed form).  An example of
this is

\begin{verbatim}
    ps(int(e**x/x,x),x,1);
\end{verbatim}
 
Note that if the integration variable is the same as the expansion
variable then REDUCE's integration package is not called; if on the
other hand the two variables are different then the integrator is
called to integrate each of the coefficients in the power series
expansion of the integrand.  The constant of integration is zero by
default.   
 
\subsection{PSEXPLIM Operator}

\index{PSEXPLIM Operator}
Syntax:

\hspace*{2em} {\tt PSEXPLIM}(UPTO:{\em integer}):{\em integer}

\hspace*{4em} or

\hspace*{2em} {\tt PSEXPLIM}():{\em integer}

Calling this operator sets an internal variable of the
TPS package to the value of UPTO (which should evaluate to an integer).
This internal variable controls how many terms of a power series are printed.
The value returned by {\tt PSEXPLIM} is the previous value of this variable.
The default value is six.
 
If {\tt PSEXPLIM} is called with no argument, the current value for
the expansion limit is returned.
 
\subsection{PSPRINTORDER Switch}

\index{PSPRINTORDER Switch}
Syntax:

\hspace*{2em} {\tt ON PSPRINTORDER}

\hspace*{4em} or

\hspace*{2em} {\tt OFF PSPRINTORDER}

When {\tt ON} this switch causes the remainder of the power series
to be printed in big-O notation.   Otherwise, three dots are printed.
 The default is {\tt ON}.

\subsection{PSORDLIM Operator}

\index{PSORDLIM operator}
Syntax:

\hspace*{2em} {\tt PSORDLIM}(UPTO:{\em integer}):{\em integer}

\hspace*{4em} or

\hspace*{2em} {\tt PSORDLIM}():{\em integer}

An internal variable is set to the value of {\tt UPTO} (which should
evaluate to an integer). The value returned is the previous value of
the variable.  The default value is 15.

If {\tt PSORDLIM} is called with no argument, the current value is
returned.

The significance of this control is that the system attempts to find
the order of the power series required, that is the order is the
degree of the first non-zero term in the power series.  If the order
is greater than the value of this variable an error message is given
and the computation aborts. This prevents infinite loops in examples
such as

\begin{verbatim}
        ps(1 - (sin x)**2 - (cos x)**2,x,0);
\end{verbatim}

where the expression being expanded is identically zero, but is not
recognized as such by REDUCE.


\subsection{PSTERM Operator}

\index{PSTERM operator}
Syntax:

\hspace*{2em} {\tt PSTERM}(TPS:{\em power series object},
NTH:{\em integer}):{\em algebraic}

The operator {\tt PSTERM} returns the NTH term of the existing
power series object TPS. If NTH does not evaluate to
an integer or TPS to a power series object an error results.  It
should be noted that an integer is treated as a power series.


\subsection{PSORDER Operator}

\index{PSORDER operator}
Syntax:

\hspace*{2em} {\tt PSORDER}(TPS:{\em power series object}):{\em integer}

The operator {\tt PSORDER} returns the order, that is the degree of
the first non-zero term, of the power series object TPS.
TPS should evaluate to a power series object or an error results. If
TPS is zero, the identifier {\tt UNDEFINED} is returned.

\subsection{PSSETORDER Operator}

\index{PSSETORDER operator}
Syntax:

\hspace*{2em} {\tt PSSETORDER}(TPS:{\em power series object},
ORD:{\em integer}):{\em integer}

The operator {\tt PSSETORDER} sets the order of the power series
TPS to the value ORD, which should evaluate to an integer. If
TPS does not evaluate to a power series object, then an error
occurs. The value returned by this operator is the previous order of
TPS, or 0 if the order of TPS was undefined.  This
operator is useful for setting the order of the power series of a
function defined by a differential equation in cases where the power
series package is inadequate to determine the order automatically.


\subsection{PSDEPVAR Operator}

\index{PSDEPVAR operator}
Syntax:

\hspace*{2em} {\tt PSDEPVAR}(TPS:{\em power series object})
:{\em identifier}

The operator {\tt PSDEPVAR} returns the expansion variable of the
power series object TPS. TPS should evaluate to a power
series object or an integer, otherwise an error results. If TPS
is an integer, the identifier {\tt UNDEFINED} is returned.

\subsection{PSEXPANSIONPT operator}

\index{PSEXPANSIONPT operator}
Syntax:

\hspace*{2em}
{\tt PSEXPANSIONPT}(TPS:{\em power series object}):{\em algebraic}

The operator {\tt PSEXPANSIONPT} returns the expansion point of the
power series object TPS. TPS should evaluate to a power
series object or an integer, otherwise an error results. If TPS
is integer, the identifier {\tt UNDEFINED} is returned. If the
expansion is about infinity, the identifier {\tt INFINITY} is
returned.

\subsection{PSFUNCTION Operator}

\index{PSFUNCTION operator}
Syntax:

\hspace*{2em}
{\tt PSFUNCTION}(TPS:{\em power series object}):{\em algebraic}

The operator {\tt PSFUNCTION} returns the function whose expansion
gave rise to the power series object TPS. TPS should
evaluate to a power series object or an integer, otherwise an error
results.

\subsection{PSCHANGEVAR Operator}

\index{PSCHANGEVAR operator}
Syntax:

\hspace*{2em} {\tt PSCHANGEVAR}(TPS:{\em power series object},
X:{\em kernel}):{\em power series object}

The operator {\tt PSCHANGEVAR} changes the dependent variable of the
power series object TPS to the variable X. TPS
should evaluate to a power series object and X to a kernel,
otherwise an error results.  Also X should not appear as a
parameter in TPS. The power series with the new dependent
variable is returned.

\subsection{PSREVERSE Operator}

\index{PSREVERSE operator}
Syntax:

\hspace*{2em}
{\tt PSREVERSE}(TPS:{\em power series object}):{\em power series}

Power series reversion.  The power series TPS is functionally
inverted.  Four cases arise:

\begin{enumerate}
\item If the order of the series is 1, then the expansion point of the
inverted series is 0. 

\item If the order is 0 {\em and} if the first order term in TPS
is non-zero, then the expansion point of the inverted series is taken
to be the coefficient of the zeroth order term in TPS.

\item If the order is -1 the expansion point of the inverted series
is the point at infinity.  In all other cases a REDUCE error is
reported because the series cannot be inverted as a power series.
Puiseux \index{Puiseux expansion} expansion would be required to
handle these cases. 

\item If the expansion point of TPS is finite it becomes the
zeroth order term in the inverted series. For expansion about 0 or the
point at infinity the order of the inverted series is one.
\end{enumerate}

If TPS is not a power series object after evaluation an error results.

\noindent Here are some examples:
\begin{verbatim}
        ps(sin x,x,0);
        psreverse(ws); % produces series for asin x about x=0.
        ps(exp x,x,0);
        psreverse ws; % produces series for log x about x=1.
        ps(sin(1/x),x,infinity);
        psreverse(ws); % series for 1/asin(x) about x=0.
\end{verbatim}

\subsection{PSCOMPOSE Operator}

\index{PSCOMPOSE operator}
Syntax:

\hspace*{2em} {\tt PSCOMPOSE}(TPS1:{\em power series},
TPS2:{\em power series}):{\em power series}

\index{power series ! composition}
{\tt PSCOMPOSE} performs power series composition.
The power series TPS1 and TPS2 are functionally composed.
That is to say that TPS2 is substituted for the expansion
variable in TPS1 and the result expressed as a power series. The
dependent variable and expansion point of the result coincide with
those of TPS2.  The following conditions apply to power series
composition:

\begin{enumerate}
\item If the expansion point of TPS1 is 0 then the order of the
TPS2 must be at least 1.

\item If the expansion point of TPS1 is finite, it should
coincide with the coefficient of the zeroth order term in TPS2.
The order of TPS2 should also be non-negative in this case.

\item If the expansion point of TPS1 is the point at infinity
then the order of TPS2 must be less than or equal to -1.

\end{enumerate}

If these conditions do not hold the series cannot be composed (with
the current algorithm terms of the inverted series would involve
infinite sums) and a REDUCE error occurs.

\noindent Examples of power series composition include the following.

\begin{verbatim}
    a:=ps(exp y,y,0);  b:=ps(sin x,x,0); 
    pscompose(a,b);
    % Produces the power series expansion of exp(sin x)
    % about x=0.

    a:=ps(exp z,z,1); b:=ps(cos x,x,0);
    pscompose(a,b);
    % Produces the power series expansion of exp(cos x)
    % about x=0.

    a:=ps(cos(1/x),x,infinity);  b:=ps(1/sin x,x,0);
    pscompose(a,b);
    % Produces the power series expansion of cos(sin x)
    % about x=0.
\end{verbatim}

\subsection{PSSUM Operator}

\index{PSSUM operator}
Syntax:

\begin{tabbing}
\hspace*{2em} {\tt PSSUM}(\=J:{\em kernel} = LOWLIM:{\em integer},
COEFF:{\em algebraic}, X:{\em kernel}, \\ 
\> ABOUT:{\em algebraic}, POWER:{\em algebraic}):{\em power series}
\end{tabbing}

The formal power series sum for J from LOWLIM to {\tt INFINITY} of 

\begin{verbatim}
      COEFF*(X-ABOUT)**POWER
\end{verbatim}

or if ABOUT is given as {\tt INFINITY}

\begin{verbatim}
      COEFF*(1/X)**POWER
\end{verbatim}

is constructed and returned. This enables power series whose general
term is known to be constructed and manipulated using the other
procedures of the power series package. 

J and X should be distinct simple kernels. The algebraics
ABOUT,  COEFF and POWER should not depend on the
expansion variable X, similarly the algebraic ABOUT should
not depend on the summation variable J.  The algebraic POWER should be
a strictly increasing integer valued function of J for J in the range
LOWLIM to {\tt INFINITY}.

\begin{verbatim}
   pssum(n=0,1,x,0,n*n);
   % Produces the power series summation for n=0 to
   % infinity of x**(n*n).

   pssum(m=1,(-1)**(m-1)/(2m-1),y,1,2m-1);
   % Produces the power series expansion of atan(y-1)
   % about y=1.

   pssum(j=1,-1/j,x,infinity,j);
   % Produces the power series expansion of log(1-1/x)
   % about the point at infinity.

   pssum(n=0,1,x,0,2n**2+3n) + pssum(n=1,1,x,0,2n**2-3n);
   % Produces the power series summation for n=-infinity
   % to +infinity of x**(2n**2+3n).
\end{verbatim}


\subsection{PSTAYLOR Operator}

Syntax:

\noindent{\tt PSTAYLOR}(EXPRN:{\em algebraic},DEPVAR:{\em kernel},
ABOUT:{\em algebraic}):{\em ps object}

\index{PSTAYLOR operator}
The {\tt PSTAYLOR} operator returns a  power series object
(a tagged domain element)
representing the univariate formal Taylor series expansion of EXPRN with
respect to the dependent variable DEPVAR about the expansion point
ABOUT.  Unlike the operator PS it directly evaluates the nth derivative of the 
expression  EXPRN wrt the variable DEPVAR at the expansion point ABOUT to find
the nth term of the series.   Poles (and other singularities) at the expansion 
point will cause an error -- PS and TAYLOR are more robust in this respect.
The PSTAYLOR operator may be useful in contexts where PS fails to build a suitable
 recurrence relation automatically and reports too deep a recursion in 
{\tt ps!:unknown!-crule}. A typical example is the expansion of the $\Gamma$ 
function about an expansion point which is not a non-positive integer.

The algebraic expression ABOUT should simplify to an expression
which is independent of the dependent variable DEPVAR, otherwise
an error will result.  

If ABOUT is the identifier {\tt INFINITY}
then the power series expansion about DEPVAR = $\infty$ is
obtained in ascending powers of 1/DEPVAR.

\subsection{PSCOPY Operator}

\index{PSCOPY operator}
Syntax:

\hspace*{2em} {\tt PSCOPY}(TPS:{\em power series}):{\em power series}

This procedure returns a copy of the power series {\tt TPS}. The copy
has no shared sub-structures in common with the original series.  This
enables substitutions to be performed on the series without
side-effects on previously computed objects. For example:

\begin{verbatim}
    clear a;
    b := ps(sin(a*x)), x, 0);
    b where a => 1;
\end{verbatim}

will result in {\tt a} being set to 1 in each of the terms of the
power series and the resulting expressions being simplified. Owing to
the way power series objects are implemented using Lisp vectors, this
has the side-effect that the value of {\tt b} is changed.  This may be
avoided by copying the series with {\tt PSCOPY} before applying the
substitution, thus:

\begin{verbatim}
    b := ps(sin(a*x)), x, 0);
    pscopy b where a => 1;
\end{verbatim}

\subsection{PSTRUNCATE Operator}

\index{PSTRUNCATE operator}
Syntax:

\hspace*{2em} {\tt PSTRUNCATE}(TPS:{\em power series} POWER: {\em integer)}
:{\em algebraic}

This procedure truncates the power series {\tt TPS} discarding terms
of order higher than {\tt POWER}. The series is extended automtically
if the value of {\tt POWER} is greater than the order of last term
calculated to date.

\begin{verbatim}
    b := ps(sin x, x, 0);
    a := pstruncate(b, 11);
\end{verbatim}

will result in {\tt a} being set to the eleventh order polynomial
resulting in truncating the series for $sin x$ after the term
involving $x^{11}$. 

If {\tt POWER} is less than the order of the series then $0$ is
returned.  If {\tt POWER} does not simplify to an integer or if 
{\tt TPS} is not a power series object then Reduce errors result.

\subsection{Arithmetic Operations}

\index{power series ! arithmetic}
As power series objects are domain elements they may be combined
together in algebraic expressions in algebraic mode of REDUCE in the
normal way.
 
For example if A and B are power series objects then the commands
such as:

\index{+ ! power series} \index{- ! power series}
\index{/ ! power series} \index{* ! power series}
\index{** ! power series}
\begin{verbatim}
    a*b;
    a/b;
    a**2+b**2;
\end{verbatim}

will produce power series objects representing the product,quotient
and the sum of the squares of the power series objects A and B
respectively.
 
\subsection{Differentiation}

\index{power series ! differentiation}
If A is a power series object depending on X then the input
{\tt df(a, x);} will produce the power series expansion of the
derivative of A with respect to X.

{\em Note} however that currently the input  {\tt int(a, x);} will
not work as intended; instead one must input
{\tt ps(int(a, x),x,0);}
in order to obtain the power series expansion of the integral of
{\tt a}.

\section{Restrictions and Known Bugs}

If A is a power series object and X is a variable
which evaluates to itself then currently expressions such as {\tt a*x}
do not evaluate to a single power series object (although the 
result is formally valid).  Instead use {\tt ps(a*x,x,0)} {\em etc.}.

Similarly expressions such as {\tt sin(A)} where {\tt A} is a PS object
currently will not be expanded. For example:

\begin{verbatim}
    a:=ps(1/(1+x),x,0);
    b:=sin a;
\end{verbatim}

will not expand {\tt sin(1/(1+x))} as a power series. In fact

\begin{verbatim}
          SIN(1 - X + X**2 - X**3 + .....)
\end{verbatim}

will be returned. However,

\begin{verbatim} 
    b:=ps(sin(a),x,0);
\end{verbatim}

or

\begin{verbatim} 
    b:=ps(sin(1/(1+x)),x,0);
\end{verbatim}

should work as intended.

The handling of functions with essential singularities is currently
erratic: usually an error message

\hspace*{2em} {\tt ***** Essential Singularity}

or

\hspace*{2em} {\tt ***** Logarithmic Singularity}

occurs but occasionally a division by
zero error or some drastic error like (for PSL) binding stack
overflow may occur.
 
There is no simple way to write the results of power series
calculation to a file and read them back into REDUCE at a later
stage.
\end{document}
