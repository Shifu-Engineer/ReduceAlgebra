%%%%%%%%%%%%%%%%%%%%%%%%%%%%%%%%%%%%%%%%%% BeginCodemist
%%% Taken from Reduce.sty

% \s{...} is a sentential form in descriptions. Enclosed \em text in <...>
\newcommand{\s}[1] {$<${\em #1}$>$}

% \meta{...} is an alternative sentential form in descriptions using \it.
%\newcommand{\meta}[1]{\mbox{$\langle$\it#1\/$\rangle$}}

% \k{...} is a keyword. Just do in bold for the moment.
\renewcommand{\k}[1] {{\bf #1}}

% \f is a function name. Just do this as tt.
\newcommand{\f}[1] {{\tt #1}}

% An example macro for numbering and indenting examples.
\newcounter{examplectr}
\newcommand{\example}{\refstepcounter{examplectr}
\noindent{\bf Example \theexamplectr}}

\part{Additional {\REDUCE} Documentation}
\setcounter{examplectr}{0}

The documentation in this section was written using to a large part
the \LaTeX\ files provided by the authors, and distributed with
\REDUCE.  There has been extensive editing and much rewriting, so
the responsibility for this part of the manual rests with the editor,
John Fitch.  It is hoped that this version of the documentation
contains sufficient information about the facilities available that a
user may be able to progress.  It deliberately avoids discussions of
algorithms or advanced use; for these the package author's own
documentation should be consulted.  In general the package
documentation will contain more examples and in some cases additional
facilities such as tracing.

