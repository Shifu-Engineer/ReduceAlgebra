\chapter[XCOLOR: Color factor in gauge theory]%
        {XCOLOR: Calculation of the color factor in non-abelian gauge
field theories}
\label{XCOLOR}
\typeout{{XCOLOR: Calculation of the color factor in non-abelian gauge
field theories}}

{\footnotesize
\begin{center}
A. Kryukov \\
Institute for Nuclear Physics, Moscow State University \\
119899, Moscow, Russia \\[0.05in]
e--mail: kryukov@npi.msu.su
\end{center}
}
\ttindex{XCOLOR}

XCOLOR calculates the colour factor in non-abelian gauge field
theories.  It provides two commands and two operators.

\noindent{\tt SUdim} integer\ttindex{SUdim}

Sets the order of the SU group.  The default value is 3.

\noindent{\tt SpTT} expression\ttindex{SpTT}

Sets the normalisation coefficient A in the equation
$Sp(T_i T_j) = A \Delta(i,j)$.  The default value is 1/2.

\noindent{\tt QG}(inQuark, outQuark, Gluon)\ttindex{QG}

Describes the quark-gluon vertex.  The parameters may be any identifiers.
The first and second of then must be in- and out- quarks correspondingly.
Third one is a gluon.

\noindent{\tt G3}(Gluon1, Gluon2, Gluon3)\ttindex{G3}

Describes the three-gluon vertex.  The parameters may be any identifiers.
The order of gluons must be clockwise.

In terms of QG and G3 operators one can input a diagram in ``color'' space as
a product of these operators.  For example
\newpage
\begin{verbatim}
                e1
             ---->---
            /        \
           /          \
          |      e2    |
        v1*............*v2
          |            |
           \          /
            \    e3  /
             ----<---
\end{verbatim}
where \verb+--->---+ is a quark and \verb+.......+ is a gluon.

The related \REDUCE\ expression is {\tt QG(e3,e1,e2)*QG(e1,e3,e2)}.

