\chapter{ALGINT: Integration of square roots}
\label{ALGINT}
\typeout{{ALGINT: Integration of square roots}}

{\footnotesize
\begin{center}
James Davenport \\
School of Mathematical Sciences \\
University of Bath \\
Bath BA2 7AY \\
England \\[0.05in]
e--mail: jhd@maths.bath.ac.uk
\end{center}
}

The package supplies no new functions, but extends the {\tt
INT}\ttindex{INT} operator for indefinite integration so it can handle
a wider range of expressions involving square roots.  When it is
loaded the controlling switch {\tt ALGINT}\ttindex{ALGINT} is turned
on.  If it is desired to revert to the standard integrator, then it
may be turned off.  The normal integrator can deal with some square
roots but in an unsystematic fashion.

\begin{verbatim}
1: load_package algint;

2: int(sqrt(sqrt(a^2+x^2)+x)/x,x);

                                 2    2
sqrt(a)*atan((sqrt(a)*sqrt(sqrt(a  + x ) + x)

                     2    2
              *sqrt(a  + x )

                                    2    2
               - sqrt(a)*sqrt(sqrt(a  + x ) + x)*a

                                    2    2
               - sqrt(a)*sqrt(sqrt(a  + x ) + x)*x)/(2
\end{verbatim}
\newpage
\begin{verbatim}
                  2                  2    2
                *a )) + 2*sqrt(sqrt(a  + x ) + x)

                          2    2
 + sqrt(a)*log(sqrt(sqrt(a  + x ) + x) - sqrt(a))

                          2    2
 - sqrt(a)*log(sqrt(sqrt(a  + x ) + x) + sqrt(a))

3: off algint;

4: int(sqrt(sqrt(a^2+x^2)+x)/x,x);

                2    2
     sqrt(sqrt(a  + x ) + x)
int(-------------------------,x)
                x

\end{verbatim}

There is also a switch {\tt TRA},\ttindex{TRA} which may be set on to
provide detailed tracing of the algorithm used.  This is not
recommended for casual use.

