
A single reaction equation is an expression of the form
\begin{quote}
 \meta{n1}\meta{s1} + \meta{n2}\meta{s2} + \ldots \texttt{->} \meta{n3}\meta{s3} + \meta{n4}\meta{s4} + \ldots
\end{quote}
 or
\begin{quote} \meta{n1}\meta{s1} + \meta{n2}\meta{s2} + \ldots \texttt{<>} \meta{n3}\meta{s3} + \meta{n4}\meta{s4} + \ldots
\end{quote}
where the \meta{si} are arbitrary names of species (\REDUCE{} symbols)
and the \meta{ni} are positive integer numbers. The number 1
can be omitted. The connector \texttt{->} describes a one way reaction,
while \texttt{<>} describes a forward and backward reaction.

A reaction system is a list of reaction equations, each of them
optionally followed by one or two expressions for the rate
constants. A rate constant can a number, a symbol or an 
arbitrary \REDUCE{} expression. If a rate constant is missing,
an automatic constant of the form RATE(n) (where n is an
integer counter) is generated. For double reactions the
first constant is used for the forward direction, the second
one for the backward direction.

The names of the species are collected in a list bound to
the \REDUCE{} variable SPECIES. This list is automatically filled
during the processing of a reaction system. The species enter
in an order corresponding to their appearance in the reaction
system and the resulting ode's will be ordered in the same manner.

If a list of species is preassigned to the variable
SPECIES either explicitly or from previous operations, the 
given order will be maintained and will dominate the formatting
process. So the ordering of the result can be easily influenced
by the user.

Syntax:

 reac2ode \{ \meta{reaction} {[},\meta{rate} {[},\meta{rate}{]}{]} 
 {[},\meta{reaction} {[},\meta{rate} {[},\meta{rate}{]}{]}{]} 
 .... 
 \};

where two rates are applicable only for \texttt{<>} reactions.

Result is a system of explicit ordinary differential
 equations with polynomial righthand sides. As side
 effect the following variables are set:

 lists:

 rates: list of the rates in the system

 species: list of the species in the system

 matrices:

 inputmat: matrix of the input coefficients

 outputmat: matrix of the output coefficients

In the matrices the row number corresponds to the input reaction 
number, while the column number corresponds to the species index.
Note: if the rates are numerical values, it will be in most cases
appropriate to switch on \REDUCE{} rounded mode for floating
point numbers. That is


on rounded;

Inputmat and outputmat can be used for linear algebra type 
investigations of the reaction system. The classical reaction 
matrix is the difference of these matrices; however, the two 
matrices contain more information than their differences because 
the appearance of a species on both sides is not reflected by 
the reaction matrix.

EXAMPLES:
This input
\begin{verbatim}
% Example taken from Feinberg (Chemical Engineering):

   species := {A1,A2,A3,A4,A5};

   reac2ode { A1 + A4 <> 2A1, rho, beta,
              A1 + A2 <> A3, gamma, epsilon,
              A3      <> A2 + A5, theta, mue};
\end{verbatim}
gives the output
\begin{verbatim}
                             2
{DF(A1,T)=RHO*A1*A4 - BETA*A1 - GAMMA*A1*A2 + EPSILON*A3,

 DF(A2,T)= - GAMMA*A1*A2 + EPSILON*A3 + THETA*A3 - MUE*A2*A5,

 DF(A3,T)=GAMMA*A1*A2 - EPSILON*A3 - THETA*A3 + MUE*A2*A5,

                                2
 DF(A4,T)= - RHO*A1*A4 + BETA*A1 ,

 DF(A5,T)=THETA*A3 - MUE*A2*A5}
\end{verbatim}
The corresponding matrices are
\begin{verbatim}
 inputmat;

[ 1 0 0 1 0 ]
[           ]
[ 1 1 0 0 0 ]
[           ]
[ 0 0 1 0 0 ]

 outputmat;

[ 2 0 0 0 0 ]
[           ] 
[ 0 0 1 0 0 ] 
[           ] 
[ 0 1 0 0 1 ] 

% computation of the classical reaction matrix as difference
% of output and input matrix:

  reactmat := outputmat-inputmat;

             [ 1   0   0   -1  0 ]
             [                   ] 
REACTMAT :=  [ -1  -1  1   0   0 ] 
             [                   ] 
             [ 0   1   -1  0   1 ]

% Example with automatic generation of rate constants
% and automatic extraction of species

   species := {};

   reac2ode { A1 + A4 <> 2A1, 
              A1 + A2 <> A3,
                   a3 <> A2 + A5};

new species: A1
new species: A4
new species: A3
new species: A2
new species: A5


               2
{DF(A1,T)= - A1 *RATE(2) + A1*A4*RATE(1) - A1*A2*RATE(3) + 

     A3*RATE(4),

            2
 DF(A4,T)=A1 *RATE(2) - A1*A4*RATE(1),

 DF(A2,T)= - A1*A2*RATE(3) - A2*A5*RATE(6) + A3*RATE(5) + A3*RATE(4),

 DF(A3,T)=A1*A2*RATE(3) + A2*A5*RATE(6) - A3*RATE(5) - A3*RATE(4),

 DF(A5,T)= - A2*A5*RATE(6) + A3*RATE(5)\}

% Example with rates computed from numerical expressions

   species := {};

   reac2ode { A1 + A4 <> 2A1, 17.3* 22.4^1.5,
                              0.04* 22.4^1.5 };

new species: A1
new species: A4

                       2
{DF(A1,T)= - 4.24065*A1 + 1834.08*A1*A4,

                    2
 DF(A4,T)=4.24065*A1 - 1834.08*A1*A4}
\end{verbatim}