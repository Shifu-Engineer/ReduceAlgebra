Some of \crlibm's functions need high precision square roots.  They
are not intended to be used outside \crlibm. In particular, we do
currently not guarantee the correct rounding of their results because
this property is not needed for our purposes. Their implementation
does not handle all possible special cases ($x < 0$, $\nan$, $\infty$
etc.) neither.

We currently provide two C macros computing the square root of a
double precision argument either in double-double precision with at
least $100$ correct bits (in faithful rounding) or in triple-double
precision with an accuracy of at least $146$ bits (in faithful
rounding). The corresponding macros are called \SqrtD~ and
\SqrtT.

The implementation of these macros was guided by the following
principles:
\begin{itemize}
\item no dependency on other \texttt{libm}s, so avoidance of
bootstrapping a Newton iteration by a double precision square root
implemented elsewhere,
\item high efficiency,
\item a small memory footprint,
\item the possible use of hardware support on some platforms in the
future.
\end{itemize}
\subsubsection{Overview of the algorithm}
The algorithm uses a combination of polynomial approximation and
Newton iteration.

After handling some special cases, the argument $x = 2^{E^\prime}
\cdot m^\prime$ is reduced into its exponent $E^\prime$ stored in
integer and its fractional part $m^\prime$ stored as a double
precision number.  This argument reduction is obviously exact. The two
values are then adjusted as follows:
\vspace{-3mm}
\begin{center}
  \begin{tabular}{cc}
    \begin{minipage}{60mm}
      $$E = \left \lbrace \begin{array}{ll} E^\prime & \mbox{ if } \exists n \in \N \mbox{ . } E^\prime = 2n\\
          E^\prime +1 & \mbox{ otherwise} \end{array} \right.$$
    \end{minipage}
    &
    \begin{minipage}{60mm}
      $$m = \left \lbrace \begin{array}{ll} m^\prime & \mbox{ if } \exists n \in \N \mbox{ . } E^\prime = 2n \\
          \frac{m^\prime}{2} & \mbox{ otherwise } \end{array} \right.$$
    \end{minipage}
  \end{tabular}
\end{center} 
One easily checks that $\frac{1}{2} \leq m \leq 2$ and that $E$ is always even. Thus
$$\sqrt{x} = \sqrt{2^E \cdot m} = 2^{\frac{E}{2}} \cdot \sqrt{m} =
2^{\frac{E}{2}} \cdot m \cdot \frac{1}{\sqrt{m}}$$ The algorithm
therefore approximates $\hat{r} = \frac{1}{\sqrt{m}}$ and reconstructs
the square root by multiplying by $m$ and exactly by
$2^{\frac{E}{2}}$.

The reciprocal square root $\hat{r}$ is approximated in two
steps. First, a polynomial approximation yields to $r_0 = \hat{r}
\cdot \left( 1 + \epsilon_1 \right)$, which is exact to about $8$
bits.  In a second step, this approximation is refined by a Newton
iteration that approximately doubles its accuracy at each step. So for
a double-double result, $4$ iterations are needed and for a
triple-double result $5$.

The initial polynomial approximation is less exact than the one
provided by Itanium's \texttt{} operation, which allows for using this
hardware assistance in the future.

\subsubsection{Special case handling}
The square root of a double precision number can never be
subnormal. In fact, if $\sqrt{x} \leq 2^{-1021}$, $x = \sqrt{x}^2 \leq
2^{-1042441}$, a value that is is not representable in double
precision. 

Concerning subnormals in argument, it to be mentioned that still
$E^\prime$ and $m^\prime$ can be found such that $x = 2^{E^\prime}
\cdot m$ exactly and $1 \leq m^\prime \leq 2$. Only the extraction
sequence must be modified: $x$ is first multiplied by $2^{52}$ where
$E^\prime$ is set to $-52$. The double number $x$ is thus no longer a
subnormal an integer handling can extract its mantissa easily. The
extraction of the exponent takes into account the preceeding bias of
$E^\prime$. The case $x = 0$ is filtered out before. Obviously
$\sqrt{0} = 0$ is returned for this argument.

The special cases $x < 0$, $x = \pm \infty$ and $x = \nan$ are not
handled since they can be easily excluded by the code using the square
root macros.

Special case handling is implemented as follows:
\begin{lstlisting}[caption={Special case handling},firstnumber=1]
/* Special case x = 0 */
if (x == 0) {
  *resh = x;
  *resl = 0;
} else {

  E = 0;

  /* Convert to integer format */
  xdb.d = x;
    
  /* Handle subnormal case */
  if (xdb.i[HI] < 0x00100000) {
    E = -52;
    xdb.d *= ((db_number) ((double) SQRTTWO52)).d; 	  /* make x a normal number */ 
  }
  
  /* Extract exponent E and mantissa m */
  E += (xdb.i[HI]>>20)-1023; 
  xdb.i[HI] = (xdb.i[HI] & 0x000fffff) | 0x3ff00000;
  m = xdb.d;
  
  /* Make exponent even */
  if (E & 0x00000001) {
    E++;
    m *= 0.5;    /* Suppose now 1/2 <= m <= 2 */
  }

  /* Construct sqrt(2^E) = 2^(E/2) */
  xdb.i[HI] = (E/2 + 1023) << 20;
  xdb.i[LO] = 0;
\end{lstlisting}

\subsubsection{Polynomial approximation}
The reciprocal square root $\hat{r} = \frac{1}{\sqrt{m}}$ is
approximated in the domain $m \in \left[ \frac{1}{2}; 2 \right]$ by a
polynomial $p\left( m \right) = \sum\limits_{i=0}^4 c_i \cdot m^i$ of
degree $4$. The polynomial's coefficients $c_0$ through $c_4$ are
stored in double precision.  The following values are used:
\begin{eqnarray*}
c_0 & = & 2.50385236695888790947606139525305479764938354492188 \\  
c_1 & = & -3.29763389114324168005509818613063544034957885742188  \\
c_2 & = & 2.75726076139124520736345402838196605443954467773438   \\
c_3 & = & -1.15233725777933848632983426796272397041320800781250  \\
c_4 & = & 0.186900066679800969104974228685023263096809387207031 
\end{eqnarray*}

The relative approximation error $\epsilon_{\mbox{\tiny approx}} =
\frac{p\left( m\right) - \hat{r}}{\hat{r}}$ is bounded by $\left \vert
\epsilon_{\mbox{\tiny approx}} \right \vert \leq 2^{-8.32}$ for $m \in
\left[ \frac{1}{2}; 2 \right]$.

The polynomial is evaluated in double precision using Horner's
scheme. There may be some cancellation in the different steps but the
relative arithmetical error $\epsilon_{\mbox{\tiny arithpoly}}$ is
always less in magnitude than $2^{-30}$. This will be shown in more
detail below.

The code implementing the polynomial approximation reads:
\begin{lstlisting}[caption={Polynomial approximation},firstnumber=1]
r0 = SQRTPOLYC0 + m * (SQRTPOLYC1 + m * (SQRTPOLYC2 + m * (SQRTPOLYC3 + m * SQRTPOLYC4)));
\end{lstlisting}
So 4 double precision multiplications and 4 additions are needed for computing the
initial approximation. They can be replaced by 4 FMA instructions, if available.

\subsubsection{Double and double-double Newton iteration}
The polynomial approximation is then refined using the following iteration scheme:
$$r_{i+1} = \frac{1}{2} \cdot r_i \cdot (3 - m \cdot r_i^2)$$
If the arithmetic operations were exact, one would obtain the following error estimate:
\begin{eqnarray*}
\epsilon_{i+1} & = & \frac{r_i - \hat{r}}{\hat{r}} \\ & = &
\frac{\frac{1}{2} \cdot r_i \cdot \left(3 - m \cdot r_i^2\right) -
\hat{r}}{\hat{r}} \\ 
& = & \frac{\frac{1}{2} \cdot \hat{r} \cdot
\left( 1 + \epsilon_i \right) \cdot \left( 3 - m \cdot \hat{r}^2 \cdot
\left( 1 + \epsilon_i \right)^2 \right) - \hat{r}}{\hat{r}} \\
& = & \frac{1}{2} \cdot \left( 1 + \epsilon_i \right) \cdot \left( 3 - m \cdot \frac{1}{m} \cdot \left( 1 + 
\epsilon_i\right)^2 \right) - 1 \\
& = & \frac{1}{2} \cdot \left( 1 + \epsilon_i \right) \cdot \left( 3 - 1 - 2 \cdot \epsilon_i - \epsilon_i^2  
\right) - 1 \\
& = & \left( 1 + \epsilon_i \right) \cdot \left( 1 - \epsilon_i - \frac{1}{2} \cdot \epsilon_i^2  
\right) - 1 \\
& = & 1 - \epsilon_i - \frac{1}{2} \cdot \epsilon_i^2 + \epsilon_i - \epsilon_i^2 - \frac{1}{2} \cdot \epsilon_i^3 - 1\\
& = & - \frac{3}{2} \cdot \epsilon_i^2 - \frac{1}{2} \cdot \epsilon_i^3
\end{eqnarray*}
So the accuracy of the approximation of the reciprocal square root is doubled at each step.

Since the initial accuracy is about $8$ bits, it is possible to iterate two times on pure double precision 
without any considerable loss of accuracy. After the two iterations about $31$ bits will be correct.
The macro implements therefore:
\begin{lstlisting}[caption={Newton iteration - double precision steps},firstnumber=1]
r1 = 0.5 * r0 * (3 - m * (r0 * r0));
r2 = 0.5 * r1 * (3 - m * (r1 * r1));
\end{lstlisting}
For these two iterations, 8 double precision multiplications and 2 additions are needed.

The next iteration steps must be performed in double-double precision
because the $53$ bit mantissa of a double cannot contain the about
$60$ bit exact value $m \cdot r_2^2 \approx 1$ before cancellation in
the substraction with $3$ and the multiplication by $r_2$.

In order to exploit maximally the parallelism in the iteration equation, we rewrite it as
\begin{eqnarray*}
r_{3} & = & \frac{1}{2} \cdot r_2 \cdot \left( 3 - m \cdot r_2^2 \right) \\
& = & \left( r_2 + \frac{1}{2} \cdot r_2 \right) - \frac{1}{2} \cdot \left( m \cdot r_2 \right) \cdot 
\left( r_2 \cdot r_2 \right)
\end{eqnarray*}
Since multiplications by integer powers of $2$ are exact, it is
possible to compute $r_2 + \frac{1}{2} \cdot r_2$ exactly as a
double-double. Concurrently it is possible to compute $m \cdot r_2$
and $r_2 \cdot r_2$ exactly as double-doubles by means of an exact
multiplication.  The multiplication $\left( m \cdot r_2 \right) \cdot
\left( r_2 \cdot r_2 \right)$ is then implemented as a double-double
multiplication.  The multiplication by $\frac{1}{2}$ of the value
obtained is exact and can be performed pairwise on the
double-double. A final double-double addition leads to $r_3 = \left(
r_2 + \frac{1}{2} \cdot r_2 \right) - \frac{1}{2} \cdot \left( m \cdot
r_2 \right) \cdot \left( r_2 \cdot r_2 \right)$. Here, massive
cancellation is no longer possible since the values added are
approximately $\frac{3}{2} \cdot r_2$ and $\frac{1}{2} \cdot r_2$.

These steps are implemented as follows:
\begin{lstlisting}[caption={Newton iteration - first double-double step},firstnumber=1]
Mul12(&r2Sqh, &r2Sql, r2, r2);    Add12(r2PHr2h, r2PHr2l, r2, 0.5 * r2);
Mul12(&mMr2h, &mMr2l, m, r2);
Mul22(&mMr2Ch, &mMr2Cl, mMr2h, mMr2l, r2Sqh, r2Sql);

MHmMr2Ch = -0.5 * mMr2Ch;
MHmMr2Cl = -0.5 * mMr2Cl;

Add22(&r3h, &r3l, r2PHr2h, r2PHr2l, MHmMr2Ch, MHmMr2Cl);
\end{lstlisting}

The next iteration step provides enough accuracy for a double-double result.
We rewrite the basic iteration equation once again as:
\begin{eqnarray*}
r_4 & = & \frac{1}{2} \cdot r_3 \cdot \left( 3 - m \cdot r_3^2 \right) \\
& = & r_3 \cdot \left( \frac{3}{2} - \frac{1}{2} \cdot m \cdot r_3^2 \right) \\
& = & r_3 \cdot \left( \frac{3}{2} - \frac{1}{2} \cdot \left( \left( m \cdot r_3^2 - 1 \right) + 1 \right) \right) \\
& = & r_3 \cdot \left( 1 - \frac{1}{2} \cdot \left( m \cdot r_3^2 - 1 \right) \right)
\end{eqnarray*}
Further, we know that $r_3$, stored as a double-double, verifies $r_3
= \hat{r} \cdot \left( 1 + \epsilon_3 \right)$ with $\left \vert
\epsilon_3 \right \vert \leq 2^{-60}$. So we check that
$$m \cdot r_3^2 = m \cdot \hat{r}^2 \cdot \left( 1 +
\epsilon_3 \right)^2 = 1 + 2 \cdot \epsilon_3 + \epsilon_3^2$$
Clearly, $\left \vert 2 \cdot \epsilon_3 + \epsilon_3^2 \right \vert <
\frac{1}{2} \mUlp\left( 1 \right)$. So when squaring $r_{3\hi} + r_{3\lo}$ in double-double precision
and multiplying it in double-double precision by $m$ produces a
double-double $mMr3Sq_\hi + mMr3Sq_\lo = m \cdot \left( r_{3\hi} +
r_{3\lo} \right)^2 \cdot \left( 1 + \epsilon \right)$, $\left \vert
\epsilon \right \vert \leq 2^{-100}$ such that $mMr3Sq_\hi = 1$ in all
cases.

So we can implement the iteration equation 
$$r_4 = r_3 \cdot \left( 1 - \frac{1}{2} \cdot \left( m \cdot r_3^2 - 1 \right) \right)$$
as follows:
\begin{lstlisting}[caption={Newton iteration - second double-double step},firstnumber=1]
Mul22(&r3Sqh, &r3Sql, r3h, r3l, r3h, r3l); 
Mul22(&mMr3Sqh, &mMr3Sql, m, 0, r3Sqh, r3Sql);

Mul22(&r4h, &r4l, r3h, r3l, 1, -0.5 * mMr3Sql);
\end{lstlisting}
We since get $r_{4\hi} + r_{4\lo} = \hat{r} \cdot \left( 1 +
\epsilon_4 \right)$ with $\left \vert \epsilon_4 \right \vert \leq
2^{-102}$, the accuracy being limited by the accuracy of the last
double-double multiplication operator.

This approximation is than multiplied by $m$ in double-double
precision, leading to an approximation $srtm_\hi + srtm_\lo = \sqrt{m}
\cdot \left( 1 + \epsilon \right)$ with $\left \vert \epsilon \right \vert \leq
2^{-100}$.

Out of this value, the square root of the initial argument can be
reconstructed by multiplying by $2^{\frac{E}{2}}$, which has already
been stored in $xdb.d$. This multiplication is exact because it cannot
produce a subnormal.

These two steps are implemented as shown below:
\begin{lstlisting}[caption={Multiplication $m \cdot \hat{r}$, reconstruction},firstnumber=1]
Mul22(&srtmh,&srtml,m,0,r4h,r4l);

/* Multiply componentwise by sqrt(2^E), which is an integer power of 2 that may not produce a subnormal */

*resh = xdb.d * srtmh;
*resl = xdb.d * srtml;
\end{lstlisting}

\subsubsection{Triple-double Newton iteration}
For producing a triple-double approximate to $\hat{r}$ with an
accuracy of at least $147$ bits, one more Newton iteration is
needed. We apply the same equation as in the last double-double step,
which reads:
$$r_5 = r_4 \cdot \left( 1 - \frac{1}{2} \cdot \left( m \cdot r_4^2 -
1 \right) \right)$$ Once again, the first component of the
triple-double number holding an approximation to $m \cdot r_4^2$ is
exactly equal to $1$. So by neglecting this component, we substract $1$ from it.
Unfortunately, a renormalization step is needed after the multiplications for
squaring $r_4$ and by $m$ because the values computed might be overlapped which would prevent us
form substracting $1$ by neglecting a component.

We implement thus:
\begin{lstlisting}[caption={Newton iteration - triple-double step},firstnumber=1]
Mul23(&r4Sqh, &r4Sqm, &r4Sql, r4h, r4l, r4h, r4l);                               
Mul133(&mMr4Sqhover, &mMr4Sqmover, &mMr4Sqlover, m, r4Sqh, r4Sqm, r4Sql);        
Renormalize3(&mMr4Sqh, &mMr4Sqm, &mMr4Sql, mMr4Sqhover, mMr4Sqmover, mMr4Sqlover);
                                                                                                                 
HmMr4Sqm = -0.5 * mMr4Sqm;                                                            
HmMr4Sql = -0.5 * mMr4Sql;                                                            
                                                                                        
Mul233(&r5h,&r5m,&r5l,r4h,r4l,1,HmMr4Sqm,HmMr4Sql);                              
\end{lstlisting}

This approximation $r_{5\hi} + r_{5\mi} + r_{5\lo} = \hat{r} \cdot
\left( 1 + \epsilon_5 \right)$, where $\left \vert \epsilon_5 \right
\vert \leq 2^{-147}$ is then multiplied by $m$ in order to obtain a
triple-double approximation of $\sqrt{m}$. Once renormalized result is
exactly multiplied by $2^{\frac{E}{2}}$ stored in $xdb.d$.  We
implement:
\begin{lstlisting}[caption={Newton iteration - triple-double step},firstnumber=1]
Mul133(&srtmhover, &srtmmover, &srtmlover,m,r5h,r5m,r5l);                     
                                                                                     
Renormalize3(&srtmh,&srtmm,&srtml,srtmhover,srtmmover,srtmlover);

(*(resh)) = xdb.d * srtmh;
(*(resm)) = xdb.d * srtmm;                                                         
(*(resl)) = xdb.d * srtml;                                                         
\end{lstlisting}

\subsubsection{Accuracy bounds}

TODO: see possibly available Gappa files meanwhile


