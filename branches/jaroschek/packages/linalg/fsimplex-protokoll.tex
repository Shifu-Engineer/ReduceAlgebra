% ----------------------------------------------------------------------
% $Id: protokoll.tex,v 1.4 2001/07/19 11:10:34 sturm Exp $
% ----------------------------------------------------------------------
% Copyright (c) 2001 Dolzmann, Gilch, Sturm, Tsirou
% ----------------------------------------------------------------------
% $Log: protokoll.tex,v $
% Revision 1.4  2001/07/19 11:10:34  sturm
% Latest results. Still computing ...
%
% Revision 1.3  2001/07/18 19:05:51  sturm
% Latest data. Still computing.
%
% Revision 1.2  2001/07/16 17:47:40  sturm
% Under construction.
%
% Revision 1.1  2001/07/16 12:32:21  gilch
% Initial check-in.
%
%
% ----------------------------------------------------------------------

\documentclass[a4paper]{article}
\usepackage[latin1]{inputenc}
\usepackage[german]{babel}
\usepackage{times}
\setlength{\parindent}{0pt}
\setlength{\parskip}{6pt plus 2pt minus 1pt}

\begin{document}
\section{Versionsbeschreibungen}
\begin{description}
\item[Version 1.1:]
	Urspr�nglich in Reduce enthaltenes Modul Simplex. Es wird gerundet 
	gerechnet und Matrizen werden als Listen von Listen mit mat-Tag 
	abgespeichert, die Zahlenwerte wurden als \textit{rounded} bzw. als 
	Integers gespeichert. \\
	Problem: \textit{reval}-Aufrufe ben�tigen mindestens zwei Drittel 
	der Gesamtlaufzeit. \textit{reval} wird vor allem bei Matrixoperationen
	benutzt.
\item[Version 1.2:]
	Jeder Aufruf eines ''reval'' wurde ausgelagert in eine 
	eigenst�ndige Prozedur. Dabei wurden auch indirekte reval-Aufrufe 
	wie ''my\_reval'' und ''my\_revlis'' durch eigene Prozeduren ersetzt. 
	In den urspr�nglichen im Modul ''Simplex'' enthaltenen Prozeduren 
	existiert somit kein direkter reval-Aufruf mehr. 
  	Fazit: Matrixmultiplikation und Skalarprodukt ben�tigen jeweils 
	mindenstens 20\% der Laufzeit.
	Problem: Uneinheitliche Darstellung der Zahlenwerte.
\item[Version 1.3:]
	Alle Zahlen in den Argumenten werden in ''rounded'' dargestellt.
	Fazit: Verlangsamt die Laufzeit unwesentlich gering.
\item[Version 1.4:]
	Die ''rounded''-Umwandlung aus Version 1.3 wurde wieder entfernt. 
	Zus�tzlich wurden eigene Zugriffsprozeduren auf SQ-Formen eingef�gt mit
	der Form \textit{sc\_proc}. 
\item[Version 1.5:]
	Es wurde die interne Repr�sentation der Argumente auf SQ-Form 
	umgestellt. Diese Umstellung geschieht vor dem Betreten der Prozedur 
	''initialise''. Dabei wurden alle relevanten Prozeduren so upgedatet, 
	dass sie mit der neuen Struktur der Argumente umgehen k�nnen. Teilweise   	wurden eigene zus�tzliche Prozeduren geschrieben, die alte ersetzen, 
	und einige ''alte'' Prozeduren werden �berfl�ssig: \\
	\\
	\begin{tabular}{ll}
	Prozedur & wurde ersetzt durch neue Prozedur \\
	\hline
	fast\_getmat & sc\_getmat \\
	fast\_setmat & sc\_setmat \\
	fast\_matrix\_augment & sc\_matrix\_augment \\
	two\_column\_scalar\_product & sc\_transpose, sc\_scalarproduct \\
	scalar\_product & sc\_scalarproduct \\
	\end{tabular} \\
	\\
	Folgende Prozeduren sind nun �berfl�ssig und werden nicht mehr
	aufgerufen: 
	\begin{itemize}
	\item fast\_my\_letmtr
	\item fast\_unchecked\_getmatelem 
	\item scalar\_product 
	\item smplx\_prepsq 
	\item two\_column\_scalar\_product 
	\end{itemize}
	Alle vorkommenden Matrizen sind nun Listen von Listen ohne 'mat-Tag.
	Fazit: Afiro ist nun doppelt so schnell, Example 6.3. nun 22 mal so 
	schnell. Das ganze Programm l�uft nun schneller, der Zeitgewinn 
	beschr�nkt sich dabei nicht auf einzelne Prozeduren. Dabei verbleibt 
	ein gro�er Teil der Laufzeit in der Prozedur \textit{sc\_getmat}.
\item[Version 1.6:]
	Patches
\item[Version 1.7:]
	\textit{sc\_matrixproduct}-Aufruf wurde ersetzt durch Prozedur 
	\textit{sc\_multm}, die auf die Prozedur \textit{multm} im Modul 
	\textit{matsm} im Package \textit{Matrix} zur�ckgreift.
	Fazit: Beschleunigung um ca. 15\%
\item[Version 1.8:]
	Matrizen werden nun als Vektoren von Vektoren dargestellt. Es wird nun
	mit genauer Arithmetik gerechnet.
	Fazit: Beschleunigung zu Version 1.7 um Faktor 15\%, im Vergleich zu 
	Version 1.0 sogar um Faktor 500-600%. 
\item[Version 1.9:]
	Die Prozeduren \textit{putv} und \textit{getv} wurden ersetzt durch
	\textit{iputv} und \textit{igetv}. �berarbeitung und Verbesserung der 
	Matrixprozeduren. 
\item[Version 1.10:]
	�berarbeitung der Matrixprozeduren und Auslagerung der Aufrufe von 
	\textit{iputv} und \textit{igetv} in Prozeduren \textit{sc\_iputv} 
	und \textit{sc\_igetv}.
	Fazit: Afiro l�uft nun um Faktor 9,10 schneller, Example 6.3 um Faktor
	77,6 und Adlittle um Faktor 1415,6.
\item[Versuch:]
	Es wurde versucht, die Matrizen als Listen von Listen ohne mat-Tag 
	darzustellen, wobei einmal die Matrizen transponiert abgespeichert 
	wurden, oder die Matrizen mit Cache gespeichert wurden oder 
	transponiert mit Cache. Dabei wurden in der Cache-Darstellung der 
	Matrizen folgendes Konstruktionsprinzip verwendet: Eine Matrix ist ein
	Paar, wobei im \textit{cdr} die eigentliche Matrix als Liste von Listen 	ohne mat-Tag steht, und im \textit{car} ein Paar von Paaren steht, 
	wobei das erste Paar die aktuelle Zeilennummer angibt und die aktuelle 		Zeile, und das zweite Paar die aktuelle Spalte mit der Restzeile der 
	aktuellen Zeile von diesem Index aus.
	Die zugeh�rigen Matrixprozeduren wurden in eigene Module ausgelagert,
	so da� eigentliche Simplex-Code nicht ver�ndert werden musste.
	Keine dieser Versuche f�hrte zu einem besseren Ergebnis im Vergleich 
	zur Vektoren-Darstellung.
\end{description}

\newpage

\section{Beispiel-Laufzeiten der verschiedenen Versionen}

\begin{table}[h]
\caption{CPU-Zeiten (in s) und Speedup-Faktoren f"ur Netlib
Benchmarks. Gemessen auf 800 MHz Athlon.}
\begin{small}
\begin{tabular}{lrrrrrrr}
\hline
 & afiro & Kappert~6.3 & adlittle & share2b & beaconfd & israel & e226\\
\hline
\textbf{v1.1} & \textbf{11.6} \textbf{(1.0)} & \textbf{45.0}
\textbf{(1.0)} &  \textbf{25.9h} \textbf{(1.0)} &\textbf{$>$72h ($>$1.0)}
& \textbf{--} & \textbf{--} & \textbf{--} \\
v1.2 & 11.5 (1.0) & -- & -- & -- & -- & -- & -- \\
v1.3 & 11.7 (1.0) & -- & --  & -- & -- & -- & -- \\
v1.4 & -- & -- & -- & -- & -- & -- & -- 	\\
v1.5 & 6.3 (1.84) & -- & -- & -- & -- & -- & -- \\
v1.6 & 6.3 (1.84) & 2.1 (21.4) & -- & -- & -- & -- & --  \\
\textbf{v1.7} & \textbf{5.3} \textbf{(2.2)} & \textbf{1.8}
\textbf{(25.0)} & \textbf{436.5 (213.6)} & \textbf{928.0 ($>$279.3)} & \textbf{26104.5} &
\textbf{20973.2} & \textbf{106888.8}  \\
v1.8 & 2.0 (5.8) & 0.9 (50.0) & -- & -- & -- & -- & --  \\
v1.9 & 1.6 (7.25) & 0.7 (64.3) & 77.7 (1200.0) & -- & -- & -- & --  \\
\textbf{v1.10} & \textbf{1.3} \textbf{(8.9)} & \textbf{0.6}
\textbf{(75.0)} & \textbf{67.1} \textbf{(1389.6)} & \textbf{173.9 ($>$1490.5)} &
\textbf{--} & \textbf{3134.2} & \textbf{12258.7} \\
\hline
\end{tabular}
\end{small}
\end{table}

Das Beispiel \textit{Kappert~6.3} ist der Diplomarbeit von Michael
Kappert entnommen. Die "ubrigen Beispiele finden sich unter diesen
Namen in der Section \texttt{lp} der \emph{Netlib Repository at UTK and
ORNL\footnote{\texttt{http://www.netlib.org/}}}.

In Klammern sind die Faktoren, um welchen die jeweilige Version im
Vergleich zu Version 1.0 beschleunigt.

\section{Neuer Reduce Simplex vs.~Kokurrenzsysteme}
\begin{table}[h]
\begin{small}
\caption{CPU-Zeiten (in s) f"ur Netlib
Benchmarks. Gemessen auf 660 MHz Pentium; v.1.10-Zeiten nach
\texttt{israel} skaliert. Minos-Zeiten auf IBM 3081K (1985).}
\begin{tabular}{lrrrrrrr}
\hline
 & afiro & Kappert~6.3 & adlittle & share2b & beaconfd & israel & e226\\
\hline
v1.10/1.11
 &\textit{1.7}&\textit{0.8}&\textit{86.1}&\textit{223.1}&--&3530.4&\textit{15724.9}\\
v1.10/1.11 (rounded) & 1.8 & 0.8& 73.8 & 140.0&--&2134.7&unbounded\\
Maple (rounded) &1.0&0.3&15.0&107.7&(1740.5)&822.6&(8138.5)\\
Mathematica &0.03&0.02&19.7&55.5&&1130.7&13218.1\\
Mathematica (rounded) &0.02&--&1.3&1.3&10.6&16.1&75.1\\
\hline
Minos (rounded) &0.5&--&1.3&1.3&1.9&5.0&9.4\\
\hline
\end{tabular}
\end{small}
\end{table}

\section{Qualified-Timing auf alle Prozeduren, die von ''simplex\_calculation'' augerufen werden in Version 1.1}

Bei AFIRO: 
\\
\begin{tabular}{lrrr}
Prozedur & Anzahl Aufrufe & Zeit & \% \\
\hline
simplex\_calculation & 2 & 11590 & 98\% \\
compute\_objective & 61 & 80 & 0\% \\
fast\_add\_rows & 3422 & 1210 & 10\% \\ 
fast\_augment\_columns & 5185 & 840 & 7\% \\
fast\_mult\_rows & 59 & 180 & 1\% \\ 
fast\_setmat & 10803 & 10 & 0\% \\
fast\_unchecked\_getmatelem & 755915 & 1460 & 11\% \\
get\_cb & 122 & 10 & 0\% \\
phiprm & 59 & 3750 & 31\% \\
reval & 20694 & 7560 & 65\% \\
rstep1 & 61 & 5780 & 48\% \\
rstep2 & 59 & 240 & 2\% \\
rstep3 & 59 & 5870 & 49\% \\
scalar\_product & 59 & 10 & 0\% \\
two\_column\_scalar\_product & 5004 & 350 & 2\% \\
\end{tabular}
\\
\\
Bei Ex. 6.3:
\\
\textit{reval} 16702 Calls, 45200ms, 95% \\

\section{Qualified-Timing auf ausgelagerte reval-Prozeduren in Version 1.2}

\begin{tabular}{lrrr}
Prozedur & Anzahl Aufrufe & Zeit & \% \\
\hline
maxmin\_reval & 1 & 0 & 0\% \\
obj\_reval & 1 & 0 & 0\% \\
eq\_reval & 1 & 10 & 0\% \\
scalar\_product\_reval & 3483 & 2440 & 20\% \\
minus\_reval & 0 & 0 & 0\% \\
scalar\_reval & 0 & 0 & 0\% \\
matrelem\_reval & 35346 & 250 & 2\% \\
simple\_reval & 3481 & 230 & 1\% \\
matrix\_product\_reval & 120 & 2480 & 20\% \\
product\_reval & 3481 & 150 & 1\% \\
list\_reval & 21606 & 10 & 0\% \\
\end{tabular} \\
\\
Die Prozeduren ''minus\_reval'' und ''scalar\_reval'' werden nur aufgerufen, wenn ein Maximierungsproblem behandelt wird. Deshalb wurden sie beim AFIRO-Beispiel nicht aufgerufen.

\section{Qualified-Timing auf alle Prozeduren in Version 1.6}

\begin{tabular}{lrrr}
Prozedur & Anzahl Aufrufe & Zeit & \% \\
\hline
add\_not\_defined\_variables & 1 & 0 & 0\% \\
add\_slacks\_to\_equations & 1 & 0 & 0\% \\
check\_minus\_b & 1 & 0 & 0\% \\
compute\_objective & 61 & 0 & 0\% \\
create\_phase1\_A1\_and\_obj\_and\_ib & 1 & 0 & 0\% \\
eq\_reval & 1 & 0 & 0\% \\
fast\_add\_rows & 3422 & 1780 & 28\% \\
fast\_augment\_columns & 5185 & 890 & 14\% \\
fast\_make\_identity & 2 & 0 & 0\% \\
fast\_mult\_rows & 59 & 0 & 0\% \\
get\_cb & 122 & 10 & 0\% \\
get\_preliminary\_variable\_list & 1 & 0 & 0\% \\
get\_X\_and\_obj\_mat & 1 & 0 & 0\% \\
initialise & 1 & 190 & 3\% \\
list\_reval & 0 &  &  \\
make\_answer\_list & 1 & 0 & 0\% \\
make\_equations\_positive & 1 & 0 & 0\% \\
matrelem\_reval & 0 &  &  \\
matrix\_product\_reval & 0 &  &  \\
maxmin\_reval & 1 & 0 & 0\% \\
minus\_reval & 0 &  &  \\
more\_initialise & 1 & 220 & 3\% \\
my\_round & 0 &  &  \\
obj\_reval & 1 & 0 & 0\% \\
phiprm & 59 & 2200 & 34\% \\
product\_reval & 0 &  &  \\
rstep1 & 61 & 3020 & 48\% \\
rstep2 & 59 & 10 & 0\% \\
rstep3 & 59 & 3180 & 50\% \\
scalar\_product\_reval & 0 &  &  \\
scalar\_reval & 0 &  &  \\
sc\_addsq & 212539 & 50 & 0\% \\
sc\_denr & 8953 & 10 & 0\% \\
sc\_geq & 5237 & 20 & 0\% \\
sc\_getmat & 3093342 & 3540 & 41\% \\
sc\_getvar & 59 & 0 & 0\% \\
sc\_getvarl & 0 &  &  \\
sc\_matrixaugment & 1 & 0 & 0\% \\
\end{tabular} \\

\newpage

\begin{tabular}{lrrr}
Prozedur & Anzahl Aufrufe & Zeit & \% \\
\hline
sc\_matrixproduct & 59 & 970 & 15\% \\
sc\_minussq & 8838 & 10 & 0\% \\
sc\_mkmatrix & 12216 & 160 & 2\% \\
sc\_multm & 61 & 50 & 0\% \\
sc\_multrow & 0 &  &  \\
sc\_multsq & 210965 & 50 & 0\% \\
sc\_negmatrix & 61 & 40 & 0\% \\
sc\_negsq & 3454 & 10 & 0\% \\
sc\_null & 38400 & 30 & 0\% \\
sc\_numr & 17965 & 0 & 0\% \\
sc\_prepsq & 5013 & 20 & 0\% \\
sc\_quotsq & 292 & 0 & 0\% \\
sc\_scalprod & 0 &  &  \\
sc\_scalarproduct & 12027 & 1580 & 25\% \\
sc\_setmat & 10803 & 30 & 0\% \\
sc\_simp & 4980 & 20 & 0\% \\
sc\_subtrsq & 5237 & 20 & 0\% \\
sc\_transpose & 12088 & 990 & 15\% \\
simp\_get\_A & 1 & 180 & 2\% \\
simple\_reval & 0 &  &  \\
simplex1 & 1 & 6340 & 101\% \\
simplex\_calculation & 2 & 6230 & 98\% \\
unique\_equation\_list & 1 & 0 & 0\% \\
\end{tabular} \\
\\
\\
Die Prozedur \textit{sc\_multrow} wurde hier von \textit{check\_minus\_b} nicht aufgerufen.
\newpage

\section{Tests mit Version 1.6}

\begin{tabular}{lrr}
Testbeispiel & Zeit V.1.6 & V.1.0\\
\hline
Example 6.3 & 2110 ms & 44630 ms \\
Example 6.4 & 50 ms & 110 ms \\
Example 6.7 (max) & 170 ms & 180 ms \\
Example 6.7 (min) & 150 ms & 200 ms \\
\hline
AFIRO mit \textit{off rounded} & 6180 ms & 6310 ms \\
\end{tabular}\\
\\
Einige Laufzeiten von Prozeduren bei Example 6.3: 
\\
\begin{tabular}{lrrr}
Prozedur & Calls & Zeit & \% \\
fast\_add\_rows & 2546 & 640 & 28 \% \\
fast\_augment\_columns & 3334 & 190 & 8 \% \\
sc\_getmat & 1473873 & 1150 ms & 34 \% \\
sc\_matrixproduct & 67 & 350 & 16 \% \\
sc\_scalarproduct & 8424 & 560 ms & 26 \% \\
sc\_transpose & 8493 & 360 & 16 \% \\
\end{tabular}
\\
\\
Die obigen Beispiele wurden der Diplomarbeit von \textit{M.Kappert} entnommen. Bei \textit{AFIRO} wurde dabei die Zeit getestet, wenn nicht \textit{rounded} eingeschaltet wird.

\section{Laufzeiten von AFIRO und Ex.6.3 bei V.1.7}

Afiro -- Example 6.3 \\

\begin{tabular}{l|rrr|rrr}
Prozedur & Calls & Zeit & \% & Calls & Zeit & \% \\
copy\_vect & 3423 & 20 & 0 \% & 2547 & 20 & 1 \% \\
fast\_add\_rows & 3422 & 30 & 1 \% & 2546 & 30 & 2 \% \\
fast\_augment\_columns & 5185 & 130 & 5 \% & 3334 & 30 & 2 \% \\
phiprm & 59 & 150 & 6 \% & 67 & 90 & 8 \% \\
rstep1 & 61 & 1440 & 66 \% & 69 & 690 & 66 \% \\
rstep2 & 59 & 10 & 0 \% & 67 & 0 & 0 \% \\
rstep3 & 59 & 460 & 20 \% & 67 & 260 & 25 \% \\
sc\_getmat & 4174104 & 1430 & 26 \% & 2010123 & 700 & 29 \% \\
sc\_matrixproduct & 120 & 770 & 36 \% & 136 & 360 & 37 \% \\
sc\_mkmatrix & 9598 & 480 & 21 \% & 14227 & 210 & 20 \% \\
sc\_scalarproduct & 15626 & 380 & 17 \% & 11115 & 160 & 15 \% \\
sc\_setmat & 1585218 & 600 & 12 \% & 767138 & 250 & 15 \% \\
sc\_transpose & 19286 & 720 & 32 \% & 13875 & 440 & 41 \% \\
simplex\_calculation & 2 & 1930 & 90 \% & 2 & 960 & 91 \% \\
\end{tabular}

\newpage

\section{Vergleich der verschiedenen Matrizenmodule(Version 1.8)}

Es werden die vorkommenden Matrizen unterschiedlich dargestellt:
\begin{itemize}
\item als Listen von Listen
\item als Listen von Listen, wobei die vorkommenden Matrizen transponiert
	gespeichert werden
\item als Listen von Listen mit Cache, wobei die aktuelle Zeile gespeichert 
 	wird und die aktuelle Restzeile vom aktuellen Spaltenindex weg
\item als Listen von Listen, mit Cache und transponiert abgespeichert
\item als Vektoren von Vektoren
\end{itemize}


\begin{tabular}{lccc}
Modul & AFIRO & Ex. 6.3 & Adlittle \\
\hline
Listen & 5130 ms & 1760 ms & 456880 ms (plus GC time: 22970 ms)  \\
Transponierte Listen & 7700 ms & 2650 ms & 627830 ms (plus GC time: 22510 ms)\\
gecachte Listen & 5080 ms & 1770 ms & 430430 (plus GC time: 22890 ms) \\
gec. transp. Listen & 6560 ms & 2400 ms & 472810 (plus GC time: 22160 ms)  \\
Vektoren & 2030 ms & 900 ms & 91970 ms (plus GC time: 2300 ms)  \\
\end{tabular}
\\
\\



\end{document}