

\subsection{Introduction}

Involutive polynomial bases are redundant Gr\"obner bases of special structure with some additional useful features in comparison 
with reduced Gr\"obner bases~\cite{GB'98}. Apart from numerous applications of involutive bases~\cite{Seiler'10} the 
involutive algorithms~\cite{Gerdt'05} provide an efficient method for computing reduced Gr\"obner bases. A reduced Gr\"obner basis 
is a well-determined subset of an involutive basis and can be easily extracted from the latter without any extra reductions. 
All this takes place not only in rings of commutative polynomials but also in Boolean rings.

Boolean Gr\"obner basis already have already revealed their value and usability in  practice. The first impressive demonstration
of practicability of Boolean Gr\"obner bases was breaking the first HFE (Hidden Fields Equations) challenge in the public
key cryptography done in~\cite{Faugere'03} by computing a Boolean Gr\"obner basis for the system of quadratic
polynomials in 80 variables. Since that time the Boolean Gr\"obner bases application area has widen drastically and nowadays there 
is also a number of quite successful examples of using Gr\"obner bases for solving SAT problems.

During our research we had developed~\cite{ISSAC'08, PaCS'08, PCA'09} Boolean involutive algorithms based on Janet and Pommaret 
divisions and applied them to computation of Boolean Gr\"obner bases. Our implementation of both divisions has experimentally 
demonstrated computational superiority of the Pommaret division implementation. This package BIBASIS is the result
of our thorough research in the field of Boolean Gr\"obner bases. BIBASIS implements the involutive algorithm based on Pommaret 
division in a multivariate Boolean ring.

In section 2 the Boolean ring and its peculiarities are shortly introduced. In section 3 we briefly argue  
why the involutive algorithm and Pommaret division are good for Boolean ring while the Buhberger's algorithm is not.
And finally in section 4 we give the full description of BIBASIS package capabilities and illustrate it by examples.

\subsection{Boolean Ring}

{\em Boolean ring} perfectly goes with its name, it is a ring of {\em Boolean functions} of $n$ variables, i.e
mappings from $\{0,1\}^n$ to $\{0,1\}^n$. Considering these variables are $\mathbf{X}:=\{x_1,\ldots,x_n\}$ and
$\mathbb{F}_2$ is the finite field of two elements $\{0,1\}$, Boolean ring can be regarded as the quotient ring
$$
\mathbb{B\,}[\mathbf{X}]:=\mathbb{F}_2[\mathbf{X}]\,/<x_1^2+x_1,\ldots,x_n^2+x_n>.
$$
Multiplication in $\mathbb{B\,}[\mathbf{X}]$ is {\em idempotent} and addition is {\em nilpotent}
$$
\forall\, b\in \mathbb{B\,}[\mathbf{X}]\ :\ \,b^2=b\,,\ b+b=0.
$$
Elements in $\mathbb{B\,}[\mathbf{X}]$ are {\em Boolean polynomials} and can be represented as finite sums
$$
\sum_j \prod_{x\in\, \Omega_j \subseteq\, \mathbf{X}} x
$$
of {\em Boolean monomials}. Each monomial is a conjunction. If set $\Omega$ is empty, then the corresponding
monomial is the unity Boolean function 1. The sum of zero monomials corresponds to zero polynomial, i.e. is
zero Boolean function 0.

\subsection{Pommaret Involutive Algorithm}

Detailed description of involutive algorithm can found in~\cite{Gerdt'05}. Here we note that result of both
involutive and Buhberger's algorithms depend on chosen monomial ordering. At that the ordering must be
admissible, i.e.
$$
\ m \neq 1 \Longleftrightarrow m \succ 1, \quad \ \ m_1 \succ m_2 \Longleftrightarrow m_1 m \succ m_2 m
\quad \ \ \forall \ m, m_1, m_2.
$$
But as one can easily check the second condition of admissibility does not hold for any monomial ordering
in Boolean ring:
$$
x_1\succ x_2\quad \xrightarrow{\ \ *x_1\ \ }\quad x_1*x_1\succ x_2*x_2\quad \xrightarrow{\ \ \ \ \ }\quad x_1 \prec x_1x_2
$$
Though $\mathbb{B\,}[\mathbf{X}]$ is a principal ideal ring, boolean singleton $\{p\}$ is not necessarily a Gr\"obner basis of
ideal $<p>$, for example:
$$
x_1,x_2\in \,<x_1x_2 + x_1 + x_2> \subset \mathbb{B\,}[x_1, x_2].
$$
That the reason why one cannot apply the Buhberger's algorithm directly in a Boolean ring, using instead a ring
$\mathbb{F}_2[\mathbf{X}]$ and {\em the field binomials} $x_1^2+x_1,\ldots,x_n^2+x_n$.

The involutive algorithm based on Janet division has the same disadvantage unlike the Pommaret division algorithm as shown 
in~\cite{ISSAC'08}. The Pommaret division  algorithm can be applied directly in a Boolean ring and admits effective data 
structures for monomial representation.

\subsection{BIBASIS Package}

The package BIBASIS implements the Pommaret division algorithm in a Boolean ring. The first step to using the package
is to load it:
\begin{verbatim}
    1: load_package bibasis;
\end{verbatim}
The current version of the BIBASIS user interface consists only of 2 functions:  
\texttt{bibasis} and
\texttt{bibasis\_print\_statistics}.

\ttindex{BIBASIS}
\noindent The \f{bibasis} is the function that performs all the computation and has the following syntax:
\begin{center}
    \texttt{bibasis(initial\_polynomial\_list, variables\_list, monomial\_ordering, reduce\_to\_groebner);}
\end{center}
Input:
\begin{itemize}
    \item \texttt{initial\_polynomial\_list} is the list of polynomials containing the known basis of initial
Boolean ideal. All given polynomials are treated modulo 2. See Example 1.
    
    \item \texttt{variables\_list} is the list of independent variables in decreasing order.
    
    \item \texttt{monomial\_ordering} is a chosen monomial ordering and the supported ones are:
        \begin{itemize}
            \item[] \texttt{lex} -- pure lexicographical ordering;
            \item[] \texttt{deglex} -- degree lexicographic ordering;
            \item[] \texttt{degrevlex} -- degree reverse lexicographic.
        \end{itemize}
        See Examples 2---4 to check that Gr\"obner (as well as involutive) basis depends on monomial ordering.
        
    \item \texttt{reduce\_to\_groebner} is a Boolean value, if it is \texttt{t} the output is the reduced
        Boolean Gr\"obner basis, if \texttt{nil}, then the reduced Boolean Pommaret basis. Examples 5,6 show distinctions between these two outputs.
\end{itemize}
Output:
\begin{itemize}
 \item The list of polynomials which constitute the reduced Boolean Gr\"obner or Pommaret basis.
\end{itemize}

\ttindex{bibasis\_print\_statistics}
\noindent The syntax of \texttt{bibasis\_print\_statistics} is simple:
\begin{center}
    \texttt{bibasis\_print\_statistics();}
\end{center}
This function prints out a brief statistics for the last invocation of \texttt{bibasis} function. See Example 7.

\subsection{Examples}


Example 1:
\begin{verbatim}
    1: load_package bibasis;
    2: bibasis({x+2*y}, {x,y}, lex, t);
    {x}

\end{verbatim}

\noindent Example 2:
\begin{verbatim}
1: load_package bibasis;
2: variables :={x0,x1,x2,x3,x4}$
3: polynomials := {x0*x3+x1*x2,x2*x4+x0}$
4: bibasis(polynomials, variables, lex, t);
{x0 + x2*x4,x2*(x1 + x3*x4)}

\end{verbatim}

\noindent Example 3:
\begin{verbatim}
1: load_package bibasis;
2: variables :={x0,x1,x2,x3,x4}$
3: polynomials := {x0*x3+x1*x2,x2*x4+x0}$
4: bibasis(polynomials, variables, deglex, t);
{x1*x2*(x3 + 1),
 x1*(x0 + x2),
 x0*(x2 + 1),
 x0*x3 + x1*x2,
 x0*(x4 + 1),
 x2*x4 + x0}

\end{verbatim}

\noindent Example 4:
\begin{verbatim}
1: load_package bibasis;
2: variables :={x0,x1,x2,x3,x4}$
3: polynomials := {x0*x3+x1*x2,x2*x4+x0}$
4: bibasis(polynomials, variables, degrevlex, t);
{x0*(x1 + x3),
 x0*(x2 + 1),
 x1*x2 + x0*x3,
 x0*(x4 + 1),
 x2*x4 + x0}

\end{verbatim}

\newpage

\noindent Example 5:
\begin{verbatim}
1: load_package bibasis;
2: variables :={x,y,z}$
3: polinomials := {x, z}$
4: bibasis(polinomials, variables, degrevlex, t);
{x,z}

\end{verbatim}

\noindent Example 6:
\begin{verbatim}
1: load_package bibasis;
2: variables :={x,y,z}$
3: polinomials := {x, z}$
4: bibasis(polinomials, variables, degrevlex, nil);
{x,z,y*z}

\end{verbatim}

\noindent Example 7:
\begin{verbatim}
1: load_package bibasis;
2: variables :={u0,u1,u2,u3,u4,u5,u6,u7,u8,u9}$
3: polinomials := {u0*u1+u1*u2+u1+u2*u3+u3*u4+u4*u5+u5*u6+u6*u7+u7*u8+u8*u9,
3:                 u0*u2+u1+u1*u3+u2*u4+u2+u3*u5+u4*u6+u5*u7+u6*u8+u7*u9,
3:                 u0*u3+u1*u2+u1*u4+u2*u5+u3*u6+u3+u4*u7+u5*u8+u6*u9,
3:                 u0*u4+u1*u3+u1*u5+u2+u2*u6+u3*u7+u4*u8+u4+u5*u9,
3:                 u0*u5+u1*u4+u1*u6+u2*u3+u2*u7+u3*u8+u4*u9+u5,
3:                 u0*u6+u1*u5+u1*u7+u2*u4+u2*u8+u3+u3*u9+u6,
3:                 u0*u7+u1*u6+u1*u8+u2*u5+u2*u9+u3*u4+u7,
3:                 u0*u8+u1*u7+u1*u9+u2*u6+u3*u5+u4+u8,
3:                 u0+u1+u2+u3+u4+u5+u6+u7+u8+u9+1}$
4: bibasis(polinomials, variables, degrevlex, t);
{u3*u6,
 u3*u7,
 u7*(u6 + 1),
 u3*u8,
 u6*u8 + u6 + u7,
 u7*u8,
 u3*(u9 + 1),
 u6*u9 + u7,
 u7*(u9 + 1),
 u8*u9 + u6 + u7 + u8,
 u0 + u3 + u6 + u9 + 1,
 u1 + u7,
 u2 + u7 + u8,
 u4 + u6 + u8,
 u5 + u6 + u7 + u8}
5: bibasis_print_statistics();
        Variables order = u0 > u1 > u2 > u3 > u4 > u5 > u6 > u7 > u8 > u9
Normal forms calculated = 216
  Non-zero normal forms = 85
        Reductions made = 4488
Time: 270 ms
GC time: 0 ms

\end{verbatim}


\begin{thebibliography}{99}

\bibitem{GB'98} V.P.Gerdt and Yu.A.Blinkov. {\em Involutive Bases of Polynomial Ideals}. 
Mathematics and Computers in Simulation, 45, 519--542, 1998; {\em Minimal Involutive Bases}, ibid. 543--560.
 
\bibitem{Seiler'10} W.M.Seiler. {\em Involution: The Formal Theory of Differential Equations and its Applications 
in Computer Algebra}. Algorithms and Computation in Mathematics, 24, Springer, 2010.  arXiv:math.AC/0501111

\bibitem{Gerdt'05} Vladimir P. Gerdt. {\em Involutive Algorithms for Computing Gr\"obner Bases}.
Computational Commutative and Non-Commutative Algebraic Geometry. IOS Press, Amsterdam, 2005, pp.199--225.

\bibitem{Faugere'03}
J.-C.Faug\`{e}re and A.Joux. Algebraic Cryptanalysis of Hidden Field Equations
(HFE) Using Gr\"obner Bases. {\em LNCS} 2729, Springer-Verlag, 2003, pp.44--60.

\bibitem{ISSAC'08}
V.P.Gerdt and M.V.Zinin. A Pommaret Division Algorithm for Computing Gr\"obner Bases in Boolean Rings.
{\em Proceedings of ISSAC 2008}, ACM Press, 2008, pp.95--102.

\bibitem{PaCS'08}
V.P.Gerdt and M.V.Zinin. Involutive Method for Computing Gr\"obner Bases over $F_2$.
{\em Programming and Computer Software}, Vol.34, No. 4, 2008, 191--203.

\bibitem{PCA'09}
Vladimir Gerdt, Mikhail Zinin and Yuri Blinkov. On computation of Boolean involutive bases,
Proceedings of International Conference Polynomial Computer Algebra 2009, pp. 17-24
(International Euler Institute, April 7-12, 2009, St. Peterburg, Russia)


\end{thebibliography}
