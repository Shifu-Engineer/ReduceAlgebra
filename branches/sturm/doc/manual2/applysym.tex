\chapter[APPLYSYM: Infinitesimal symmetries]{APPLYSYM: Infinitesimal symmetries of differential equations}
\label{APPLYSYM}
\typeout{[APPLYSYM: Infinitesimal symmetries]}

{\footnotesize
\begin{center}
Thomas Wolf \\
School of Mathematical Sciences, Queen Mary and Westfield College \\
University of London \\
London E1 4NS, England \\[0.05in]
e--mail: T.Wolf@maths.qmw.ac.uk
\end{center}
}

The investigation of infinitesimal symmetries of differential equations
(DEs) with computer algebra programs attracted considerable attention
over the last years.  The package {\tt APPLYSYM} concentrates on the
implementation of applying symmetries for calculating similarity
variables to perform a point transformation which lowers the order of
an ODE or effectively reduces the number of explicitly occuring
independent variables of a PDE(-system) and for generalising given
special solutions of ODEs/PDEs with new constant parameters.

A prerequisite for applying symmetries is the solution of first order
quasilinear PDEs.  The corresponding program 
{\tt QUASILINPDE}\ttindex{QUASILINPDE} can as well be used without
{\tt APPLYSYM}\ttindex{APPLYSYM} for solving first order PDEs which are 
linear in their first order derivative and otherwise at most rationally
non-linear.  The following two PDEs are equations (2.40) and (3.12) 
taken from E. Kamke, "Loesungsmethoden und Loesungen von Differential-
gleichungen, Partielle Differentialgleichungen erster Ordnung", 
B.G. Teubner, Stuttgart (1979).
\newpage
{\small
\begin{verbatim}
------------------------ Equation 2.40 ------------------------

                                     2              3      4
The quasilinear PDE:  0 = df(z,x)*x*y  + 2*df(z,y)*y  - 2*x

      2          2  2
 + 4*x *y*z - 2*y *z .
The equivalent characteristic system:

              3    4      2        2  2
0=2*(df(z,y)*y  - x  + 2*x *y*z - y *z )

   2
0=y *(2*df(x,y)*y - x)

for the functions: x(y)  z(y)  .
The general solution of the PDE is given through

                4           2        2
        log(y)*x  - log(y)*x *y*z - y *z   sqrt(y)*x
0 = ff(----------------------------------,-----------)
                   4    2                      y
                  x  - x *y*z

with arbitrary function ff(..).

------------------------ Equation 3.12 ------------------------

The quasilinear PDE:  0 = df(w,x)*x + df(w,y)*a*x + df(w,y)*b*y

 + df(w,z)*c*x + df(w,z)*d*y + df(w,z)*f*z.
The equivalent characteristic system:

0=df(w,x)*x


0=df(z,x)*x - c*x - d*y - f*z


0=df(y,x)*x - a*x - b*y

for the functions: z(x)  y(x)  w(x)  .
The general solution of the PDE is given through

        a*x + b*y - y
0 = ff(---------------,( - a*d*x + b*c*x + b*f*z - b*z - c*f*x
           b      b
          x *b - x

                            2            f        f      f  2    f
           - d*f*y + d*y - f *z + f*z)/(x *b*f - x *b - x *f  + x *f)

       ,w)

with arbitrary function ff(..).
\end{verbatim}
}
The program {\tt DETRAFO}\ttindex{DETRAFO} can be used to perform
point transformations of ODEs/PDEs (and -systems).

For detailed explanations the user is
referred to the paper {\em Programs for Applying Symmetries of PDEs}
by Thomas Wolf, supplied as part of the Reduce documentation as {\tt
applysym.tex} and published in the Proceedings of ISSAC'95 - 7/95
Montreal, Canada, ACM Press (1995).

