\hoffset -.5cm
\voffset 2cm
%\documentstyle[12pt]{book}
%\pagestyle{empty}
\parindent=0pt
\addtolength{\topmargin}{-2cm}
\addtolength{\textheight}{1cm}
%\addtolength{\textwidth}{2cm}
%\setlength{\parskip}{2mm}
%\begin{document}
%\special{postscriptfile ziblogo.ps}
\thispagestyle{empty}
%\vspace*{.5cm}
\begin{center}
{\Large \bf The PSL Users Manual}\\[.5cm]
{\large Version 4.2}\\[1.0cm]
by \\[1.0cm]
{\bf Herbert Melenk} \  and \ {\bf Winfried Neun} \\[.5cm] 
{\normalsize
Konrad--Zuse--Zentrum f\"ur Informationstechnik Berlin\\
Division of Symbolic Computing\\
Takustrasse 7\\
D--14195 Berlin-Dahlem\\[0.5cm]
URL: http://www.zib.de\\[1.1cm]
based on earlier Versions\\[.5cm]
by\\[.5cm]
The Utah Symbolic Computation Group\\
Department of Computer Science\\
University of Utah\\
Salt Lake City, Utah 84112\\[.5cm]
and\\[.5cm]
Hewlett-Packard Company\\
Computer Research Center}
\end{center}
%\vfill
%\vspace*{2mm}
\begin{quote}
{\footnotesize
\begin{tabular}{llll}
Copyright & \copyright & 1997 & Konrad--Zuse--Zentrum Berlin \\
& & & University of Utah \\
& & & Hewlett--Packard Company \\
& & & All rights reserved.
\end{tabular}
}
\end{quote}
\vspace*{.5cm}
\begin{quote}
{\footnotesize Registered system holders may reproduce all or any part of
this publication for internal purposes, provided that the source of the
material is clearly acknowledged, and the copyright notice is retained.}
\end{quote}
\hrule
\vspace*{3mm}
(M 3004.02)

%\end{document}


\pagestyle{empty}
\newpage
\begin{center}
{ABSTRACT}\\[4mm]
\end{center}
\begin{quote}

This manual describes the primitive data structures, facilities and
functions present in the Portable Standard LISP
(PSL)  system.    It  describes  the  implementation details and
functions of interest to a PSL programmer.  Except for  a  small
number  of  hand-coded  routines  for I/O and efficient function
calling,  PSL  is  written   entirely   in   itself,   using   a
machine-oriented  mode  of PSL, called SYSLisp, to perform word,
byte, and efficient  integer  and  string  operations.   PSL  is
compiled  by  an enhanced version of the Portable LISP Compiler,
and currently runs on many platforms, from personal computers up
to super computers.
\end{quote}

\newpage
\tableofcontents
\newpage
\setcounter{page}{0}
\pagestyle{headings}
